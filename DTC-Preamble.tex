
%%%%%%% Standard Packages
\usepackage{amsmath}       % I think this gives me some symbols
\usepackage{amsthm}        % Does theorem stuff
\usepackage{amssymb}       % more symbols and fonts
\usepackage{amsfonts}
\usepackage[all]{xy}
\usepackage{xspace}
\usepackage{calc}

\usepackage{ifthen}		% For if-then-else statements in making commands. 

\setlength{\topskip}{0pt}
\setlength{\footskip}{30pt}
\headheight=0pt
\topmargin=0pt
\headsep=18pt
\textheight=603pt %% 792pt to page, 648 is 9in
\textwidth=420pt  %% 612pt to page, 468pt is 6.5in
\oddsidemargin=25pt
\evensidemargin=25pt

\pagestyle{plain}


%%%%%% Adds hyperlinks
\usepackage[colorlinks, linkcolor=black, citecolor=blue,
	% pagebackref,
 	%bookmarksnumbered=true
	]{hyperref}
	
	
	
%%%%%% Tikz !!! Commands and Macros %%%%%%%%%%%%%
\usepackage{tikz}
\usetikzlibrary{matrix}


%%%% These draw triple or quadruple set of arrows of length 0.5 cm
\DeclareMathOperator{\righttriplearrows} {{\; \tikz{ \foreach \y in {0, 0.1, 0.2} { \draw [-stealth] (0, \y) -- +(0.5, 0);}} \; }}
\DeclareMathOperator{\lefttriplearrows} {{\; \tikz{ \foreach \y in {0, 0.1, 0.2} { \draw [stealth-] (0, \y) -- +(0.5, 0);}} \; }}
\DeclareMathOperator{\rightquadarrows} {{\; \tikz{ \foreach \y in {0, 0.1, 0.2, 0.3} { \draw [-stealth] (0, \y) -- +(0.5, 0);}} \; }}
\DeclareMathOperator{\leftquadarrows} {{\; \tikz{ \foreach \y in {0, 0.1, 0.2, 0.3} { \draw [stealth-] (0, \y) -- +(0.5, 0);}} \; }}

%%%%%%% End TikZ Commands and Macros %%%%%%%%%%%%%



%%%%%%%%%%%%%%%%%%%%%% Theorem Styles and Counters %%%%%%%%%%%%%%%%%%%%%%%%%%
% These all use the same "theorem" counter. 
\theoremstyle{plain} %%% Plain Theorem Styles.
\newtheorem{theorem}{Theorem}[section]
\newtheorem{lemma}[theorem]{Lemma}
\newtheorem{corollary}[theorem]{Corollary}          
\newtheorem{proposition}[theorem]{Proposition}              

\theoremstyle{definition} %%%% Definition-like Commands  
\newtheorem{definition}[theorem]{Definition}

\theoremstyle{remark}  %%%% Remark-like Commands
\newtheorem{remark}[theorem]{Remark}
\newtheorem{example}[theorem]{Example}
\newtheorem{conjecture}[theorem]{Conjecture}
%%%%%%%%%%%%%%%%%%%%%% End Theorem Styles and Counters %%%%%%%%%%%%%%%%%%%%%%%%%%

%%% Special Commands %%%

%% Category of modules commands.
%% Syntax:  \Mod{A}{B}
%% Makes a math command for the category of left, right, and bimodules. If either A or B is empty, it will omit that portion. This uses the ifthen package.
\newcommand{\Mod}[2]  
{
  \ifthenelse{\equal{#1}{}}{  			% If A is empty ...
		\ifthenelse{\equal{#2}{}}		% And B is empty ...
			{\mathrm{Mod}}{ 			% just print "Mod"
				{\mathrrm{Mod}\textrm{-}#2}		% Else, print "Mod-B"
			}
	}{									% but if A is not empty,
		\ifthenelse{\equal{#2}{}}		% and B is empty,
			{{#1\textrm{-}\mathrm{Mod}}}{		% print "A-mod",
				{{#1\textrm{-}\mathrrm{Mod}\textrm{-}#2}}	% otherwise print "A-Mod-B".
			}
	}
}

%% Bimodule command.  Syntax \bimod{C}{M}{D}

\newcommand{\bimod}[3]
{{}_{#1} {#2}_{#3}}

%% Left and right dual commands.

\newcommand{\ld}[1]
{{}^*{#1}}
\newcommand{\rd}[1]
{{#1}^*}
\newcommand{\ldd}[1]
{{}^{**}{#1}}
\newcommand{\rdd}[1]
{{#1}^{**}}
\newcommand{\ldddd}[1]
{{}^{****}{#1}}
\newcommand{\rdddd}[1]
{{#1}^{****}}



%%%% Misc symbols %%%%%

\newcommand{\nn}{\nonumber}
\newcommand{\nid}{\noindent}
\newcommand{\ra}{\rightarrow}
\newcommand{\dra}{\Rightarrow}
\newcommand{\la}{\leftarrow}
\newcommand{\xra}{\xrightarrow}
\newcommand{\xla}{\xleftarrow}
\newcommand{\hra}{\hookrightarrow}
\newcommand{\lra}{\looparrowright}
\DeclareMathOperator{\Hom}{Hom}
\DeclareMathOperator{\hofib}{hofib}
\renewcommand{\mp}{mp}
\newcommand{\op}{op}
\newcommand{\mop}{mop}
\newcommand{\id}{id}


\newcommand{\dtimes}{\boxtimes} %Deligne tensor product
\newcommand{\adj}{\dashv} % Adjoint
\newcommand{\ambadj}{\vdash \dashv} %Ambi adjoint

%\newcommand{\om}{\mathrm{Hom}}

\newcommand{\Bord}{\mathrm{Bord}}
\newcommand{\FrBord}{\mathrm{FrBord}}
\newcommand{\StrBord}{\mathrm{StrBord}}
\newcommand{\OrpoBord}{\mathrm{OrpoBord}}
\newcommand{\Or}{Or}
\newcommand{\Orpo}{Orpo}
\newcommand{\Spin}{Spin}
\newcommand{\String}{String}
\newcommand{\Frame}{Frame}

\newcommand{\Ab}{\mathrm{Ab}}
\newcommand{\Algd}{\mathrm{Algd}}
\newcommand{\Cat}{\mathrm{Cat}}
\newcommand{\Vect}{\mathrm{Vect}}
\newcommand{\TC}{\mathrm{TC}}
\newcommand{\ev}{\mathrm{ev}}
\newcommand{\coev}{\mathrm{coev}}

\def\cA{\mathcal A}\def\cB{\mathcal B}\def\cC{\mathcal C}\def\cD{\mathcal D}
\def\cE{\mathcal E}\def\cF{\mathcal F}\def\cG{\mathcal G}\def\cH{\mathcal H}
\def\cI{\mathcal I}\def\cJ{\mathcal J}\def\cK{\mathcal K}\def\cL{\mathcal L}
\def\cM{\mathcal M}\def\cN{\mathcal N}\def\cO{\mathcal O}\def\cP{\mathcal P}
\def\cQ{\mathcal Q}\def\cR{\mathcal R}\def\cS{\mathcal S}\def\cT{\mathcal T}
\def\cU{\mathcal U}\def\cV{\mathcal V}\def\cW{\mathcal W}\def\cX{\mathcal X}
\def\cY{\mathcal Y}\def\cZ{\mathcal Z}

\def\AA{\mathbb A}\def\BB{\mathbb B}\def\CC{\mathbb C}\def\DD{\mathbb D}
\def\EE{\mathbb E}\def\FF{\mathbb F}\def\GG{\mathbb G}\def\HH{\mathbb H}
\def\II{\mathbb I}\def\JJ{\mathbb J}\def\KK{\mathbb K}\def\LL{\mathbb L}
\def\MM{\mathbb M}\def\NN{\mathbb N}\def\OO{\mathbb O}\def\PP{\mathbb P}
\def\QQ{\mathbb Q}\def\RR{\mathbb R}\def\SS{\mathbb S}\def\TT{\mathbb T}
\def\UU{\mathbb U}\def\VV{\mathbb V}\def\WW{\mathbb W}\def\XX{\mathbb X}
\def\YY{\mathbb Y}\def\ZZ{\mathbb Z}

%%%%%%%%%


%%%%% Commenting Commands
\setlength{\marginparwidth}{1.0in}
\definecolor{CSPcolor}{rgb}{0.0,0.5,0.75}	% Textcolor for CSP
\definecolor{NScolor}{rgb}{0.5,0.0,0.5}		% Textcolor for NS
\definecolor{CDcolor}{rgb}{0.8,0.0,0.2}		% Textcolor for CD
%\begin{center}
%{\color{MyDarkBlue}This color is MyDarkBlue}
%\end{center}
\newcommand{\CSP}[1]{\marginpar{\tiny\color{CSPcolor}{ #1}}}
\newcommand{\CD}[1]{\marginpar{\tiny\color{CDcolor}{ #1}}}
\newcommand{\NS}[1]{\marginpar{\tiny\color{NScolor}{ #1}}}

\newcommand{\CSPcomm}[1]{{\color{CSPcolor}{#1}}}
\newcommand{\CDcomm}[1]{{\color{CDcolor}{#1}}}
\newcommand{\NScomm}[1]{{\color{NScolor}{#1}}}






