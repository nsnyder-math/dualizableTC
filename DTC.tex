
%%% The DTC.tex file
%%% Authors: Christopher Douglas, Christopher Schommer-Pries, and Noah Snyder

\documentclass{amsart}


%%%%%%% Standard Packages
\usepackage{amsmath}       % I think this gives me some symbols
\usepackage{amsthm}        % Does theorem stuff
\usepackage{amssymb}       % more symbols and fonts
\usepackage{amsfonts}
\usepackage[all]{xy}
\usepackage{xspace}
\usepackage{calc}



\setlength{\topskip}{0pt}
\setlength{\footskip}{30pt}
\headheight=0pt
\topmargin=0pt
\headsep=18pt
\textheight=603pt %% 792pt to page, 648 is 9in
\textwidth=420pt  %% 612pt to page, 468pt is 6.5in
\oddsidemargin=25pt
\evensidemargin=25pt

\pagestyle{plain}


%%%%%% Adds hyperlinks
\usepackage[colorlinks, linkcolor=black, citecolor=blue,
	% pagebackref,
 	%bookmarksnumbered=true
	]{hyperref}
	
	
	
%%%%%% Tikz !!! Commands and Macros %%%%%%%%%%%%%
\usepackage{tikz}
\usetikzlibrary{matrix}


%%%% These draw triple or quadruple set of arrows of length 0.5 cm
\DeclareMathOperator{\righttriplearrows} {{\; \tikz{ \foreach \y in {0, 0.1, 0.2} { \draw [-stealth] (0, \y) -- +(0.5, 0);}} \; }}
\DeclareMathOperator{\lefttriplearrows} {{\; \tikz{ \foreach \y in {0, 0.1, 0.2} { \draw [stealth-] (0, \y) -- +(0.5, 0);}} \; }}
\DeclareMathOperator{\rightquadarrows} {{\; \tikz{ \foreach \y in {0, 0.1, 0.2, 0.3} { \draw [-stealth] (0, \y) -- +(0.5, 0);}} \; }}
\DeclareMathOperator{\leftquadarrows} {{\; \tikz{ \foreach \y in {0, 0.1, 0.2, 0.3} { \draw [stealth-] (0, \y) -- +(0.5, 0);}} \; }}

%%%%%%% End TikZ Commands and Macros %%%%%%%%%%%%%



%%%%%%%%%%%%%%%%%%%%%% Theorem Styles and Counters %%%%%%%%%%%%%%%%%%%%%%%%%%
% These all use the same "theorem" counter. 
\theoremstyle{plain} %%% Plain Theorem Styles.
\newtheorem{theorem}{Theorem}[section]
\newtheorem{lemma}[theorem]{Lemma}
\newtheorem{corollary}[theorem]{Corollary}          
\newtheorem{proposition}[theorem]{Proposition}              

\theoremstyle{definition} %%%% Definition-like Commands  
\newtheorem{definition}[theorem]{Definition}

\theoremstyle{remark}  %%%% Remark-like Commands
\newtheorem{remark}[theorem]{Remark}
\newtheorem{example}[theorem]{Example}
%%%%%%%%%%%%%%%%%%%%%% End Theorem Styles and Counters %%%%%%%%%%%%%%%%%%%%%%%%%%

%%%% Misc symbols %%%%%

\newcommand{\nn}{\nonumber}
\newcommand{\nid}{\noindent}
\newcommand{\ra}{\rightarrow}
\newcommand{\la}{\leftarrow}
\newcommand{\xra}{\xrightarrow}
\newcommand{\xla}{\xleftarrow}

\newcommand{\Bord}{\mathrm{Bord}}
\newcommand{\Vect}{\mathrm{Vect}}
\newcommand{\TC}{\mathrm{TC}}

\def\cA{\mathcal A}\def\cB{\mathcal B}\def\cC{\mathcal C}\def\cD{\mathcal D}
\def\cE{\mathcal E}\def\cF{\mathcal F}\def\cG{\mathcal G}\def\cH{\mathcal H}
\def\cI{\mathcal I}\def\cJ{\mathcal J}\def\cK{\mathcal K}\def\cL{\mathcal L}
\def\cM{\mathcal M}\def\cN{\mathcal N}\def\cO{\mathcal O}\def\cP{\mathcal P}
\def\cQ{\mathcal Q}\def\cR{\mathcal R}\def\cS{\ess}\def\cT{\mathcal T}
\def\cU{\mathcal U}\def\cV{\mathcal V}\def\cW{\mathcal W}\def\cX{\mathcal X}
\def\cY{\mathcal Y}\def\cZ{\mathcal Z}

\def\AA{\mathbb A}\def\BB{\mathbb B}\def\CC{\mathbb C}\def\DD{\mathbb D}
\def\EE{\mathbb E}\def\FF{\mathbb F}\def\GG{\mathbb G}\def\HH{\mathbb H}
\def\II{\mathbb I}\def\JJ{\mathbb J}\def\KK{\mathbb K}\def\LL{\mathbb L}
\def\MM{\mathbb M}\def\NN{\mathbb N}\def\OO{\mathbb O}\def\PP{\mathbb P}
\def\QQ{\mathbb Q}\def\RR{\mathbb R}\def\SS{\mathbb S}\def\TT{\mathbb T}
\def\UU{\mathbb U}\def\VV{\mathbb V}\def\WW{\mathbb W}\def\XX{\mathbb X}
\def\YY{\mathbb Y}\def\ZZ{\mathbb Z}

%%%%%%%%%















%%%%%%%
% Outline
%%%%%%%
%
%
% 0. Abstract
%
% 1. Introduction
% 1.1. Background and motivation
% 1.2. Results
% 
% 2. Tensor categories
% 2.1. Linear categories
% 2.2. Tensor products and colimits of linear categories
% 2.3. Tensor category bimodules and bimodule composition
% 2.4. The 3-category of tensor categories.
% 
% 3. Dualizability and fusion categories
% 3.1. Dualizability in 3-categories
% 3.2. Fusion categories
% 3.3. Fusion categories are dualizable
% 3.3.1. Functors of finite semisimple module categories have duals
% 3.3.2. Indecomposable modules with braided commutant have duals
% 	[Prop: Given C fusion, C--M--Vect indecomposable with C' braided, then M has an ambiadjoint.]
% 3.3.3. Fusion categories have duals
% 3.4. Dualizable tensor categories are fusion
% 
% 4. The Serre automorphism of a fusion category
% 4.1. The double dual is the Serre automorphism
% 4.1.1. 3-framed 1-manifolds and the Serre automorphism
% 4.1.2. Computing the Serre automorphism
% 	[Thm: Serre(C) = [**].]
% 4.2. The quadruple dual is trivial
% 	[Bimodulification Lemma]
% 	[Thm: If C is dualizable, that is fusion, then ****=1.]
% 
% 5. Pivotality as a descent condition
% 5.1. Structure groups of 3-manifolds
% 5.2. Fusion category TFTs are string
% 5.3. Pivotal fusion category TFTs are orpo
% 	[Thm: A fusion category is pivotal if and only if the associated TFT is orpo.]
% 5.4. Structure groups of fusion category TFTs.
% 	[Conj: All TC-valued TFTs are orpo.] [This conj is equivalent to ENO.]
% 	[Conj: All TC-valued orpo TFTs are oriented.] [Sketch: Drinfeld centers of pivotal fusion categories are anomaly free modular, therefore oriented 123; pushout to show oriented as 0123.]
% 
% 
%%%%%%%



\begin{document}

\title{Dualizable Tensor Categories}

\begin{abstract}

\end{abstract}
	
\author{Christopher L. Douglas}
\address{Department of Mathematics, University of California, Berkeley, CA 94720, USA}
\email{cdouglas@math.berkeley.edu}
	
\author{Christopher Schommer-Pries}
\address{Department of Mathematics \\
	Harvard University\\
	1 Oxford St.\\
	Cambridge, MA 02138} % Current Address
\email{schommerpries.chris.math@gmail.com}

\author{Noah Snyder}
\address{}
\email{}

\maketitle	
\tableofcontents
%%%%%%%%

\section{Introduction}

\subsection{Background and motivation}

\subsection{Results}

\section{Tensor categories}
\CD{The organization of this section might well change as we decide what exactly we should include.}
%\marginpar{\small CD: The organization of this section might well change as we decide what exactly we should include.}

\CSP{A Test: CSP's comment color}
\NS{Another test: NS's comment color}




\subsection{Linear categories}

A category with direct sums is a category with finite products and coproducts in which the unique morphism from the initial object to the terminal object is an isomorphism (and both are hence zero objects), and for which the canonical morphism,
\begin{equation*}
	\left( \begin{array}{cc} id_A & 0 \\ 0 & id_B \end{array}\right) : A \coprod B \to A \prod B
\end{equation*}
is an isomorphism.  Here `0' denotes the unique map which factors $X \to 0 \to Y$. We call this common object the direct sum of objects $A$ and $B$.

The Baer sum equips the hom sets in any category with direct sums with a commutative addition. A category with direct sums is an {\em additive category} if the hom sets are in fact abelian groups. Thus an additive category is canonically enriched in the category of abelian groups. An additive functor is a functor which preserves direct sums. 

\begin{definition}
	Let $R$ be a commutative ring. The 2-category of {\em $R$-algebroids} is the 2-category of categories enriched in $\Mod{R}{}$, the category of $R$-modules. An {\em $R$-linear category} is an additive category enriched in $\Mod{R}{}$ in a way extending the canonical enrichment in $\Mod{\ZZ}{} = \Ab$. These form an evident 2-category, $\Cat_R$ in which the 1-morphisms are the additive $R$-enriched functors. 
\end{definition}

\begin{remark}
	If $R = k$ is a field, and $\cA$ is an additive category, then the enrichment in $\Vect_k = \Mod{k}{}$ is unique, if it exists. 
\end{remark}

\begin{definition}
	An additive category A is {\em idempotent complete} \cite[1.2.1, 1.2.2]{Karoubi68} if for
	every idempotent $p: A \to A$, i.e. $p^2 = p$, there is a decomposition $A \cong K \oplus I$ of $A$ such that $p = \left( \begin{array}{cc} 0 & 0 \\ 0 & 1 \end{array}\right)$ is projection onto $I$. 
\end{definition}

Let $\Cat_R^\text{IC}$ denote the subcategory of idempotently complete $R$-linear categories. We have adjunctions:
\begin{align*}
		L:  & \Algd_R   \rightleftarrows \Cat_R : U \\
		\hat{(-)}:  & \Cat_R \rightleftarrows \Cat_R^\textrm{IC} : U. 
\end{align*}
where in each case $U$ denotes the forgetful functor. Both forgetful functors are faithful and $U: \Cat_R^\textrm{IC} \to \Cat_R$ is full. The functor $L$ formally adjoins direct sums. The functor $\hat{(-)}$ is known as {\em idempotent completion}. It admits an elementary description. The objects of $\hat \cA$ are pairs $(A, p)$ where $A \in \cA$ is an object and $p: A \to A$ is an idempotent. The morphisms from $(A, p)$ to $(B,q)$ are given by,
\begin{equation*}
	q \circ \hom_\cA(A, B) \circ p
\end{equation*} 
with the composition induced from $\cA$.
add construction of idempotent completion. 

%%%% Not clear we need this. 

%\begin{definition}
%	A linear category $C$ is a {\em retract} of a linear category $D$ if there exists linear functors $r: D \to C$ and $i: C \to D$ such that the composition $r \circ i: C \to C$ is naturally isomorphic to the identity functor of $C$. 
%\end{definition}

\begin{definition}[Kapranov-Voevodsky \cite{KV94}]
	A {\em finite 2-vector space} is a linear category isomorphic to $\bigoplus_I \Vect$,
	 a finite direct sum of the category $\Vect$ with itself. 
\end{definition}

\subsection{Tensor products and colimits of linear categories}

{\color{CSPcolor}
\cite{Tambara01} has a construction of the tensor product of additive categories. And the Idem. completion of this gives the tensor product 

}
\subsection{Tensor category bimodules and bimodule composition}


\begin{proposition}
	Let $R$ be a commutative ring and let $\cA$ be an additive category. Then the following structures are equivalent (in the case that $R= k$ is a field these are properties): 
	\begin{enumerate}
		\item $\cA$ is $R$-linear category,
		\item $\cA$ is an $\Mod{R}{}^\textrm{free}$-module category, where $\Mod{R}{}^\textrm{free}$ is the category of finitely generated free $R$-modules.  
	\end{enumerate}
	Moreover if the above structure is present, then the following implications hold: 
	\begin{enumerate}
		\item [(a)] If $\cA$ is idempotent complete then it is an $\Mod{R}{}^\textrm{proj}$-module category, where $\Mod{R}{}^\textrm{proj}$ is the category of finitely generated projective $R$-modules.  
		\item [(b)] If $\cA$ is abelian, then it is an $\Mod{R}{}^\textrm{fg}$-module category, where $\Mod{R}{}^\textrm{fp}$ is the category of finitely presented $R$-modules.
		
	\end{enumerate}
\end{proposition}

\begin{proof}
	% Note that the two notions of finitely presented by f.g. free modules and finitely presented by f.g. projective modules are the same, and in particular include the f.g. projective modules.  
\end{proof}




\CSP{convention on left/right: view S-R-bimodules as functors from R-mod to S-mod.}
\begin{proposition}
	Let $S$ be a tensor categories and let $M$ be a left $S$-module category.  If ${}_S M$ admits a left-adjoint as an $S$-$\Vect$-bimodule category, then there exist linear functors  $i: M \to \oplus_I \Vect$ and $r: \oplus_I \Vect \to M$, where $I$ is a finite index set, and  a split short exact sequence of endofunctors of $M$,
	\begin{equation*}
		0 \to k \to ri \to id_M \to 0.
	\end{equation*}
	
	
%	$M$ is a retract of a finite 2-vector space. 
	% There is also a version for left module cats... If $(-) \otimes_S M_R$ is a right-adjoint, then ${}_SM$ is a finitely generated projective $S$-module.
\end{proposition}

{\color{CSPcolor} This automatically implies that all homs in M are finite dimensional. 
This will be used later in the special case that $S = C \boxtimes C^{mp}$ and $M=C$ to prove things about  dualizible tensor categories.}

\begin{proof}
	The bimodule category ${}_S M_\Vect$ admits a left-adjoint. This consists of a bimodule category ${}_{\Vect}N_S$ together with bimodule functors,
	\begin{align*}
		\varepsilon: & N \otimes_S M \to \Vect, \\
		\eta: & {}_S S_S \to {}_S M \boxtimes N_S. 
	\end{align*}
	These must satisfy the adjunction equations,
	\begin{align*}
		(id_M \boxtimes \varepsilon) \circ (\eta \otimes_S id_M) &\cong id_M, \\
		(\varepsilon \boxtimes id_N) \circ (id_N \otimes_S \eta) & \cong id_N.
	\end{align*}
	The left-module category ${}_S S$ is cyclic, generated by the object $1 \in S$. Hence the bimodule category ${}_S S_S$ is also cyclic, generated by the same object. This implies that, up to isomorphism, the functor $\eta$ is determined by the image $\eta(1) \in M \boxtimes N $. Any object in the Deligne tensor product $X \boxtimes Y$ is a finite direct sum of primitive objects $(x \boxtimes y, p)$, with $x \in X$ and $y \in Y$, and $p$ and idempotent of $x \boxtimes y$.
	 Thus the functor $\eta$ is determined by a finite direct sum,
	\begin{equation*}
		\eta(1) = \sum_{i \in I} (m_i \boxtimes n_i , p_i)
	\end{equation*} 
	where $I$ is a finite index set and for all $i \in I$,  $m_i \in M$, $n_i \in N$, and $p_i: m_i \boxtimes n_i \to m_i \boxtimes n_i$ an idempotent. The adjunction equations become the pair of natural isomorphisms,
	\begin{align*}
		\sum_{i \in I}   (id_M \boxtimes \varepsilon) \circ (p_i \otimes_S id_a) (m_i \boxtimes n_i \otimes_S a) & \cong a, \\
		\sum_{i \in I}   ( \varepsilon \boxtimes id_N) \circ (id_b \otimes_S p_i) (b \otimes_S m_i \boxtimes  n_i) & \cong b,
	\end{align*}
	for all $a \in M$ and $b \in N$. We will only use the first of these. 
	
We now construct two functors, $i: M \to \oplus_I \Vect$ and $r: \oplus_I \Vect \to M$. These are defined by,
\begin{align*}
	i: M & \to \oplus_I \Vect \\
	a & \mapsto (\varepsilon(n_i \otimes_S a) )_{i \in I}
\end{align*}	
and 
\begin{align*}
	r: \oplus_I \Vect & \to M \\
	(v_i)_{i \in I} & \mapsto \sum_{i \in I} m_i \cdot v_i.
\end{align*}	
The first adjunction equation shows that the idempotents $p_i$ give rise to a split natural transformation $p: r \circ i \to id_M$. We let $k$ denote the kernel of $p$.
\end{proof}

{\color{CSPcolor} If we don't idempotently complete, the above proposition becomes stronger and says that the identity functor is equal to $r \circ i$. This should imply split semi-simplicity. However if instead we complete with respect to all sub-objects, then we just get a short exact sequence
\begin{equation*}
	0 \to k \to ri \to id_M \to 0.
\end{equation*}
It is not necessarily split. 
 }

\begin{corollary}
	If as in the above proposition ${}_S M$ is admits a dual, and the linear functor $\varepsilon:N \otimes_S M \to \Vect $ admits a right-adjoint, then every object in $M$ is projective.
\end{corollary}

\begin{proof}
	
\end{proof}

\subsection{The 3-category of tensor categories.}

\section{Dualizability and fusion categories}

\subsection{Dualizability in 3-categories}

\subsection{Fusion categories}

\subsection{Fusion categories are dualizable}

\subsubsection{Functors of finite semisimple module categories have duals}

\subsubsection{Indecomposable modules with braided commutant have duals} .\\

	[Prop: Given C fusion, C--M--Vect indecomposable with C' braided, then M has an ambiadjoint.]

\subsubsection{Fusion categories have duals}

\begin{theorem}
	Fusion Categories are fully-dualizable. 
\end{theorem}
	
\begin{proof}[Proof Sketch]
We must show the following conditions for a Fusion Category C to be fully dualizable: 
	\begin{enumerate}
		\item $C$ must have a dual (it is automatically both a left and right dual, since TC is symmetric monoidal).
		\item The adjunction 1-morphisms (certain bimodule categories) used in the above duality must themselves have duals (both left and right duals), which in turn must themselves have duals, and so on.  In fact, we show that the relevant 1-morphisms have ambidextrous adjoints, so we do not need to worry about an infinite chain of adjunctions.
		\item The adjunctions of the above 1-morphism dualities, must have duals, and their duals must have duals, and so on.  Again, we show that in fact the adjoints are ambidextrous.
	\end{enumerate}

Condition 1: The dual of $C$ is $C^{mp}$ (the one with only the tensor structure opposite). The dualizing bimodule categories are:
\begin{equation*}
	{}_{C \boxtimes C^{mp}} C_{Vect}, \qquad \text{ and } \qquad {}_{Vect} C_{C^{mp} \boxtimes C}	
\end{equation*} 
These satisfy the necessary adjunction for C to be dualizable. 
Condition 2 will be proven by the two propositions below. Note that $C$ is indecomposable as a $C \boxtimes C^{mp}$-module.  
	
Condition 3 was established in the ENO Part II blip `Solution to item (1)'.  It uses the fact that C is semisimple with finitely many simples. 	
	
\end{proof}	
	
\begin{proposition}
	Let $C$ be a fusion category, and let ${}_{C} M_D$ be a bimodule which is $C$-indecomposible. Let $C'$ be the commutant of $C$ acting on $M$, and let $i: D \to C'$ be the induced tensor functor (warning! it may not be an inclusion in the usual sense). Then, the bimodule category ${}_{C} M_{C'}$ is invertible, with inverse ${}_{C'} N_C$ (See Lemma \ref{Lma:}). In this case, 
\begin{enumerate}
	\item the maps of bimodules,
	\begin{equation*}
		{C --- M \otimes_D N ---C }  ===> {C --- M x_C' N ---C }  == {C---C ---C}
		{D ---D--- D} ===> {D --- C' ----D} == {D --- N x_C M --- D}
	\end{equation*}	
	form the unit and counit of an (say "left") adjunction between $C--M--D$ and $D--N--C$.
	\item  Moreover, if there exist maps ("conditional expectation maps")
	\begin{align*}
		lambda: D---C'---D  ==> D---D---D \\
		mu: C' --- C' ---C'  ===>  C' --- C' x_D C' --- C' 
	\end{align*}
	making D-- C' ---C' and C' ---C' ---D into a ("right") adjunction, then the composites 
	\begin{align*}	
		{C --- M x_D N ---C }  <=mu== {C --- M x_C' N ---C }  == {C---C ---C}
		{D ---D--- D} <=lambda== {D --- C' ----D} == {D --- N x_C M --- D}
	\end{align*}
	form the units of a ("right") adjucntion between C--M--D and D--N--C.
\end{enumerate}	
\end{proposition}	

\begin{proof}
	...
\end{proof}
	


\subsection{Dualizable tensor categories are fusion}



\section{The Serre automorphism of a fusion category}

\subsection{The double dual is the Serre automorphism}

\subsubsection{3-framed 1-manifolds and the Serre automorphism}

\subsubsection{Computing the Serre automorphism} .\\

	[Thm: Serre(C) = [**].]

\subsection{The quadruple dual is trivial} .\\

	[Bimodulification Lemma]
	
	[Thm: If C is dualizable, that is fusion, then ****=1.]

\section{Pivotality as a descent condition}

\subsection{Structure groups of 3-manifolds}

\subsection{Fusion category TFTs are string}

\subsection{Pivotal fusion category TFTs are orpo} .\\

	[Thm: A fusion category is pivotal if and only if the associated TFT is orpo.]

\subsection{Structure groups of fusion category TFTs.} .\\

	[Conj: All TC-valued TFTs are orpo.] [This conj is equivalent to ENO.]
	
	[Conj: All TC-valued orpo TFTs are oriented.] [Sketch: Drinfeld centers of pivotal fusion categories are anomaly free modular, therefore oriented 123; pushout to show oriented as 0123.]


%%%%%%%%

\nid ---------------------\\
---------------------
%%% I left these as demonstartion, please comment out. -CSP
\CD{I haven't tried to fit what CSP wrote below into the above outline structure.}
\CD{Known how to get these margin notes to fit?}
%\marginpar{\small CD: I haven't tried to fit what CSP wrote below into the above outline structure.}
%\marginpar{\small CD: Known how to get these margin notes to fit?}


%% The Bibliography
\bibliographystyle{amsalpha}
\bibliography{DTCreferences}

\end{document}
