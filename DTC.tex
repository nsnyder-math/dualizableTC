
%%% The DTC.tex file
%%% Authors: Christopher Douglas, Christopher Schommer-Pries, and Noah Snyder

\documentclass{amsart}


%%%%%%% Standard Packages
\usepackage{amsmath}       % I think this gives me some symbols
\usepackage{amsthm}        % Does theorem stuff
\usepackage{amssymb}       % more symbols and fonts
\usepackage{amsfonts}
\usepackage[all]{xy}
\usepackage{xspace}
\usepackage{calc}



\setlength{\topskip}{0pt}
\setlength{\footskip}{30pt}
\headheight=0pt
\topmargin=0pt
\headsep=18pt
\textheight=603pt %% 792pt to page, 648 is 9in
\textwidth=420pt  %% 612pt to page, 468pt is 6.5in
\oddsidemargin=25pt
\evensidemargin=25pt

\pagestyle{plain}


%%%%%% Adds hyperlinks
\usepackage[colorlinks, linkcolor=black, citecolor=blue,
	% pagebackref,
 	%bookmarksnumbered=true
	]{hyperref}
	
	
	
%%%%%% Tikz !!! Commands and Macros %%%%%%%%%%%%%
\usepackage{tikz}
\usetikzlibrary{matrix}


%%%% These draw triple or quadruple set of arrows of length 0.5 cm
\DeclareMathOperator{\righttriplearrows} {{\; \tikz{ \foreach \y in {0, 0.1, 0.2} { \draw [-stealth] (0, \y) -- +(0.5, 0);}} \; }}
\DeclareMathOperator{\lefttriplearrows} {{\; \tikz{ \foreach \y in {0, 0.1, 0.2} { \draw [stealth-] (0, \y) -- +(0.5, 0);}} \; }}
\DeclareMathOperator{\rightquadarrows} {{\; \tikz{ \foreach \y in {0, 0.1, 0.2, 0.3} { \draw [-stealth] (0, \y) -- +(0.5, 0);}} \; }}
\DeclareMathOperator{\leftquadarrows} {{\; \tikz{ \foreach \y in {0, 0.1, 0.2, 0.3} { \draw [stealth-] (0, \y) -- +(0.5, 0);}} \; }}

%%%%%%% End TikZ Commands and Macros %%%%%%%%%%%%%



%%%%%%%%%%%%%%%%%%%%%% Theorem Styles and Counters %%%%%%%%%%%%%%%%%%%%%%%%%%
% These all use the same "theorem" counter. 
\theoremstyle{plain} %%% Plain Theorem Styles.
\newtheorem{theorem}{Theorem}[section]
\newtheorem{lemma}[theorem]{Lemma}
\newtheorem{corollary}[theorem]{Corollary}          
\newtheorem{proposition}[theorem]{Proposition}              

\theoremstyle{definition} %%%% Definition-like Commands  
\newtheorem{definition}[theorem]{Definition}

\theoremstyle{remark}  %%%% Remark-like Commands
\newtheorem{remark}[theorem]{Remark}
\newtheorem{example}[theorem]{Example}
%%%%%%%%%%%%%%%%%%%%%% End Theorem Styles and Counters %%%%%%%%%%%%%%%%%%%%%%%%%%

%%%% Misc symbols %%%%%

\newcommand{\nn}{\nonumber}
\newcommand{\nid}{\noindent}
\newcommand{\ra}{\rightarrow}
\newcommand{\la}{\leftarrow}
\newcommand{\xra}{\xrightarrow}
\newcommand{\xla}{\xleftarrow}

\newcommand{\Bord}{\mathrm{Bord}}
\newcommand{\Vect}{\mathrm{Vect}}
\newcommand{\TC}{\mathrm{TC}}

\def\cA{\mathcal A}\def\cB{\mathcal B}\def\cC{\mathcal C}\def\cD{\mathcal D}
\def\cE{\mathcal E}\def\cF{\mathcal F}\def\cG{\mathcal G}\def\cH{\mathcal H}
\def\cI{\mathcal I}\def\cJ{\mathcal J}\def\cK{\mathcal K}\def\cL{\mathcal L}
\def\cM{\mathcal M}\def\cN{\mathcal N}\def\cO{\mathcal O}\def\cP{\mathcal P}
\def\cQ{\mathcal Q}\def\cR{\mathcal R}\def\cS{\ess}\def\cT{\mathcal T}
\def\cU{\mathcal U}\def\cV{\mathcal V}\def\cW{\mathcal W}\def\cX{\mathcal X}
\def\cY{\mathcal Y}\def\cZ{\mathcal Z}

\def\AA{\mathbb A}\def\BB{\mathbb B}\def\CC{\mathbb C}\def\DD{\mathbb D}
\def\EE{\mathbb E}\def\FF{\mathbb F}\def\GG{\mathbb G}\def\HH{\mathbb H}
\def\II{\mathbb I}\def\JJ{\mathbb J}\def\KK{\mathbb K}\def\LL{\mathbb L}
\def\MM{\mathbb M}\def\NN{\mathbb N}\def\OO{\mathbb O}\def\PP{\mathbb P}
\def\QQ{\mathbb Q}\def\RR{\mathbb R}\def\SS{\mathbb S}\def\TT{\mathbb T}
\def\UU{\mathbb U}\def\VV{\mathbb V}\def\WW{\mathbb W}\def\XX{\mathbb X}
\def\YY{\mathbb Y}\def\ZZ{\mathbb Z}

%%%%%%%%%















%%%%%%%
% Outline
%%%%%%%
%
%
% 0. Abstract
%
% 1. Introduction
% 1.1. Background and motivation
% 1.2. Results
% 
% 2. Tensor categories
% 2.1. Linear categories
% 2.2. Tensor products and colimits of linear categories
% 2.3. Tensor category bimodules and bimodule composition
% 2.4. The 3-category of tensor categories.
% 
% 3. Dualizability and fusion categories
% 3.1. Dualizability in 3-categories
% 3.2. Fusion categories
% 3.3. Fusion categories are dualizable
% 3.3.1. Functors of finite semisimple module categories have duals
% 3.3.2. Indecomposable modules with braided commutant have duals
% 	[Prop: Given C fusion, C--M--Vect indecomposable with C' braided, then M has an ambiadjoint.]
% 3.3.3. Fusion categories have duals
% 3.4. Dualizable tensor categories are fusion
% 
% 4. The Serre automorphism of a fusion category
% 4.1. The double dual is the Serre automorphism
% 4.1.1. 3-framed 1-manifolds and the Serre automorphism
% 4.1.2. Computing the Serre automorphism
% 	[Thm: Serre(C) = [**].]
% 4.2. The quadruple dual is trivial
% 	[Bimodulification Lemma]
% 	[Thm: If C is dualizable, that is fusion, then ****=1.]
% 
% 5. Pivotality as a descent condition
% 5.1. Structure groups of 3-manifolds
% 5.2. Fusion category TFTs are string
% 5.3. Pivotal fusion category TFTs are orpo
% 	[Thm: A fusion category is pivotal if and only if the associated TFT is orpo.]
% 5.4. Structure groups of fusion category TFTs.
% 	[Conj: All TC-valued TFTs are orpo.] [This conj is equivalent to ENO.]
% 	[Conj: All TC-valued orpo TFTs are oriented.] [Sketch: Drinfeld centers of pivotal fusion categories are anomaly free modular, therefore oriented 123; pushout to show oriented as 0123.]
% 
% 
%%%%%%%



\begin{document}

\title{Dualizable Tensor Categories}

\begin{abstract}

\end{abstract}
	
\author{Christopher L. Douglas}
\address{Department of Mathematics, University of California, Berkeley, CA 94720, USA}
\email{cdouglas@math.berkeley.edu}
	
\author{Christopher Schommer-Pries}
\address{Department of Mathematics \\
	Harvard University\\
	1 Oxford St.\\
	Cambridge, MA 02138} % Current Address
\email{schommerpries.chris.math@gmail.com}

\author{Noah Snyder}
\address{}
\email{}

\maketitle	
\tableofcontents
%%%%%%%%

\section{Introduction}

\subsection{Background and motivation}

\subsection{Results}

\section{Tensor categories}
\CD{The organization of this section might well change as we decide what exactly we should include.}
%\marginpar{\small CD: The organization of this section might well change as we decide what exactly we should include.}

\CSP{A Test: CSP's comment color}
\NS{Another test: NS's comment color}




\subsection{Linear categories}

\begin{definition}
	A linear category $C$ is a {\em retract} of a linear category $D$ if there exists linear functors $r: D \to C$ and $i: C \to D$ such that the composition $r \circ i: C \to C$ is naturally isomorphic to the identity functor of $C$. 
\end{definition}

\begin{definition}[Kapranov-Voevodsky \cite{KV94}]
	A {\em finite 2-vector space} is a linear category isomorphic to $\bigoplus_I \Vect$,
	 a finite direct sum of the category $\Vect$ with itself. 
\end{definition}

\subsection{Tensor products and colimits of linear categories}

\subsection{Tensor category bimodules and bimodule composition}


\CSP{convention on left/right: view S-R-bimodules as functors from R-mod to S-mod.}
\begin{proposition}
	Let $S$ be a tensor categories and let $M$ be a left $S$-module category.  If ${}_S M$ admits a left-adjoint as an $S$-$\Vect$-bimodule category, then $M$ is a retract of a finite 2-vector space. 
	% There is also a version for left module cats... If $(-) \otimes_S M_R$ is a right-adjoint, then ${}_SM$ is a finitely generated projective $S$-module.
\end{proposition}

{\color{CSPcolor} This automatically implies that all homs in M are finite dimensional. I believe that it also implies that M is semi-simple with finitely many simples, hence is a finite 2-vector space (if our ground ring is $\CC$).

This will be used later in the special case that $S = C \boxtimes C^{mp}$ and $M=C$ to prove that a dualizible tensor category is semi-simple with finitely many simples.}

\begin{proof}
	
\end{proof}

\subsection{The 3-category of tensor categories.}

\section{Dualizability and fusion categories}

\subsection{Dualizability in 3-categories}

\subsection{Fusion categories}

\subsection{Fusion categories are dualizable}

\subsubsection{Functors of finite semisimple module categories have duals}

\subsubsection{Indecomposable modules with braided commutant have duals} .\\

	[Prop: Given C fusion, C--M--Vect indecomposable with C' braided, then M has an ambiadjoint.]

\subsubsection{Fusion categories have duals}

\begin{theorem}
	Fusion Categories are fully-dualizable. 
\end{theorem}
	
\begin{proof}[Proof Sketch]
We must show the following conditions for a Fusion Category C to be fully dualizable: 
	\begin{enumerate}
		\item $C$ must have a dual (it is automatically both a left and right dual, since TC is symmetric monoidal).
		\item The adjunction 1-morphisms (certain bimodule categories) used in the above duality must themselves have duals (both left and right duals), which in turn must themselves have duals, and so on.  In fact, we show that the relevant 1-morphisms have ambidextrous adjoints, so we do not need to worry about an infinite chain of adjunctions.
		\item The adjunctions of the above 1-morphism dualities, must have duals, and their duals must have duals, and so on.  Again, we show that in fact the adjoints are ambidextrous.
	\end{enumerate}

Condition 1: The dual of $C$ is $C^{mp}$ (the one with only the tensor structure opposite). The dualizing bimodule categories are:
\begin{equation*}
	{}_{C \boxtimes C^{mp}} C_{Vect}, \qquad \text{ and } \qquad {}_{Vect} C_{C^{mp} \boxtimes C}	
\end{equation*} 
These satisfy the necessary adjunction for C to be dualizable. 
Condition 2 will be proven by the two propositions below. Note that $C$ is indecomposable as a $C \boxtimes C^{mp}$-module.  
	
Condition 3 was established in the ENO Part II blip `Solution to item (1)'.  It uses the fact that C is semisimple with finitely many simples. 	
	
\end{proof}	
	
\begin{proposition}
	Let $C$ be a fusion category, and let ${}_{C} M_D$ be a bimodule which is $C$-indecomposible. Let $C'$ be the commutant of $C$ acting on $M$, and let $i: D \to C'$ be the induced tensor functor (warning! it may not be an inclusion in the usual sense). Then, the bimodule category ${}_{C} M_{C'}$ is invertible, with inverse ${}_{C'} N_C$ (See Lemma \ref{Lma:}). In this case, 
\begin{enumerate}
	\item the maps of bimodules,
	\begin{equation*}
		{C --- M \otimes_D N ---C }  ===> {C --- M x_C' N ---C }  == {C---C ---C}
		{D ---D--- D} ===> {D --- C' ----D} == {D --- N x_C M --- D}
	\end{equation*}	
	form the unit and counit of an (say "left") adjunction between $C--M--D$ and $D--N--C$.
	\item  Moreover, if there exist maps ("conditional expectation maps")
	\begin{align*}
		lambda: D---C'---D  ==> D---D---D \\
		mu: C' --- C' ---C'  ===>  C' --- C' x_D C' --- C' 
	\end{align*}
	making D-- C' ---C' and C' ---C' ---D into a ("right") adjunction, then the composites 
	\begin{align*}	
		{C --- M x_D N ---C }  <=mu== {C --- M x_C' N ---C }  == {C---C ---C}
		{D ---D--- D} <=lambda== {D --- C' ----D} == {D --- N x_C M --- D}
	\end{align*}
	form the units of a ("right") adjucntion between C--M--D and D--N--C.
\end{enumerate}	
\end{proposition}	

\begin{proof}
	...
\end{proof}
	


\subsection{Dualizable tensor categories are fusion}



\section{The Serre automorphism of a fusion category}

\subsection{The double dual is the Serre automorphism}

\subsubsection{3-framed 1-manifolds and the Serre automorphism}

\subsubsection{Computing the Serre automorphism} .\\

	[Thm: Serre(C) = [**].]

\subsection{The quadruple dual is trivial} .\\

	[Bimodulification Lemma]
	
	[Thm: If C is dualizable, that is fusion, then ****=1.]

\section{Pivotality as a descent condition}

\subsection{Structure groups of 3-manifolds}

\subsection{Fusion category TFTs are string}

\subsection{Pivotal fusion category TFTs are orpo} .\\

	[Thm: A fusion category is pivotal if and only if the associated TFT is orpo.]

\subsection{Structure groups of fusion category TFTs.} .\\

	[Conj: All TC-valued TFTs are orpo.] [This conj is equivalent to ENO.]
	
	[Conj: All TC-valued orpo TFTs are oriented.] [Sketch: Drinfeld centers of pivotal fusion categories are anomaly free modular, therefore oriented 123; pushout to show oriented as 0123.]


%%%%%%%%

\nid ---------------------\\
---------------------
%%% I left these as demonstartion, please comment out. -CSP
\CD{I haven't tried to fit what CSP wrote below into the above outline structure.}
\CD{Known how to get these margin notes to fit?}
%\marginpar{\small CD: I haven't tried to fit what CSP wrote below into the above outline structure.}
%\marginpar{\small CD: Known how to get these margin notes to fit?}


%% The Bibliography
\bibliographystyle{amsalpha}
\bibliography{DTCreferences}

\end{document}
