
\CD{cf ENOPartII "Solution to item (1)"}

\begin{lemma}
Let $F: \bimod{\cC}{\cM}{\cD} \ra \bimod{\cC}{\cN}{\cD}$ be a functor of bimodule categories, and suppose the underlying functor $\tilde{F}: \cM \ra \cN$ of linear categories has an ambidextrous adjoint $\tilde{G}: \cN \ra \cM$.  Then $F$ has an ambidextrous adjoint $G: \bimod{\cC}{\cN}{\cD} \ra \bimod{\cC}{\cM}{\cD}$.
\end{lemma}

\begin{proof}
\CDcomm{Look at the wave}
\end{proof}

\begin{lemma}
Let $F: \bimod{\cC}{\cM}{\cD} \ra \bimod{\cC}{\cN}{\cD}$ be a functor of bimodule categories.  If $\cM$ and $\cN$ are semisimple categories with finitely many simple objects, then the functor $\tilde{F}: \cM \ra \cN$ of linear categories underlying $F$ has an ambidextrous adjoint.
\end{lemma}

\begin{proof}
\CDcomm{The ambidextrous adjoint $\rd{F}$ is given by the transpose of the entrywise dual of $F$.  That is, write $F$ as a matrix of vector spaces in terms of a chosen basis of simple objects for $M$ and $N$, then take the matrix of dual vector spaces, and take the transpose.}
...
\end{proof}

\begin{proposition} \label{prop-functadj}
A functor $F: \bimod{\cC}{\cM}{\cD} \ra \bimod{\cC}{\cN}{\cD}$ of bimodules categories has an ambidextrous adjoint if the linear categories $\cM$ and $\cN$ are semisimple with finitely many simple objects.
\end{proposition}

This proposition follows from the above two lemmas.


\subsubsection{Indecomposable modules with braided fusion commutant have adjoints} \label{sec-df-modules}


\begin{lemma} \label{lemma-invertible}
Let $\cC$ be a fusion category and $\cM$ an indecomposable $\cC$-module.  Let $\cC'$ denote the commutant of $\cC$ acting on $\cM$.  In this case the bimodule $\bimod{\cC}{\cM}{\cC'}$ is invertible, \CDcomm{with inverse $\bimod{\cC'}{\Hom_{\cC}(\cM,\cC)}{\cC}$}.
\end{lemma}

\begin{proof}
\CDcomm{See ENOPartIII Apr 7 "Lemma: When C is fusion" for a sketch.}

\CDcomm{Can this be shown only using that C is semisimple?}

\CDcomm{Crucial point: the inverse is given by $\Hom_C(M,C)$, right?.  Emphasize this.}
\end{proof}

In the situation of this lemma we will abbreviate the inverse \CDcomm{$\bimod{\cC'}{\Hom_{\cC}(\cM,\cC)}{\cC}$} of $\bimod{\cC}{\cM}{\cC'}$ by $\bimod{\cC'}{\cN}{\cC}$.

\begin{lemma} \label{lemma-conditional}
Let $\cC$ be a fusion category and $\cM$ an indecomposable $\cC$-module such that the commutant $\cC'$ of the $\cC$ action on $\cM$ is bradied fusion.  In this case there exist maps
\begin{itemize}
\item[] $\lambda : \bimod{\Vect}{\cC' \dtimes_{\cC'} \cC'}{\Vect} \dra \bimod{\Vect}{\Vect}{\Vect}$
\item[] $\mu : \bimod{\cC'}{\cC'}{\cC'} \dra \bimod{\cC'}{\cC' \dtimes \cC'}{\cC'}$
\end{itemize}
that form an adjunction
$$\bimod{\Vect}{\cC'}{\cC'} \adj \bimod{\cC'}{\cC'}{\Vect}.$$
\end{lemma}
\nid We refer to $\lambda$ and $\mu$ as "conditional expectation" maps.
\CD{If we have anything to say about the existence of conditional expectations when neither tensor category is $\Vect$, then this proposition can be generalized to that case.  In that case, the theorem below might also be able to be generalized.}
\CD{Do you need to assume fusion for the commutant, or is something weaker enough?}
\CD{Do you need to assume $\cC$ is fusion?}

\begin{proof}
\CD{cf ENOPartIII "Proposition3".}
The first map $\lambda : \cC' \ra \Vect$ is defined by $\lambda(x) = \Hom_{\cC'}(1,x)$.  The second map $\mu : \cC' \ra \cC' \dtimes \cC'$ is determined, using the left $\cC'$-module structure, by the condition that
$$\mu(1) = \sum_{\sigma \in \cI} \ld{\sigma} \dtimes \sigma.$$
Here $\cI$ is a basis of simple objects of $\cC'$.

By construction $\mu$ is a left module map, but we also need to give $\mu$ the structure of a right module map.  It is sufficient to check that for $\tau \in \cC'$ simple, there is an isomorphism
$$\sum_{\sigma \in \cI} {\ld{\sigma}} \dtimes \sigma \tau  \cong  \sum_{\sigma \in \cI} \tau {(\ld{\sigma})} \dtimes \sigma$$
\CD{Is that right that there is no further condition, ie you pick the module structure to be whatever iso you want on each simple, and that's it?}
Let $N^a_{bc}$ denote the vector spaces defining the tensor structure on the tensor category $\cC'$.  We have a series of isomorphisms
\begin{align}
\sum_{\sigma \in \cI} {\ld{\sigma}} \dtimes \sigma \tau
& =
\sum_{\sigma,\rho \in \cI} N^{\rho}_{\sigma \tau} {\ld{\sigma} \dtimes \rho} \nn \\
& \cong{(1)}
\sum_{\sigma,\rho \in \cI} N^{\ld{\sigma}}_{\tau {(\ld{\rho}})} {\ld{\sigma} \dtimes \rho} \nn \\
& \cong{(2)}
\sum_{\sigma,\rho \in \cI} N^{\ld{\sigma}}_{{(\ld{\rho})} \tau} {\ld{\sigma} \dtimes \rho} \nn \\
& \cong{(3)}
\sum_{\sigma,\rho \in \cI} N^{\rho}_{\sigma \tau} {\rho \dtimes \rd{\sigma}} \nn \\
& =
\sum_{\sigma \in \cI} \sigma \tau \dtimes {\rd{\sigma}} \nn \\
& \cong{(4)}
\sum_{\sigma \in \cI} \tau {(\ld{\sigma})} \dtimes \sigma \nn 
\end{align}
The first isomorphism $N^{\rho}_{\sigma \tau} \cong N^{\ld{\sigma}}_{\tau {(\ld{\rho}})}$ exists by the standard properties of structure constants for fusion categories. \CDcomm{say more?}
Because $\cC'$ is braided, the constant $N^{\ld{\sigma}}_{\tau {(\ld{\rho}})}$ is isomorphic to $N^{\ld{\sigma}}_{{(\ld{\rho}}) \tau}$, giving the second isomorphism.  Reindexing the sum by substituting $\rd{\rho}$ for $\sigma$ and $\rd{\sigma}$ for $\rho$ provides the third isomorphism.  Braiding $\sigma$ and $\tau$ and then substituting $\ld{\sigma}$ for $\sigma$ produces the fourth isomorphism.  \CD{Was \emph{that} the argument? Using the braiding twice?}

Finally we need to know that the maps $\lambda$ and $\mu$ do indeed satisfy the adjunction S-relations.  The first relation is the composite
$$\bimod{\Vect}{\cC'}{\cC'} = \cC' \dtimes_{\cC'} \cC' \ra \cC' \dtimes_{\cC'} \cC' \dtimes \cC' \ra \Vect \dtimes \cC' = \cC'$$
sending $1$ to $\sum_{\sigma \in \cI} \Hom(1,\ld{\sigma}) \sigma = Hom(1, \ld{1}) 1 = Hom(1 \cdot 1, 1) 1 = 1$.
The second relation is the composite
$$\bimod{\cC'}{\cC'}{\Vect} = \cC' \dtimes_{\cC'} \cC' \ra \cC' \dtimes \cC' \dtimes_{\cC'} \cC' \ra \cC' \dtimes \Vect = \cC'$$
sending $1$ to $\sum_{\sigma \in \cI} \ld{\sigma} \Hom(1,\sigma) = \ld{1} = 1$.  Both maps are indeed equivalent to the identity.


\end{proof}


\begin{theorem} \label{thm-indecompbraided}
Let $\cC$ be a tensor category and $\cM$ an indecomposable $\cC$-module.  If the commutant $\cC'^{\cM}$ is a braided fusion category, then the bimodule $\bimod{\cC}{\cM}{\Vect}$ has an ambidextrous adjoint, \CDcomm{namely $\bimod{\Vect}{\Hom_{\cC}(\cM,\cC)}{\cC}$}.
\end{theorem}

\CD{cf ENOPartIII "Proof details" etc.}
\begin{proof}
Let $\bimod{\cC'}{\cN}{\cC}$ abbreviate the inverse \CDcomm{$\bimod{\cC'}{\Hom_{\cC}(\cM,\cC)}{\cC}$} of the bimodule $\bimod{\cC}{\cM}{\cC'}$ provided by Lemma~\ref{lemma-invertible}.  We will construct an ambidextrous adjunction
$$\bimod{\cC}{\cM}{\Vect} \ambadj \bimod{\Vect}{\cN}{\cC}.$$

First we build the adjunction
$$\bimod{\cC}{\cM}{\Vect} \adj \bimod{\Vect}{\cN}{\cC}$$
as follows.  Write the bimodules $\bimod{\cC}{\cM}{\Vect}$ and $\bimod{\Vect}{\cN}{\cC}$ as tensor products:
$$\cM = \cM \dtimes_{\cC'} \cC'$$
$$\cN = \cC' \dtimes_{\cC'} \cN$$
The desired adjunction is the composite of the following two adjunctions:
\begin{enumerate}
\item $\bimod{\cC'}{\cC'}{\Vect} \adj \bimod{\Vect}{\cC'}{\cC'}$
\item $\bimod{\cC}{\cM}{\cC'} \adj \bimod{\cC'}{\cN}{\cC}$
\end{enumerate}
The bimodules $\bimod{\cC}{\cM}{\cC'}$ and $\bimod{\cC'}{\cN}{\cC}$ are inverse by construction, therefore adjoint as needed.  The unit and counit for the first adjunction are given by
\begin{itemize}
\item[] $\phi: \cC' \dtimes \cC' \ra \cC' \dtimes_{\cC'} \cC' = \cC'$
\item[] $\psi: \Vect \ra \cC' = \cC' \dtimes_{\cC'} \cC'$
\end{itemize}
The S-relations for this unit and counit can be checked as follows:
\begin{itemize}
\item[] $\bimod{\cC'}{\cC'}{\Vect} = \cC' \dtimes \Vect \ra \cC' \dtimes \cC' = \cC' \dtimes \cC' \dtimes_{\cC'} \cC' \ra \cC' \dtimes_{\cC'} \cC' \dtimes_{\cC'} \cC' = \cC' \dtimes_{\cC'} \cC' = \cC'$
\item[] $\bimod{\Vect}{\cC'}{\cC'} = \Vect \dtimes \cC' \ra \cC' \dtimes \cC' = \cC' \dtimes_{\cC'} \cC' \dtimes \cC' \ra \cC' \dtimes_{\cC'} \cC' \dtimes_{\cC'} \cC' \ra \cC' \dtimes_{\cC'} \cC' = \cC'$
\end{itemize}
\CD{The above adjunction could be generalized to $\cD$ instead of $\Vect$.}

Explicitly, the unit and counit for the adjunction $\bimod{\cC}{\cM}{\Vect} \adj \bimod{\Vect}{\cN}{\cC}$ are respectively the composites:
\begin{itemize}
\item[] $\cM \dtimes \cN = (\cM \dtimes_{\cC'} \cC') \dtimes (\cC' \dtimes_{\cC'} \cN) \xra{\phi} \cM \dtimes_{\cC'} \cC' \dtimes_{\cC'} \cN = \cM \dtimes_{\cC'} \cN \cong \cC$
\item[] $\Vect \xra{\psi} \cC' \dtimes_{\cC'} \cC' = \cC' \dtimes_{\cC'} \cC' \dtimes_{\cC'} \cC' \cong \cC' \dtimes_{\cC'} \cN \dtimes_{\cC} \cM \dtimes_{\cC'} \cC' = \cN \dtimes_{\cC} \cM$
\end{itemize}

Second we construct an adjunction
$$\bimod{\Vect}{\cN}{\cC} \adj \bimod{\cC}{\cM}{\Vect}.$$
Again this adjunction is constructed as the composite of two adjunctions, namely
\begin{enumerate}
\item $\bimod{\Vect}{\cC'}{\cC'} \adj \bimod{\cC'}{\cC'}{\Vect}$
\item $\bimod{\cC'}{\cN}{\cC} \adj \bimod{\cC}{\cM}{\cC'}$
\end{enumerate}
The bimodules $\bimod{\cC'}{\cN}{\cC}$ and $\bimod{\cC}{\cM}{\cC'}$ are inverse, so again adjoint as needed.  The unit and counit for the first adjunction, namely
\begin{itemize}
\item[] $\lambda : \bimod{\Vect}{\cC' \dtimes_{\cC'} \cC'}{\Vect} \dra \bimod{\Vect}{\Vect}{\Vect}$
\item[] $\mu : \bimod{\cC'}{\cC'}{\cC'} \dra \bimod{\cC'}{\cC' \dtimes \cC'}{\cC'}$,
\end{itemize}
are provided by Lemma~\ref{lemma-conditional}.
\CD{Since this adjunction depends on the lemma, which depends on $\cD = \Vect$, we don't know how to generalize this part at the moment.}
\end{proof}


     
\subsubsection{Fusion categories have duals} \label{sec-df-categories}

\begin{theorem} \label{thm-fcd}
[... fusion categories have duals ...]
\end{theorem}


\begin{remark}
In non-zero characteristic, it is not the case that all fusion categories are dualizable.  In particular, if the global dimension of a fusion category is zero, then the category cannot be dualizable.
\end{remark}

Recalling the discussion of dualizability in 3-categories from Section~\ref{sec-lft-dual}, the theorem follows from the following three propositions.

\begin{proposition}
Every tensor category $\cC \in \TC$ has a dual in the homotopy category of $\TC$, namely the monoidal opposite category $\cC^{\mp}$.
\end{proposition}
%\nid Here the tensor category $\cC^{\mp}$ has the same underlying category as $\cC$ but with the opposite tensor structure.

\begin{proof}
The evaluation of the duality is $\cC$ as a $\cC \dtimes \cC^{\mp}$--$\Vect$ bimodule.  The coevaluation of the duality is $\cC$ as a $\Vect$--$\cC^{\mp} \dtimes \cC$ bimodule.
\end{proof}

\begin{proposition} \label{prop-moduleadj}
Let $\cC$ be a fusion category.  The evaluation $\bimod{\cC \dtimes \cC^{\mp}}{\cC}{\Vect}$ and coevaluation $\bimod{\Vect}{\cC}{\cC^{\mp} \dtimes \cC}$ of the duality between $\cC$ and $\cC^{\mp}$ both have ambidextrous adjoints.
\end{proposition}

\begin{proof}
The evaluation category $\cC$ is indecomposable as a $\cC \dtimes \cC^{\mp}$-module, and the commutant of this module structure is the Drinfeld center $Z(\cC)$ which is braided fusion, by [...].  Theorem~\ref{thm-indecompbraided} therefore ensures that this module has an ambidextrous adjoint.  \CDcomm{The argument for the coevaluation is analogous.} \CD{It is analogous, right??}
\end{proof}

Let $\bimod{\cC \dtimes \cC^{\mp}}{\cC}{\Vect} \ambadj \bimod{\Vect}{\cD_1}{\cC \dtimes \cC^{\mp}}$ and $\bimod{\Vect}{\cC}{\cC^{\mp} \dtimes \cC} \ambadj \bimod{\cC^{\mp} \dtimes \cC}{\cD_2}{\Vect}$ denote the ambidextrous adjoints provided by Proposition~\ref{prop-moduleadj}.

\begin{proposition}
For $\cC$ a fusion category, the units and counits of the four adjunctions $\bimod{\cC \dtimes \cC^{\mp}}{\cC}{\Vect} \adj \cD_1$, $\cD_1 \adj \bimod{\cC \dtimes \cC^{\mp}}{\cC}{\Vect}$, $\bimod{\Vect}{\cC}{\cC^{\mp} \dtimes \cC} \adj \cD_2$, and $\cD_2 \adj \bimod{\Vect}{\cC}{\cC^{\mp} \dtimes \cC}$ all have ambidextrous adjoints.
\end{proposition}

\begin{proof}
\CDcomm{Need to know that the inverse bimodule provided by Lemma~\ref{lemma-invertible} is finite semisimple.  Given that, this follows from Proposition~\ref{prop-functadj}. (This uses the fact that $A \dtimes_{\cC} B$ is finite semisimple if $A,B,\cC$ all are.)}
...
\end{proof}

\subsection{Examples of dualization structures} \label{sec-df-examples}

For a variety of fusion categories, we explicitly describe the dualization structure provided by Theorem~\ref{thm-fcd}, namely the dual categories and the adjunctions and higher adjunctions for the adjunctions, and so on.  Throughout we use implicitly that the adjunctions are all ambidextrous. 

\begin{example}
$Rep(Z/2) = \{\CC[x]/(x^2-1)\}-\mod$ (Symmetric)
[Start this by copying the content from Wave "Rep(Z/2)", ENO Part III, on Jun 22.]
\end{example}

\begin{example}
Here we give an example of a tensor category that is not fusion, namely $\{\CC[x]/x^2\}-\mod$ and highlight the failure of dualizability. \CDcomm{Prove that this category is not dualizable.  Cf Wave ENO Part III on May 4 search for "An example to consider", and Wave ENO Part III on Jun 17, search for "is still not fully dualizable".}
\end{example}

\begin{example}
$Vect[G,\lambda]$ for some simple $G$ and $\lambda$?
\end{example}

\begin{example}
Fibonacci category (Modular)
\end{example}

\begin{example}
$D_4$ (or even part of $E_6$) as small non-braided category.  $Z(D_4) = A_5 [x] Z/3$.
\end{example}


%\subsubsection{Duals of 0-morphisms} All tensor categories have dual tensor categories.
%\subsubsection{Duals of 1-morphisms} 
%\subsubsection{Duals of 2-morphisms}

%
% CD: I have commented out this subsection, as I think it will really end up being in DTCII.
% 
% \subsection{Dualizable tensor categories are fusion} \label{sec-df-dtcf}
%
% \CD{This section might be omitted in a future version of this paper, or the title modified as appropriate.}








