
%%% The DTC.tex file
%%% Authors: Christopher Douglas, Christopher Schommer-Pries, and Noah Snyder

\documentclass{amsart}


%%%%%%% Standard Packages
\usepackage{amsmath}       % I think this gives me some symbols
\usepackage{amsthm}        % Does theorem stuff
\usepackage{amssymb}       % more symbols and fonts
\usepackage{amsfonts}
\usepackage[all]{xy}
\usepackage{xspace}
\usepackage{calc}



\setlength{\topskip}{0pt}
\setlength{\footskip}{30pt}
\headheight=0pt
\topmargin=0pt
\headsep=18pt
\textheight=603pt %% 792pt to page, 648 is 9in
\textwidth=420pt  %% 612pt to page, 468pt is 6.5in
\oddsidemargin=25pt
\evensidemargin=25pt

\pagestyle{plain}


%%%%%% Adds hyperlinks
\usepackage[colorlinks, linkcolor=black, citecolor=blue,
	% pagebackref,
 	%bookmarksnumbered=true
	]{hyperref}
	
	
	
%%%%%% Tikz !!! Commands and Macros %%%%%%%%%%%%%
\usepackage{tikz}
\usetikzlibrary{matrix}


%%%% These draw triple or quadruple set of arrows of length 0.5 cm
\DeclareMathOperator{\righttriplearrows} {{\; \tikz{ \foreach \y in {0, 0.1, 0.2} { \draw [-stealth] (0, \y) -- +(0.5, 0);}} \; }}
\DeclareMathOperator{\lefttriplearrows} {{\; \tikz{ \foreach \y in {0, 0.1, 0.2} { \draw [stealth-] (0, \y) -- +(0.5, 0);}} \; }}
\DeclareMathOperator{\rightquadarrows} {{\; \tikz{ \foreach \y in {0, 0.1, 0.2, 0.3} { \draw [-stealth] (0, \y) -- +(0.5, 0);}} \; }}
\DeclareMathOperator{\leftquadarrows} {{\; \tikz{ \foreach \y in {0, 0.1, 0.2, 0.3} { \draw [stealth-] (0, \y) -- +(0.5, 0);}} \; }}

%%%%%%% End TikZ Commands and Macros %%%%%%%%%%%%%



%%%%%%%%%%%%%%%%%%%%%% Theorem Styles and Counters %%%%%%%%%%%%%%%%%%%%%%%%%%
% These all use the same "theorem" counter. 
\theoremstyle{plain} %%% Plain Theorem Styles.
\newtheorem{theorem}{Theorem}[section]
\newtheorem{lemma}[theorem]{Lemma}
\newtheorem{corollary}[theorem]{Corollary}          
\newtheorem{proposition}[theorem]{Proposition}              

\theoremstyle{definition} %%%% Definition-like Commands  
\newtheorem{definition}[theorem]{Definition}

\theoremstyle{remark}  %%%% Remark-like Commands
\newtheorem{remark}[theorem]{Remark}
\newtheorem{example}[theorem]{Example}
%%%%%%%%%%%%%%%%%%%%%% End Theorem Styles and Counters %%%%%%%%%%%%%%%%%%%%%%%%%%

%%%% Misc symbols %%%%%

\newcommand{\nn}{\nonumber}
\newcommand{\nid}{\noindent}
\newcommand{\ra}{\rightarrow}
\newcommand{\la}{\leftarrow}
\newcommand{\xra}{\xrightarrow}
\newcommand{\xla}{\xleftarrow}

\newcommand{\Bord}{\mathrm{Bord}}
\newcommand{\Vect}{\mathrm{Vect}}
\newcommand{\TC}{\mathrm{TC}}

\def\cA{\mathcal A}\def\cB{\mathcal B}\def\cC{\mathcal C}\def\cD{\mathcal D}
\def\cE{\mathcal E}\def\cF{\mathcal F}\def\cG{\mathcal G}\def\cH{\mathcal H}
\def\cI{\mathcal I}\def\cJ{\mathcal J}\def\cK{\mathcal K}\def\cL{\mathcal L}
\def\cM{\mathcal M}\def\cN{\mathcal N}\def\cO{\mathcal O}\def\cP{\mathcal P}
\def\cQ{\mathcal Q}\def\cR{\mathcal R}\def\cS{\ess}\def\cT{\mathcal T}
\def\cU{\mathcal U}\def\cV{\mathcal V}\def\cW{\mathcal W}\def\cX{\mathcal X}
\def\cY{\mathcal Y}\def\cZ{\mathcal Z}

\def\AA{\mathbb A}\def\BB{\mathbb B}\def\CC{\mathbb C}\def\DD{\mathbb D}
\def\EE{\mathbb E}\def\FF{\mathbb F}\def\GG{\mathbb G}\def\HH{\mathbb H}
\def\II{\mathbb I}\def\JJ{\mathbb J}\def\KK{\mathbb K}\def\LL{\mathbb L}
\def\MM{\mathbb M}\def\NN{\mathbb N}\def\OO{\mathbb O}\def\PP{\mathbb P}
\def\QQ{\mathbb Q}\def\RR{\mathbb R}\def\SS{\mathbb S}\def\TT{\mathbb T}
\def\UU{\mathbb U}\def\VV{\mathbb V}\def\WW{\mathbb W}\def\XX{\mathbb X}
\def\YY{\mathbb Y}\def\ZZ{\mathbb Z}

%%%%%%%%%















% 0. Abstract
%
% 1. Introduction
% 1.1. Background and motivation
% 1.2. Results
% 1.3. Acknowledgments
% 
%
% 2. Tensor categories
% 2.1. Linear categories
% 2.2. Tensor products and colimits of linear categories
% 2.3. Tensor category bimodules and bimodule composition
% 2.4. The 3-category of tensor categories.
% 2.5. Fusion categories
%
% 3. Local field theory in dimension 3
% 3.1 Dualizability in 3-categories
% 3.2 Structure groups of 3-manifolds
%
%
% 4. Dualizability and fusion categories
% 4.1. Fusion categories are dualizable
% 4.1.1 Duals of 0-morphisms
% 4.1.2 Duals of 1-morphisms
% 4.1.2 Duals of 2-morphisms
% 4.2. [Dualizable tensor categories are fusion] --- [title modified as appropriate]
%
%%% Old outline:
%%4.1.1. Functors of finite semisimple module categories have duals
%%4.1.2. Indecomposable modules with braided commutant have duals
%%     [Prop: Given C fusion, C--M--Vect indecomposable with C' braided, then M has an ambiadjoint.]
%%4.1.3. Fusion categories have duals 
%%%
%
% 5. The Serre automorphism of a fusion category
% 5.1. The double dual is the Serre automorphism
% 5.1.1. 3-framed 1-manifolds and the Serre automorphism
% 5.1.2. Computing the Serre automorphism
%     [Thm: Serre(C) = [**].]
% 5.2. The quadruple dual is trivial
%     [Bimodulification Lemma]
%     [Thm: If C is dualizable, that is fusion, then ****=1.]
%
%6. Pivotality as a descent condition
%6.1. Fusion category TFTs are string
%6.2. Pivotal fusion category TFTs are orpo    
	%[Thm: A fusion category is pivotal if and only if the associated TFT is orpo.]
%6.3. Structure groups of fusion category TFTs.
   % [Conj: All TC-valued TFTs are orpo.] [This conj is equivalent to ENO.]
   % [Conj: All TC-valued orpo TFTs are oriented.] [Sketch: Drinfeld centers of pivotal fusion categories are anomaly free modular, therefore oriented 123; pushout to show oriented as 0123.]



\begin{document}

\title{Dualizable Tensor Categories}

\begin{abstract}

\end{abstract}
	
\author{Christopher L. Douglas}
\address{Department of Mathematics, University of California, Berkeley, CA 94720, USA}
\email{cdouglas@math.berkeley.edu}
	
\author{Christopher Schommer-Pries}
\address{Department of Mathematics \\
%	Harvard University\\
%	1 Oxford St.\\
%	Cambridge, MA 02138
} % Current Address
\email{schommerpries.chris.math@gmail.com}

\author{Noah Snyder}
\address{}
\email{nsnyder@math.berkeley.edu}

\maketitle	
\tableofcontents
%%%%%%%%

\section{Introduction}
% 1. Introduction
% 1.1. Background and motivation
% 1.2. Results
% 1.3. Acknowledgments

\CDcomm{The story: ``Fusion categories provided local field theories, and the structure (pivotality) of the fusion category corresponds to the structure (spinness) of the local field theory."}

\CDcomm{We are aiming to keep this paper to 30 pages.}


\subsection{Background and motivation}

\subsection{Results}.

%%

The first half of the paper, sections~\ref{sec-lft} and~\ref{sec-dfc}, focuses on [local field theory in dimension three and the dualizability of fusion categories].

The main theorem:
\begin{theorem}
Fusion categories are dualizable.
\end{theorem}
%!% Keep this statement this short and snappy in the introduction.  In the main text it can be fleshed out with more precision about the ambient 3-category.

A key application of this theorem is the construction of a plethora of local field theories:
\begin{corollary}
For any fusion category there is a local topological quantum field theory whose value on a point is that fusion category.
\end{corollary}

In particular, the theorem provides localizations of Turaev-Viro field theories:
\begin{corollary}
There is a local field theory whose value on a circle is the center of the fusion category of representations of a loop group at (any nondegenerate?) level.
\end{corollary}
Of course, the fusion category of representations of a loop group is merely an example, and can be replaced by any fusion category here.  This result is related to recent work of Kirillov and Balsam~\cite{kirillovbalsam}, which constructs a semi-local (that is, $1+1+1$-dimensional) version of Turaev-Viro theory.  In particular, our $0+1+1+1$-dimensional theory has the same value on a circle as the Kirillov-Balsam theory.  
%!% Add?: Assuming widely believed statements about the classification of 123 theories, it follows that our theory agrees with KB on 123 manifolds.
%!% Add mention of how sphericality comes in?

%%

The second half of the paper, sections~\ref{sec-serre} and~\ref{sec-pivot}, focuses on [Serre/doubledual/pivotality].

The main theorem:
\begin{theorem}
The Serre automorphism of a fusion category $\cC$ is the bimodule associated to the double dual functor $**: \cC \ra \cC$.
\end{theorem}

Because the Serre automorphism is necessarily order 2, this theorem provides a simple topological proof of the following generalization (?) of a theorem of ENO:
\begin{corollary}
The quadruple dual functor $****: \cC \ra \cC$ on a fusion category is naturally isomorphic to the identity functor.
\end{corollary}

A key insight resulting from the field-theoretic perspective on fusion categories is that the ENO conjecture, namely that fusion categories are pivotal, is equivalent to the spin-independence of the topological field theories associated to fusion categories:
\begin{theorem}
A fusion category $\cC$ is pivotal if and only if the local field theory associated to $\cC$ is independent of spin structure.
\end{theorem}
A precise formulation of this result, in terms of a descent condition for the bordism structure group of the local field theory, is given in section~\ref{sec-pivot-...}.
%\begin{theorem}
%A fusion category $\cC$ is pivotal if and only if the tensor-category-valued local field theory $F_{\cC} : \StringBord_0^3 \ra \TC$ associated to $\cC$ descends to a field theory on oriented $p_1$ bordism.
%\end{theorem}



\subsection*{Acknowledgments}
Andr\'e Henriques, Scott Morrison, Kevin Walker.

\CD{CD's comment color}
\CSP{CSP's comment color}
\NS{NS's comment color}


\section{Tensor categories} \label{sec-tc}

\CD{The organization of this section might well change as we decide what exactly we should include.}

\CDcomm{Our goal is to define $TC(3)$, ie [...].  We are not trying to include a huge discussion of all the different variations $TC(i)$, which can occur in DTCII.}

\subsection{Linear categories} \label{sec-tc-lincat}
.

[Fix the ground ring to be $\CC$?  Linear categories will mean linear over $\CC$?]

\subsection{Tensor products and colimits of linear categories} \label{sec-tc-tensorprod}

\subsection{Tensor category bimodules and bimodule composition} \label{sec-tc-bimod}

\subsection{The 3-category of tensor categories} \label{sec-tc-threecat}
.

\CDcomm{Formalism for 3-categories.}

\CDcomm{Formalism for symmetric monoidal 3-categories.}
A symmetric monoidal category is the same as a functor $Fin_* \to \Cat$ which sends coproducts to products. A symmetric monoidal 3-category is a functor $Fin_* \to 3\text{-}\Cat$ which sends coproducts to products.

\CDcomm{Definition of TC.}
[Objects of TC are \emph{idempotent complete linear categories}.]


\subsection{Multi-Fusion categories} \label{sec-tc-fusion}

\CD{CD continues to vote for using 'fusion' rather than 'multifusion'}



\section{Local Field Theory in Dimension Three} \label{sec-lft}

\subsection{Dualizability in 3-categories} \label{sec-lft-dual}
.

[Adjunction convention: $\bimod{C}{M}{D} \adj  \bimod{D}{N}{C}$ if there are $M \dtimes_D N \ra C$ and $D \ra N \dtimes_C M$ satisfying the S relations.] 

\CSPcomm{Set up terminology for evaluation and coevaluation.}


\CSP{This proposition is stated as a remark, without proof in Jacob's paper.}

\begin{proposition}[Lurie, Remark 3.4.22]
	Let $C$ be a symmetric monoidal 3-category. Let $f: x \to y$ be a 1-morphism in $C$, and suppose that $f$ admits a right adjoint $f^R$,  so we have unit and counit maps $u:id_x \to f^R \circ f$ and $v:f \circ f^R \to id_y$. If $u$ and $v$ admit left adjoints $u^L$ and $v^L$, the $u^L$ and $v^L$ exhibit $f^R$ also as a left adjoint to $f$. 
\end{proposition}

\begin{proof}
	....
\end{proof}

\subsection{Structure groups of 3-manifolds} \label{sec-lft-struc}



\section{Dualizability and fusion categories} \label{sec-dualfusion}

%!% CD will begin prototyping this section next.

\subsection{Fusion categories are dualizable} \label{sec-df-fcd}


%!% CD: I forgot why we switched the outline to "Duals of 0/1/2-morphisms", so I switched it back to the geodesic ordering 2/1/0.

\subsubsection{Functors of finite semisimple module categories have adjoints} \label{sec-df-functors}

\CD{cf ENOPartII "Solution to item (1)"}

\begin{lemma}
Let $F: \bimod{\cC}{\cM}{\cD} \ra \bimod{\cC}{\cN}{\cD}$ be a functor of bimodule categories, and suppose the underlying functor $\tilde{F}: \cM \ra \cN$ of linear categories has an ambidextrous adjoint $\tilde{G}: \cN \ra \cM$.  Then $F$ has an ambidextrous adjoint $G: \bimod{\cC}{\cN}{\cD} \ra \bimod{\cC}{\cM}{\cD}$.
\end{lemma}

\begin{proof}
\CDcomm{Look at the wave}
\end{proof}

\begin{lemma}
Let $F: \bimod{\cC}{\cM}{\cD} \ra \bimod{\cC}{\cN}{\cD}$ be a functor of bimodule categories.  If $\cM$ and $\cN$ are semisimple categories with finitely many simple objects, then the functor $\tilde{F}: \cM \ra \cN$ of linear categories underlying $F$ has an ambidextrous adjoint.
\end{lemma}

\begin{proof}
\CDcomm{The ambidextrous adjoint $F^*$ is given by the transpose of the entrywise dual of $F$.  That is, write $F$ as a matrix of vector spaces in terms of a chosen basis of simple objects for $M$ and $N$, then take the matrix of dual vector spaces, and take the transpose.}
...
\end{proof}

\begin{proposition} \label{prop-functadj}
A functor $F: \bimod{\cC}{\cM}{\cD} \ra \bimod{\cC}{\cN}{\cD}$ of bimodules categories has an ambidextrous adjoint if the linear categories $\cM$ and $\cN$ are semisimple with finitely many simple objects.
\end{proposition}

This proposition follows from the above two lemmas.


\subsubsection{Indecomposable modules with braided fusion commutant have adjoints} \label{sec-df-modules}


\begin{lemma} \label{lemma-invertible}
Let $\cC$ be a fusion category and $\cM$ an indecomposable $\cC$-module.  Let $\cC'$ denote the commutant of $\cC$ acting on $\cM$.  In this case the bimodule $\bimod{\cC}{\cM}{\cC'}$ is invertible.
\end{lemma}

\begin{proof}
\CDcomm{See ENOPartIII Apr 7 "Lemma: When C is fusion" for a sketch.}
\CDcomm{Can this be shown only using that C is semisimple?}
\end{proof}

In the situation of this lemma we will denote the inverse of $\bimod{\cC}{\cM}{\cC'}$ by $\bimod{\cC'}{\cN}{\cC}$.

\begin{lemma} \label{lemma-conditional}
Let $\cC$ be a fusion category and $\cM$ an indecomposable $\cC$-module such that the commutant $\cC'$ of the $\cC$ action on $\cM$ is bradied fusion.  In this case there exist maps
\begin{itemize}
\item[] $\lambda : \bimod{\Vect}{\cC' \dtimes_{\cC'} \cC'}{\Vect} \dra \bimod{\Vect}{\Vect}{\Vect}$
\item[] $\mu : \bimod{\cC'}{\cC'}{\cC'} \dra \bimod{\cC'}{\cC' \dtimes \cC'}{\cC'}$
\end{itemize}
that form an adjunction
$$\bimod{\Vect}{\cC'}{\cC'} \adj \bimod{\cC'}{\cC'}{\Vect}.$$
\end{lemma}
\nid We refer to $\lambda$ and $\mu$ as "conditional expectation" maps.
\CD{If we have anything to say about the existence of conditional expectations when neither tensor category is $\Vect$, then this proposition can be generalized to that case.  In that case, the theorem below might also be able to be generalized.}
\CD{Do you need to assume fusion for the commutant, or is something weaker enough?}
\CD{Do you need to assume $\cC$ is fusion?}

\begin{proof}
\CD{cf ENOPartIII "Proposition3".}
The first map $\lambda : \cC' \ra \Vect$ is defined by $\lambda(x) = \Hom_{\cC'}(1,x)$.  The second map $\mu : \cC' \ra \cC' \dtimes \cC'$ is determined, using the left $\cC'$-module structure, by the condition that
$$\mu(1) = \sum_{\sigma \in \cI} {}^*\sigma \dtimes \sigma.$$
Here $\cI$ is a basis of simple objects of $\cC'$.

By construction $\mu$ is a left module map, but we also need to give $\mu$ the structure of a right module map.  It is sufficient to check that for $\tau \in \cC'$ simple, there is an isomorphism
$$\sum_{\sigma \in \cI} {{}^*\sigma} \dtimes \sigma \tau  \cong  \sum_{\sigma \in \cI} \tau {({}^*\sigma)} \dtimes \sigma$$
\CD{Is that right that there is no further condition, ie you pick the module structure to be whatever iso you want on each simple, and that's it?}
Let $N^a_{bc}$ denote the vector spaces defining the tensor structure on the tensor category $\cC'$.  We have a series of isomorphisms
\begin{align}
\sum_{\sigma \in \cI} {{}^*\sigma} \dtimes \sigma \tau
& =
\sum_{\sigma,\rho \in \cI} N^{\rho}_{\sigma \tau} {{}^*\sigma \dtimes \rho} \nn \\
& \cong{(1)}
\sum_{\sigma,\rho \in \cI} N^{{}^*\sigma}_{\tau {({}^*\rho})} {{}^*\sigma \dtimes \rho} \nn \\
& \cong{(2)}
\sum_{\sigma,\rho \in \cI} N^{{}^*\sigma}_{{({}^*\rho)} \tau} {{}^*\sigma \dtimes \rho} \nn \\
& \cong{(3)}
\sum_{\sigma,\rho \in \cI} N^{\rho}_{\sigma \tau} {\rho \dtimes \sigma^*} \nn \\
& =
\sum_{\sigma \in \cI} \sigma \tau \dtimes {\sigma^*} \nn \\
& \cong{(4)}
\sum_{\sigma \in \cI} \tau {({}^*\sigma)} \dtimes \sigma \nn 
\end{align}
The first isomorphism $N^{\rho}_{\sigma \tau} \cong N^{{}^*\sigma}_{\tau {({}^*\rho})}$ exists by the standard properties of structure constants for fusion categories. \CDcomm{say more?}
Because $\cC'$ is braided, the constant $N^{{}^*\sigma}_{\tau {({}^*\rho})}$ is isomorphic to $N^{{}^*\sigma}_{{({}^*\rho}) \tau}$, giving the second isomorphism.  Reindexing the sum by substituting $\rho^*$ for $\sigma$ and $\sigma^*$ for $\rho$ provides the third isomorphism.  Braiding $\sigma$ and $\tau$ and then substituting ${}^*\sigma$ for $\sigma$ produces the fourth isomorphism.  \CD{Was \emph{that} the argument? Using the braiding twice?}

Finally we need to know that the maps $\lambda$ and $\mu$ do indeed satisfy the adjunction S-relations.  The first relation is the composite
$$\bimod{\Vect}{\cC'}{\cC'} = \cC' \dtimes_{\cC'} \cC' \ra \cC' \dtimes_{\cC'} \cC' \dtimes \cC' \ra \Vect \dtimes \cC' = \cC'$$
sending $1$ to $\sum_{\sigma \in \cI} \Hom(1,{}^*\sigma) \sigma = Hom(1, {}^*1) 1 = Hom(1 \cdot 1, 1) 1 = 1$.
The second relation is the composite
$$\bimod{\cC'}{\cC'}{\Vect} = \cC' \dtimes_{\cC'} \cC' \ra \cC' \dtimes \cC' \dtimes_{\cC'} \cC' \ra \cC' \dtimes \Vect = \cC'$$
sending $1$ to $\sum_{\sigma \in \cI} {}^*\sigma \Hom(1,\sigma) = {}^*1 = 1$.  Both maps are indeed equivalent to the identity.


\end{proof}


\begin{theorem} \label{thm-indecompbraided}
Let $\cC$ be a tensor category and $\cM$ an indecomposable $\cC$-module.  If the commutant $\cC'^{\cM}$ is a braided fusion category, then the bimodule $\bimod{\cC}{\cM}{\Vect}$ has an ambidextrous adjoint.
\end{theorem}

\CD{cf ENOPartIII "Proof details" etc.}
\begin{proof}
Let $\bimod{\cC'}{\cN}{\cC}$ denote the inverse of the bimodule $\bimod{\cC}{\cM}{\cC'}$ provided by Lemma~\ref{lemma-invertible}.  We will construct an ambidextrous adjunction
$$\bimod{\cC}{\cM}{\Vect} \ambadj \bimod{\Vect}{\cN}{\cC}.$$

First we build the adjunction
$$\bimod{\cC}{\cM}{\Vect} \adj \bimod{\Vect}{\cN}{\cC}$$
as follows.  Write the bimodules $\bimod{\cC}{\cM}{\Vect}$ and $\bimod{\Vect}{\cN}{\cC}$ as tensor products:
$$\cM = \cM \dtimes_{\cC'} \cC'$$
$$\cN = \cC' \dtimes_{\cC'} \cN$$
The desired adjunction is the composite of the following two adjunctions:
\begin{enumerate}
\item $\bimod{\cC'}{\cC'}{\Vect} \adj \bimod{\Vect}{\cC'}{\cC'}$
\item $\bimod{\cC}{\cM}{\cC'} \adj \bimod{\cC'}{\cN}{\cC}$
\end{enumerate}
The bimodules $\bimod{\cC}{\cM}{\cC'}$ and $\bimod{\cC'}{\cN}{\cC}$ are inverse by construction, therefore adjoint as needed.  The unit and counit for the first adjunction are given by
\begin{itemize}
\item[] $\phi: \cC' \dtimes \cC' \ra \cC' \dtimes_{\cC'} \cC' = \cC'$
\item[] $\psi: \Vect \ra \cC' = \cC' \dtimes_{\cC'} \cC'$
\end{itemize}
The S-relations for this unit and counit can be checked as follows:
\begin{itemize}
\item[] $\bimod{\cC'}{\cC'}{\Vect} = \cC' \dtimes \Vect \ra \cC' \dtimes \cC' = \cC' \dtimes \cC' \dtimes_{\cC'} \cC' \ra \cC' \dtimes_{\cC'} \cC' \dtimes_{\cC'} \cC' = \cC' \dtimes_{\cC'} \cC' = \cC'$
\item[] $\bimod{\Vect}{\cC'}{\cC'} = \Vect \dtimes \cC' \ra \cC' \dtimes \cC' = \cC' \dtimes_{\cC'} \cC' \dtimes \cC' \ra \cC' \dtimes_{\cC'} \cC' \dtimes_{\cC'} \cC' \ra \cC' \dtimes_{\cC'} \cC' = \cC'$
\end{itemize}
\CD{The above adjunction could be generalized to $\cD$ instead of $\Vect$.}

Explicitly, the unit and counit for the adjunction $\bimod{\cC}{\cM}{\Vect} \adj \bimod{\Vect}{\cN}{\cC}$ are respectively the composites:
\begin{itemize}
\item[] $\cM \dtimes \cN = (\cM \dtimes_{\cC'} \cC') \dtimes (\cC' \dtimes_{\cC'} \cN) \xra{\phi} \cM \dtimes_{\cC'} \cC' \dtimes_{\cC'} \cN = \cM \dtimes_{\cC'} \cN \cong \cC$
\item[] $\Vect \xra{\psi} \cC' \dtimes_{\cC'} \cC' = \cC' \dtimes_{\cC'} \cC' \dtimes_{\cC'} \cC' \cong \cC' \dtimes_{\cC'} \cN \dtimes_{\cC} \cM \dtimes_{\cC'} \cC' = \cN \dtimes_{\cC} \cM$
\end{itemize}

Second we construct an adjunction
$$\bimod{\Vect}{\cN}{\cC} \adj \bimod{\cC}{\cM}{\Vect}.$$
Again this adjunction is constructed as the composite of two adjunctions, namely
\begin{enumerate}
\item $\bimod{\Vect}{\cC'}{\cC'} \adj \bimod{\cC'}{\cC'}{\Vect}$
\item $\bimod{\cC'}{\cN}{\cC} \adj \bimod{\cC}{\cM}{\cC'}$
\end{enumerate}
The bimodules $\bimod{\cC'}{\cN}{\cC}$ and $\bimod{\cC}{\cM}{\cC'}$ are inverse, so again adjoint as needed.  The unit and counit for the first adjunction, namely
\begin{itemize}
\item[] $\lambda : \bimod{\Vect}{\cC' \dtimes_{\cC'} \cC'}{\Vect} \dra \bimod{\Vect}{\Vect}{\Vect}$
\item[] $\mu : \bimod{\cC'}{\cC'}{\cC'} \dra \bimod{\cC'}{\cC' \dtimes \cC'}{\cC'}$,
\end{itemize}
are provided by Lemma~\ref{lemma-conditional}.
\CD{Since this adjunction depends on the lemma, which depends on $\cD = \Vect$, we don't know how to generalize this part at the moment.}
\end{proof}


     
\subsubsection{Fusion categories have duals} \label{sec-df-categories}

\begin{theorem} \label{thm-fcd}
[... fusion categories have duals ...]
\end{theorem}


\begin{remark}
In non-zero characteristic, it is not the case that all fusion categories are dualizable.  In particular, if the global dimension of a fusion category is zero, then the category cannot be dualizable.
\end{remark}

Recalling the discussion of dualizability in 3-categories from Section~\ref{sec-lft-dual}, the theorem follows from the following three propositions.

\begin{proposition}
Every tensor category $\cC \in \TC$ has a dual in the homotopy category of $\TC$, namely the monoidal opposite category $\cC^{\mp}$.
\end{proposition}
%\nid Here the tensor category $\cC^{\mp}$ has the same underlying category as $\cC$ but with the opposite tensor structure.

\begin{proof}
The unit of the duality is $\cC$ as a $\cC \dtimes \cC^{\mp}$--$\Vect$ bimodule.  The counit of the duality is $\cC$ as a $\Vect$--$\cC^{\mp} \dtimes \cC$ bimodule.
\end{proof}

\begin{proposition} \label{prop-moduleadj}
Let $\cC$ be a fusion category.  The unit $\bimod{\cC \dtimes \cC^{\mp}}{\cC}{\Vect}$ and counit $\bimod{\Vect}{\cC}{\cC^{\mp} \dtimes \cC}$ of the duality between $\cC$ and $\cC^{\mp}$ both have ambidextrous adjoints.
\end{proposition}

\begin{proof}
The unit category $\cC$ is indecomposable as a $\cC \dtimes \cC^{\mp}$-module, and the commutant of this module structure is the Drinfeld center $Z(\cC)$ which is braided fusion, by [...].  Theorem~\ref{thm-indecompbraided} therefore ensures that this module has an ambidextrous adjoint.  The argument for the counit is analogous.
\end{proof}

Let $\bimod{\cC \dtimes \cC^{\mp}}{\cC}{\Vect} \ambadj \bimod{\Vect}{\cD_1}{\cC \dtimes \cC^{\mp}}$ and $\bimod{\Vect}{\cC}{\cC^{\mp} \dtimes \cC} \ambadj \bimod{\cC^{\mp} \dtimes \cC}{\cD_2}{\Vect}$ denote the ambidextrous adjoints provided by Proposition~\ref{prop-moduleadj}.

\begin{proposition}
For $\cC$ a fusion category, the units and counits of the four adjunctions $\bimod{\cC \dtimes \cC^{\mp}}{\cC}{\Vect} \adj \cD_1$, $\cD_1 \adj \bimod{\cC \dtimes \cC^{\mp}}{\cC}{\Vect}$, $\bimod{\Vect}{\cC}{\cC^{\mp} \dtimes \cC} \adj \cD_2$, and $\cD_2 \adj \bimod{\Vect}{\cC}{\cC^{\mp} \dtimes \cC}$ all have ambidextrous adjoints.
\end{proposition}

\begin{proof}
\CDcomm{Need to know that the inverse bimodule provided by Lemma~\ref{lemma-invertible} is finite semisimple.  Given that, this follows from Proposition~\ref{prop-functadj}. (This uses the fact that $A \dtimes_{\cC} B$ is finite semisimple if $A,B,\cC$ all are.)}
...
\end{proof}


\subsection{Examples of dualization structures} \label{sec-df-examples}

For a variety of fusion categories, we explicitly describe the dualization structure provided by Theorem~\ref{thm-fcd}, namely the dual categories and the adjunctions and higher adjunctions for the adjunctions, and so on.  Throughout we use implicitly that the adjunctions are all ambidextrous. 

\begin{example}
$Rep(Z/2) = \{\CC[x]/(x^2-1)\}-\mod$ (Symmetric)
[Start this by copying the content from Wave "Rep(Z/2)", ENO Part III, on Jun 22.]
\end{example}

\begin{example}
Here we give an example of a tensor category that is not fusion, namely $\{\CC[x]/x^2\}-\mod$ and highlight the failure of dualizability. \CDcomm{Prove that this category is not dualizable.  Cf Wave ENO Part III on May 4 search for "An example to consider", and Wave ENO Part III on Jun 17, search for "is still not fully dualizable".}
\end{example}

\begin{example}
$Vect[G,\lambda]$ for some simple $G$ and $\lambda$?
\end{example}

\begin{example}
Fibonacci category (Modular)
\end{example}

\begin{example}
$D_4$ (or even part of $E_6$) as small non-braided category.  $Z(D_4) = A_5 [x] Z/3$.
\end{example}


%\subsubsection{Duals of 0-morphisms} All tensor categories have dual tensor categories.
%\subsubsection{Duals of 1-morphisms} 
%\subsubsection{Duals of 2-morphisms}

%
% CD: I have commented out this subsection, as I think it will really end up being in DTCII.
% 
% \subsection{Dualizable tensor categories are fusion} \label{sec-df-dtcf}
%
% \CD{This section might be omitted in a future version of this paper, or the title modified as appropriate.}








%%%%%%%%%%

\section{The Serre automorphism of a fusion category} \label{sec-serre}


\subsection{The double dual is the Serre automorphism} \label{sec-serre-dd}

\subsubsection{3-framed 1-manifolds and the Serre automorphism} \label{sec-serre-oneman}

\subsubsection{Computing the Serre automorphism} \label{sec-serre-comp}
\begin{theorem}
Serre(C) = [**].
\end{theorem}

\subsection{The quadruple dual is trivial} \label{sec-serre-quad}

\begin{lemma}
[Bimodulification Lemma]
\end{lemma}

\begin{theorem} 
If C is dualizable, that is fusion, then $****=1$.
\end{theorem}




%%%%%%

\section{Pivotality as a descent condition} \label{sec-pivot}

\subsection{Fusion category TFTs are string} \label{sec-pivot-string}

\subsection{Pivotal fusion category TFTs are orpo} \label{sec-pivot-orpo}

\subsection{Structure groups of fusion category TFTs} \label{sec-pivot-struc}

\begin{theorem}
A fusion category is pivotal if and only if the associated TFT is orpo.
\end{theorem}

In particular, the ENO conjecture can be reformulated as follows:
\begin{conjecture}
Every local framed topological field theory with values in tensor categories descends to an oriented $p_1$ field theory.
\end{conjecture}

\begin{conjecture}
All TC-valued orpo TFTs are oriented.
\end{conjecture}
\CD{Do we think we can prove this or should we leave it as a conjecture?}

\begin{proof}[Sketch]
Drinfeld centers of pivotal fusion categories are anomaly free modular (ref Mueger), therefore oriented 123; pushout to show oriented as 0123.
\end{proof}


%%%%%%%%

%%% I left these as demonstartion, please comment out. -CSP
%\CD{I haven't tried to fit what CSP wrote below into the above outline structure.}
%\CD{Known how to get these margin notes to fit?}
%\marginpar{\small CD: I haven't tried to fit what CSP wrote below into the above outline structure.}
%\marginpar{\small CD: Known how to get these margin notes to fit?}


%% The Bibliography
\bibliographystyle{amsalpha}
\bibliography{DTCreferences}

\end{document}
