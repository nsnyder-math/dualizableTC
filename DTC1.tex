
%%% The DTC.tex file
%%% Authors: Christopher Douglas, Christopher Schommer-Pries, and Noah Snyder

\documentclass{amsart}


%%%%%%% Standard Packages
\usepackage{amsmath}       % I think this gives me some symbols
\usepackage{amsthm}        % Does theorem stuff
\usepackage{amssymb}       % more symbols and fonts
\usepackage{amsfonts}
\usepackage[all]{xy}
\usepackage{xspace}
\usepackage{calc}



\setlength{\topskip}{0pt}
\setlength{\footskip}{30pt}
\headheight=0pt
\topmargin=0pt
\headsep=18pt
\textheight=603pt %% 792pt to page, 648 is 9in
\textwidth=420pt  %% 612pt to page, 468pt is 6.5in
\oddsidemargin=25pt
\evensidemargin=25pt

\pagestyle{plain}


%%%%%% Adds hyperlinks
\usepackage[colorlinks, linkcolor=black, citecolor=blue,
	% pagebackref,
 	%bookmarksnumbered=true
	]{hyperref}
	
	
	
%%%%%% Tikz !!! Commands and Macros %%%%%%%%%%%%%
\usepackage{tikz}
\usetikzlibrary{matrix}


%%%% These draw triple or quadruple set of arrows of length 0.5 cm
\DeclareMathOperator{\righttriplearrows} {{\; \tikz{ \foreach \y in {0, 0.1, 0.2} { \draw [-stealth] (0, \y) -- +(0.5, 0);}} \; }}
\DeclareMathOperator{\lefttriplearrows} {{\; \tikz{ \foreach \y in {0, 0.1, 0.2} { \draw [stealth-] (0, \y) -- +(0.5, 0);}} \; }}
\DeclareMathOperator{\rightquadarrows} {{\; \tikz{ \foreach \y in {0, 0.1, 0.2, 0.3} { \draw [-stealth] (0, \y) -- +(0.5, 0);}} \; }}
\DeclareMathOperator{\leftquadarrows} {{\; \tikz{ \foreach \y in {0, 0.1, 0.2, 0.3} { \draw [stealth-] (0, \y) -- +(0.5, 0);}} \; }}

%%%%%%% End TikZ Commands and Macros %%%%%%%%%%%%%



%%%%%%%%%%%%%%%%%%%%%% Theorem Styles and Counters %%%%%%%%%%%%%%%%%%%%%%%%%%
% These all use the same "theorem" counter. 
\theoremstyle{plain} %%% Plain Theorem Styles.
\newtheorem{theorem}{Theorem}[section]
\newtheorem{lemma}[theorem]{Lemma}
\newtheorem{corollary}[theorem]{Corollary}          
\newtheorem{proposition}[theorem]{Proposition}              

\theoremstyle{definition} %%%% Definition-like Commands  
\newtheorem{definition}[theorem]{Definition}

\theoremstyle{remark}  %%%% Remark-like Commands
\newtheorem{remark}[theorem]{Remark}
\newtheorem{example}[theorem]{Example}
%%%%%%%%%%%%%%%%%%%%%% End Theorem Styles and Counters %%%%%%%%%%%%%%%%%%%%%%%%%%

%%%% Misc symbols %%%%%

\newcommand{\nn}{\nonumber}
\newcommand{\nid}{\noindent}
\newcommand{\ra}{\rightarrow}
\newcommand{\la}{\leftarrow}
\newcommand{\xra}{\xrightarrow}
\newcommand{\xla}{\xleftarrow}

\newcommand{\Bord}{\mathrm{Bord}}
\newcommand{\Vect}{\mathrm{Vect}}
\newcommand{\TC}{\mathrm{TC}}

\def\cA{\mathcal A}\def\cB{\mathcal B}\def\cC{\mathcal C}\def\cD{\mathcal D}
\def\cE{\mathcal E}\def\cF{\mathcal F}\def\cG{\mathcal G}\def\cH{\mathcal H}
\def\cI{\mathcal I}\def\cJ{\mathcal J}\def\cK{\mathcal K}\def\cL{\mathcal L}
\def\cM{\mathcal M}\def\cN{\mathcal N}\def\cO{\mathcal O}\def\cP{\mathcal P}
\def\cQ{\mathcal Q}\def\cR{\mathcal R}\def\cS{\ess}\def\cT{\mathcal T}
\def\cU{\mathcal U}\def\cV{\mathcal V}\def\cW{\mathcal W}\def\cX{\mathcal X}
\def\cY{\mathcal Y}\def\cZ{\mathcal Z}

\def\AA{\mathbb A}\def\BB{\mathbb B}\def\CC{\mathbb C}\def\DD{\mathbb D}
\def\EE{\mathbb E}\def\FF{\mathbb F}\def\GG{\mathbb G}\def\HH{\mathbb H}
\def\II{\mathbb I}\def\JJ{\mathbb J}\def\KK{\mathbb K}\def\LL{\mathbb L}
\def\MM{\mathbb M}\def\NN{\mathbb N}\def\OO{\mathbb O}\def\PP{\mathbb P}
\def\QQ{\mathbb Q}\def\RR{\mathbb R}\def\SS{\mathbb S}\def\TT{\mathbb T}
\def\UU{\mathbb U}\def\VV{\mathbb V}\def\WW{\mathbb W}\def\XX{\mathbb X}
\def\YY{\mathbb Y}\def\ZZ{\mathbb Z}

%%%%%%%%%















% 0. Abstract
%
% 1. Introduction
% 1.1. Background and motivation
% 1.2. Results
% 1.3. Acknowledgments
% 
%
% 2. Tensor categories
% 2.1. Linear categories
% 2.2. Tensor products and colimits of linear categories
% 2.3. Tensor category bimodules and bimodule composition
% 2.4. The 3-category of tensor categories.
% 2.5. Fusion categories
%
% 3. Local field theory in dimension 3
% 3.1 Dualizability in 3-categories
% 3.2 Structure groups of 3-manifolds
%
%
% 4. Dualizability and fusion categories
% 4.1. Fusion categories are dualizable
% 4.1.1 Duals of 0-morphisms
% 4.1.2 Duals of 1-morphisms
% 4.1.2 Duals of 2-morphisms
% 4.2. [Dualizable tensor categories are fusion] --- [title modified as appropriate]
%
%%% Old outline:
%%4.1.1. Functors of finite semisimple module categories have duals
%%4.1.2. Indecomposable modules with braided commutant have duals
%%     [Prop: Given C fusion, C--M--Vect indecomposable with C' braided, then M has an ambiadjoint.]
%%4.1.3. Fusion categories have duals 
%%%
%
% 5. The Serre automorphism of a fusion category
% 5.1. The double dual is the Serre automorphism
% 5.1.1. 3-framed 1-manifolds and the Serre automorphism
% 5.1.2. Computing the Serre automorphism
%     [Thm: Serre(C) = [**].]
% 5.2. The quadruple dual is trivial
%     [Bimodulification Lemma]
%     [Thm: If C is dualizable, that is fusion, then ****=1.]
%
%6. Pivotality as a descent condition
%6.1. Fusion category TFTs are string
%6.2. Pivotal fusion category TFTs are orpo    
	%[Thm: A fusion category is pivotal if and only if the associated TFT is orpo.]
%6.3. Structure groups of fusion category TFTs.
   % [Conj: All TC-valued TFTs are orpo.] [This conj is equivalent to ENO.]
   % [Conj: All TC-valued orpo TFTs are oriented.] [Sketch: Drinfeld centers of pivotal fusion categories are anomaly free modular, therefore oriented 123; pushout to show oriented as 0123.]



\begin{document}

\title{Dualizable Tensor Categories}

\begin{abstract}

\end{abstract}
	
\author{Christopher L. Douglas}
\address{Department of Mathematics, University of California, Berkeley, CA 94720, USA}
\email{cdouglas@math.berkeley.edu}
	
\author{Christopher Schommer-Pries}
\address{Department of Mathematics \\
%	Harvard University\\
%	1 Oxford St.\\
%	Cambridge, MA 02138
} % Current Address
\email{schommerpries.chris.math@gmail.com}

\author{Noah Snyder}
\address{}
\email{nsnyder@math.berkeley.edu}

\maketitle	
\tableofcontents
%%%%%%%%

\section{Introduction}
% 1. Introduction
% 1.1. Background and motivation
% 1.2. Results
% 1.3. Acknowledgments

\CDcomm{The story: ``Fusion categories provided local field theories, and the structure (pivotality) of the fusion category corresponds to the structure (spinness) of the local field theory."}

\CDcomm{We are aiming to keep this paper to 30 pages.}


\subsection{Background and motivation}

\subsection{Results}.

%%

The first half of the paper, sections~\ref{sec-lft} and~\ref{sec-dfc}, focuses on [local field theory in dimension three and the dualizability of fusion categories].

The main theorem:
\begin{theorem}
Fusion categories are dualizable.
\end{theorem}
%!% Keep this statement this short and snappy in the introduction.  In the main text it can be fleshed out with more precision about the ambient 3-category.

A key application of this theorem is the construction of a plethora of local field theories:
\begin{corollary}
For any fusion category there is a local topological quantum field theory whose value on a point is that fusion category.
\end{corollary}

In particular, the theorem provides localizations of Turaev-Viro field theories:
\begin{corollary}
There is a local field theory whose value on a circle is the center of the fusion category of representations of a loop group at (any nondegenerate?) level.
\end{corollary}
Of course, the fusion category of representations of a loop group is merely an example, and can be replaced by any fusion category here.  This result is related to recent work of Kirillov and Balsam~\cite{kirillovbalsam}, which constructs a semi-local (that is, $1+1+1$-dimensional) version of Turaev-Viro theory.  In particular, our $0+1+1+1$-dimensional theory has the same value on a circle as the Kirillov-Balsam theory.  
%!% Add?: Assuming widely believed statements about the classification of 123 theories, it follows that our theory agrees with KB on 123 manifolds.
%!% Add mention of how sphericality comes in?

%%

The second half of the paper, sections~\ref{sec-serre} and~\ref{sec-pivot}, focuses on [Serre/doubledual/pivotality].

The main theorem:
\begin{theorem}
The Serre automorphism of a fusion category $\cC$ is the bimodule associated to the double dual functor $**: \cC \ra \cC$.
\end{theorem}

Because the Serre automorphism is necessarily order 2, this theorem provides a simple topological proof of the following generalization (?) of a theorem of ENO:
\begin{corollary}
The quadruple dual functor $****: \cC \ra \cC$ on a fusion category is naturally isomorphic to the identity functor.
\end{corollary}

A key insight resulting from the field-theoretic perspective on fusion categories is that the ENO conjecture, namely that fusion categories are pivotal, is equivalent to the spin-independence of the topological field theories associated to fusion categories:
\begin{theorem}
A fusion category $\cC$ is pivotal if and only if the local field theory associated to $\cC$ is independent of spin structure.
\end{theorem}
A precise formulation of this result, in terms of a descent condition for the bordism structure group of the local field theory, is given in section~\ref{sec-pivot-...}.
%\begin{theorem}
%A fusion category $\cC$ is pivotal if and only if the tensor-category-valued local field theory $F_{\cC} : \StringBord_0^3 \ra \TC$ associated to $\cC$ descends to a field theory on oriented $p_1$ bordism.
%\end{theorem}



\subsection*{Acknowledgments}
Andr\'e Henriques, Scott Morrison, Kevin Walker.

\CD{CD's comment color}
\CSP{CSP's comment color}
\NS{NS's comment color}


\section{Tensor categories} \label{sec-tc}

\CD{The organization of this section might well change as we decide what exactly we should include.}

\CDcomm{Our goal is to define $TC(3)$, ie [...].  We are not trying to include a huge discussion of all the different variations $TC(i)$, which can occur in DTCII.}

\subsection{Linear categories} \label{sec-tc-lincat}
.

[Fix the ground ring to be $\CC$?  Linear categories will mean linear over $\CC$?]

\subsection{Tensor products and colimits of linear categories} \label{sec-tc-tensorprod}

\subsection{Tensor category bimodules and bimodule composition} \label{sec-tc-bimod}

\subsection{The 3-category of tensor categories} \label{sec-tc-threecat}
.

\CDcomm{Formalism for 3-categories.}

\CDcomm{Formalism for symmetric monoidal 3-categories.}
A symmetric monoidal category is the same as a functor $Fin_* \to \Cat$ which sends coproducts to products. A symmetric monoidal 3-category is a functor $Fin_* \to 3\text{-}\Cat$ which sends coproducts to products.

\CDcomm{Definition of TC.}
[Objects of TC are \emph{idempotent complete linear categories}.]


\subsection{Multi-Fusion categories} \label{sec-tc-fusion}

\CD{CD continues to vote for using 'fusion' rather than 'multifusion'}

\begin{definition}
A fusion category is a tensor category $\cC$ satisfying the following conditions:
\begin{enumerate}
\item For all objects $a, b \in \cC$, the vector space $\Hom(a,b)$ is finite dimensional.
\item The category $\cC$ is semisimple, with finitely many simple objects.
\item For every object $a \in \cC$, there exists an object $\ld{a} \in \cC$ that is a left dual for $a$ and there exists an object $\rd{a} \in \cC$ that is a right dual for $a$.
\end{enumerate}
\end{definition}

\CD{We need to write a macro for texing left and right duals in a way that looks nice and isn't a pain to type.}

\begin{remark}
Here for brevity we use ``fusion category" to refer to what has elsewhere gone under the name ``multi-fusion category".
\end{remark}

\begin{proposition}
If $\cC$ is a fusion category, then there exist monoidal functors $\ld{(-)} : \cC \ra \cC^{\op}$ and $\rd{(-)} : \cC \ra \cC^{\op}$ whose values on any object $a \in \cC$ are respectively a left dual object $\ld{(a)}$ and a right dual object $\rd{(a)}$ for $a$.
\end{proposition}

\begin{proof}
Define the functor $\ld{(-)}$ on objects by picking for each object $a \in \cC$ a left dual object $\ld{a} \in \cC$.  Also pick a unit map $u: \ld{a} \; a \ra 1$ and a counit map $v: 1 \ra a \; \ld{a}$ giving $\ld{a}$ the structure of a left dual to $a$.  Define the functor $\ld{(-)}$ on a morphism $f: a \ra b$ by $\ld{(f)} := (u_b \cdot \id_{\ld{a}}) (\id_{\ld{b}} \cdot f \cdot \id_{\ld{a}}) (\id_{\ld{b}} \cdot v_a)$.  (This morphism $\ld{(f)}$ is called the \emph{mate} of $F$.)  Next we want to see that this functor is monoidal.  Observe that the morphisms $\ld{b} \; \ld{a} \; a \; b \xra{u_a} \ld{b} \; b \xra{u_b} 1$ and $1 \xra{v_a} a \; \ld{a} \xra{v_b} a \; b \; \ld{b} \; \ld{a}$ show that $(\ld{b} \; \ld{a})$ is a left dual to $(a b)$.  There is therefore a uniquely determined isomorphism from $(\ld{b} \; \ld{a})$ to $\ld{(a b)}$.  These isomorphisms provide the left dual functor with a monoidal structure.  The right dual functor is analogous.  % In principle, need to check naturality and hexagon for those isos.
\end{proof}
% This fact is asserted in Bruce's thesis; the functor part is mentioned in Selinger Eg 4.4.


\section{Local Field Theory in Dimension Three} \label{sec-lft}

\subsection{Dualizability in 3-categories} \label{sec-lft-dual}
.

[Adjunction convention: $\bimod{C}{M}{D} \adj  \bimod{D}{N}{C}$ if there are $M \dtimes_D N \ra C$ and $D \ra N \dtimes_C M$ satisfying the S relations.] 

\CSPcomm{Set up terminology for evaluation and coevaluation.}


\CSP{This proposition is stated as a remark, without proof in Jacob's paper.}

\begin{proposition}[Lurie, Remark 3.4.22]
	Let $C$ be a symmetric monoidal 3-category. Let $f: x \to y$ be a 1-morphism in $C$, and suppose that $f$ admits a right adjoint $f^R$,  so we have unit and counit maps $u:id_x \to f^R \circ f$ and $v:f \circ f^R \to id_y$. If $u$ and $v$ admit left adjoints $u^L$ and $v^L$, the $u^L$ and $v^L$ exhibit $f^R$ also as a left adjoint to $f$. 
\end{proposition}

\begin{proof}
	....
\end{proof}

\subsection{Structure groups of 3-manifolds} \label{sec-lft-struc}



\section{Dualizability and fusion categories} \label{sec-dualfusion}

%!% CD will begin prototyping this section next.

\subsection{Fusion categories are dualizable} \label{sec-df-fcd}


%!% CD: I forgot why we switched the outline to "Duals of 0/1/2-morphisms", so I switched it back to the geodesic ordering 2/1/0.

\subsubsection{Functors of finite semisimple module categories have adjoints} \label{sec-df-functors}

\CD{cf ENOPartII "Solution to item (1)"}

\begin{lemma}
Let $F: \bimod{\cC}{\cM}{\cD} \ra \bimod{\cC}{\cN}{\cD}$ be a functor of bimodule categories, and suppose the underlying functor $\tilde{F}: \cM \ra \cN$ of linear categories has an ambidextrous adjoint $\tilde{G}: \cN \ra \cM$.  Then $F$ has an ambidextrous adjoint $G: \bimod{\cC}{\cN}{\cD} \ra \bimod{\cC}{\cM}{\cD}$.
\end{lemma}

\begin{proof}
\CDcomm{Look at the wave}
\end{proof}

\begin{lemma}
Let $F: \bimod{\cC}{\cM}{\cD} \ra \bimod{\cC}{\cN}{\cD}$ be a functor of bimodule categories.  If $\cM$ and $\cN$ are semisimple categories with finitely many simple objects, then the functor $\tilde{F}: \cM \ra \cN$ of linear categories underlying $F$ has an ambidextrous adjoint.
\end{lemma}

\begin{proof}
\CDcomm{The ambidextrous adjoint $\rd{F}$ is given by the transpose of the entrywise dual of $F$.  That is, write $F$ as a matrix of vector spaces in terms of a chosen basis of simple objects for $M$ and $N$, then take the matrix of dual vector spaces, and take the transpose.}
...
\end{proof}

\begin{proposition} \label{prop-functadj}
A functor $F: \bimod{\cC}{\cM}{\cD} \ra \bimod{\cC}{\cN}{\cD}$ of bimodules categories has an ambidextrous adjoint if the linear categories $\cM$ and $\cN$ are semisimple with finitely many simple objects.
\end{proposition}

This proposition follows from the above two lemmas.


\subsubsection{Indecomposable modules with braided fusion commutant have adjoints} \label{sec-df-modules}


\begin{lemma} \label{lemma-invertible}
Let $\cC$ be a fusion category and $\cM$ an indecomposable $\cC$-module.  Let $\cC'$ denote the commutant of $\cC$ acting on $\cM$.  In this case the bimodule $\bimod{\cC}{\cM}{\cC'}$ is invertible, \CDcomm{with inverse $\bimod{\cC'}{\Hom_{\cC}(\cM,\cC)}{\cC}$}.
\end{lemma}

\begin{proof}
\CDcomm{See ENOPartIII Apr 7 "Lemma: When C is fusion" for a sketch.}

\CDcomm{Can this be shown only using that C is semisimple?}

\CDcomm{Crucial point: the inverse is given by $\Hom_C(M,C)$, right?.  Emphasize this.}
\end{proof}

In the situation of this lemma we will abbreviate the inverse \CDcomm{$\bimod{\cC'}{\Hom_{\cC}(\cM,\cC)}{\cC}$} of $\bimod{\cC}{\cM}{\cC'}$ by $\bimod{\cC'}{\cN}{\cC}$.

\begin{lemma} \label{lemma-conditional}
Let $\cC$ be a fusion category and $\cM$ an indecomposable $\cC$-module such that the commutant $\cC'$ of the $\cC$ action on $\cM$ is bradied fusion.  In this case there exist maps
\begin{itemize}
\item[] $\lambda : \bimod{\Vect}{\cC' \dtimes_{\cC'} \cC'}{\Vect} \dra \bimod{\Vect}{\Vect}{\Vect}$
\item[] $\mu : \bimod{\cC'}{\cC'}{\cC'} \dra \bimod{\cC'}{\cC' \dtimes \cC'}{\cC'}$
\end{itemize}
that form an adjunction
$$\bimod{\Vect}{\cC'}{\cC'} \adj \bimod{\cC'}{\cC'}{\Vect}.$$
\end{lemma}
\nid We refer to $\lambda$ and $\mu$ as "conditional expectation" maps.
\CD{If we have anything to say about the existence of conditional expectations when neither tensor category is $\Vect$, then this proposition can be generalized to that case.  In that case, the theorem below might also be able to be generalized.}
\CD{Do you need to assume fusion for the commutant, or is something weaker enough?}
\CD{Do you need to assume $\cC$ is fusion?}

\begin{proof}
\CD{cf ENOPartIII "Proposition3".}
The first map $\lambda : \cC' \ra \Vect$ is defined by $\lambda(x) = \Hom_{\cC'}(1,x)$.  The second map $\mu : \cC' \ra \cC' \dtimes \cC'$ is determined, using the left $\cC'$-module structure, by the condition that
$$\mu(1) = \sum_{\sigma \in \cI} \ld{\sigma} \dtimes \sigma.$$
Here $\cI$ is a basis of simple objects of $\cC'$.

By construction $\mu$ is a left module map, but we also need to give $\mu$ the structure of a right module map.  It is sufficient to check that for $\tau \in \cC'$ simple, there is an isomorphism
$$\sum_{\sigma \in \cI} {\ld{\sigma}} \dtimes \sigma \tau  \cong  \sum_{\sigma \in \cI} \tau {(\ld{\sigma})} \dtimes \sigma$$
\CD{Is that right that there is no further condition, ie you pick the module structure to be whatever iso you want on each simple, and that's it?}
Let $N^a_{bc}$ denote the vector spaces defining the tensor structure on the tensor category $\cC'$.  We have a series of isomorphisms
\begin{align}
\sum_{\sigma \in \cI} {\ld{\sigma}} \dtimes \sigma \tau
& =
\sum_{\sigma,\rho \in \cI} N^{\rho}_{\sigma \tau} {\ld{\sigma} \dtimes \rho} \nn \\
& \cong{(1)}
\sum_{\sigma,\rho \in \cI} N^{\ld{\sigma}}_{\tau {(\ld{\rho}})} {\ld{\sigma} \dtimes \rho} \nn \\
& \cong{(2)}
\sum_{\sigma,\rho \in \cI} N^{\ld{\sigma}}_{{(\ld{\rho})} \tau} {\ld{\sigma} \dtimes \rho} \nn \\
& \cong{(3)}
\sum_{\sigma,\rho \in \cI} N^{\rho}_{\sigma \tau} {\rho \dtimes \rd{\sigma}} \nn \\
& =
\sum_{\sigma \in \cI} \sigma \tau \dtimes {\rd{\sigma}} \nn \\
& \cong{(4)}
\sum_{\sigma \in \cI} \tau {(\ld{\sigma})} \dtimes \sigma \nn 
\end{align}
The first isomorphism $N^{\rho}_{\sigma \tau} \cong N^{\ld{\sigma}}_{\tau {(\ld{\rho}})}$ exists by the standard properties of structure constants for fusion categories. \CDcomm{say more?}
Because $\cC'$ is braided, the constant $N^{\ld{\sigma}}_{\tau {(\ld{\rho}})}$ is isomorphic to $N^{\ld{\sigma}}_{{(\ld{\rho}}) \tau}$, giving the second isomorphism.  Reindexing the sum by substituting $\rd{\rho}$ for $\sigma$ and $\rd{\sigma}$ for $\rho$ provides the third isomorphism.  Braiding $\sigma$ and $\tau$ and then substituting $\ld{\sigma}$ for $\sigma$ produces the fourth isomorphism.  \CD{Was \emph{that} the argument? Using the braiding twice?}

Finally we need to know that the maps $\lambda$ and $\mu$ do indeed satisfy the adjunction S-relations.  The first relation is the composite
$$\bimod{\Vect}{\cC'}{\cC'} = \cC' \dtimes_{\cC'} \cC' \ra \cC' \dtimes_{\cC'} \cC' \dtimes \cC' \ra \Vect \dtimes \cC' = \cC'$$
sending $1$ to $\sum_{\sigma \in \cI} \Hom(1,\ld{\sigma}) \sigma = Hom(1, \ld{1}) 1 = Hom(1 \cdot 1, 1) 1 = 1$.
The second relation is the composite
$$\bimod{\cC'}{\cC'}{\Vect} = \cC' \dtimes_{\cC'} \cC' \ra \cC' \dtimes \cC' \dtimes_{\cC'} \cC' \ra \cC' \dtimes \Vect = \cC'$$
sending $1$ to $\sum_{\sigma \in \cI} \ld{\sigma} \Hom(1,\sigma) = \ld{1} = 1$.  Both maps are indeed equivalent to the identity.


\end{proof}


\begin{theorem} \label{thm-indecompbraided}
Let $\cC$ be a tensor category and $\cM$ an indecomposable $\cC$-module.  If the commutant $\cC'^{\cM}$ is a braided fusion category, then the bimodule $\bimod{\cC}{\cM}{\Vect}$ has an ambidextrous adjoint, \CDcomm{namely $\bimod{\Vect}{\Hom_{\cC}(\cM,\cC)}{\cC}$}.
\end{theorem}

\CD{cf ENOPartIII "Proof details" etc.}
\begin{proof}
Let $\bimod{\cC'}{\cN}{\cC}$ abbreviate the inverse \CDcomm{$\bimod{\cC'}{\Hom_{\cC}(\cM,\cC)}{\cC}$} of the bimodule $\bimod{\cC}{\cM}{\cC'}$ provided by Lemma~\ref{lemma-invertible}.  We will construct an ambidextrous adjunction
$$\bimod{\cC}{\cM}{\Vect} \ambadj \bimod{\Vect}{\cN}{\cC}.$$

First we build the adjunction
$$\bimod{\cC}{\cM}{\Vect} \adj \bimod{\Vect}{\cN}{\cC}$$
as follows.  Write the bimodules $\bimod{\cC}{\cM}{\Vect}$ and $\bimod{\Vect}{\cN}{\cC}$ as tensor products:
$$\cM = \cM \dtimes_{\cC'} \cC'$$
$$\cN = \cC' \dtimes_{\cC'} \cN$$
The desired adjunction is the composite of the following two adjunctions:
\begin{enumerate}
\item $\bimod{\cC'}{\cC'}{\Vect} \adj \bimod{\Vect}{\cC'}{\cC'}$
\item $\bimod{\cC}{\cM}{\cC'} \adj \bimod{\cC'}{\cN}{\cC}$
\end{enumerate}
The bimodules $\bimod{\cC}{\cM}{\cC'}$ and $\bimod{\cC'}{\cN}{\cC}$ are inverse by construction, therefore adjoint as needed.  The unit and counit for the first adjunction are given by
\begin{itemize}
\item[] $\phi: \cC' \dtimes \cC' \ra \cC' \dtimes_{\cC'} \cC' = \cC'$
\item[] $\psi: \Vect \ra \cC' = \cC' \dtimes_{\cC'} \cC'$
\end{itemize}
The S-relations for this unit and counit can be checked as follows:
\begin{itemize}
\item[] $\bimod{\cC'}{\cC'}{\Vect} = \cC' \dtimes \Vect \ra \cC' \dtimes \cC' = \cC' \dtimes \cC' \dtimes_{\cC'} \cC' \ra \cC' \dtimes_{\cC'} \cC' \dtimes_{\cC'} \cC' = \cC' \dtimes_{\cC'} \cC' = \cC'$
\item[] $\bimod{\Vect}{\cC'}{\cC'} = \Vect \dtimes \cC' \ra \cC' \dtimes \cC' = \cC' \dtimes_{\cC'} \cC' \dtimes \cC' \ra \cC' \dtimes_{\cC'} \cC' \dtimes_{\cC'} \cC' \ra \cC' \dtimes_{\cC'} \cC' = \cC'$
\end{itemize}
\CD{The above adjunction could be generalized to $\cD$ instead of $\Vect$.}

Explicitly, the unit and counit for the adjunction $\bimod{\cC}{\cM}{\Vect} \adj \bimod{\Vect}{\cN}{\cC}$ are respectively the composites:
\begin{itemize}
\item[] $\cM \dtimes \cN = (\cM \dtimes_{\cC'} \cC') \dtimes (\cC' \dtimes_{\cC'} \cN) \xra{\phi} \cM \dtimes_{\cC'} \cC' \dtimes_{\cC'} \cN = \cM \dtimes_{\cC'} \cN \cong \cC$
\item[] $\Vect \xra{\psi} \cC' \dtimes_{\cC'} \cC' = \cC' \dtimes_{\cC'} \cC' \dtimes_{\cC'} \cC' \cong \cC' \dtimes_{\cC'} \cN \dtimes_{\cC} \cM \dtimes_{\cC'} \cC' = \cN \dtimes_{\cC} \cM$
\end{itemize}

Second we construct an adjunction
$$\bimod{\Vect}{\cN}{\cC} \adj \bimod{\cC}{\cM}{\Vect}.$$
Again this adjunction is constructed as the composite of two adjunctions, namely
\begin{enumerate}
\item $\bimod{\Vect}{\cC'}{\cC'} \adj \bimod{\cC'}{\cC'}{\Vect}$
\item $\bimod{\cC'}{\cN}{\cC} \adj \bimod{\cC}{\cM}{\cC'}$
\end{enumerate}
The bimodules $\bimod{\cC'}{\cN}{\cC}$ and $\bimod{\cC}{\cM}{\cC'}$ are inverse, so again adjoint as needed.  The unit and counit for the first adjunction, namely
\begin{itemize}
\item[] $\lambda : \bimod{\Vect}{\cC' \dtimes_{\cC'} \cC'}{\Vect} \dra \bimod{\Vect}{\Vect}{\Vect}$
\item[] $\mu : \bimod{\cC'}{\cC'}{\cC'} \dra \bimod{\cC'}{\cC' \dtimes \cC'}{\cC'}$,
\end{itemize}
are provided by Lemma~\ref{lemma-conditional}.
\CD{Since this adjunction depends on the lemma, which depends on $\cD = \Vect$, we don't know how to generalize this part at the moment.}
\end{proof}


     
\subsubsection{Fusion categories have duals} \label{sec-df-categories}

\begin{theorem} \label{thm-fcd}
[... fusion categories have duals ...]
\end{theorem}


\begin{remark}
In non-zero characteristic, it is not the case that all fusion categories are dualizable.  In particular, if the global dimension of a fusion category is zero, then the category cannot be dualizable.
\end{remark}

Recalling the discussion of dualizability in 3-categories from Section~\ref{sec-lft-dual}, the theorem follows from the following three propositions.

\begin{proposition}
Every tensor category $\cC \in \TC$ has a dual in the homotopy category of $\TC$, namely the monoidal opposite category $\cC^{\mp}$.
\end{proposition}
%\nid Here the tensor category $\cC^{\mp}$ has the same underlying category as $\cC$ but with the opposite tensor structure.

\begin{proof}
The evaluation of the duality is $\cC$ as a $\cC \dtimes \cC^{\mp}$--$\Vect$ bimodule.  The coevaluation of the duality is $\cC$ as a $\Vect$--$\cC^{\mp} \dtimes \cC$ bimodule.
\end{proof}

\begin{proposition} \label{prop-moduleadj}
Let $\cC$ be a fusion category.  The evaluation $\bimod{\cC \dtimes \cC^{\mp}}{\cC}{\Vect}$ and coevaluation $\bimod{\Vect}{\cC}{\cC^{\mp} \dtimes \cC}$ of the duality between $\cC$ and $\cC^{\mp}$ both have ambidextrous adjoints.
\end{proposition}

\begin{proof}
The evaluation category $\cC$ is indecomposable as a $\cC \dtimes \cC^{\mp}$-module, and the commutant of this module structure is the Drinfeld center $Z(\cC)$ which is braided fusion, by [...].  Theorem~\ref{thm-indecompbraided} therefore ensures that this module has an ambidextrous adjoint.  \CDcomm{The argument for the coevaluation is analogous.} \CD{It is analogous, right??}
\end{proof}

Let $\bimod{\cC \dtimes \cC^{\mp}}{\cC}{\Vect} \ambadj \bimod{\Vect}{\cD_1}{\cC \dtimes \cC^{\mp}}$ and $\bimod{\Vect}{\cC}{\cC^{\mp} \dtimes \cC} \ambadj \bimod{\cC^{\mp} \dtimes \cC}{\cD_2}{\Vect}$ denote the ambidextrous adjoints provided by Proposition~\ref{prop-moduleadj}.

\begin{proposition}
For $\cC$ a fusion category, the units and counits of the four adjunctions $\bimod{\cC \dtimes \cC^{\mp}}{\cC}{\Vect} \adj \cD_1$, $\cD_1 \adj \bimod{\cC \dtimes \cC^{\mp}}{\cC}{\Vect}$, $\bimod{\Vect}{\cC}{\cC^{\mp} \dtimes \cC} \adj \cD_2$, and $\cD_2 \adj \bimod{\Vect}{\cC}{\cC^{\mp} \dtimes \cC}$ all have ambidextrous adjoints.
\end{proposition}

\begin{proof}
\CDcomm{Need to know that the inverse bimodule provided by Lemma~\ref{lemma-invertible} is finite semisimple.  Given that, this follows from Proposition~\ref{prop-functadj}. (This uses the fact that $A \dtimes_{\cC} B$ is finite semisimple if $A,B,\cC$ all are.)}
...
\end{proof}


\subsection{Examples of dualization structures} \label{sec-df-examples}

For a variety of fusion categories, we explicitly describe the dualization structure provided by Theorem~\ref{thm-fcd}, namely the dual categories and the adjunctions and higher adjunctions for the adjunctions, and so on.  Throughout we use implicitly that the adjunctions are all ambidextrous. 

\begin{example}
$Rep(Z/2) = \{\CC[x]/(x^2-1)\}-\mod$ (Symmetric)
[Start this by copying the content from Wave "Rep(Z/2)", ENO Part III, on Jun 22.]
\end{example}

\begin{example}
Here we give an example of a tensor category that is not fusion, namely $\{\CC[x]/x^2\}-\mod$ and highlight the failure of dualizability. \CDcomm{Prove that this category is not dualizable.  Cf Wave ENO Part III on May 4 search for "An example to consider", and Wave ENO Part III on Jun 17, search for "is still not fully dualizable".}
\end{example}

\begin{example}
$Vect[G,\lambda]$ for some simple $G$ and $\lambda$?
\end{example}

\begin{example}
Fibonacci category (Modular)
\end{example}

\begin{example}
$D_4$ (or even part of $E_6$) as small non-braided category.  $Z(D_4) = A_5 [x] Z/3$.
\end{example}


%\subsubsection{Duals of 0-morphisms} All tensor categories have dual tensor categories.
%\subsubsection{Duals of 1-morphisms} 
%\subsubsection{Duals of 2-morphisms}

%
% CD: I have commented out this subsection, as I think it will really end up being in DTCII.
% 
% \subsection{Dualizable tensor categories are fusion} \label{sec-df-dtcf}
%
% \CD{This section might be omitted in a future version of this paper, or the title modified as appropriate.}








%%%%%%%%%%

\section{The Serre automorphism of a fusion category} \label{sec-serre}


\CDcomm{(1a) Any fusion category is dualizable; (1b) any dualizable category has a Serre automorphism; (2) any fusion category has a monoidal functor $*: \cC \ra \cC^{\op}$.  Therefore for any fusion category it makes sense to compare the Serre bimodule and the bimodule associated to the tensor functor $** : \cC \ra \cC$.}


\subsection{The double dual is the Serre automorphism} \label{sec-serre-dd}



\subsubsection{n-framed 1-manifolds and the Serre automorphism} \label{sec-serre-oneman}

Recall that an n-framed k-manifold $(M,\tau)$ is a k-manifold $M$ equipped with a trivialization $\tau$ of $TM \oplus \RR^{n-k}$.  For $m < n$, an m-framed k-manifold $(M,\tau)$ is naturally n-framed by the trivialization $\tau \oplus \gamma$ where $\gamma$ is the canonical trivialization of $\RR^{n-m}$.  A convenient way to encode $(k+1)$-framed k-manifolds is as normally-oriented immersed k-manifolds in $\RR^{k+1}$.  That is, given an immersion $i: M^k \lra \RR^{k+1}$, the sum $TM \oplus \nu(M,\RR^{k+1})$ of the tangent and normal bundles is canonically trivialized, and a normal orientation for the immersion trivializes $\nu(M,\RR^{k+1})$, providing a trivialization of $TM \oplus \RR$ as desired.  More generally, by the same reasoning any coframed immersed k-manifold in $\RR^n$ is naturally n-framed.

We now define the Serre automorphism.  We presume $n \geq 2$ throughout.  For any point $p \in \RR^n$ we can equip the embedding $p \hra \RR^n$ with the canonical coframing, therefore n-framing.  We refer to such a point as the standard positively-oriented point---it is an object of $\FBord_0^n$, and we denote it by $s$.  Consider the normally-oriented immersed 1-manifold $\cS \lra \RR^2$ in Figure~\ref{fig-serre}.  This manifold is 2-framed, therefore n-framed for any $n \geq 2$.  It can be viewed as a morphism in $\FBord_0^n$ from $s$ to $s$.  This automorphism is called the \emph{universal Serre automorphism}.  

\CDcomm{[Figure of loop in R2, normally oriented] --- IMPT: for the upward normal orientation, the loop should go *down* (according to my random convention).  Is there an intrinsic way to distinguish S from its inverse?}

\begin{proposition}
When $n \geq 3$, the universal Serre automorphism $\cS: s \ra s$ in $\FBord_0^n$ is an involution; that is, $\cS \circ \cS : s \ra s$ is equivalent to the identity automorphism.
\end{proposition}

\begin{proof}
\CDcomm{[Just draw the bordisms back and forth and indicate the bordism from the two composites to the identity? Is there a clear way to say what those bordisms of 2-manifolds immersed in $\RR^3$ are?]}
\end{proof}
%This can be seen by observing that there is a normally-oriented immersed bordism in $\RR^3$ from the square of the Serre 1-manifold to the trivial 1-manifold.

Whenever $a$ is a dualizable object in a symmetric monoidal n-category $\cA$, there is a functor $\cF_a: \FBord_0^n \ra \cA$ taking the standard positively-oriented point to $a$.  The image of the universal Serre automorphism under this functor is called the Serre automorphism of $a$ and is denoted $\cS_a : a \ra a$.  This automorphism is again an involution.

\begin{corollary} \label{cor-serreinvol}
Let $a$ be a dualizable object of a symmetric monoidal n-category $\cA$.  Provided $n \geq 3$, the square $\cS_a^2 : a \ra a$ of the Serre automorphism $\cS_a$ of $a$ is equivalent to the identity of $a$.
\end{corollary}

\CD{Do we want to sort out the precise conditions needed, ie that $a \in \cA$ only needs to be 2-dualizable, or 2.5-dualizable?  This would slightly generalize the $****=1$ result.  cf ENOII.}


\subsubsection{Computing the Serre automorphism} \label{sec-serre-comp}


For $a \in \cA$ be a dualizable object of a symmetric monoidal n-category, let $\ev : a \otimes a^{\vee} \ra 1$ and $\coev: 1 \ra a^{\vee} \otimes a$ denote the evaluation and coevaluation maps for the duality between $a \in \cA$ and its dual object $a^{\vee} \in \cA$.  Let $\rd{\ev}: 1 \ra a \otimes a^{\vee}$ be the right adjoint to $\ev$ and let $\ld{\coev}: a^{\vee} \otimes a \ra 1$ be the left adjoint to $\coev$.  Let $\tau: a \otimes a \ra a \otimes a$ denote the symmetric monoidal switch.

\begin{proposition} \label{prop-serrecomp}
Let $a \in \cA$ be a dualizable object of the symmetric monoidal n-category $\cA$, for $n \geq 2$.  The Serre automorphism $\cS_a : a \ra a$ is equivalent to both of the following composites:
\begin{align}
\cS_a & \simeq (\id_a \otimes \rd{\ev}) (\tau \otimes \id_{a^{\vee}}) (\id_a \otimes \ev) \nn\\
\cS_a & \simeq (\coev \otimes \id_a) (\id_{a^{\vee}} \otimes \tau) (\ld{\coev} \otimes \id_a) \nn
\end{align}
\end{proposition}
\CD{the composition order here is geometric}

The expressions given hold in the universal case of framed bordism, and we leave the proof as an excercise in adjunctions of 2-framed 1-manifolds.

\begin{remark}
If $n \geq 3$, then the adjunctions of 1-morphisms are ambidextrous, and so the equations for the Serre automorphism given in the proposition could as well have used $\ld{\ev}$ and $\rd{\coev}$ instead.
\end{remark}

%%%

We now specialize to our case of interest, namely where the dualizable object in question is a fusion category $\cC \in \TC$.

%\CD{Where did we establish that $Hom_D(M,D)$ is a left/right adjoint to M, etc, and under what conditions? --- the identification of Serre depended on this!}

\begin{theorem} \label{thm-serre}
If $\cC$ is a fusion category, then the Serre bimodule $\bimod{\cC}{\cS}{\cC}$ for $\cC$ is Morita equivalent to the $\cC$-$\cC$ bimodule associated to the left double dual monoidal functor $\ldd{(-)}: \cC \ra \cC$.
\end{theorem}

\begin{proof}

%Recall that the evaluation and coevaluation of the duality between $\cC$ and $\cC^{\mp}$ are the bimodules $\ev_{\cC} = \bimod{\cC \otimes \cC^{\mp}}{\cC}{\Vect}$ and $\coev_{\cC} = \bimod{\Vect}{\cC}{\cC^{\mp} \otimes \cC}$.  By Theorem~\ref{thm-indecompbraided} and Proposition~\ref{prop-moduleadj}, these bimodules have adjoints, namely $\rd{\ev} = \bimod{\Vect}{\Hom_{(\cC \otimes \cC^{\mp})-\mod}(\cC,\cC \otimes \cC^{\mp})}{\cC \otimes \cC^{\mp}}$ and $\ld{\coev} = \bimod{\cC^{\mp} \otimes \cC}{\Hom_{\mod-(\cC^{\mp} \otimes \cC)}(\cC,\cC^{\mp} \otimes \cC)}{\Vect}$.

Recall that the evaluation of the duality between $\cC$ and $\cC^{\mp}$ is the bimodule $\ev_{\cC} = \bimod{\cC \otimes \cC^{\mp}}{\cC}{\Vect}$.  By Theorem~\ref{thm-indecompbraided} and Proposition~\ref{prop-moduleadj}, this bimodule has an adjoint, namely $\rd{\ev} = \bimod{\Vect}{\Hom_{\cC \otimes \cC^{\mp}}(\cC,\cC \otimes \cC^{\mp})}{\cC \otimes \cC^{\mp}}$.  By Proposition~\ref{prop-serrecomp}, the Serre bimodule $\bimod{\cC}{\cS}{\cC}$ can be expressed as
\begin{equation} \nn
\cS \simeq (\cC \otimes \Hom_{\cC \otimes \cC^{\mp}}(\cC,\cC \otimes \cC^{\mp})) \dtimes_{\cC \otimes \cC \otimes \cC^{\mp}} (\cC \otimes \cC \otimes \cC^{\mp}) \dtimes_{\cC \otimes \cC \otimes \cC^{\mp}} (\cC \otimes \cC)
\end{equation}
which can be compacted into the expression
\begin{equation} \nn
\cS \simeq (\cC \otimes \Hom_{\cC \otimes \cC^{\mp}}(\cC,\cC \otimes \cC^{\mp})) \dtimes_{\cC \otimes \cC \otimes \cC^{\mp}} (\cC \otimes \cC)
\end{equation}
where the $\cC \otimes \cC \otimes \cC^{\mp}$ action on $(\cC \otimes \Hom_{\cC \otimes \cC^{\mp}}(\cC,\cC \otimes \cC^{\mp}))$ is the tensor of the expected right actions, but the $\cC \otimes \cC \otimes \cC^{\mp}$ action on $(\cC \otimes \cC)$ is given by switching the two $\cC$ factors of $\cC \otimes \cC \otimes \cC^{\mp}$ and then acting by the tensor of the expected left actions.

The bimodule associated to the double dual functor $\ldd{(-)}: \cC \ra \cC$ can be written as $\bimod{\cC}{\cD}{\cC} := \bimod{\cC}{\cC \dtimes_{\cC} \cC}{\cC}$; here the $\cC$-$\cC$ bimodule structure on ${\cC \dtimes_{\cC} \cC}$ is standard, and the tensor $\dtimes_{\cC}$ occurs with respect to the standard right action of $\cC$ on $\cC$ and with respect to the \emph{right} double dual left action of $\cC$ on $\cC$, that is the left action induced by the functor $\rd{(-)}: \cC \ra \cC$.

As the double dual bimodule is cyclic, the equivalence between the double dual bimodule and the Serre bimodule can be specified by its value on the unit:
\begin{align}
\cC \dtimes_{\cC} \cC 
& \xra{\phi} 
(\cC \otimes \Hom_{\cC \otimes \cC^{\mp}}(\cC, \cC \otimes \cC^{\mp})) \dtimes_{\cC \otimes \cC \otimes \cC^{\mp}} (\cC \otimes \cC) 
\nn\\
1 \dtimes 1 
& \mapsto 
(1 \otimes (1 \mapsto \sum_{\sigma \in \cI} (\sigma \otimes \ld{\sigma}))) \dtimes (1 \otimes 1)
\nn
\end{align}
As before, $\cI$ indexes the simple objects of $\cC$.  To check that this map is well defined, we need to know, since $a \otimes 1 \simeq 1 \otimes \ldd{a} \in \cC \dtimes_{\cC} \cC$, that $a \phi(1 \otimes 1) \simeq \phi(1 \otimes 1) \ldd{a}$.  \CDcomm{[Recheck the rest of this proof and add more explanation if necessary.]}  This is true provided $\sum_{\sigma \in \cI} \sigma \otimes a \; \ld{\sigma} \simeq \sum \sigma \; \ldd{a} \otimes \ld{\sigma}$.  That equivalence holds because of the isomorphism of structure constants,
\begin{equation}
N^{\ld{\tau}}_{a, \ld{\sigma}} = N^{\ldd{\sigma}}_{\ldd{\tau}, a} = N^{\sigma}_{\tau, \ldd{a}} \nn
\end{equation}
\CDcomm{[Comment about why this map is an equivalence.]}

\CD{The proof is slightly sketchy on the distinction between equality and iso and how much naturality is needed of the isos.}


\end{proof}

%!% \CDcomm{Comment: The above proof uses the expression for the adjoint that is Hom over $\cC \otimes \cC^{\mp}$.  I did it this way because the proofs in section 4 produce that as the adjoint.  However, we know, I think, that when an adjoint exists, the adjoint is also equal to $Hom_{\Vect}(\cC,\Vect)$---see ENOII "Next steps on adjoints", though that is written for right adjoints, so would show that the $\Vect$-Hom is an adjoint for the coevaluation.  In any case, we could instead have explained why the $\Vect$-Hom is also an adjoint, and then used that expression to calculate Serre.  The calculation of Serre is probably more straightforward then, as some of the fiddling has been moved into comparing the two adjoint expressions.  At a minimum, we should probably add a remark, attempted below.}

\begin{remark}
The Serre automorphism of a fusion category can be identified with the double dual in a slightly different way, as follows.  Observe that $\Hom_{(\Vect)-\mathrm{mod}}(\cC,\Vect)$ is a right adjoint to the coevaluation bimodule $\coev_{\cC} = \bimod{\Vect}{\cC}{\cC^{\mp} \otimes \cC}$.  \CDcomm{[sentence or more explaning why, or perhaps this fact will be a proposition earlier in the paper]}  Because the adjoints are ambidextrous, it is also a left adjoint, and so can be used in the second expression for the Serre automorphism in Proposition~\ref{serrecomp}.  This gives the equivalence
\begin{equation} \nn
\cS \simeq (\cC \otimes \cC) \dtimes_{\cC^{\mp} \otimes \cC \otimes \cC} (\Hom_{(\Vect)-\mathrm{mod}}(\cC,\Vect) \otimes \cC)
\end{equation}
Here the left action of $\cC^{\mp} \otimes \cC \otimes \cC$ is the expected one, but the right action is twisted by switching the two $\cC$ factors.  As before let $\cD = \cC \dtimes_{\cC} \cC$ be the double dual bimodule, where the right action of $\cC$ on $\cC$ is standard, but the left action of $\cC$ on $\cC$ is via the right double dual functor.  Now define an equivalence between the Serre bimodule $\cS$ and the double dual bimodule $\cD$, as follows:
\begin{align}
(\cC \otimes \cC) \dtimes_{\cC^{\mp} \otimes \cC \otimes \cC} (\Hom(\cC,\Vect) \otimes \cC) & \ra \cC \dtimes_{\cC} \cC \nn\\
(m, n) \dtimes (e_p, q) & \mapsto (n \rd{p}, mq)
\end{align}
Here $e_p \in \Hom(\cC,\Vect)$ is defined by $e_p(r) = \Hom(p,r)$.  As in the above proof, you need to check that this map is well defined, and that it is an equivalence.
\end{remark}

\subsection{The quadruple dual is trivial} \label{sec-serre-quad}

\begin{lemma}[Bimodulification Lemma]
Let $f: \cC \ra \cD$ and $g: \cC \ra \cD$ be tensor functors with associated bimodules $\bimod{\cC}{\cM(f)}{\cD}$ and $\bimod{\cC}{\cM(g)}{\cD}$.  If there is a Morita equivalence between $\bimod{\cC}{\cM(f)}{\cD}$ and $\bimod{\cC}{\cM(g)}{\cD}$ such that [\CDcomm{what conditions go here?}], then $f$ and $g$ are naturally equivalence as monoidal functors.
\end{lemma}
\CD{cf "bimodulatification trivial" in ENOII.}

\begin{theorem} \label{thm-quaddual}
If $\cC$ is a fusion category, then the quadruple dual functor $****: \cC \ra \cC$ is naturally equivalent as a monoidal functor to the identity.
\end{theorem}
%\CDcomm{To state the theorem as "dualizable => ****=1", we'd need to show that a dualizable tensor category has a dualization functor $*$, which is a piece of the converse, ie of DTCII.}

\begin{proof}
...
\end{proof}



%%%%%%

\section{Pivotality as a descent condition} \label{sec-pivot}

\subsection{Fusion category TFTs are string} \label{sec-pivot-string}

\subsection{Pivotal fusion category TFTs are orpo} \label{sec-pivot-orpo}

\subsection{Structure groups of fusion category TFTs} \label{sec-pivot-struc}

\begin{theorem}
A fusion category is pivotal if and only if the associated TFT is orpo.
\end{theorem}

In particular, the ENO conjecture can be reformulated as follows:
\begin{conjecture}
Every local framed topological field theory with values in tensor categories descends to an oriented $p_1$ field theory.
\end{conjecture}

\begin{conjecture}
All TC-valued orpo TFTs are oriented.
\end{conjecture}
\CD{Do we think we can prove this or should we leave it as a conjecture?}

\begin{proof}[Sketch]
Drinfeld centers of pivotal fusion categories are anomaly free modular (ref Mueger), therefore oriented 123; pushout to show oriented as 0123.
\end{proof}


%%%%%%%%

%%% I left these as demonstartion, please comment out. -CSP
%\CD{I haven't tried to fit what CSP wrote below into the above outline structure.}
%\CD{Known how to get these margin notes to fit?}
%\marginpar{\small CD: I haven't tried to fit what CSP wrote below into the above outline structure.}
%\marginpar{\small CD: Known how to get these margin notes to fit?}


%% The Bibliography
\bibliographystyle{amsalpha}
\bibliography{DTCreferences}

\end{document}
