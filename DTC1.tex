
%%% The DTC.tex file
%%% Authors: Christopher Douglas, Christopher Schommer-Pries, and Noah Snyder

\documentclass{amsart}


%%%%%%% Standard Packages
\usepackage{amsmath}       % I think this gives me some symbols
\usepackage{amsthm}        % Does theorem stuff
\usepackage{amssymb}       % more symbols and fonts
\usepackage{amsfonts}
\usepackage[all]{xy}
\usepackage{xspace}
\usepackage{calc}



\setlength{\topskip}{0pt}
\setlength{\footskip}{30pt}
\headheight=0pt
\topmargin=0pt
\headsep=18pt
\textheight=603pt %% 792pt to page, 648 is 9in
\textwidth=420pt  %% 612pt to page, 468pt is 6.5in
\oddsidemargin=25pt
\evensidemargin=25pt

\pagestyle{plain}


%%%%%% Adds hyperlinks
\usepackage[colorlinks, linkcolor=black, citecolor=blue,
	% pagebackref,
 	%bookmarksnumbered=true
	]{hyperref}
	
	
	
%%%%%% Tikz !!! Commands and Macros %%%%%%%%%%%%%
\usepackage{tikz}
\usetikzlibrary{matrix}


%%%% These draw triple or quadruple set of arrows of length 0.5 cm
\DeclareMathOperator{\righttriplearrows} {{\; \tikz{ \foreach \y in {0, 0.1, 0.2} { \draw [-stealth] (0, \y) -- +(0.5, 0);}} \; }}
\DeclareMathOperator{\lefttriplearrows} {{\; \tikz{ \foreach \y in {0, 0.1, 0.2} { \draw [stealth-] (0, \y) -- +(0.5, 0);}} \; }}
\DeclareMathOperator{\rightquadarrows} {{\; \tikz{ \foreach \y in {0, 0.1, 0.2, 0.3} { \draw [-stealth] (0, \y) -- +(0.5, 0);}} \; }}
\DeclareMathOperator{\leftquadarrows} {{\; \tikz{ \foreach \y in {0, 0.1, 0.2, 0.3} { \draw [stealth-] (0, \y) -- +(0.5, 0);}} \; }}

%%%%%%% End TikZ Commands and Macros %%%%%%%%%%%%%



%%%%%%%%%%%%%%%%%%%%%% Theorem Styles and Counters %%%%%%%%%%%%%%%%%%%%%%%%%%
% These all use the same "theorem" counter. 
\theoremstyle{plain} %%% Plain Theorem Styles.
\newtheorem{theorem}{Theorem}[section]
\newtheorem{lemma}[theorem]{Lemma}
\newtheorem{corollary}[theorem]{Corollary}          
\newtheorem{proposition}[theorem]{Proposition}              

\theoremstyle{definition} %%%% Definition-like Commands  
\newtheorem{definition}[theorem]{Definition}

\theoremstyle{remark}  %%%% Remark-like Commands
\newtheorem{remark}[theorem]{Remark}
\newtheorem{example}[theorem]{Example}
%%%%%%%%%%%%%%%%%%%%%% End Theorem Styles and Counters %%%%%%%%%%%%%%%%%%%%%%%%%%

%%%% Misc symbols %%%%%

\newcommand{\nn}{\nonumber}
\newcommand{\nid}{\noindent}
\newcommand{\ra}{\rightarrow}
\newcommand{\la}{\leftarrow}
\newcommand{\xra}{\xrightarrow}
\newcommand{\xla}{\xleftarrow}

\newcommand{\Bord}{\mathrm{Bord}}
\newcommand{\Vect}{\mathrm{Vect}}
\newcommand{\TC}{\mathrm{TC}}

\def\cA{\mathcal A}\def\cB{\mathcal B}\def\cC{\mathcal C}\def\cD{\mathcal D}
\def\cE{\mathcal E}\def\cF{\mathcal F}\def\cG{\mathcal G}\def\cH{\mathcal H}
\def\cI{\mathcal I}\def\cJ{\mathcal J}\def\cK{\mathcal K}\def\cL{\mathcal L}
\def\cM{\mathcal M}\def\cN{\mathcal N}\def\cO{\mathcal O}\def\cP{\mathcal P}
\def\cQ{\mathcal Q}\def\cR{\mathcal R}\def\cS{\ess}\def\cT{\mathcal T}
\def\cU{\mathcal U}\def\cV{\mathcal V}\def\cW{\mathcal W}\def\cX{\mathcal X}
\def\cY{\mathcal Y}\def\cZ{\mathcal Z}

\def\AA{\mathbb A}\def\BB{\mathbb B}\def\CC{\mathbb C}\def\DD{\mathbb D}
\def\EE{\mathbb E}\def\FF{\mathbb F}\def\GG{\mathbb G}\def\HH{\mathbb H}
\def\II{\mathbb I}\def\JJ{\mathbb J}\def\KK{\mathbb K}\def\LL{\mathbb L}
\def\MM{\mathbb M}\def\NN{\mathbb N}\def\OO{\mathbb O}\def\PP{\mathbb P}
\def\QQ{\mathbb Q}\def\RR{\mathbb R}\def\SS{\mathbb S}\def\TT{\mathbb T}
\def\UU{\mathbb U}\def\VV{\mathbb V}\def\WW{\mathbb W}\def\XX{\mathbb X}
\def\YY{\mathbb Y}\def\ZZ{\mathbb Z}

%%%%%%%%%















% 0. Abstract
% 1. Introduction
% 1.1. Background and motivation
% 1.2. Results
% 1.3. Acknowledgments
% 
%
% 2. Tensor categories
% 2.1. Linear categories
% 2.2. Tensor products and colimits of linear categories
% 2.3. Tensor category bimodules and bimodule composition
% 2.4. The 3-category of tensor categories.
% 2.5. Fusion categories
%
% 3. Local field theory in dimension 3
% 3.1 Dualizability in 3-categories
% 3.2 Structure groups of 3-manifolds
%
%
% 4. Dualizability and fusion categories
% 4.1. Fusion categories are dualizable
% 4.1.1 Duals of 0-morphisms
% 4.1.2 Duals of 1-morphisms
% 4.1.2 Duals of 2-morphisms
% 4.2. [Dualizable tensor categories are fusion] --- [title modified as appropriate]
%
%%% Old outline:
%%4.1.1. Functors of finite semisimple module categories have duals
%%4.1.2. Indecomposable modules with braided commutant have duals
%%     [Prop: Given C fusion, C--M--Vect indecomposable with C' braided, then M has an ambiadjoint.]
%%4.1.3. Fusion categories have duals 
%%%
%
% 5. The Serre automorphism of a fusion category
% 5.1. The double dual is the Serre automorphism
% 5.1.1. 3-framed 1-manifolds and the Serre automorphism
% 5.1.2. Computing the Serre automorphism
%     [Thm: Serre(C) = [**].]
% 5.2. The quadruple dual is trivial
%     [Bimodulification Lemma]
%     [Thm: If C is dualizable, that is fusion, then ****=1.]
%
%6. Pivotality as a descent condition
%6.1. Fusion category TFTs are string
%6.2. Pivotal fusion category TFTs are orpo    
	%[Thm: A fusion category is pivotal if and only if the associated TFT is orpo.]
%6.3. Structure groups of fusion category TFTs.
   % [Conj: All TC-valued TFTs are orpo.] [This conj is equivalent to ENO.]
   % [Conj: All TC-valued orpo TFTs are oriented.] [Sketch: Drinfeld centers of pivotal fusion categories are anomaly free modular, therefore oriented 123; pushout to show oriented as 0123.]



\begin{document}

\title{Dualizable Tensor Categories}

\begin{abstract}

\end{abstract}
	
\author{Christopher L. Douglas}
\address{Department of Mathematics, University of California, Berkeley, CA 94720, USA}
\email{cdouglas@math.berkeley.edu}
	
\author{Christopher Schommer-Pries}
\address{Department of Mathematics \\
%	Harvard University\\
%	1 Oxford St.\\
%	Cambridge, MA 02138
} % Current Address
\email{schommerpries.chris.math@gmail.com}

\author{Noah Snyder}
\address{}
\email{}

\maketitle	
\tableofcontents
%%%%%%%%

\section{Introduction}
% 1. Introduction
% 1.1. Background and motivation
% 1.2. Results
% 1.3. Acknowledgments
\subsection{Background and motivation}

\subsection{Results}

\subsection{Acknowledgments}
Andr\'e Henriques, Scott Morrison, Kevin Walker.


\section{Tensor categories}
\CD{The organization of this section might well change as we decide what exactly we should include.  (CD's comment color)}
\CSP{CSP's comment color}
\NS{NS's comment color}

\subsection{Linear categories}
\subsection{Tensor products and colimits of linear categories}
\subsection{Tensor category bimodules and bimodule composition}
\subsection{The 3-category of tensor categories}

A symmetric monoidal category is the same as a functor $Fin_* \to \Cat$ which sends coproducts to products. A symmetric monoidal 3-category is a functor $Fin_* \to 3\text{-}\Cat$ which sends coproducts to products.

\subsection{Multi-Fusion categories}
\CD{CD continues to vote for using 'fusion' rather than 'multifusion'}



\section{Local Field Theory in Dimension Three}

\subsection{Local Field Theory in Dimension Three}
\subsection{Dualizability in 3-categories}

\CSP{This proposition is stated as a remark, without proof in Jacob's paper.}

\begin{proposition}[Lurie, Remark 3.4.22]
	Let $C$ be a symmetric monoidal 3-category. Let $f: x \to y$ be a 1-morphism in $C$, and suppose that $f$ admits a right adjoint $f^R$,  so we have unit and counit maps $u:id_x \to f^R \circ f$ and $v:f \circ f^R \to id_y$. If $u$ and $v$ admit left adjoints $u^L$ and $v^L$, the $u^L$ and $v^L$ exhibit $f^R$ also as a left adjoint to $f$. 
\end{proposition}

\begin{proof}
	....
\end{proof}

\subsection{Structure groups of 3-manifolds}



\section{Dualizability and fusion categories}

%!% CD will begin prototyping this section next.

\subsection{Fusion categories are dualizable}
\subsubsection{Duals of 0-morphisms} All tensor categories have dual tensor categories.

{\color{CSPcolor} Set up terminology for evaluation and coevaluation.}

\subsubsection{Duals of 1-morphisms} 
\subsubsection{Duals of 2-morphisms}
\subsection{Dualizable tensor categories are fusion}
\CD{This section might be omitted in a future version of this paper, or the title modified as appropriate.}


%%%%%%%%%%

\section{The Serre automorphism of a fusion category}


\subsection{The double dual is the Serre automorphism}

\subsubsection{3-framed 1-manifolds and the Serre automorphism}

\subsubsection{Computing the Serre automorphism}
\begin{theorem}
Serre(C) = [**].
\end{theorem}

\subsection{The quadruple dual is trivial}

[Bimodulification Lemma]

\begin{theorem} 
If C is dualizable, that is fusion, then ****=1.
\end{theorem}




%%%%%%

\section{Pivotality as a descent condition}

\subsection{Fusion category TFTs are string}

\subsection{Pivotal fusion category TFTs are orpo}

\begin{theorem}
A fusion category is pivotal if and only if the associated TFT is orpo.
\end{theorem}

\subsection{Structure groups of fusion category TFTs}
   [Conj: All TC-valued TFTs are orpo.] [This conj is equivalent to ENO.]
   [Conj: All TC-valued orpo TFTs are oriented.] [Sketch: Drinfeld centers of pivotal fusion categories are anomaly free modular, therefore oriented 123; pushout to show oriented as 0123.]


%%%%%%%%

\nid ---------------------\\
---------------------
%%% I left these as demonstartion, please comment out. -CSP
\CD{I haven't tried to fit what CSP wrote below into the above outline structure.}
\CD{Known how to get these margin notes to fit?}
%\marginpar{\small CD: I haven't tried to fit what CSP wrote below into the above outline structure.}
%\marginpar{\small CD: Known how to get these margin notes to fit?}


%% The Bibliography
\bibliographystyle{amsalpha}
\bibliography{DTCreferences}

\end{document}
