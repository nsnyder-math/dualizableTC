\documentclass{amsart}


%%%%%%% Standard Packages
\usepackage{amsmath}       % I think this gives me some symbols
\usepackage{amsthm}        % Does theorem stuff
\usepackage{amssymb}       % more symbols and fonts
\usepackage{amsfonts}
\usepackage[all]{xy}
\usepackage{xspace}
\usepackage{calc}



\setlength{\topskip}{0pt}
\setlength{\footskip}{30pt}
\headheight=0pt
\topmargin=0pt
\headsep=18pt
\textheight=603pt %% 792pt to page, 648 is 9in
\textwidth=420pt  %% 612pt to page, 468pt is 6.5in
\oddsidemargin=25pt
\evensidemargin=25pt

\pagestyle{plain}


%%%%%% Adds hyperlinks
\usepackage[colorlinks, linkcolor=black, citecolor=blue,
	% pagebackref,
 	%bookmarksnumbered=true
	]{hyperref}
	
	
	
%%%%%% Tikz !!! Commands and Macros %%%%%%%%%%%%%
\usepackage{tikz}
\usetikzlibrary{matrix}


%%%% These draw triple or quadruple set of arrows of length 0.5 cm
\DeclareMathOperator{\righttriplearrows} {{\; \tikz{ \foreach \y in {0, 0.1, 0.2} { \draw [-stealth] (0, \y) -- +(0.5, 0);}} \; }}
\DeclareMathOperator{\lefttriplearrows} {{\; \tikz{ \foreach \y in {0, 0.1, 0.2} { \draw [stealth-] (0, \y) -- +(0.5, 0);}} \; }}
\DeclareMathOperator{\rightquadarrows} {{\; \tikz{ \foreach \y in {0, 0.1, 0.2, 0.3} { \draw [-stealth] (0, \y) -- +(0.5, 0);}} \; }}
\DeclareMathOperator{\leftquadarrows} {{\; \tikz{ \foreach \y in {0, 0.1, 0.2, 0.3} { \draw [stealth-] (0, \y) -- +(0.5, 0);}} \; }}

%%%%%%% End TikZ Commands and Macros %%%%%%%%%%%%%



%%%%%%%%%%%%%%%%%%%%%% Theorem Styles and Counters %%%%%%%%%%%%%%%%%%%%%%%%%%
% These all use the same "theorem" counter. 
\theoremstyle{plain} %%% Plain Theorem Styles.
\newtheorem{theorem}{Theorem}[section]
\newtheorem{lemma}[theorem]{Lemma}
\newtheorem{corollary}[theorem]{Corollary}          
\newtheorem{proposition}[theorem]{Proposition}              

\theoremstyle{definition} %%%% Definition-like Commands  
\newtheorem{definition}[theorem]{Definition}

\theoremstyle{remark}  %%%% Remark-like Commands
\newtheorem{remark}[theorem]{Remark}
\newtheorem{example}[theorem]{Example}
%%%%%%%%%%%%%%%%%%%%%% End Theorem Styles and Counters %%%%%%%%%%%%%%%%%%%%%%%%%%

%%%% Misc symbols %%%%%

\newcommand{\nn}{\nonumber}
\newcommand{\nid}{\noindent}
\newcommand{\ra}{\rightarrow}
\newcommand{\la}{\leftarrow}
\newcommand{\xra}{\xrightarrow}
\newcommand{\xla}{\xleftarrow}

\newcommand{\Bord}{\mathrm{Bord}}
\newcommand{\Vect}{\mathrm{Vect}}
\newcommand{\TC}{\mathrm{TC}}

\def\cA{\mathcal A}\def\cB{\mathcal B}\def\cC{\mathcal C}\def\cD{\mathcal D}
\def\cE{\mathcal E}\def\cF{\mathcal F}\def\cG{\mathcal G}\def\cH{\mathcal H}
\def\cI{\mathcal I}\def\cJ{\mathcal J}\def\cK{\mathcal K}\def\cL{\mathcal L}
\def\cM{\mathcal M}\def\cN{\mathcal N}\def\cO{\mathcal O}\def\cP{\mathcal P}
\def\cQ{\mathcal Q}\def\cR{\mathcal R}\def\cS{\ess}\def\cT{\mathcal T}
\def\cU{\mathcal U}\def\cV{\mathcal V}\def\cW{\mathcal W}\def\cX{\mathcal X}
\def\cY{\mathcal Y}\def\cZ{\mathcal Z}

\def\AA{\mathbb A}\def\BB{\mathbb B}\def\CC{\mathbb C}\def\DD{\mathbb D}
\def\EE{\mathbb E}\def\FF{\mathbb F}\def\GG{\mathbb G}\def\HH{\mathbb H}
\def\II{\mathbb I}\def\JJ{\mathbb J}\def\KK{\mathbb K}\def\LL{\mathbb L}
\def\MM{\mathbb M}\def\NN{\mathbb N}\def\OO{\mathbb O}\def\PP{\mathbb P}
\def\QQ{\mathbb Q}\def\RR{\mathbb R}\def\SS{\mathbb S}\def\TT{\mathbb T}
\def\UU{\mathbb U}\def\VV{\mathbb V}\def\WW{\mathbb W}\def\XX{\mathbb X}
\def\YY{\mathbb Y}\def\ZZ{\mathbb Z}

%%%%%%%%%
















\usetikzlibrary{calc}

\usepgflibrary{decorations.pathreplacing}


 \tikzset{external/force remake}


\begin{document}
	
	\title{Sandbox}
	
	\maketitle

\section{NNN}

\begin{figure}[t]
\cb{
\begin{tikzpicture}
\draw[linestyle] (0,0) -- (\evlength+\loopsize,0)
arc (-90:180:\loopsize)
-- +(0,-\evlengthv-\loopsize-\loopsize)
arc (-180:90:\loopsize)
-- +(-\evlength-\loopsize,0);
\end{tikzpicture}
}
\end{figure}

\begin{figure}[t]
\cb{
\begin{tikzpicture}
[scale=.1]
\draw[linestyle] (0,0) -- (\evlength+\loopsize,0)
arc (-90:180:\loopsize)
-- +(0,-\evlengthv-\loopsize-\loopsize)
arc (-180:90:\loopsize)
-- +(-\evlength-\loopsize,0);
\end{tikzpicture}
}
\end{figure}


\end{document}










%/tikz/external/optimize=false
\begin{figure}[ht]
\cb{
\begin{tikzpicture}
%%% curved version
%{ [xshift=-3cm,scale=.1]
%\draw[linestyle,fuzzleft]
%(0,5) to [out=180, in=70] (-4,2.5)
%	to [looseness=1.6, out=-110, in=-90] (-6,2.5)
%	to [looseness=1.6, out=90, in=110] (-4,2.5)
%	to [out=-70, in=70] (-4,-2.5)
%	to [looseness=1.6, out=-110, in=-90] (-6,-2.5)
%	to [looseness=1.6, out=90, in=110] (-4,-2.5)
%	to [out=-70, in=180] (0,-5);
%\begin{pgfonlayer}{background}
%	\draw[->,outstyle] (0,5) -- +(0:10*\arrowlength);
%	\draw[->,outstyle] (0,-5) -- +(0:10*\arrowlength);
%\end{pgfonlayer}
%\node at (0,-10) {$coev^{LL}$};
%}
%%%
		\node at (.5,0) {$\cdots$};
		\node at (1,0) {$\dashv$};
{ [xshift=1.85cm]
		\node at (0,-1.1) {$\coev^{LL}$};
			\draw[linestyle,fuzzleft] (0.5, 0.5) -- (-.25,0.5)
				to [looseness=2,out = 180, in = 90] (0, 0.75)
				-- (0, -0.75)
				to [looseness=2,out = 270, in = 180] (-.25,-0.5)
				-- (0.5, -0.5);
			\begin{pgfonlayer}{background}
				\draw[->,outstyle] (0.5,0.5) -- +(0:\arrowlength);
				\draw[->,outstyle] (0.5,-0.5) -- +(0:\arrowlength);
			\end{pgfonlayer}
}
{ [xshift=3cm]
\node at (0,0) {$\dashv$};
}
{ [xshift=3.75cm]
\draw[linestyle,fuzzright]
(0,.5) .. controls (.66,.5) and (.66,-.5) .. (0,-.5);
}
{ [xshift=4cm]
\node at (0,-1) {$\coev^L$};
}
{ [xshift=5cm]
\node at (0,0) {$\dashv$};
}
{ [xshift=6.15cm]
\draw[linestyle,fuzzleft]
(0,.5) .. controls (-.66,.5) and (-.66,-.5) .. (0,-.5);
\begin{pgfonlayer}{background}
	\draw[->,outstyle] (0,.5) -- +(0:\arrowlength);
	\draw[->,outstyle] (0,-.5) -- +(0:\arrowlength);
\end{pgfonlayer}
}
{ [xshift=6cm]
\node at (0,-1) {$\coev$};
}
{ [xshift=7cm]
\node at (0,0) {$\dashv$};
}
%%% curved version, won't compile
{ [xshift=1.5cm,scale=.1]
\draw[linestyle,fuzzleft]%, /pgf/fpu,/pgf/fpu/output format=fixed]
(0,5) to [out=0, in=110] (4,2.5)
	to [looseness=1.6, out=-70, in=-90] (6,2.5)
	to [looseness=1.6, out=90, in=70] (4,2.5)
	to [out=-110, in=110] (4,-2.5)
	to [looseness=1.6, out=-70, in=-90] (6,-2.5);
%	to [looseness=1.6, out=90, in=70] (4,-2.5)
%	to [out=-110, in=0] (0,-5);
\node at (0,-10) {$coev^{R}$};
}
%%%
{ [xshift=7.85cm]
		\draw [linestyle,fuzzright] (-.25,0.5) -- (0.5, 0.5) 
			to [looseness=2,out=0, in=90] (0.25, 0.75)
			-- (0.25, -0.75)
			to [looseness=2,out=-90, in=0] (0.5, -0.5)
			-- (-.25,-0.5);
}
{ [xshift=8cm]
		\node at (0, -1.1) {$\coev^{R}$};
}
\node at (9,0) {$\dashv$};
\node at (9.5,0) {$\cdots$};
\end{tikzpicture}
}
\caption{Two infinite chains of adjunctions of 2-framed bordisms.} \label{fig:adjointchains}
\end{figure}

\end{document}

%\section{Bordism pics}
%
%%\begin{figure}[htbp]
%%	\begin{center}
%%		\begin{tikzpicture}[
%%			% This decoration will be used to make portions of curves into "bouckground curves". It allows us to dash a portion of the curve.
%%			decoration={border, 
%%				segment length = 4pt, 	% determines the distance between consecutive ticks.
%%				amplitude = 2pt, 		% determines the length of the ticks.
%%				angle = 0  				% determines the angle between the ticks and the line of the path. 
%%				}, 
%%			% Styles
%%			contour line/.style={thin, blue}
%%				]
%%				
%%			% Colors
%%			\colorlet{surfacecolor1}{blue!15}
%%			\colorlet{surfacecolor2}{blue!10}	
%%			
%%			% The graphic
%%			\draw [step = 1cm, help lines] (-0.9,-0.9) grid (11.9, 10.9);
%%			
%%			% First we fill the surfaces with the background colors.
%%			% region 1
%%			\fill [color = surfacecolor1] (0, 2.5) .. controls (0, 4) and (2,6.5) .. (1.75,8)
%%				arc (180:0:2cm and 3cm)
%%				%.. controls (3, 9) and (3,10) .. (4,10)
%%				%.. controls (4.5,10) and (6,9.5) .. (6,8)
%%				to [out = 270, in = 90] (6, 5.5)
%%				-- (6,0.5)
%%				arc (360: 180: 1cm and 0.5cm)
%%				-- (4, 6)
%%				arc (90:180:2cm and 3.5cm)
%%				%.. controls (1, 2) and (1.5, 2) .. (1, 2)
%%				to [out = 210, in = 0] (1,2)
%%				arc (270: 180: 1cm and 0.5cm);
%%			
%%			% region 2
%%			\fill [color = surfacecolor1] (7, 6.5) parabola bend (8.5, 5.8) (8.25, 5.8) 
%%				to [out = 45, in = 260] (9, 7)
%%				parabola bend (7.75, 6.4) (7, 6.5);
%%			
%%			% region 3
%%			\fill [color = surfacecolor2] (4, 6) arc (90:180:2cm and 3.5cm)
%%				-- (4, 1.5) -- (4,6);
%%			
%%			% region 4
%%			\fill [color = surfacecolor2] (6, 5.5) parabola (7, 6.5) parabola bend (8.5, 5.8) (11, 6)
%%				to [out = 240, in = 30] (8, 2.5) -- (6, 1.5) -- (6, 5.5);
%%			
%%			
%%			% Now we draw the lines of the surface
%%			%first around region 1
%%			\draw (0, 2.5) .. controls (0, 4) and (2,6.5) .. (1.75,8)
%%				arc (180:0:2cm and 3cm)
%%				%.. controls (3, 9) and (3,10) .. (4,10)
%%				%.. controls (4.5,10) and (6,9.5) .. (6,8)
%%				to [out = 270, in = 90] (6, 5.5)
%%				-- (6,0.5)
%%				arc (360: 180: 1cm and 0.5cm)
%%				-- (4, 6)
%%				arc (90:180:2cm and 3.5cm)
%%				%.. controls (1, 2) and (1.5, 2) .. (1, 2)
%%				to [out = 210, in = 0] (1,2)
%%				arc (270: 180: 1cm and 0.5cm);
%%			
%%			% then the rest 
%%			\draw (2,2.5) -- (4,1.5) 
%%				decorate {-- (5,1) to [out = -30, in = 135] (6,0.5)
%%					  (4,0.5) to [out = 45, in = 210] (5,1) -- (6,1.5)} 
%%				-- (8, 2.5) to [out = 30, in = 240] (11, 6);
%%				% then off to cusp points
%%			\draw decorate { (5, 1) arc (0:90:1cm and 5cm)};
%%			\draw decorate {(2,2.5) -- (7,5) to [out = 30, in = 225] (8.25, 5.8)} to [out = 45, in = 260] (9, 7) ;
%%			\draw decorate {(2,2.5) to [out = 150, in = 0] (1,3) arc (90:180:1cm and 0.5cm)};
%%			% cusp
%%			\draw (4, 6) to [out = 90, in = 270] (4.5, 8.5);
%%			\draw (6, 5.5) decorate { parabola (4.5, 8.5)} (6, 5.5) parabola (7, 6.5)
%%				parabola bend (7.75, 6.4) (9, 7);
%%			\draw (7, 6.5) parabola bend (8.5, 5.8) (11, 6);
%%			
%%%		% now the contour lines
%%%		% the first contour line
%%%		\draw [contour line] (2.25, 10) arc (180:360:1.5cm and 0.5cm)
%%%			decorate {(2.25, 10) arc (180:0:1.5cm and 0.5cm)};
%%%		
%%%		
%%%		% the second contour line
%%%		
%%%		% the third contour line
%%%		\draw [contour line] 
%%%			decorate {(1.25, 5.75) arc (180:90: 1cm and 0.5cm) to [out = 0, in = 150] (3.5,5.75)}
%%%			 -- (4, 5.5) 
%%%			decorate {-- (4.5,5.25) to [out = -30, in = 135] (6,5)}
%%%			arc (360: 180: 1cm and 0.5cm)
%%%			decorate { to [out = 45, in = 210] (4.5,5.25)};
%%%		
%%%		\draw [contour line] 
%%%		 (1.25, 5.75) arc (180: 270: 1cm and 0.5cm) to [out = 0, in = 210] (3.25,5.75)
%%%		 	 to [out = 30, in = 150] (5.25, 6.15) to [out = -30,in = 30] (4.5, 5.25);
%%			
%%			%\draw (0,3) decorate {to (2,3)} -- (3,4);
%%				
%%		\end{tikzpicture}
%%		\end{center}	
%%	\caption{Without contour lines}
%%	%\label{fig:label}
%%\end{figure}
%
%%\begin{figure}[htbp]
%%	\begin{center}
%%		\begin{tikzpicture}[
%%			% This decoration will be used to make portions of curves into "bouckground curves". It allows us to dash a portion of the curve.
%%			decoration={border, 
%%				segment length = 4pt, 	% determines the distance between consecutive ticks.
%%				amplitude = 2pt, 		% determines the length of the ticks.
%%				angle = 0  				% determines the angle between the ticks and the line of the path. 
%%				}, 
%%			% Styles
%%			contour line/.style={thin, blue}
%%				]
%%				
%%			% Colors
%%			\colorlet{surfacecolor1}{blue!15}
%%			\colorlet{surfacecolor2}{blue!10}	
%%			
%%			% The graphic
%%			\draw [step = 1cm, help lines] (-0.9,-0.9) grid (11.9, 10.9);
%%			
%%			% First we fill the surfaces with the background colors.
%%			% region 1
%%			\fill [color = surfacecolor1] (0, 2.5) .. controls (0, 4) and (2,6.5) .. (1.75,8)
%%				arc (180:0:2cm and 3cm)
%%				%.. controls (3, 9) and (3,10) .. (4,10)
%%				%.. controls (4.5,10) and (6,9.5) .. (6,8)
%%				to [out = 270, in = 90] (6, 5.5)
%%				-- (6,0.5)
%%				arc (360: 180: 1cm and 0.5cm)
%%				-- (4, 6)
%%				arc (90:180:2cm and 3.5cm)
%%				%.. controls (1, 2) and (1.5, 2) .. (1, 2)
%%				to [out = 210, in = 0] (1,2)
%%				arc (270: 180: 1cm and 0.5cm);
%%			
%%			% region 2
%%			\fill [color = surfacecolor1] (7, 6.5) parabola bend (8.5, 5.8) (8.25, 5.8) 
%%				to [out = 45, in = 260] (9, 7)
%%				parabola bend (7.75, 6.4) (7, 6.5);
%%			
%%			% region 3
%%			\fill [color = surfacecolor2] (4, 6) arc (90:180:2cm and 3.5cm)
%%				-- (4, 1.5) -- (4,6);
%%			
%%			% region 4
%%			\fill [color = surfacecolor2] (6, 5.5) parabola (7, 6.5) parabola bend (8.5, 5.8) (11, 6)
%%				to [out = 240, in = 30] (8, 2.5) -- (6, 1.5) -- (6, 5.5);
%%			
%%			
%%			% Now we draw the lines of the surface
%%			%first around region 1
%%			\draw (0, 2.5) .. controls (0, 4) and (2,6.5) .. (1.75,8)
%%				arc (180:0:2cm and 3cm)
%%				%.. controls (3, 9) and (3,10) .. (4,10)
%%				%.. controls (4.5,10) and (6,9.5) .. (6,8)
%%				to [out = 270, in = 90] (6, 5.5)
%%				-- (6,0.5)
%%				arc (360: 180: 1cm and 0.5cm)
%%				-- (4, 6)
%%				arc (90:180:2cm and 3.5cm)
%%				%.. controls (1, 2) and (1.5, 2) .. (1, 2)
%%				to [out = 210, in = 0] (1,2)
%%				arc (270: 180: 1cm and 0.5cm);
%%			
%%			% then the rest 
%%			\draw (2,2.5) -- (4,1.5) 
%%				decorate {-- (5,1) to [out = -30, in = 135] (6,0.5)
%%					  (4,0.5) to [out = 45, in = 210] (5,1) -- (6,1.5)} 
%%				-- (8, 2.5) to [out = 30, in = 240] (11, 6);
%%				% then off to cusp points
%%			\draw decorate { (5, 1) arc (0:90:1cm and 5cm)};
%%			\draw decorate {(2,2.5) -- (7,5) to [out = 30, in = 225] (8.25, 5.8)} to [out = 45, in = 260] (9, 7) ;
%%			\draw decorate {(2,2.5) to [out = 150, in = 0] (1,3) arc (90:180:1cm and 0.5cm)};
%%			% cusp
%%			\draw (4, 6) to [out = 90, in = 270] (4.5, 8.5);
%%			\draw (6, 5.5) decorate { parabola (4.5, 8.5)} (6, 5.5) parabola (7, 6.5)
%%				parabola bend (7.75, 6.4) (9, 7);
%%			\draw (7, 6.5) parabola bend (8.5, 5.8) (11, 6);
%%			
%%			% now the contour lines
%%			% the first contour line
%%			\draw [contour line] (2.3, 10) arc (180:360:1.45cm and 0.5cm)
%%				decorate {(2.3, 10) arc (180:0:1.45cm and 0.5cm)};
%%			
%%			
%%			% the second contour line
%%			\draw [contour line] (1.75,8) arc (180:320:1.45cm and 0.5cm)
%%				decorate {to [out = 30,in = 135](4.65, 7.6) to [out = -45, in = 45] (4.2, 7.1) }
%%				arc (180:270:0.5cm and 0.2cm) to [out = 0, in = -90] (5.75,8)
%%			;
%%			\draw [contour line] decorate {(1.75,8) arc (180:0:2cm and 0.6cm)};
%%			
%%			% the third contour line
%%			\draw [contour line] 
%%				decorate {(1.25, 5.75) arc (180:90: 1cm and 0.5cm) to [out = 0, in = 150] (3.5,5.75)}
%%				 -- (4, 5.5) 
%%				decorate {-- (4.5,5.25) to [out = -30, in = 135] (6,5)}
%%				arc (360: 180: 1cm and 0.5cm)
%%				decorate { to [out = 45, in = 210] (4.5,5.25)};
%%			
%%			\draw [contour line] 
%%			 (1.25, 5.75) arc (180: 270: 1cm and 0.5cm) to [out = 0, in = 210] (3.25,5.75)
%%			 	 to [out = 30, in = 150] (5.25, 6.15) to [out = -30,in = 30] (4.5, 5.25);
%%			
%%			%\draw (0,3) decorate {to (2,3)} -- (3,4);
%%				
%%		\end{tikzpicture}
%%		\end{center}	
%%	\caption{with contour lines}
%%	%\label{fig:label}
%%\end{figure}
%
%\begin{figure}[htbp]
%	\begin{center}
%		\begin{tikzpicture}[
%			% This decoration will be used to make portions of curves into "bouckground curves". It allows us to dash a portion of the curve.
%			decoration={border, 
%				segment length = 4pt, 	% determines the distance between consecutive ticks.
%				amplitude = 2pt, 		% determines the length of the ticks.
%				angle = 0  				% determines the angle between the ticks and the line of the path. 
%				}, 
%			% Styles
%			contour line/.style={thin, blue}
%				]
%				
%			% Colors
%			\colorlet{surfacecolor1}{blue!15}
%			\colorlet{surfacecolor2}{blue!10}	
%			
%			% The graphic
%			\draw [step = 1cm, help lines] (-3.9,-0.9) grid (11.9, 10.9);
%			
%			% First we fill the surfaces with the background colors.
%			% region 1
%			\fill [color = surfacecolor1] (0, 2.5) .. controls (0, 4) and (2,6.5) .. (1.75,8)
%				arc (180:0:2cm and 3cm)
%				%.. controls (3, 9) and (3,10) .. (4,10)
%				%.. controls (4.5,10) and (6,9.5) .. (6,8)
%				to [out = 270, in = 90] (6, 5.5)
%				-- (6,0.5)
%				arc (360: 180: 1cm and 0.5cm)
%				-- (4, 6)
%				arc (90:180:2cm and 3.5cm)
%				%.. controls (1, 2) and (1.5, 2) .. (1, 2)
%				to [out = 210, in = 0] (1,2)
%				arc (270: 180: 1cm and 0.5cm);
%			
%			% region 2
%			\fill [color = surfacecolor1] (7, 6.5) parabola bend (8.5, 5.8) (8.25, 5.8) 
%				to [out = 45, in = 260] (9, 7)
%				parabola bend (7.75, 6.4) (7, 6.5);
%			
%			% region 3
%			\fill [color = surfacecolor2] (4, 6) arc (90:180:2cm and 3.5cm)
%				-- (4, 1.5) -- (4,6);
%			
%			% region 4
%			\fill [color = surfacecolor2] (6, 5.5) parabola (7, 6.5) parabola bend (8.5, 5.8) (11, 6)
%				to [out = 240, in = 30] (8, 2.5) -- (6, 1.5) -- (6, 5.5);
%			
%			
%			% Now we draw the lines of the surface
%			%first around region 1
%			\draw (0, 2.5) .. controls (0, 4) and (2,6.5) .. (1.75,8)
%				arc (180:0:2cm and 3cm)
%				%.. controls (3, 9) and (3,10) .. (4,10)
%				%.. controls (4.5,10) and (6,9.5) .. (6,8)
%				to [out = 270, in = 90] (6, 5.5)
%				-- (6,0.5)
%				arc (360: 180: 1cm and 0.5cm)
%				-- (4, 6)
%				arc (90:180:2cm and 3.5cm)
%				%.. controls (1, 2) and (1.5, 2) .. (1, 2)
%				to [out = 210, in = 0] (1,2)
%				arc (270: 180: 1cm and 0.5cm);
%			
%			% then the rest 
%			\draw (2,2.5) -- (4,1.5) 
%				decorate {-- (5,1) to [out = -30, in = 135] (6,0.5)
%					  (4,0.5) to [out = 45, in = 210] (5,1) -- (6,1.5)} 
%				-- (8, 2.5) to [out = 30, in = 240] (11, 6);
%				% then off to cusp points
%			\draw decorate { (5, 1) arc (0:90:1cm and 5cm)};
%			\draw decorate {(2,2.5) -- (7,5) to [out = 30, in = 225] (8.25, 5.8)} to [out = 45, in = 260] (9, 7) ;
%			\draw decorate {(2,2.5) to [out = 150, in = 0] (1,3) arc (90:180:1cm and 0.5cm)};
%			% cusp
%			\draw (4, 6) to [out = 90, in = 270] (4.5, 8.5);
%			\draw (6, 5.5) decorate { parabola (4.5, 8.5)} (6, 5.5) parabola (7, 6.5)
%				parabola bend (7.75, 6.4) (9, 7);
%			\draw (7, 6.5) parabola bend (8.5, 5.8) (11, 6);
%			
%			% now the contour lines
%			% the first contour line
%			\draw [contour line] (2.3, 10) arc (180:360:1.45cm and 0.5cm)
%				decorate {(2.3, 10) arc (180:0:1.45cm and 0.5cm)};
%			
%			
%			% the second contour line
%			\draw [contour line] (1.75,8) arc (180:320:1.45cm and 0.5cm)
%				decorate {to [out = 30,in = 135](4.65, 7.6) to [out = -45, in = 45] (4.2, 7.1) }
%				arc (180:270:0.5cm and 0.2cm) to [out = 0, in = -90] (5.75,8)
%			;
%			\draw [contour line] decorate {(1.75,8) arc (180:0:2cm and 0.6cm)};
%			
%			% the third contour line
%			\draw [contour line] 
%				decorate {(1.25, 5.75) arc (180:90: 1cm and 0.5cm) to [out = 0, in = 150] (3.5,5.75)}
%				 -- (4, 5.5) 
%				decorate {-- (4.5,5.25) to [out = -30, in = 135] (6,5)}
%				arc (360: 180: 1cm and 0.5cm)
%				decorate { to [out = 45, in = 210] (4.5,5.25)};
%			
%			\draw [contour line] 
%			 (1.25, 5.75) arc (180: 270: 1cm and 0.5cm) to [out = 0, in = 210] (3.25,5.75)
%			 	 to [out = 30, in = 150] (5.25, 6.15) to [out = -30,in = 30] (4.5, 5.25);
%			
%			
%			\begin{scope}[xshift = -6cm]
%				% now the contour lines
%				% the first contour line
%				\draw [contour line] (2.3, 10) arc (180:360:1.45cm and 0.5cm)
%					 {(2.3, 10) arc (180:0:1.45cm and 0.5cm)};
%			
%			
%				% the second contour line
%				\draw [contour line] (1.75,8) arc (180:320:1.45cm and 0.5cm)
%					 {to [out = 30,in = 135](4.65, 7.6) to [out = -45, in = 45] (4.2, 7.1) }
%					arc (180:270:0.5cm and 0.2cm) to [out = 0, in = -90] (5.75,8)
%				;
%				\draw [contour line]  {(1.75,8) arc (180:0:2cm and 0.6cm)};
%			
%				% the third contour line
%				\draw [contour line] 
%					 {(1.25, 5.75) arc (180:90: 1cm and 0.5cm) to [out = 0, in = 150] (3.5,5.75)}
%					 -- (4, 5.5) 
%					 {-- (4.5,5.25) to [out = -30, in = 135] (6,5)}
%					arc (360: 180: 1cm and 0.5cm)
%					 { to [out = 45, in = 210] (4.5,5.25)};
%			
%				\draw [contour line] 
%				 (1.25, 5.75) arc (180: 270: 1cm and 0.5cm) to [out = 0, in = 210] (3.25,5.75)
%				 	 to [out = 30, in = 150] (5.25, 6.15) to [out = -30,in = 30] (4.5, 5.25);
%			\end{scope}
%			
%			
%			
%							
%		\end{tikzpicture}
%		\end{center}	
%	\caption{with contour lines and translated contour lines}
%	%\label{fig:label}
%\end{figure}
%
%
%
%\begin{figure}[htbp]
%	\begin{center}
%		\begin{tikzpicture}[
%			yscale=0.5,
%			% This decoration will be used to make portions of curves into "bouckground curves". It allows us to dash a portion of the curve.
%			decoration={border, 
%				segment length = 4pt, 	% determines the distance between consecutive ticks.
%				amplitude = 2pt, 		% determines the length of the ticks.
%				angle = 0  				% determines the angle between the ticks and the line of the path. 
%				}, 
%			% Styles
%			contour line/.style={thin, blue}
%				]
%				
%			% Colors
%			\colorlet{surfacecolor1}{black!10}
%			\colorlet{surfacecolor2}{black!5}	
%			
%			% The graphic
%			\draw [step = 1cm, help lines] (-0.9,-0.9) grid (11.9, 10.9);
%			
%			% First we fill the surfaces with the background colors.
%			% region 1
%			\fill [color = surfacecolor1] (0, 2.5) .. controls (0, 4) and (2,6.5) .. (1.75,8)
%				arc (180:0:2cm and 3cm)
%				%.. controls (3, 9) and (3,10) .. (4,10)
%				%.. controls (4.5,10) and (6,9.5) .. (6,8)
%				to [out = 270, in = 90] (6, 5.5)
%				-- (6,0.5)
%				arc (360: 180: 1cm and 0.5cm)
%				-- (4, 6)
%				arc (90:180:2cm and 3.5cm)
%				%.. controls (1, 2) and (1.5, 2) .. (1, 2)
%				to [out = 210, in = 0] (1,2)
%				arc (270: 180: 1cm and 0.5cm);
%			
%			% region 2
%			\fill [color = surfacecolor1] (7, 6.5) parabola bend (8.5, 5.8) (8.25, 5.8) 
%				to [out = 45, in = 260] (9, 7)
%				parabola bend (7.75, 6.4) (7, 6.5);
%			
%			% region 3
%			\fill [color = surfacecolor2] (4, 6) arc (90:180:2cm and 3.5cm)
%				-- (4, 1.5) -- (4,6);
%			
%			% region 4
%			\fill [color = surfacecolor2] (6, 5.5) parabola (7, 6.5) parabola bend (8.5, 5.8) (11, 6)
%				to [out = 240, in = 30] (8, 2.5) -- (6, 1.5) -- (6, 5.5);
%			
%			
%			% Now we draw the lines of the surface
%			%first around region 1
%			\draw (0, 2.5) .. controls (0, 4) and (2,6.5) .. (1.75,8)
%				arc (180:0:2cm and 3cm)
%				%.. controls (3, 9) and (3,10) .. (4,10)
%				%.. controls (4.5,10) and (6,9.5) .. (6,8)
%				to [out = 270, in = 90] (6, 5.5)
%				-- (6,0.5)
%				arc (360: 180: 1cm and 0.5cm)
%				-- (4, 6)
%				arc (90:180:2cm and 3.5cm)
%				%.. controls (1, 2) and (1.5, 2) .. (1, 2)
%				to [out = 210, in = 0] (1,2)
%				arc (270: 180: 1cm and 0.5cm);
%			
%			% then the rest 
%			\draw (2,2.5) -- (4,1.5) 
%				decorate {-- (5,1) to [out = -30, in = 135] (6,0.5)
%					  (4,0.5) to [out = 45, in = 210] (5,1) -- (6,1.5)} 
%				-- (8, 2.5) to [out = 30, in = 240] (11, 6);
%				% then off to cusp points
%			\draw decorate { (5, 1) arc (0:90:1cm and 5cm)};
%			\draw decorate {(2,2.5) -- (7,5) to [out = 30, in = 225] (8.25, 5.8)} to [out = 45, in = 260] (9, 7) ;
%			\draw decorate {(2,2.5) to [out = 150, in = 0] (1,3) arc (90:180:1cm and 0.5cm)};
%			% cusp
%			\draw (4, 6) to [out = 90, in = 270] (4.5, 8.5);
%			\draw (6, 5.5) decorate { parabola (4.5, 8.5)} (6, 5.5) parabola (7, 6.5)
%				parabola bend (7.75, 6.4) (9, 7);
%			\draw (7, 6.5) parabola bend (8.5, 5.8) (11, 6);
%			
%%		% now the contour lines
%%		% the first contour line
%%		\draw [contour line] (2.25, 10) arc (180:360:1.5cm and 0.5cm)
%%			decorate {(2.25, 10) arc (180:0:1.5cm and 0.5cm)};
%%		
%%		
%%		% the second contour line
%		\draw [contour line] (1.75,8) arc (180:320:1.45cm and 0.5cm)
%			decorate {to [out = 30,in = 135](4.65, 7.6) to [out = -45, in = 45] (4.2, 7.1) }
%			arc (180:270:0.5cm and 0.2cm) to [out = 0, in = -90] (5.75,8)
%		;
%		\draw [contour line] decorate {(1.75,8) arc (180:0:2cm and 0.6cm)};
%%		% the third contour line
%%		\draw [contour line] 
%%			decorate {(1.25, 5.75) arc (180:90: 1cm and 0.5cm) to [out = 0, in = 150] (3.5,5.75)}
%%			 -- (4, 5.5) 
%%			decorate {-- (4.5,5.25) to [out = -30, in = 135] (6,5)}
%%			arc (360: 180: 1cm and 0.5cm)
%%			decorate { to [out = 45, in = 210] (4.5,5.25)};
%%		
%%		\draw [contour line] 
%%		 (1.25, 5.75) arc (180: 270: 1cm and 0.5cm) to [out = 0, in = 210] (3.25,5.75)
%%		 	 to [out = 30, in = 150] (5.25, 6.15) to [out = -30,in = 30] (4.5, 5.25);
%			
%			%\draw (0,3) decorate {to (2,3)} -- (3,4);
%				
%		\end{tikzpicture}
%		\end{center}	
%	\caption{Squashed version, with only one contour line. 
%	 Maybe for the intro?}
%	%\label{fig:label}
%\end{figure}
%
%
%\begin{figure}[htbp]
%	\begin{center}
%		\begin{tikzpicture}[
%			yscale=0.5,
%			% This decoration will be used to make portions of curves into "bouckground curves". It allows us to dash a portion of the curve.
%			decoration={border, 
%				segment length = 4pt, 	% determines the distance between consecutive ticks.
%				amplitude = 2pt, 		% determines the length of the ticks.
%				angle = 0  				% determines the angle between the ticks and the line of the path. 
%				}, 
%			% Styles
%			contour line/.style={thin, blue}
%				]
%				
%			% Colors
%			\colorlet{surfacecolor1}{black!15}
%			\colorlet{surfacecolor2}{black!10}	
%			
%			% The graphic
%			\draw [step = 1cm, help lines] (-0.9,-0.9) grid (11.9, 10.9);
%			
%			% First we fill the surfaces with the background colors.
%			% region 1
%			\fill [color = surfacecolor1] (0, 2.5) .. controls (0, 4) and (2,6.5) .. (1.75,8)
%				arc (180:0:2cm and 3cm)
%				%.. controls (3, 9) and (3,10) .. (4,10)
%				%.. controls (4.5,10) and (6,9.5) .. (6,8)
%				to [out = 270, in = 90] (6, 5.5)
%				-- (6,0.5)
%				arc (360: 180: 1cm and 0.5cm)
%				-- (4, 6)
%				arc (90:180:2cm and 3.5cm)
%				%.. controls (1, 2) and (1.5, 2) .. (1, 2)
%				to [out = 210, in = 0] (1,2)
%				arc (270: 180: 1cm and 0.5cm);
%			
%			% region 2
%			\fill [color = surfacecolor1] (7, 6.5) parabola bend (8.5, 5.8) (8.25, 5.8) 
%				to [out = 45, in = 260] (9, 7)
%				parabola bend (7.75, 6.4) (7, 6.5);
%			
%			% region 3
%			\fill [color = surfacecolor2] (4, 6) arc (90:180:2cm and 3.5cm)
%				-- (4, 1.5) -- (4,6);
%			
%			% region 4
%			\fill [color = surfacecolor2] (6, 5.5) parabola (7, 6.5) parabola bend (8.5, 5.8) (11, 6)
%				to [out = 240, in = 30] (8, 2.5) -- (6, 1.5) -- (6, 5.5);
%			
%			
%			% Now we draw the lines of the surface
%			%first around region 1
%			\draw (0, 2.5) .. controls (0, 4) and (2,6.5) .. (1.75,8)
%				arc (180:0:2cm and 3cm)
%				%.. controls (3, 9) and (3,10) .. (4,10)
%				%.. controls (4.5,10) and (6,9.5) .. (6,8)
%				to [out = 270, in = 90] (6, 5.5)
%				-- (6,0.5)
%				arc (360: 180: 1cm and 0.5cm)
%				-- (4, 6)
%				arc (90:180:2cm and 3.5cm)
%				%.. controls (1, 2) and (1.5, 2) .. (1, 2)
%				to [out = 210, in = 0] (1,2)
%				arc (270: 180: 1cm and 0.5cm);
%			
%			% then the rest 
%			\draw (2,2.5) -- (4,1.5) 
%				decorate {-- (5,1) to [out = -30, in = 135] (6,0.5)
%					  (4,0.5) to [out = 45, in = 210] (5,1) -- (6,1.5)} 
%				-- (8, 2.5) to [out = 30, in = 240] (11, 6);
%				% then off to cusp points
%			\draw decorate { (5, 1) arc (0:90:1cm and 5cm)};
%			\draw decorate {(2,2.5) -- (7,5) to [out = 30, in = 225] (8.25, 5.8)} to [out = 45, in = 260] (9, 7) ;
%			\draw decorate {(2,2.5) to [out = 150, in = 0] (1,3) arc (90:180:1cm and 0.5cm)};
%			% cusp
%			\draw (4, 6) to [out = 90, in = 270] (4.5, 8.5);
%			\draw (6, 5.5) decorate { parabola (4.5, 8.5)} (6, 5.5) parabola (7, 6.5)
%				parabola bend (7.75, 6.4) (9, 7);
%			\draw (7, 6.5) parabola bend (8.5, 5.8) (11, 6);
%			
%%		% now the contour lines
%%		% the first contour line
%%		\draw [contour line] (2.25, 10) arc (180:360:1.5cm and 0.5cm)
%%			decorate {(2.25, 10) arc (180:0:1.5cm and 0.5cm)};
%%		
%%		
%%		% the second contour line
%		\draw [contour line] (1.75,8) arc (180:320:1.45cm and 0.5cm)
%			decorate {to [out = 30,in = 135](4.65, 7.6) to [out = -45, in = 45] (4.2, 7.1) }
%			arc (180:270:0.5cm and 0.2cm) to [out = 0, in = -90] (5.75,8)
%		;
%		\draw [contour line] decorate {(1.75,8) arc (180:0:2cm and 0.6cm)};
%%		% the third contour line
%%		\draw [contour line] 
%%			decorate {(1.25, 5.75) arc (180:90: 1cm and 0.5cm) to [out = 0, in = 150] (3.5,5.75)}
%%			 -- (4, 5.5) 
%%			decorate {-- (4.5,5.25) to [out = -30, in = 135] (6,5)}
%%			arc (360: 180: 1cm and 0.5cm)
%%			decorate { to [out = 45, in = 210] (4.5,5.25)};
%%		
%%		\draw [contour line] 
%%		 (1.25, 5.75) arc (180: 270: 1cm and 0.5cm) to [out = 0, in = 210] (3.25,5.75)
%%		 	 to [out = 30, in = 150] (5.25, 6.15) to [out = -30,in = 30] (4.5, 5.25);
%			
%			%\draw (0,3) decorate {to (2,3)} -- (3,4);
%				
%		\end{tikzpicture}
%		\end{center}	
%	\caption{Squashed version, with only one contour line, grayscale.}
%	%\label{fig:label}
%\end{figure}
%
%%
%%\begin{figure}[htbp]
%%	\begin{center}
%%		\begin{tikzpicture}[
%%			% This decoration will be used to make portions of curves into "bouckground curves". It allows us to dash a portion of the curve.
%%			decoration={border, 
%%				segment length = 4pt, 	% determines the distance between consecutive ticks.
%%				amplitude = 2pt, 		% determines the length of the ticks.
%%				angle = 0  				% determines the angle between the ticks and the line of the path. 
%%				}, 
%%			% Styles
%%			contour line/.style={thin, blue}
%%				]
%%				
%%			% Colors
%%			\colorlet{surfacecolor1}{blue!15}
%%			\colorlet{surfacecolor2}{blue!10}	
%%			
%%			% The graphic
%%			\draw [step = 1cm, help lines] (-3.9,-0.9) grid (11.9, 10.9);
%%			
%%			% First we fill the surfaces with the background colors.
%%			% region 1
%%			\fill [color = surfacecolor1] (0, 2.5) .. controls (0, 4) and (2,6.5) .. (1.75,8)
%%				arc (180:0:2cm and 3cm)
%%				%.. controls (3, 9) and (3,10) .. (4,10)
%%				%.. controls (4.5,10) and (6,9.5) .. (6,8)
%%				to [out = 270, in = 90] (6, 5.5)
%%				-- (6,0.5)
%%				arc (360: 180: 1cm and 0.5cm)
%%				-- (4, 6)
%%				arc (90:180:2cm and 3.5cm)
%%				%.. controls (1, 2) and (1.5, 2) .. (1, 2)
%%				to [out = 210, in = 0] (1,2)
%%				arc (270: 180: 1cm and 0.5cm);
%%			
%%			% region 2
%%			\fill [color = surfacecolor1] (7, 6.5) parabola bend (8.5, 5.8) (8.25, 5.8) 
%%				to [out = 45, in = 260] (9, 7)
%%				parabola bend (7.75, 6.4) (7, 6.5);
%%			
%%			% region 3
%%			\fill [color = surfacecolor2] (4, 6) arc (90:180:2cm and 3.5cm)
%%				-- (4, 1.5) -- (4,6);
%%			
%%			% region 4
%%			\fill [color = surfacecolor2] (6, 5.5) parabola (7, 6.5) parabola bend (8.5, 5.8) (11, 6)
%%				to [out = 240, in = 30] (8, 2.5) -- (6, 1.5) -- (6, 5.5);
%%			
%%			
%%			% Now we draw the lines of the surface
%%			%first around region 1
%%			\draw (0, 2.5) .. controls (0, 4) and (2,6.5) .. (1.75,8)
%%				arc (180:0:2cm and 3cm)
%%				%.. controls (3, 9) and (3,10) .. (4,10)
%%				%.. controls (4.5,10) and (6,9.5) .. (6,8)
%%				to [out = 270, in = 90] (6, 5.5)
%%				-- (6,0.5)
%%				arc (360: 180: 1cm and 0.5cm)
%%				-- (4, 6)
%%				arc (90:180:2cm and 3.5cm)
%%				%.. controls (1, 2) and (1.5, 2) .. (1, 2)
%%				to [out = 210, in = 0] (1,2)
%%				arc (270: 180: 1cm and 0.5cm);
%%			
%%			% then the rest 
%%			\draw (2,2.5) -- (4,1.5) 
%%				decorate {-- (5,1) to [out = -30, in = 135] (6,0.5)
%%					  (4,0.5) to [out = 45, in = 210] (5,1) -- (6,1.5)} 
%%				-- (8, 2.5) to [out = 30, in = 240] (11, 6);
%%				% then off to cusp points
%%			\draw decorate { (5, 1) arc (0:90:1cm and 5cm)};
%%			\draw decorate {(2,2.5) -- (7,5) to [out = 30, in = 225] (8.25, 5.8)} to [out = 45, in = 260] (9, 7) ;
%%			\draw decorate {(2,2.5) to [out = 150, in = 0] (1,3) arc (90:180:1cm and 0.5cm)};
%%			% cusp
%%			\draw (4, 6) to [out = 90, in = 270] (4.5, 8.5);
%%			\draw (6, 5.5) decorate { parabola (4.5, 8.5)} (6, 5.5) parabola (7, 6.5)
%%				parabola bend (7.75, 6.4) (9, 7);
%%			\draw (7, 6.5) parabola bend (8.5, 5.8) (11, 6);
%%			
%%%			% now the contour lines
%%%			% the first contour line
%%%			\draw [contour line] (2.3, 10) arc (180:360:1.45cm and 0.5cm)
%%%				decorate {(2.3, 10) arc (180:0:1.45cm and 0.5cm)};
%%%			
%%%			
%%%			% the second contour line
%%%			\draw [contour line] (1.75,8) arc (180:320:1.45cm and 0.5cm)
%%%				decorate {to [out = 30,in = 135](4.65, 7.6) to [out = -45, in = 45] (4.2, 7.1) }
%%%				arc (180:270:0.5cm and 0.2cm) to [out = 0, in = -90] (5.75,8)
%%%			;
%%%			\draw [contour line] decorate {(1.75,8) arc (180:0:2cm and 0.6cm)};
%%%			
%%%			% the third contour line
%%%			\draw [contour line] 
%%%				decorate {(1.25, 5.75) arc (180:90: 1cm and 0.5cm) to [out = 0, in = 150] (3.5,5.75)}
%%%				 -- (4, 5.5) 
%%%				decorate {-- (4.5,5.25) to [out = -30, in = 135] (6,5)}
%%%				arc (360: 180: 1cm and 0.5cm)
%%%				decorate { to [out = 45, in = 210] (4.5,5.25)};
%%%			
%%%			\draw [contour line] 
%%%			 (1.25, 5.75) arc (180: 270: 1cm and 0.5cm) to [out = 0, in = 210] (3.25,5.75)
%%%			 	 to [out = 30, in = 150] (5.25, 6.15) to [out = -30,in = 30] (4.5, 5.25);
%%			
%%			
%%			\begin{scope}[xshift = -6cm]
%%				% now the contour lines
%%				% the first contour line
%%				\draw [contour line] (2.3, 10) arc (180:360:1.45cm and 0.5cm)
%%					 {(2.3, 10) arc (180:0:1.45cm and 0.5cm)};
%%			
%%			
%%				% the second contour line
%%				\draw [contour line] (1.75,8) arc (180:320:1.45cm and 0.5cm)
%%					 {to [out = 30,in = 135](4.65, 7.6) to [out = -45, in = 45] (4.2, 7.1) }
%%					arc (180:270:0.5cm and 0.2cm) to [out = 0, in = -90] (5.75,8)
%%				;
%%				\draw [contour line]  {(1.75,8) arc (180:0:2cm and 0.6cm)};
%%			
%%				% the third contour line
%%				\draw [contour line] 
%%					 {(1.25, 5.75) arc (180:90: 1cm and 0.5cm) to [out = 0, in = 150] (3.5,5.75)}
%%					 -- (4, 5.5) 
%%					 {-- (4.5,5.25) to [out = -30, in = 135] (6,5)}
%%					arc (360: 180: 1cm and 0.5cm)
%%					 { to [out = 45, in = 210] (4.5,5.25)};
%%			
%%				\draw [contour line] 
%%				 (1.25, 5.75) arc (180: 270: 1cm and 0.5cm) to [out = 0, in = 210] (3.25,5.75)
%%				 	 to [out = 30, in = 150] (5.25, 6.15) to [out = -30,in = 30] (4.5, 5.25);
%%			\end{scope}
%%					
%%							
%%		\end{tikzpicture}
%%		\end{center}	
%%	\caption{with translated contour lines only}
%%	%\label{fig:label}
%%\end{figure}
%
%
%\begin{figure}[htbp]
%	\begin{center}
%		\begin{tikzpicture}[
%			% This decoration will be used to make portions of curves into "bouckground curves". It allows us to dash a portion of the curve.
%			decoration={border, 
%				segment length = 4pt, 	% determines the distance between consecutive ticks.
%				amplitude = 2pt, 		% determines the length of the ticks.
%				angle = 0  				% determines the angle between the ticks and the line of the path. 
%				}, 
%			% Styles
%			contour line/.style={thin, blue}
%				]
%				
%			% Colors
%			\colorlet{surfacecolor1}{blue!15}
%			\colorlet{surfacecolor2}{blue!10}	
%			
%			% The graphic
%			%\draw [step = 1cm, help lines] (-3.9,-0.9) grid (11.9, 10.9);
%			
%				% adding fuzz
%				\draw [fuzzright] (0,2.5) arc (180:270: 1cm and 0.5cm) to [out = 0, in = 210] (2,2.5)
%					-- (4, 1.5);
%				\draw [fuzzright] (4, 0.5) arc (180:360: 1cm and 0.5cm);
%				\draw [fuzzright] (6, 1.5) -- (8, 2.5) to [out = 30, in = 240] (11, 6);
%				\draw [fuzzright]	 (8.25, 5.8) to [out = 45, in = 260] (9, 7);
%			
%			% First we fill the surfaces with the background colors.
%			% region 1
%			\fill [color = surfacecolor1] (0, 2.5) .. controls (0, 4) and (2,6.5) .. (1.75,8)
%				arc (180:0:2cm and 3cm)
%				%.. controls (3, 9) and (3,10) .. (4,10)
%				%.. controls (4.5,10) and (6,9.5) .. (6,8)
%				to [out = 270, in = 90] (6, 5.5)
%				-- (6,0.5)
%				arc (360: 180: 1cm and 0.5cm)
%				-- (4, 6)
%				arc (90:180:2cm and 3.5cm)
%				%.. controls (1, 2) and (1.5, 2) .. (1, 2)
%				to [out = 210, in = 0] (1,2)
%				arc (270: 180: 1cm and 0.5cm);
%			
%			% region 2
%			\fill [color = surfacecolor1] (7, 6.5) parabola bend (8.5, 5.8) (8.25, 5.8) 
%				to [out = 45, in = 260] (9, 7)
%				parabola bend (7.75, 6.4) (7, 6.5);
%			
%			% region 3
%			\fill [color = surfacecolor2] (4, 6) arc (90:180:2cm and 3.5cm)
%				-- (4, 1.5) -- (4,6);
%			
%			% region 4
%			\fill [color = surfacecolor2] (6, 5.5) parabola (7, 6.5) parabola bend (8.5, 5.8) (11, 6)
%				to [out = 240, in = 30] (8, 2.5) -- (6, 1.5) -- (6, 5.5);
%			
%			
%			% Now we draw the lines of the surface
%			%first around region 1
%			\draw (0, 2.5) .. controls (0, 4) and (2,6.5) .. (1.75,8)
%				arc (180:0:2cm and 3cm)
%				%.. controls (3, 9) and (3,10) .. (4,10)
%				%.. controls (4.5,10) and (6,9.5) .. (6,8)
%				to [out = 270, in = 90] (6, 5.5)
%				-- (6,0.5)
%				arc (360: 180: 1cm and 0.5cm)
%				-- (4, 6)
%				arc (90:180:2cm and 3.5cm)
%				%.. controls (1, 2) and (1.5, 2) .. (1, 2)
%				to [out = 210, in = 0] (1,2)
%				arc (270: 180: 1cm and 0.5cm);
%			
%			% then the rest 
%			\draw (2,2.5) -- (4,1.5) 
%				decorate {-- (5,1) to [out = -30, in = 135] (6,0.5)
%					  (4,0.5) to [out = 45, in = 210] (5,1) -- (6,1.5)} 
%				-- (8, 2.5) to [out = 30, in = 240] (11, 6);
%				% then off to cusp points
%			\draw decorate { (5, 1) arc (0:90:1cm and 5cm)};
%			\draw decorate {(2,2.5) -- (7,5) to [out = 30, in = 225] (8.25, 5.8)} to [out = 45, in = 260] (9, 7) ;
%			\draw decorate {(2,2.5) to [out = 150, in = 0] (1,3) arc (90:180:1cm and 0.5cm)};
%			% cusp
%			\draw (4, 6) to [out = 90, in = 270] (4.5, 8.5);
%			\draw (6, 5.5) decorate { parabola (4.5, 8.5)} (6, 5.5) parabola (7, 6.5)
%				parabola bend (7.75, 6.4) (9, 7);
%			\draw (7, 6.5) parabola bend (8.5, 5.8) (11, 6);
%			
%			% now the contour lines
%			% the first contour line
%			\draw [contour line] (2.3, 10) arc (180:360:1.45cm and 0.5cm)
%				decorate {(2.3, 10) arc (180:0:1.45cm and 0.5cm)};
%			
%			
%			% the second contour line
%			\draw [contour line] (1.75,8) arc (180:320:1.45cm and 0.5cm)
%				decorate {to [out = 30,in = 135](4.65, 7.6) to [out = -45, in = 45] (4.2, 7.1) }
%				arc (180:270:0.5cm and 0.2cm) to [out = 0, in = -90] (5.75,8)
%			;
%			\draw [contour line] decorate {(1.75,8) arc (180:0:2cm and 0.6cm)};
%			
%			% the third contour line
%			\draw [contour line] 
%				decorate {(1.25, 5.75) arc (180:90: 1cm and 0.5cm) to [out = 0, in = 150] (3.5,5.75)}
%				 -- (4, 5.5) 
%				decorate {-- (4.5,5.25) to [out = -30, in = 135] (6,5)}
%				arc (360: 180: 1cm and 0.5cm)
%				decorate { to [out = 45, in = 210] (4.5,5.25)};
%			
%			\draw [contour line] 
%			 (1.25, 5.75) arc (180: 270: 1cm and 0.5cm) to [out = 0, in = 210] (3.25,5.75)
%			 	 to [out = 30, in = 150] (5.25, 6.15) to [out = -30,in = 30] (4.5, 5.25);
%			
%			
%			\begin{scope}[xshift = -4cm, yshift =2.5cm, scale = 0.75]
%				% now the contour lines
%				% the first contour line
%				\draw [contour line] (2.3, 10) arc (180:360:1.45cm and 0.5cm)
%					 {(2.3, 10) arc (180:0:1.45cm and 0.5cm)};
%			
%			
%				% the second contour line
%				\draw [contour line] (1.75,8) arc (180:320:1.45cm and 0.5cm)
%					 {to [out = 30,in = 135](4.65, 7.6) to [out = -45, in = 45] (4.2, 7.1) }
%					arc (180:270:0.5cm and 0.2cm) to [out = 0, in = -90] (5.75,8)
%				;
%				\draw [contour line]  {(1.75,8) arc (180:0:2cm and 0.6cm)};
%			
%				% the third contour line
%				\draw [contour line] 
%					 {(1.25, 5.75) arc (180:90: 1cm and 0.5cm) to [out = 0, in = 150] (3.5,5.75)}
%					 -- (4, 5.5) 
%					 {-- (4.5,5.25) to [out = -30, in = 135] (6,5)}
%					arc (360: 180: 1cm and 0.5cm)
%					 { to [out = 45, in = 210] (4.5,5.25)};
%			
%				\draw [contour line] 
%				 (1.25, 5.75) arc (180: 270: 1cm and 0.5cm) to [out = 0, in = 210] (3.25,5.75)
%				 	 to [out = 30, in = 150] (5.25, 6.15) to [out = -30,in = 30] (4.5, 5.25);
%			\end{scope}
%			
%			% now for some braces
%			\draw [thick, decoration={brace, amplitude=3pt}]  decorate {(6.5, 10.5) -- (6.5,6.5)};
%			\node at (7.5,8.5) {\small some math};
%			\draw [thick, decoration={brace, amplitude=3pt}]  decorate {(11.5, 6.5) -- (11.5,1.5)};	
%			\node at (12, 4) {\small more};			
%			
%		
%				
%						
%							
%		\end{tikzpicture}
%		\end{center}	
%	\caption{with contour lines, smaller translated contour lines, braces (for math notation), and fuzz. (i.e all the trimmings). Also no grid lines. }
%	%\label{fig:label}
%\end{figure}
%
%
%\begin{figure}[htbp]
%	\begin{center}
%		\begin{tikzpicture}[
%			yscale = 0.5,
%			% This decoration will be used to make portions of curves into "bouckground curves". It allows us to dash a portion of the curve.
%			decoration={border, 
%				segment length = 4pt, 	% determines the distance between consecutive ticks.
%				amplitude = 2pt, 		% determines the length of the ticks.
%				angle = 0  				% determines the angle between the ticks and the line of the path. 
%				}, 
%			% Styles
%			contour line/.style={thin, blue}
%				]
%				
%			% Colors
%			\colorlet{surfacecolor1}{blue!15}
%			\colorlet{surfacecolor2}{blue!10}	
%			
%			% The graphic
%			\draw [step = 1cm, help lines] (-0.9,-0.9) grid (11.9, 10.9);
%			
%			% First we fill the surfaces with the background colors.
%			% region 1
%			\fill [color = surfacecolor1] (0, 2.5) .. controls (0, 4) and (2,6.5) .. (1.75,8)
%				arc (180:0:2cm and 3cm)
%				%.. controls (3, 9) and (3,10) .. (4,10)
%				%.. controls (4.5,10) and (6,9.5) .. (6,8)
%				to [out = 270, in = 90] (6, 5.5)
%				-- (6,0.5)
%				arc (360: 180: 1cm and 0.5cm)
%				-- (4, 6)
%				arc (90:180:2cm and 3.5cm)
%				%.. controls (1, 2) and (1.5, 2) .. (1, 2)
%				to [out = 210, in = 0] (1,2)
%				arc (270: 180: 1cm and 0.5cm);
%			
%			% region 2
%			\fill [color = surfacecolor1] (7, 6.5) parabola bend (8.5, 5.8) (8.25, 5.8) 
%				to [out = 45, in = 260] (9, 7)
%				parabola bend (7.75, 6.4) (7, 6.5);
%			
%			% region 3
%			\fill [color = surfacecolor2] (4, 6) arc (90:180:2cm and 3.5cm)
%				-- (4, 1.5) -- (4,6);
%			
%			% region 4
%			\fill [color = surfacecolor2] (6, 5.5) parabola (7, 6.5) parabola bend (8.5, 5.8) (11, 6)
%				to [out = 240, in = 30] (8, 2.5) -- (6, 1.5) -- (6, 5.5);
%			
%			
%			% Now we draw the lines of the surface
%			%first around region 1
%			\draw (0, 2.5) .. controls (0, 4) and (2,6.5) .. (1.75,8)
%				arc (180:0:2cm and 3cm)
%				%.. controls (3, 9) and (3,10) .. (4,10)
%				%.. controls (4.5,10) and (6,9.5) .. (6,8)
%				to [out = 270, in = 90] (6, 5.5)
%				-- (6,0.5)
%				arc (360: 180: 1cm and 0.5cm)
%				-- (4, 6)
%				arc (90:180:2cm and 3.5cm)
%				%.. controls (1, 2) and (1.5, 2) .. (1, 2)
%				to [out = 210, in = 0] (1,2)
%				arc (270: 180: 1cm and 0.5cm);
%			
%			% then the rest 
%			\draw (2,2.5) -- (4,1.5) 
%				decorate {-- (5,1) to [out = -30, in = 135] (6,0.5)
%					  (4,0.5) to [out = 45, in = 210] (5,1) -- (6,1.5)} 
%				-- (8, 2.5) to [out = 30, in = 240] (11, 6);
%				% then off to cusp points
%			\draw decorate { (5, 1) arc (0:90:1cm and 5cm)};
%			\draw decorate {(2,2.5) -- (7,5) to [out = 30, in = 225] (8.25, 5.8)} to [out = 45, in = 260] (9, 7) ;
%			\draw decorate {(2,2.5) to [out = 150, in = 0] (1,3) arc (90:180:1cm and 0.5cm)};
%			% cusp
%			\draw (4, 6) to [out = 90, in = 270] (4.5, 8.5);
%			\draw (6, 5.5) decorate { parabola (4.5, 8.5)} (6, 5.5) parabola (7, 6.5)
%				parabola bend (7.75, 6.4) (9, 7);
%			\draw (7, 6.5) parabola bend (8.5, 5.8) (11, 6);
%			
%			% now the contour lines
%			% the first contour line
%			\draw [contour line] (2.3, 10) arc (180:360:1.45cm and 0.5cm)
%				decorate {(2.3, 10) arc (180:0:1.45cm and 0.5cm)};
%			
%			
%			% the second contour line
%			\draw [contour line] (1.75,8) arc (180:320:1.45cm and 0.5cm)
%				decorate {to [out = 30,in = 135](4.65, 7.6) to [out = -45, in = 45] (4.2, 7.1) }
%				arc (180:270:0.5cm and 0.2cm) to [out = 0, in = -90] (5.75,8)
%			;
%			\draw [contour line] decorate {(1.75,8) arc (180:0:2cm and 0.6cm)};
%			
%			% the third contour line
%			\draw [contour line] 
%				decorate {(1.25, 5.75) arc (180:90: 1cm and 0.5cm) to [out = 0, in = 150] (3.5,5.75)}
%				 -- (4, 5.5) 
%				decorate {-- (4.5,5.25) to [out = -30, in = 135] (6,5)}
%				arc (360: 180: 1cm and 0.5cm)
%				decorate { to [out = 45, in = 210] (4.5,5.25)};
%			
%			\draw [contour line] 
%			 (1.25, 5.75) arc (180: 270: 1cm and 0.5cm) to [out = 0, in = 210] (3.25,5.75)
%			 	 to [out = 30, in = 150] (5.25, 6.15) to [out = -30,in = 30] (4.5, 5.25);
%			
%			%\draw (0,3) decorate {to (2,3)} -- (3,4);
%				
%		\end{tikzpicture}
%		\end{center}	
%	\caption{squashed, with all contour lines}
%	%\label{fig:label}
%\end{figure}
%
%
%
%
%
%
%
%The 1-morphism $ev^L$ of $\FrBord_2$ itself has a left adjoint $ev^{LL}$, and similarly $ev^R$ has a right adjoint, and so on.  The 1-morphism $coev$ also has a chain of left and right adjoints.  We therefore have two infinite series of adjunctions, as follows:
%
%
%
%
%\begin{figure}[htbp]
%	\begin{center}
%		\begin{tikzpicture}[
%			% This decoration will be used to make portions of curves into "bouckground curves". It allows us to dash a portion of the curve.
%			%xscale = 0.75,
%			decoration={border, 
%				segment length = 4pt, 	% determines the distance between consecutive ticks.
%				amplitude = 2pt, 		% determines the length of the ticks.
%				angle = 0  				% determines the angle between the ticks and the line of the path. 
%				}, 
%			% Styles
%			contour line/.style={thin, blue}
%				]
%				
%			% Colors
%			\colorlet{surfacecolor1}{black!12}
%			\colorlet{surfacecolor2}{black!5}	
%			
%			% The graphic
%			%\draw [step = 1cm, help lines] (-0.9,-0.9) grid (11.9, 10.9);
%			
%			\begin{scope}[scale=0.75]			
%				% adding fuzz
%				\draw [fuzzright] (0,2.5) arc (180:270: 1cm and 0.5cm) to [out = 0, in = 210] (2,2.5)
%					-- (4.5, 1.25);
%				\draw [fuzzleft] (0,2.5) arc (180:90: 1cm and 0.5cm);
%				\draw [fuzzright] (4, 0.5) arc (180:360: 1cm and 0.5cm);
%				\draw [fuzzleft] (4, 0.5) arc (180:90: 1cm and 0.5cm);
%				
%				\draw [fuzzright] (6, 1.5) -- (8, 2.5) to [out = 30, in = 240] (11, 6);
%				\draw [fuzzright]	 (8.25, 5.8) to [out = 45, in = 260] (9, 7);
%				% and small arrows
%				\begin{pgfonlayer}{background}
%					\draw[->,outstyle] (11,6) -- +(30:\arrowlength);
%					\draw[->,outstyle] (9,7) -- +(45:\arrowlength);
%				\end{pgfonlayer}
%				
%			% First we fill the surfaces with the background colors.
%			% region 1
%			\fill [color = surfacecolor1] (0, 2.5) .. controls (0, 4) and (2,6.5) .. (1.75,8)
%				arc (180:0:2cm and 3cm)
%				%.. controls (3, 9) and (3,10) .. (4,10)
%				%.. controls (4.5,10) and (6,9.5) .. (6,8)
%				to [out = 270, in = 90] (6, 5.5)
%				-- (6,0.5)
%				arc (360: 180: 1cm and 0.5cm)
%				-- (4, 6)
%				arc (90:180:2cm and 3.5cm)
%				%.. controls (1, 2) and (1.5, 2) .. (1, 2)
%				to [out = 210, in = 0] (1,2)
%				arc (270: 180: 1cm and 0.5cm);
%			
%			% region 2
%			\fill [color = surfacecolor1] (7, 6.5) parabola bend (8.5, 5.8) (8.25, 5.8) 
%				to [out = 45, in = 260] (9, 7)
%				parabola bend (7.75, 6.4) (7, 6.5);
%			
%			% region 3
%			\fill [color = surfacecolor2] (4, 6) arc (90:180:2cm and 3.5cm)
%				-- (4, 1.5) -- (4,6);
%			
%			% region 4
%			\fill [color = surfacecolor2] (6, 5.5) parabola (7, 6.5) parabola bend (8.5, 5.8) (11, 6)
%				to [out = 240, in = 30] (8, 2.5) -- (6, 1.5) -- (6, 5.5);
%			
%			
%			% Now we draw the lines of the surface
%			%first around region 1
%			\draw (0, 2.5) .. controls (0, 4) and (2,6.5) .. (1.75,8)
%				arc (180:0:2cm and 3cm)
%				%.. controls (3, 9) and (3,10) .. (4,10)
%				%.. controls (4.5,10) and (6,9.5) .. (6,8)
%				to [out = 270, in = 90] (6, 5.5)
%				-- (6,0.5)
%				arc (360: 180: 1cm and 0.5cm)
%				-- (4, 6)
%				arc (90:180:2cm and 3.5cm)
%				%.. controls (1, 2) and (1.5, 2) .. (1, 2)
%				to [out = 210, in = 0] (1,2)
%				arc (270: 180: 1cm and 0.5cm);
%			
%			% then the rest 
%			\draw (2,2.5) -- (4,1.5) 
%				decorate {-- (5,1) to [out = -30, in = 135] (6,0.5)
%					  (4,0.5) to [out = 45, in = 210] (5,1) -- (6,1.5)} 
%				-- (8, 2.5) to [out = 30, in = 240] (11, 6);
%				% then off to cusp points
%			\draw decorate { (5, 1) arc (0:90:1cm and 5cm)};
%			\draw decorate {(2,2.5) -- (7,5) to [out = 30, in = 225] (8.25, 5.8)} to [out = 45, in = 260] (9, 7) ;
%			\draw decorate {(2,2.5) to [out = 150, in = 0] (1,3) arc (90:180:1cm and 0.5cm)};
%			% cusp
%			\draw (4, 6) to [out = 90, in = 270] (4.5, 8.5);
%			\draw (6, 5.5) decorate { parabola (4.5, 8.5)} (6, 5.5) parabola (7, 6.5)
%				parabola bend (7.75, 6.4) (9, 7);
%			\draw (7, 6.5) parabola bend (8.5, 5.8) (11, 6);
%			
%			% now the contour lines
%			% the first contour line
%			\draw [contour line] (2.3, 10) arc (180:360:1.45cm and 0.5cm)
%				decorate {(2.3, 10) arc (180:0:1.45cm and 0.5cm)};
%			
%			
%			% the second contour line
%			\draw [contour line] (1.75,8) arc (180:320:1.45cm and 0.5cm)
%				decorate {to [out = 30,in = 135](4.65, 7.6) to [out = -45, in = 45] (4.2, 7.1) }
%				arc (180:270:0.5cm and 0.2cm) to [out = 0, in = -90] (5.75,8)
%			;
%			\draw [contour line] decorate {(1.75,8) arc (180:0:2cm and 0.6cm)};
%			
%			% the third contour line
%			\draw [contour line] 
%				decorate {(1.25, 5.75) arc (180:90: 1cm and 0.5cm) to [out = 0, in = 150] (3.5,5.75)}
%				 -- (4, 5.5) 
%				decorate {-- (4.5,5.25) to [out = -30, in = 135] (6,5)}
%				arc (360: 180: 1cm and 0.5cm)
%				decorate { to [out = 45, in = 210] (4.5,5.25)};
%			
%			\draw [contour line] 
%			 (1.25, 5.75) arc (180: 270: 1cm and 0.5cm) to [out = 0, in = 210] (3.25,5.75)
%			 	 to [out = 30, in = 150] (5.25, 6.15) to [out = -30,in = 30] (4.5, 5.25);
%			
%			
%%			\begin{scope}[xshift = -4cm, yshift =2.5cm, scale = 0.75]
%%				% now the contour lines
%%				% the first contour line
%%				\draw [contour line] (2.3, 10) arc (180:360:1.45cm and 0.5cm)
%%					 {(2.3, 10) arc (180:0:1.45cm and 0.5cm)};
%%			
%%			
%%				% the second contour line
%%				\draw [contour line] (1.75,8) arc (180:320:1.45cm and 0.5cm)
%%					 {to [out = 30,in = 135](4.65, 7.6) to [out = -45, in = 45] (4.2, 7.1) }
%%					arc (180:270:0.5cm and 0.2cm) to [out = 0, in = -90] (5.75,8)
%%				;
%%				\draw [contour line]  {(1.75,8) arc (180:0:2cm and 0.6cm)};
%%			
%%				% the third contour line
%%				\draw [contour line] 
%%					 {(1.25, 5.75) arc (180:90: 1cm and 0.5cm) to [out = 0, in = 150] (3.5,5.75)}
%%					 -- (4, 5.5) 
%%					 {-- (4.5,5.25) to [out = -30, in = 135] (6,5)}
%%					arc (360: 180: 1cm and 0.5cm)
%%					 { to [out = 45, in = 210] (4.5,5.25)};
%%			
%%				\draw [contour line] 
%%				 (1.25, 5.75) arc (180: 270: 1cm and 0.5cm) to [out = 0, in = 210] (3.25,5.75)
%%				 	 to [out = 30, in = 150] (5.25, 6.15) to [out = -30,in = 30] (4.5, 5.25);
%%			\end{scope}
%		\end{scope}
%			
%			% now for some braces
%		%	\draw [thick, decoration={brace, amplitude=3pt}]  decorate {(6.5, 10.5) -- (6.5,6.5)};
%		%	\node at (7.5,8.5) {\small some math};
%		%	\draw [thick, decoration={brace, amplitude=3pt}]  decorate {(11.5, 6.5) -- (11.5,1.5)};	
%		%	\node at (12, 4) {\small more};			
%			
%		
%		% Now the 2D diagrams
%		\begin{scope}[xshift = 7cm, yshift = 2cm]
%			
%						% ev^L
%						\draw[linestyle,fuzzright] (4,7) to [looseness=1.6,out = 180, in = 180] (4,6);
%						\begin{pgfonlayer}{background}
%							\draw[->,outstyle] (4,7) -- +(0:\arrowlength);
%							\draw[->,outstyle] (4,6) -- +(0:\arrowlength);
%						\end{pgfonlayer}
%						
%						% the first morphism	
%						\draw [thick, ->] (3, 5.75) -- (3, 4.75);
%							\draw[linestyle, fill=\fillcolor, yshift = 0.5*\smcirclerad] (3.5,5.25) 
%								to [looseness=1.6, out=180, in=190] +(90:0.75*\smcirclerad)
%								to [looseness=1.6, out=10, in=-10] +(90:-2.5*\smcirclerad)
%								to [looseness=1.6, out=170, in=180] +(90:.75*\smcirclerad)
%								to [looseness=1.55, out=0, in=0] +(90:\smcirclerad);
%						\node at (4,5.5) {\small $R$};
%
%						% ev^R ev + ev^L
%						\draw[linestyle,fuzzright] (4,4.5) to [looseness=1.6,out = 180, in = 180] (4,3.5);
%							\begin{pgfonlayer}{background}
%								\draw[->,outstyle] (4,4.5) -- +(0:\arrowlength);
%								\draw[->,outstyle] (4,3.5) -- +(0:\arrowlength);
%							\end{pgfonlayer}
%						\draw[yshift = \smcirclerad, linestyle, fuzzright] (2,4) 
%							to [looseness=1.6, out=180, in=190] +(90:1.5*\smcirclerad)
%							to [looseness=1.6, out=10, in=-10] +(90:-5*\smcirclerad)
%							to [looseness=1.6, out=170, in=180] +(90:1.5*\smcirclerad)
%							to [looseness=1.55, out=0, in=0] +(90:2*\smcirclerad);
%
%						% the second morphism	
%						\draw [thick, ->] (3, 3.25) -- (3, 2.25);
%						\node at (3.5,2.75) {\small Isom};
%							
%						% ev^R ev + ev^L (deformed)
%						\draw[linestyle,fuzzright] (4,2) to [looseness=1.6,out = 180, in = 180] (4,1);
%								\begin{pgfonlayer}{background}
%									\draw[->,outstyle] (4,2) -- +(0:\arrowlength);
%									\draw[->,outstyle] (4,1) -- +(0:\arrowlength);
%								\end{pgfonlayer}
%						\draw[linestyle,fuzzright] (2.5, 2) -- (1.75,2)
%								to [looseness=2,out = 180, in = 90] (2, 2.25)
%									-- (2, 0.75)
%									to [looseness=2,out = 270, in = 180] (1.75,1)
%									-- (2.5, 1) arc (-90:90:0.5cm);
%						
%						% the third morphism			
%						\draw [thick, ->] (3, 0.75) -- (3, -0.25);
%						\begin{scope}[xshift = 3.5cm, yshift = 0.0cm, scale = 0.65]
%							\filldraw[linestyle,fill=\fillcolor] 
%							(0,0) .. controls (.25,.25) and (.75,.25) .. (1,0)
%								.. controls (.75,.25) and (.75,.75) .. (1,1)
%								.. controls (.75,.75) and (.25,.75) .. (0,1)
%								.. controls (.25,.75) and (.25,.25) .. (0,0);
%						\draw[linestyle,fuzzright]
%							(0,0) .. controls (.25,.25) and (.75,.25) .. (1,0);
%						\draw[linestyle,fuzzleft]
%							(0,1) .. controls (.25,.75) and (.75,.75) .. (1,1);
%						\begin{pgfonlayer}{background}
%							\draw[->,outstyle] (1,1) -- +(45:\arrowlength);
%							\draw[->,outstyle] (1,0) -- +(-45:\arrowlength);
%						\end{pgfonlayer}
%						\end{scope}
%						
%
%						% ev^R
%							\draw[linestyle,fuzzright] (4, -0.5) -- (1.75, -0.5)
%								to [looseness=2,out = 180, in = 90] (2, -0.25)
%								-- (2, -1.75)
%								to [looseness=2,out = 270, in = 180] (1.75,-1.5)
%								-- (4, -1.5);
%							\begin{pgfonlayer}{background}
%								\draw[->,outstyle] (4,-0.5) -- +(0:\arrowlength);
%								\draw[->,outstyle] (4,-1.5) -- +(0:\arrowlength);
%							\end{pgfonlayer}
%
%		\end{scope}	
%			
%		% names of 2D slices
%		\node (A) at (6.5, 8.5) {$\ev^L$};
%		\node (B) at (6.5, 6) {$\ev^L \circ \ev \circ \ev^R$};		
%		\node (C) at (6.5, 1) {$\ev^R$};
%		\draw [->, gray, thin] (A) -- +(3.5,0);
%		\draw [->, gray, thin] (B) -- +(2,0);
%		\draw [->, gray, thin] (C) -- +(2,0);
%		% names of arrows
%		\node at (9, 7.25) {$ id_{\ev^L} \circledcirc v_2^R$};
%		\node at (9, 2.25) {$ v_1 \circledcirc id_{\ev^R}$};
%		
%		\node at (6, -1) {$\cR = (v_1 \circledcirc id_{\ev^R})\circ (id_{\ev^L} \circledcirc v_2^R)$};				
%							
%		\end{tikzpicture}
%		\end{center}	
%	\caption{The Radford Bordism.  }
%	%\label{fig:label}
%\end{figure}
%
%
%
%\begin{figure}[htpb]
%	\begin{center} 
%		\begin{tikzpicture}[
%	%		fuzzright/.style={line width=\fuzzwidth,\fuzzcolor}
%	%        preaction={draw,line width=\fuzzwidth,\fuzzcolor,decoration={curveto, amplitude=0,raise=-.3*\fuzzwidth}}}%,
%	        %fuzzleft/.style={preaction={draw,line width=\fuzzwidth,\fuzzcolor,decorate,decoration={curveto,amplitude=0,raise=.5*\fuzzwidth}}}
%			]
%		
%	
%			% ev^L
%			\draw[linestyle,fuzzright] (4,7) to [looseness=1.6,out = 180, in = 180] (4,6);
%			\begin{pgfonlayer}{background}
%				\draw[->,outstyle] (4,7) -- +(0:\arrowlength);
%				\draw[->,outstyle] (4,6) -- +(0:\arrowlength);
%			\end{pgfonlayer}
%	
%%			\draw[linestyle, fill=\fillcolor] (2,4) 
%%				to [looseness=1.6, out=180, in=190] +(90:1.5*\smcirclerad)
%%				to [looseness=1.6, out=10, in=-10] +(90:-5*\smcirclerad)
%%				to [looseness=1.6, out=170, in=180] +(90:1.5*\smcirclerad)
%%				to [looseness=1.55, out=0, in=0] +(90:2*\smcirclerad);
%			
%			% ev^R ev + ev^L
%			\draw[linestyle,fuzzright] (4,4.5) to [looseness=1.6,out = 180, in = 180] (4,3.5);
%				\begin{pgfonlayer}{background}
%					\draw[->,outstyle] (4,4.5) -- +(0:\arrowlength);
%					\draw[->,outstyle] (4,3.5) -- +(0:\arrowlength);
%				\end{pgfonlayer}
%			\draw[yshift = \smcirclerad, linestyle, fuzzright] (2,4) 
%				to [looseness=1.6, out=180, in=190] +(90:1.5*\smcirclerad)
%				to [looseness=1.6, out=10, in=-10] +(90:-5*\smcirclerad)
%				to [looseness=1.6, out=170, in=180] +(90:1.5*\smcirclerad)
%				to [looseness=1.55, out=0, in=0] +(90:2*\smcirclerad);
%	
%			
%			% ev^R ev + ev^L (deformed)
%			\draw[linestyle,fuzzright] (4,2) to [looseness=1.6,out = 180, in = 180] (4,1);
%					\begin{pgfonlayer}{background}
%						\draw[->,outstyle] (4,2) -- +(0:\arrowlength);
%						\draw[->,outstyle] (4,1) -- +(0:\arrowlength);
%					\end{pgfonlayer}
%			\draw[linestyle,fuzzright] (2.5, 2) -- (1.75,2)
%					to [looseness=2,out = 180, in = 90] (2, 2.25)
%						-- (2, 0.75)
%						to [looseness=2,out = 270, in = 180] (1.75,1)
%						-- (2.5, 1) arc (-90:90:0.5cm);
%			
%			
%	
%			% ev^R
%				\draw[linestyle,fuzzright] (4, -0.5) -- (1.75, -0.5)
%					to [looseness=1.7,out = 180, in = 90] (2, -0.25)
%					-- (2, -1.75)
%					to [looseness=1.7,out = 270, in = 180] (1.75,-1.5)
%					-- (4, -1.5);
%				\begin{pgfonlayer}{background}
%					\draw[->,outstyle] (4,-0.5) -- +(0:\arrowlength);
%					\draw[->,outstyle] (4,-1.5) -- +(0:\arrowlength);
%				\end{pgfonlayer}
%		
%			
%		
%		\end{tikzpicture}
%	\end{center}
%	\caption{.... }
%\end{figure}
%

%\tikzexternaldisable

\section{mmm}

\begin{proof}
	This proof may be substantially shortened if one assumes that $F$ is either right exact or left exact, but here we will only assume that it is additive. 
	Since $\cC$ is finite there exists a short exact sequence in $\cC$:
	\begin{equation*}
		0 \to k \to p \to 1 \to 0
	\end{equation*}
	with $p \in \cC$ projective. Let
	\begin{equation*}
		0 \to x \to y \to z \to 0
	\end{equation*}
	be a short exact sequence in $\cM$. Tensoring these together we obtain the following $3$x$3$ diagram. 
	\begin{center}
	\begin{tikzpicture}
			\matrix (m) [matrix of math nodes, row sep = 0.5cm, column sep = 0.75cm, text height=1.5ex, text depth=0.25ex]
			{
				  & 0 & 0 & 0 & \\
				0 & k \otimes x & k \otimes y & k \otimes z & 0 \\
				0 & p \otimes x & p \otimes y & p \otimes z & 0 \\
				0 &  x &  y &  z & 0 \\ 
				  & 0 & 0 & 0 & \\
			};
			\foreach \x in {2,3,4}{ % coloumns
				\foreach \y/\z in {1/2,2/3,3/4,4/5}{
					\draw [->] (m-\y-\x) -- (m-\z-\x);
				};		
			};
			\foreach \y in {2,3,4}{ % rows
				\foreach \x/\z in {1/2,2/3,3/4,4/5}{
					\draw [->] (m-\y-\x) -- (m-\y-\z);
				};		
			};
	\end{tikzpicture}
	\end{center}
	By Cor.~\ref{cor:biexact_action}, all of the columns and rows are exact. Moreover, since $\cM$ is exact, $p \otimes z$ is projective, and hence the middle row is a split exact sequence. 
		\begin{equation*}
		0 \to p \otimes x \to p \otimes y \to p \otimes z \to 0 
	\end{equation*}
	Applying $F$, we obtain the following commutative diagram. The solid sequences are exact: the columns are exact by Cor.~\ref{cor:biexact_action}, and the middle row is exact because any additive functor preserves split short exact sequences. 
	\begin{center}
	\begin{tikzpicture}
			\matrix (m) [matrix of math nodes, row sep = 0.5cm, column sep = 0.75cm, text height=1.5ex, text depth=0.25ex]
			{
				  & 0 & 0 & 0 & \\
				0 & k \otimes F(x) & k \otimes F(y) & k \otimes F(z) & 0 \\
				0 & p \otimes F(x) & p \otimes F(y) & p \otimes F(z) & 0 \\
				0 &  F(x) &  F(y) &  F(z) & 0 \\ 
				  & 0 & 0 & 0 & \\
			};
			\foreach \x in {2,3,4}{ % coloumns
				\foreach \y/\z in {1/2,2/3,3/4,4/5}{
					\draw [->] (m-\y-\x) -- (m-\z-\x);
				};		
			};
			\foreach \y in {2,4}{ % rows
				\foreach \x/\z in {1/2,2/3,3/4,4/5}{
					\draw [->, dashed] (m-\y-\x) -- node [above] {} (m-\y-\z);
				};		
			};
			\foreach \x/\z/\text in {1/2,2/3,3/4,4/5}{
				\draw [->] (m-3-\x) -- (m-3-\z);
			};
	\end{tikzpicture}
	\end{center}
	A diagram chase in the lower-right-hand corner shows that $F(y) \to F(z) \to 0$ is exact. Hence, since $\cC$ is rigid,
		$k \otimes F(y) \to k \otimes F(z) \to 0$
	is exact as well. Now we may apply the sharp $3 \times 3$-lemma \cite[Lem. 2]{MR1004230} to conclude that $F(x) \to F(y) \to F(z)$ is also exact. In other words we have shown that $F$ is right exact. In particular
	\begin{equation*}
		k \otimes F(x) \to k \otimes F(y) \to k \otimes F(z) \to 0
	\end{equation*} 
	is exact. Now we may apply the snake lemma to the above diagram to conclude that $F$ is in fact exact. 
\end{proof}




\end{document}
	
	