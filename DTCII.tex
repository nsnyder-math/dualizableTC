%%% Authors: Christopher Douglas, Christopher Schommer-Pries, and Noah Snyder

\documentclass{amsart}


%%%%%%% Standard Packages
\usepackage{amsmath}       % I think this gives me some symbols
\usepackage{amsthm}        % Does theorem stuff
\usepackage{amssymb}       % more symbols and fonts
\usepackage{amsfonts}
\usepackage[all]{xy}
\usepackage{xspace}
\usepackage{calc}



\setlength{\topskip}{0pt}
\setlength{\footskip}{30pt}
\headheight=0pt
\topmargin=0pt
\headsep=18pt
\textheight=603pt %% 792pt to page, 648 is 9in
\textwidth=420pt  %% 612pt to page, 468pt is 6.5in
\oddsidemargin=25pt
\evensidemargin=25pt

\pagestyle{plain}


%%%%%% Adds hyperlinks
\usepackage[colorlinks, linkcolor=black, citecolor=blue,
	% pagebackref,
 	%bookmarksnumbered=true
	]{hyperref}
	
	
	
%%%%%% Tikz !!! Commands and Macros %%%%%%%%%%%%%
\usepackage{tikz}
\usetikzlibrary{matrix}


%%%% These draw triple or quadruple set of arrows of length 0.5 cm
\DeclareMathOperator{\righttriplearrows} {{\; \tikz{ \foreach \y in {0, 0.1, 0.2} { \draw [-stealth] (0, \y) -- +(0.5, 0);}} \; }}
\DeclareMathOperator{\lefttriplearrows} {{\; \tikz{ \foreach \y in {0, 0.1, 0.2} { \draw [stealth-] (0, \y) -- +(0.5, 0);}} \; }}
\DeclareMathOperator{\rightquadarrows} {{\; \tikz{ \foreach \y in {0, 0.1, 0.2, 0.3} { \draw [-stealth] (0, \y) -- +(0.5, 0);}} \; }}
\DeclareMathOperator{\leftquadarrows} {{\; \tikz{ \foreach \y in {0, 0.1, 0.2, 0.3} { \draw [stealth-] (0, \y) -- +(0.5, 0);}} \; }}

%%%%%%% End TikZ Commands and Macros %%%%%%%%%%%%%



%%%%%%%%%%%%%%%%%%%%%% Theorem Styles and Counters %%%%%%%%%%%%%%%%%%%%%%%%%%
% These all use the same "theorem" counter. 
\theoremstyle{plain} %%% Plain Theorem Styles.
\newtheorem{theorem}{Theorem}[section]
\newtheorem{lemma}[theorem]{Lemma}
\newtheorem{corollary}[theorem]{Corollary}          
\newtheorem{proposition}[theorem]{Proposition}              

\theoremstyle{definition} %%%% Definition-like Commands  
\newtheorem{definition}[theorem]{Definition}

\theoremstyle{remark}  %%%% Remark-like Commands
\newtheorem{remark}[theorem]{Remark}
\newtheorem{example}[theorem]{Example}
%%%%%%%%%%%%%%%%%%%%%% End Theorem Styles and Counters %%%%%%%%%%%%%%%%%%%%%%%%%%

%%%% Misc symbols %%%%%

\newcommand{\nn}{\nonumber}
\newcommand{\nid}{\noindent}
\newcommand{\ra}{\rightarrow}
\newcommand{\la}{\leftarrow}
\newcommand{\xra}{\xrightarrow}
\newcommand{\xla}{\xleftarrow}

\newcommand{\Bord}{\mathrm{Bord}}
\newcommand{\Vect}{\mathrm{Vect}}
\newcommand{\TC}{\mathrm{TC}}

\def\cA{\mathcal A}\def\cB{\mathcal B}\def\cC{\mathcal C}\def\cD{\mathcal D}
\def\cE{\mathcal E}\def\cF{\mathcal F}\def\cG{\mathcal G}\def\cH{\mathcal H}
\def\cI{\mathcal I}\def\cJ{\mathcal J}\def\cK{\mathcal K}\def\cL{\mathcal L}
\def\cM{\mathcal M}\def\cN{\mathcal N}\def\cO{\mathcal O}\def\cP{\mathcal P}
\def\cQ{\mathcal Q}\def\cR{\mathcal R}\def\cS{\ess}\def\cT{\mathcal T}
\def\cU{\mathcal U}\def\cV{\mathcal V}\def\cW{\mathcal W}\def\cX{\mathcal X}
\def\cY{\mathcal Y}\def\cZ{\mathcal Z}

\def\AA{\mathbb A}\def\BB{\mathbb B}\def\CC{\mathbb C}\def\DD{\mathbb D}
\def\EE{\mathbb E}\def\FF{\mathbb F}\def\GG{\mathbb G}\def\HH{\mathbb H}
\def\II{\mathbb I}\def\JJ{\mathbb J}\def\KK{\mathbb K}\def\LL{\mathbb L}
\def\MM{\mathbb M}\def\NN{\mathbb N}\def\OO{\mathbb O}\def\PP{\mathbb P}
\def\QQ{\mathbb Q}\def\RR{\mathbb R}\def\SS{\mathbb S}\def\TT{\mathbb T}
\def\UU{\mathbb U}\def\VV{\mathbb V}\def\WW{\mathbb W}\def\XX{\mathbb X}
\def\YY{\mathbb Y}\def\ZZ{\mathbb Z}

%%%%%%%%%















\begin{document}

\title{``Dualizable Tensor Categories II: homotopy $SO(3)$-actions"}

\author{Christopher L. Douglas}
\address{Mathematical Institute\\ University of Oxford\\ Oxford OX1 3LB\\ United Kingdom}
\email{cdouglas@maths.ox.ac.uk}
\urladdr{http://people.maths.ox.ac.uk/cdouglas}
      	
\author{Christopher Schommer-Pries}
\address{Department of Mathematics\\ Massachusetts Institute of Technology\\ Cambridge, MA 02139\\ USA}
\email{schommerpries.chris.math@gmail.com}
\urladdr{http://sites.google.com/site/chrisschommerpriesmath}

\author{Noah Snyder}
\address{Department of Mathematics\\ Columbia University\\ New York, NY 10027\\ USA}
\email{nsnyder@math.columbia.edu}
\urladdr{http://www.math.columbia.edu/\!\raisebox{-1mm}{~}nsnyder/}


\maketitle

\tableofcontents

\section*{Sketch scratch}

This purpose of this paper is to explain what it means, cellularly, to give a homotopy $SO(3)$-action on a homotopy 3-type.  That is, you say, $BSO(3)$ has cells $e_2, e_3, e_4, e_5$, where $e_3$ is attached to $e_2$ by $2$, and $e_4$ is attached to $e_2$ by a generator $q: S^3 \ra S^2$ and $e_5$ is attached to $e_3$ by yadda.  And to give a map $BSO(3) \rightarrow BAut(X)$ is to give an element $S \in Map(S^2,BAut(X))$ and a trivialization $A$ of $q S \in Map(S^3,BAut(X))$ and a trivialization $R$ of $2S$ such that a canonical element called $\frac{\eta q S}{2}$ of $\pi_4(BAut(X))$ vanishes.  (Note that the choice of 2-divisor of $\eta q S$ depends on $R$.  Here $\eta: S^4 \ra S^3$ is the generator.)

Along the way, we also discuss what it means to have actions by $Orp(3)$, $SO(2)$, $\Omega S^2$, $\Omega \Sigma \RP^2$.

\CDcomm{We should also precisely describe the data required to give a G-homotopy fixed point, for each of the discussed groups.  And for each group we should also explicitly describe the map $G \ra Aut(X)$ in terms of the given data defining $BG \ra BAut(X)$.}

\subsection{General remarks}

Why? The cobordism hypothesis tells us that $Hom^{\otimes}(GBord_n,C) = ((dC)^{\sim})^{hG}$, where $dC$ is the maximal symmetric monoidal sub-n-category of C with duals, and the $\sim$ denotes the maximal sub-n-groupoid.  Note that $(dC)^\sim$ is a homotopy $n$-type.  Thus we want to describe explicitly the structure of $G$ actions on $n$-types.  (In fact, the description we give will apply to actions on discrete $n$-categories that are not necessarily $n$-groupoids \CDcomm{(right?)}, but such a case is not relevant for cobordism hypothesis applications and we will not mention it explicitly.)

If $C$ is a discrete $n$-category (eg an $n$-type), let $Aut(C)$ denote the $n$-groupoid of automorphisms of $C$ --- the objects are the automorphisms of $C$ in the discrete $(n+1)$-category of $n$-categories, the morphisms are the automorphisms of automorphisms, etc.  Thus $Aut(C)$ is an $n$-type and $BAut(C)$ is an $(n+1)$-type.  By definition, an action of $G$ on $C$ is a map $BG \ra BAut(C)$.  Observe that if a map $A \ra B$ is a $k$-equivalence (i.e. an isomorphism on $\pi_i$ for $i \leq k$) and $X$ is a $k$-type (i.e. has $\pi_i = 0$ for $i > k$), then the induced map $[B,X] \ra [A,X]$ is an isomorphism.  As a result, we will only be concerned with the $(n+1)$-type of $BG$ when we are considering actions on $n$-categories.

... the techniques are for the most part elementary, but we believe the results are of fundamental importance for the study of 3-dimensional quantum topology.


\subsection{Warmup: $SO(2)$ actions}

An $SO(2)$ action on a 3-type $X$ is a map $\alpha: BSO(2) \ra BAut(X)$.  The 5-skeleton of $BSO(2)$ has cells $e_2$ and $e_4$.  The data of the map $\alpha$ on $e_2$ is an element of $\pi_2(BAut(X)) = \pi_1(Aut(X))$.  We denote this element $S$ and call it the Serre automorphism.  The 4-cell is attached to the 2-cell by $\eta : S^3 \ra S^2$.  The data of an extension of $S$ to the 4-cell and therefore to all of $BSO(2)$ is given by a trivialization of the composite $\eta S : S^3 \ra BAut(X)$.

\begin{proposition}
The composite $\eta S : S^3 \ra BAut(X)$ can be computed as follows: draw a figure 8, read the cap and cup as the equivalences $1 \cong S \cdot S^{-1}$ and $S^{-1} \cdot S \cong 1$ and the cross as the braiding of elements in $2Aut(X)$, and evaluate the picture in $\Omega^2(Aut(X)) \simeq \Omega^3 BAut(X)$.
\end{proposition}

\begin{proof}
\CDcomm{This would be clear stably, but requires explanation unstably.}
\end{proof}

To summarize:

\begin{proposition}
The set of homotopy classes of $SO(2)$ actions on a 3-type $X$ is the set of homotopy classes of pairs $(S,E)$, where $S: S^1 \ra Aut(X)$ is a map and $E : D^3 \ra Aut(X)$ is a null homotopy of the figure 8 composite on $S$.
\end{proposition}

\CDcomm{More explanation is really necessary to even make the statement clear.  Eg, make clear in what sense is the figure 8 composite an actual map to be trivialized rather than merely a homotopy class of maps.  (If it were just a homotopy class, it would look like E is an element of $\pi_3$ of $Aut(X)$, which is wrong, since it lives in a torsor.)  For instance, do this by picking equivalences $\Omega^2(Aut(X)) \simeq \Omega^3 BAut(X)$ etc?}

We can restrict attention to 3-types that happen to be 2-types.

\begin{corollary}
The set of homotopy classes of $SO(2)$ actions on a 2-type $X$ is the set of homotopy classes of maps $S: S^1 \ra Aut(X)$ such that the figure 8 composite on $S$ is null.
\end{corollary}

\section{Introduction}
.

\CDcomm{I think much of what is currently in section I of DTCIV, introducing and discussing the 4-types of BQuad and BOrp(3), will need to be moved here for those spaces to seem less crazy.  How much?}

\CDcomm{Annoyingly, we probably also need some of the facts in the Appendix to DTCIV.  Nevertheless, I think that appendix should stay there, and we will just state the occasional fact we need here independently.}

\section{Cell structures}

\CDcomm{This section is ``pure homotopy theory", ie focuses on determining the cell structures only.  All issues about interpreting the attaching maps and computing them explicitly are left to later sections.}

As we only care about the 4-types of our spaces, for each space we will describe a cell complex that captures the 4-type in question.  More specifically, for concreteness we will construct 4-truncations of our spaces, in the following sense.

\begin{definition}
By a \emph{$k$-truncation} of a space $X$, we will mean a $(k+1)$-dimensional cell complex, denoted $_k X$, together with a $k$-equivalence $_{k} X \ra X$.
\end{definition}
\nid Note that the homotopy type of a $k$-truncation is not uniquely determined, but the $k$-type is.  Moreover the homotopy class of the $k$-equivalence $_k X \ra X$ is not, in general, uniquely determined.  Though it is by no means a necessity, it is sometimes reassuring to know that a given truncation is as small as possible: we call a $k$-truncation \emph{minimal} if it has the minimum number of cells among all $k$-truncations.  Note that the cell structure of a minimal $k$-truncation is not in general uniquely determined, though it happens that it is in all the examples we consider.

%\CD{For better or worse the notation here is meant to suggest that $_4 X$ is a subobject, as opposed to the usual ``4-quotient" $X_4$ obtained by adding cells to $X$.  This notation is a bit questionable.  Note that $_4 X$ is not a 4-type.}

\subsection{Of the spaces $B\Omega S^2$ and $B\Quad$}.

Needless to say, the 4-truncation of $B\Omega S^2 \simeq S^2$ is simply $S^2$, and this is obviously minimal.

\begin{definition}
The space $B\Quad$ is the homotopy fiber of the map $c_1^2: BSO(2) \ra K(\ZZ,4)$ classifying the square of the first chern class.
\end{definition}
%\CD{Maybe Quad will already have been introduced earlier.}


\begin{proposition}
The minimal 4-trunctation of $BQuad$ is the 2-cell complex $S^2 \cup_{\eta q} e_5$.  Here the attaching map is the composite of the generator $\eta: S^4 \ra S^3$ and the generator $q: S^3 \ra S^2$.
\end{proposition}

\begin{proof}
From the long exact sequence of the fibration $K(\ZZ,3) \ra B\Quad \ra BSO(2)$, we see that the first four homotopy groups ($\pi_1$--$\pi_4$) of $B\Quad$ are $0$, $\ZZ$, $\ZZ$, $0$, and from the corresponding Serre spectral sequence, that the first six cohomology groups ($H^1$--$H^6$) are $0$, $\ZZ$, $0$, $0$, $0$, $\ZZ/2$.  From the cohomology, we see there is a 4-truncation with a single 2-cell and a single 5-cell, and the homotopy groups force the attaching map to be $\eta q$.  This truncation is clearly minimal.
\end{proof}

\subsection{Of the space $BSO(2)$}.

The 4-truncation of $BSO(2)$ is $S^2 \cup_q e_4$, and this is certainly minimal.

\subsection{Of the spaces $B \Omega \Sigma \RP^2$ and $B Orp(3)$}.

The space $B \Omega \Sigma \RP^2 \simeq \Sigma \RP^2 \simeq S^2 \cup_2 e_3$ is its own minimal 4-truncation.

\begin{definition}
The space $BOrp(3)$ is the homotopy fiber of the composite $BSO(3) \ra BSO \xra{p_1} K(\ZZ,4)$, where $p_1$ is the first Pontryagin class.
\end{definition}

We will need the following homotopy-theoretic fact.
\begin{proposition}
The fourth homotopy group of the suspension of the real projective plane is $\ZZ/4$, generated by a class called $\frac{\eta q}{2}$ in deference to the fact that $2 \cdot \frac{\eta q}{2}$ is equal to the homotopy class of the composite $S^4 \xra{\eta} S^3 \xra{q} S^2 \xra{\text{inc}} \Sigma \RP^2$.
\end{proposition}
\CD{Do we have a homotopy theoretic description of the generator?} %Later we will certainly give the geometric description.
\CDcomm{Need to add a reference or proof here.}

\begin{proposition} \label{prop-borp}
The minimal 4-truncation of $BOrp(3)$ is the 3-cell complex $(S^2 \cup_2 e_3) \cup_{\frac{\eta q}{2}} e_5$.
\end{proposition}
\begin{proof}
The homotopy groups of $S^3$ determine the homotopy groups of $SO(3)$ which determine the homotopy groups of $BSO(3)$ as $0$, $\ZZ/2$, $0$, $\ZZ$, $\ZZ/2$ for $\pi_1$ through $\pi_5$.  Using the slightly surprising fact that the composite $BSO(3) \ra BSO \xra{p_1} K(\ZZ,4)$ induces multiplication by 4 on $\pi_4$, this in turn determines the homotopy groups of $BOrp(3)$ as $0$, $\ZZ/2$, $\ZZ/4$, $0$, $\ZZ/2$.  

The Serre spectral sequence for $SO(3) \ra * \ra BSO(3)$ shows the cohomology of $BSO(3)$ is $0$, $0$, $\ZZ/2$, $\ZZ$, $0$, $\ZZ/2$ for $H^1$ through $H^6$, and the Serre spectral sequence for $K(\ZZ,3) \ra BOrp(3) \ra BSO(3)$ then gives the cohomology of $BOrp(3)$ as $0$, $0$, $\ZZ/2$, $0$, $0$, $\ZZ/4$.  (The extension in degree 6 is resolved by the corresponding mod 2 spectral sequence.)

From the cohomology of $BOrp(3)$ we see that there is a map $\Sigma \RP^2 \ra BOrp(3)$ which is an isomorphism on $H_{\leq 4}$.  By Hurewicz it follows that this map is an isomorphism on $\pi_{\leq 3}$.  To correct the $\pi_4$ discrepancy, we need to attach a 5-cell to kill $\pi_4(\Sigma \RP^2)$.  Because $\pi_4(BOrp(3))$ is zero, we may pick an extension of the map $\Sigma \RP^2 \ra BOrp(3)$ to a map $(S^2 \cup_2 e_3) \cup_{\frac{\eta q}{2}} e_5 \ra BOrp(3)$.  This latter map is now a 4-equivalence, as desired.  This truncation is clearly minimal.  Note that there were two possible choices of the extension, but both yield 4-equivalences and we will have no need to differentiate them.
\end{proof}

\subsection{Of the space $BSO(3)$}.


\begin{definition}
A commutative square is a pushout of $n$-types, also called an $n$-pushout, if the canonical map from the pushout to the target object is an $n$-equivalence.
\end{definition}

\begin{proposition}
The commutative square
\begin{equation}\label{s2-pushout} \tag{S}
\xymatrix{
B\Omega S^2 \ar[r] \ar[d] & BOrp(3) \ar[d] \\
BSO(2) \ar[r] & BSO(3)
}
\end{equation}
is a $4$-pushout.
\end{proposition}
\begin{proof}
We need to show that the map $P := BSO(2) \cup_{S^2} BOrp(3) \ra BSO(3)$ is a 4-equivalence.  By the exact sequence of the pair $(P, BOrp(3))$, the homology of $P$ in degrees 1 through 5 is $0$, $\ZZ/2$, $0$, $\ZZ$, $\ZZ/4$.  Now consider the map $P \ra BSO(3)$ in homology.  The maps $H_2(BOrp(3)) \ra H_2(P)$ and $H_5(BOrp(3)) \ra H_5(P)$ are isomorphisms, and $H_2(BOrp(3)) \ra H_2(BSO(3))$ is an isomorphism and $H_5(BOrp(3)) \ra H_5(BSO(3))$ is surjective by the Serre spectral sequence of $K(\ZZ,3) \ra BOrp(3) \ra BSO(3)$.  The map $H_4(BSO(2)) \ra H_4(P)$ is an isomorphism, and the map $H_4(BSO(2)) \ra H_4(BSO(3))$ is an isomorphism (by for instance noting that the mod 2 reduction of $p_1$ is $w_2^2$ and that considered with $\ZZ[\frac{1}{2}]$ coefficients, the pullback of $p_1$ to $BU(1)$ is $c_1^2$).  It follows that $H_{\leq 5}(BSO(3),P) = 0$, thus $\pi_{\leq 5}(BSO(3),P) = 0$ and so the map $P \ra BSO(3)$ is a 4-equivalence, as desired.
\end{proof}

%\CD{I suggest that in writing we try to be fairly careful to denote the difference between a space $X$ and its $k$-truncation or $k$-quotient, ie never leave the truncation/quotient implicit.}

\begin{corollary} \label{cor-bso3}
The minimal 4-truncation of $BSO(3)$ is the 4-cell complex $(S^2 \cup_2 e_3) \cup_q e_4 \cup_{\frac{\eta q}{2}} e_5$, where $q: S^3 \ra S^2$ and $\frac{\eta q}{2}: S^4 \ra (S^2 \cup_2 e_3)$ are generators.
\end{corollary}

\begin{proof}
By the proposition, the map $P := BSO(2) \cup_{S^2} BOrp(3) \ra BSO(3)$ is a 4-equivalence.  Note that $_4 BSO(2) \cup_{S^2} {}_4 BOrp(3)$ is a 4-truncation of $P$ and therefore provides a 4-truncation of $BSO(3)$.  Using Proposition~\ref{prop-borp} the cell structure of that space is as described.  

That this truncation is uniquely minimal is seen as follows.  The first four homology groups of $BSO(3)$ force any truncation to have at least one cell in each of the dimensions 2, 3, and 4.  If the truncation has only three cells, the attaching map of the 3-cell must be multiplication by 2, and the third homotopy group of $BSO(3)$ forces the attaching map of the 4-cell to be a generator; call the resulting complex $H := (S^2 \cup_2 e_3) \cup_q e_4$.  \CDcomm{(Proof that H is not yet a 4-truncation, eg that $\pi_4(H)$ has a nontrivial 2-torsion element.)}  We therefore know that a minimal truncation must have exactly 4-cells.  If it has two 2-cells or two 3-cells, the homology in dimension 2 or 3 respectively will not match that of $BSO(3)$, preventing it from being a 4-truncation.  If the putative 4-cell truncation $T$ has two 4-cells, the homotopy exact sequence of the pair $(BSO(3), T)$ shows that $\pi_4(T) \ra \pi_4(BSO(3))$ cannot be an isomorphism.  A minimal truncation must therefore have a single cell in each of dimensions 2, 3, 4, and 5, and \CDcomm{whatever the proof was about H should make it clear that you have to attach by eta q / 2}.
\end{proof}

Because the fourth and fifth cohomology groups of $BQuad$ are trivial, the composite $BQuad \ra BSO(2) \ra BSO(3) \xra{p_1} K(\ZZ,4)$ is trivial and the map $BQuad \ra BSO(3)$ lifts uniquely to a map $BQuad \ra BOrp(3)$.

\begin{corollary}
The commutative square
\begin{equation} \label{quad-pushout} \tag{Q}
\xymatrix{
BQuad \ar[r] \ar[d] & BOrp(3) \ar[d] \\
BSO(2) \ar[r] & BSO(3)
}
\end{equation}
is also a $4$-pushout.
\end{corollary}
\begin{proof}
Though the map $B\Omega S^2 \ra BQuad$ is not a 4-equivalence, the difference of 4-types is a single additional 5-cell in $BQuad$, which results in an additional 6-cell of $BSO(2) \cup_{BQuad} BOrp(3)$ compared to $BSO(2) \cup_{B\Omega S^2} BOrp(3)$; that 6-cell does not affect the 4-type of the pushout.
\end{proof}

\nid Though the square~\eqref{s2-pushout} is more elementary, the square~\eqref{quad-pushout} encodes important conceptual information not visible in~\eqref{s2-pushout}.  Specifically, a cellular model of the pushout $BSO(2) \cup_{BQuad} BOrp(3)$ has, in addition to cells $e_2$, $e_3$, $e_4$, and $e_5$ described in Corollary~\ref{cor-bso3}, a 6-cell $e_6$ attached as follows.  First note that the 5-cell of $BQuad$ maps by multiplication by $2$ to the 5-cell of $BOrp(3)$ and by the cone on $\eta$ to the 4-cell of $BSO(2)$.  % This is not at all clear, but is forced if you know 2 gamma = eta q.  Maybe rephrase in a less flippant way?
Now decompose the boundary $\partial e_6$ as the union $a_5 \cup (S^4 \times I) \cup b_5$, for 5-cells $a_5$ and $b_5$.  The cell $a_5$ wraps twice onto $e_5$, and the cell $b_5$ maps by the cone on $\eta: S^4 \ra S^3$ onto $e_4$, while the cylinder $S^4 \times I$ performs a homotopy between $2 \gamma : S^4 \ra \Sigma \RP^2$ and $\eta q: S^4 \ra S^2$ in $\Sigma \RP^2$.  In words, the 4-cell of $BSO(3)$ trivializes $q$ while the 5-cell trivializes $\gamma$, and the 6-cell witnesses the equivalence of $\eta$ times the trivialization of $q$ with $2$ times the trivialization of $\gamma$.

\begin{remark}
There is a commutative square (but not a 4-pushout),
\begin{equation}\label{quadmoore} \tag{M}
\xymatrix{
B\Omega S^2 \ar[r] \ar[d] & B\Omega \Sigma \RP^2 \ar[d] \\
BQuad \ar[r] & BOrp(3)
}
\end{equation}
% It isn't a 4-pushout because the pushout is Sigma RP2 with a 5-cell killing eta q, but you know eta q is twice the generator of pi_4 Sigma RP2, so there is pi_4 left in the pushout, which is no present in BOrp(3).
We saw that $B\Omega S^2$ and $BQuad$ were interchangable in the pushout square.  The space $BOrp(3)$ cannot be similarly replaced by $B\Omega \Sigma \RP^2$.  Nevertheless it will sometimes be useful to consider building an $Orp(3)$ action by first building an $\Omega \Sigma \RP^2$ action.
\end{remark}


\CDcomm{Later on, we'll describe the meaning of a Quad action, etc, and then refer back to see that S leads to one description, while Q leads to a slightly lengthier description that ties together the trivialization of q and the condition on eta q / 2.}

\section{Homotopy actions}

\CDcomm{This section describes the meaning of a homotopy action for each group.  This is where the explicit translation to computable, i.e. categorically readable, expressions occurs.  Also, for each group, the map $G \ra Aut(X)$ is described.}

\subsection{By the groups $\Omega S^2$ and $\Quad$}


\subsection{By the group $SO(2)$}


\subsection{By the groups $\Omega \Sigma \RP^2$ and $Orp(3)$}


\subsection{By the group $SO(3)$}


\section{Homotopy fixed points}

\CDcomm{This section describes the meaning of a homotopy fixed point explicitly for each group.}


\subsection{For $\Omega S^2$ and $\Quad$ actions}


\subsection{For $SO(2)$ actions}


\subsection{For $\Omega \Sigma \RP^2$ and $Orp(3)$ actions}


\subsection{For $SO(3)$ actions}

.






\end{document}











