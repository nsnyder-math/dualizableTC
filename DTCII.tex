%%% Authors: Christopher Douglas, Christopher Schommer-Pries, and Noah Snyder

\documentclass{amsart}


%%%%%%% Standard Packages
\usepackage{amsmath}       % I think this gives me some symbols
\usepackage{amsthm}        % Does theorem stuff
\usepackage{amssymb}       % more symbols and fonts
\usepackage{amsfonts}
\usepackage[all]{xy}
\usepackage{xspace}
\usepackage{calc}



\setlength{\topskip}{0pt}
\setlength{\footskip}{30pt}
\headheight=0pt
\topmargin=0pt
\headsep=18pt
\textheight=603pt %% 792pt to page, 648 is 9in
\textwidth=420pt  %% 612pt to page, 468pt is 6.5in
\oddsidemargin=25pt
\evensidemargin=25pt

\pagestyle{plain}


%%%%%% Adds hyperlinks
\usepackage[colorlinks, linkcolor=black, citecolor=blue,
	% pagebackref,
 	%bookmarksnumbered=true
	]{hyperref}
	
	
	
%%%%%% Tikz !!! Commands and Macros %%%%%%%%%%%%%
\usepackage{tikz}
\usetikzlibrary{matrix}


%%%% These draw triple or quadruple set of arrows of length 0.5 cm
\DeclareMathOperator{\righttriplearrows} {{\; \tikz{ \foreach \y in {0, 0.1, 0.2} { \draw [-stealth] (0, \y) -- +(0.5, 0);}} \; }}
\DeclareMathOperator{\lefttriplearrows} {{\; \tikz{ \foreach \y in {0, 0.1, 0.2} { \draw [stealth-] (0, \y) -- +(0.5, 0);}} \; }}
\DeclareMathOperator{\rightquadarrows} {{\; \tikz{ \foreach \y in {0, 0.1, 0.2, 0.3} { \draw [-stealth] (0, \y) -- +(0.5, 0);}} \; }}
\DeclareMathOperator{\leftquadarrows} {{\; \tikz{ \foreach \y in {0, 0.1, 0.2, 0.3} { \draw [stealth-] (0, \y) -- +(0.5, 0);}} \; }}

%%%%%%% End TikZ Commands and Macros %%%%%%%%%%%%%



%%%%%%%%%%%%%%%%%%%%%% Theorem Styles and Counters %%%%%%%%%%%%%%%%%%%%%%%%%%
% These all use the same "theorem" counter. 
\theoremstyle{plain} %%% Plain Theorem Styles.
\newtheorem{theorem}{Theorem}[section]
\newtheorem{lemma}[theorem]{Lemma}
\newtheorem{corollary}[theorem]{Corollary}          
\newtheorem{proposition}[theorem]{Proposition}              

\theoremstyle{definition} %%%% Definition-like Commands  
\newtheorem{definition}[theorem]{Definition}

\theoremstyle{remark}  %%%% Remark-like Commands
\newtheorem{remark}[theorem]{Remark}
\newtheorem{example}[theorem]{Example}
%%%%%%%%%%%%%%%%%%%%%% End Theorem Styles and Counters %%%%%%%%%%%%%%%%%%%%%%%%%%

%%%% Misc symbols %%%%%

\newcommand{\nn}{\nonumber}
\newcommand{\nid}{\noindent}
\newcommand{\ra}{\rightarrow}
\newcommand{\la}{\leftarrow}
\newcommand{\xra}{\xrightarrow}
\newcommand{\xla}{\xleftarrow}

\newcommand{\Bord}{\mathrm{Bord}}
\newcommand{\Vect}{\mathrm{Vect}}
\newcommand{\TC}{\mathrm{TC}}

\def\cA{\mathcal A}\def\cB{\mathcal B}\def\cC{\mathcal C}\def\cD{\mathcal D}
\def\cE{\mathcal E}\def\cF{\mathcal F}\def\cG{\mathcal G}\def\cH{\mathcal H}
\def\cI{\mathcal I}\def\cJ{\mathcal J}\def\cK{\mathcal K}\def\cL{\mathcal L}
\def\cM{\mathcal M}\def\cN{\mathcal N}\def\cO{\mathcal O}\def\cP{\mathcal P}
\def\cQ{\mathcal Q}\def\cR{\mathcal R}\def\cS{\ess}\def\cT{\mathcal T}
\def\cU{\mathcal U}\def\cV{\mathcal V}\def\cW{\mathcal W}\def\cX{\mathcal X}
\def\cY{\mathcal Y}\def\cZ{\mathcal Z}

\def\AA{\mathbb A}\def\BB{\mathbb B}\def\CC{\mathbb C}\def\DD{\mathbb D}
\def\EE{\mathbb E}\def\FF{\mathbb F}\def\GG{\mathbb G}\def\HH{\mathbb H}
\def\II{\mathbb I}\def\JJ{\mathbb J}\def\KK{\mathbb K}\def\LL{\mathbb L}
\def\MM{\mathbb M}\def\NN{\mathbb N}\def\OO{\mathbb O}\def\PP{\mathbb P}
\def\QQ{\mathbb Q}\def\RR{\mathbb R}\def\SS{\mathbb S}\def\TT{\mathbb T}
\def\UU{\mathbb U}\def\VV{\mathbb V}\def\WW{\mathbb W}\def\XX{\mathbb X}
\def\YY{\mathbb Y}\def\ZZ{\mathbb Z}

%%%%%%%%%














\usepackage{MnSymbol} % Needed this for a couple symbols, but it breaks \dtimes, so is not included in the general preamble.


%\tikzexternaldisable


\begin{document}

\title{``Dualizable Tensor Categories II: homotopy $SO(3)$-actions"}

\author{Christopher L. Douglas}
\address{Mathematical Institute\\ University of Oxford\\ Oxford OX1 3LB\\ United Kingdom}
\email{cdouglas@maths.ox.ac.uk}
\urladdr{http://people.maths.ox.ac.uk/cdouglas}
      	
\author{Christopher Schommer-Pries}
\address{Department of Mathematics\\ Massachusetts Institute of Technology\\ Cambridge, MA 02139\\ USA}
\email{schommerpries.chris.math@gmail.com}
\urladdr{http://sites.google.com/site/chrisschommerpriesmath}

\author{Noah Snyder}
\address{Department of Mathematics\\ Columbia University\\ New York, NY 10027\\ USA}
\email{nsnyder@math.columbia.edu}
\urladdr{http://www.math.columbia.edu/\!\raisebox{-1mm}{~}nsnyder/}


\maketitle

\tikzexternaldisable

\tableofcontents

\CDcomm{To do items.
\begin{itemize}
\item Mention DTCI plus this implies we have constructed an $\Omega S^2$ action and an $\Omega \Sigma \RP^2$ action on $TC$.  DTCIII will go further.
\item Similarly, in ho fixed point section, discuss that pivotality is $\Omega S^2$ fixed point.  For sphericality, either do it ($\Omega \Sigma \RP^2$ fixed point) or forward ref to DTCIV, considering it may require translating the usual notion of sphericality to our notion.  (Same fixed points for $\Omega S^2$ and $Quad$ and for $\Omega \Sigma \RP^2$ and $\Orp(3)$.
\item Add category theory perspective/interpretation, building on Joyal/Street view of figure 8.
\item (Research to do: what is W?)
\end{itemize}
}

\section*{Sketch scratch}

This purpose of this paper is to explain what it means, cellularly, to give a homotopy $SO(3)$-action on a homotopy 3-type.  That is, you say, $BSO(3)$ has cells $e_2, e_3, e_4, e_5$, where $e_3$ is attached to $e_2$ by $2$, and $e_4$ is attached to $e_2$ by a generator $q: S^3 \ra S^2$ and $e_5$ is attached to $e_3$ by yadda.  And to give a map $BSO(3) \rightarrow BAut(X)$ is to give an element $S \in Map(S^2,BAut(X))$ and a trivialization $A$ of $q S \in Map(S^3,BAut(X))$ and a trivialization $R$ of $2S$ such that a canonical element called $\frac{\eta q S}{2}$ of $\pi_4(BAut(X))$ vanishes.  (Note that the choice of 2-divisor of $\eta q S$ depends on $R$.  Here $\eta: S^4 \ra S^3$ is the generator.)

Along the way, we also discuss what it means to have actions by $Orp(3)$, $SO(2)$, $\Omega S^2$, $\Omega \Sigma \RP^2$.

\CDcomm{We should also precisely describe the data required to give a G-homotopy fixed point, for each of the discussed groups.  And for each group we should also explicitly describe the map $G \ra Aut(X)$ in terms of the given data defining $BG \ra BAut(X)$.}

\subsection{General remarks}

Why? The cobordism hypothesis tells us that $Hom^{\otimes}(GBord_n,C) = ((dC)^{\sim})^{hG}$, where $dC$ is the maximal symmetric monoidal sub-n-category of C with duals, and the $\sim$ denotes the maximal sub-n-groupoid.  Note that $(dC)^\sim$ is a homotopy $n$-type.  Thus we want to describe explicitly the structure of $G$ actions on $n$-types.  (In fact, the description we give will apply to actions on discrete $n$-categories that are not necessarily $n$-groupoids \CDcomm{(right?)}, but such a case is not relevant for cobordism hypothesis applications and we will not mention it explicitly.)

If $C$ is a discrete $n$-category (eg an $n$-type), let $Aut(C)$ denote the $n$-groupoid of automorphisms of $C$ --- the objects are the automorphisms of $C$ in the discrete $(n+1)$-category of $n$-categories, the morphisms are the automorphisms of automorphisms, etc.  Thus $Aut(C)$ is an $n$-type and $BAut(C)$ is an $(n+1)$-type.  By definition, an action of $G$ on $C$ is a map $BG \ra BAut(C)$.  Observe that if a map $A \ra B$ is a $k$-equivalence (i.e. an isomorphism on $\pi_i$ for $i \leq k$) and $X$ is a $k$-type (i.e. has $\pi_i = 0$ for $i > k$), then the induced map $[B,X] \ra [A,X]$ is an isomorphism.  As a result, we will only be concerned with the $(n+1)$-type of $BG$ when we are considering actions on $n$-categories.

... the techniques are for the most part elementary, but we believe the results are of fundamental importance for the study of 3-dimensional quantum topology.


\section{Introduction}
.

\CDcomm{I think much of what is currently in section I of DTCIV, introducing and discussing the 4-types of BQuad and BOrp(3), will need to be moved here for those spaces to seem less crazy.  How much?}

\CDcomm{Annoyingly, we probably also need some of the facts in the Appendix to DTCIV.  Nevertheless, I think that appendix should stay there, and we will just state the occasional fact we need here independently.}


\subsection*{Acknowledgments}
CSP would like to thank Peter Teichner for several fruitful conversations about the difference between $\Omega \Sigma \RR \PP^2$-actions and $SO(3)\langle p_1 \rangle$-actions.



\section{Cell structures}

\CDcomm{This section is ``pure homotopy theory", ie focuses on determining the cell structures only.  All issues about interpreting the attaching maps and computing them explicitly are left to later sections.}

As we only care about the 4-types of our spaces, for each space we will describe a cell complex that captures the 4-type in question.  More specifically, for concreteness we will construct 4-truncations of our spaces, in the following sense.

\begin{definition}
By a \emph{$k$-truncation} of a space $X$, we will mean a $(k+1)$-dimensional cell complex, denoted $_k X$, together with a $k$-equivalence $_{k} X \ra X$.
\end{definition}
\nid Note that the homotopy type of a $k$-truncation is not uniquely determined, but the $k$-type is.  Moreover the homotopy class of the $k$-equivalence $_k X \ra X$ is not, in general, uniquely determined.  Though it is by no means a necessity, it is sometimes reassuring to know that a given truncation is as small as possible: we call a $k$-truncation \emph{minimal} if it has the minimum number of cells among all $k$-truncations.  Note that the cell structure of a minimal $k$-truncation is not in general uniquely determined, though it happens that it is in all the examples we consider.

%\CD{For better or worse the notation here is meant to suggest that $_4 X$ is a subobject, as opposed to the usual ``4-quotient" $X_4$ obtained by adding cells to $X$.  This notation is a bit questionable.  Note that $_4 X$ is not a 4-type.}

\subsection{Of the spaces $B\Omega S^2$ and $B\Quad$}.

Needless to say, the 4-truncation of $B\Omega S^2 \simeq S^2$ is simply $S^2$, and this is obviously minimal.

\begin{definition}
The space $B\Quad$ is the homotopy fiber of the map $c_1^2: BSO(2) \ra K(\ZZ,4)$ classifying the square of the first chern class.
\end{definition}
%\CD{Maybe Quad will already have been introduced earlier.}


\begin{proposition} \label{prop-quadtrunc}
The minimal 4-trunctation of $BQuad$ is the 2-cell complex $S^2 \cup_{\eta q} e_5$.  Here the attaching map is the composite of the generator $\eta: S^4 \ra S^3$ and the generator $q: S^3 \ra S^2$.
\end{proposition}

\begin{proof}
From the long exact sequence of the fibration $K(\ZZ,3) \ra B\Quad \ra BSO(2)$, we see that the first four homotopy groups ($\pi_1$--$\pi_4$) of $B\Quad$ are $0$, $\ZZ$, $\ZZ$, $0$, and from the corresponding Serre spectral sequence, that the first six cohomology groups ($H^1$--$H^6$) are $0$, $\ZZ$, $0$, $0$, $0$, $\ZZ/2$.  From the cohomology, we see there is a 4-truncation with a single 2-cell and a single 5-cell, and the homotopy groups force the attaching map to be $\eta q$.  This truncation is clearly minimal.
\end{proof}

\subsection{Of the space $BSO(2)$}.

The 4-truncation of $BSO(2)$ is $S^2 \cup_q e_4$, and this is certainly minimal.

\subsection{Of the spaces $B \Omega \Sigma \RP^2$ and $B Orp(3)$}.

The space $B \Omega \Sigma \RP^2 \simeq \Sigma \RP^2 \simeq S^2 \cup_2 e_3$ is its own minimal 4-truncation.

\begin{definition}
The space $BOrp(3)$ is the homotopy fiber of the composite $BSO(3) \ra BSO \xra{p_1} K(\ZZ,4)$, where $p_1$ is the first Pontryagin class.
\end{definition}

We will need the following homotopy-theoretic fact.
\begin{proposition}
The fourth homotopy group of the suspension of the real projective plane is $\ZZ/4$, generated by a class called $\frac{\eta q}{2}$ in deference to the fact that $2 \cdot \frac{\eta q}{2}$ is equal to the homotopy class of the composite $S^4 \xra{\eta} S^3 \xra{q} S^2 \xra{\text{inc}} \Sigma \RP^2$.
\end{proposition}
\CD{Do we have a homotopy theoretic description of the generator?} %Later we will certainly give the geometric description.
\CDcomm{Need to add a reference or proof here.}

\begin{proposition} \label{prop-borptrunc}
The minimal 4-truncation of $BOrp(3)$ is the 3-cell complex $(S^2 \cup_2 e_3) \cup_{\frac{\eta q}{2}} e_5$.
\end{proposition}
\begin{proof}
The homotopy groups of $S^3$ determine the homotopy groups of $SO(3)$ which determine the homotopy groups of $BSO(3)$ as $0$, $\ZZ/2$, $0$, $\ZZ$, $\ZZ/2$ for $\pi_1$ through $\pi_5$.  Using the slightly surprising fact that the composite $BSO(3) \ra BSO \xra{p_1} K(\ZZ,4)$ induces multiplication by 4 on $\pi_4$, this in turn determines the homotopy groups of $BOrp(3)$ as $0$, $\ZZ/2$, $\ZZ/4$, $0$, $\ZZ/2$.  

The Serre spectral sequence for $SO(3) \ra * \ra BSO(3)$ shows the cohomology of $BSO(3)$ is $0$, $0$, $\ZZ/2$, $\ZZ$, $0$, $\ZZ/2$ for $H^1$ through $H^6$, and the Serre spectral sequence for $K(\ZZ,3) \ra BOrp(3) \ra BSO(3)$ then gives the cohomology of $BOrp(3)$ as $0$, $0$, $\ZZ/2$, $0$, $0$, $\ZZ/4$.  (The extension in degree 6 is resolved by the corresponding mod 2 spectral sequence.)

From the cohomology of $BOrp(3)$ we see that there is a map $\Sigma \RP^2 \ra BOrp(3)$ which is an isomorphism on $H_{\leq 4}$.  By Hurewicz it follows that this map is an isomorphism on $\pi_{\leq 3}$.  To correct the $\pi_4$ discrepancy, we need to attach a 5-cell to kill $\pi_4(\Sigma \RP^2)$.  Because $\pi_4(BOrp(3))$ is zero, we may pick an extension of the map $\Sigma \RP^2 \ra BOrp(3)$ to a map $(S^2 \cup_2 e_3) \cup_{\frac{\eta q}{2}} e_5 \ra BOrp(3)$.  This latter map is now a 4-equivalence, as desired.  This truncation is clearly minimal.  Note that there were two possible choices of the extension, but both yield 4-equivalences and we will have no need to differentiate them.
\end{proof}

\subsection{Of the space $BSO(3)$}.


\begin{definition}
A commutative square is a pushout of $n$-types, also called an $n$-pushout, if the canonical map from the pushout to the target object is an $n$-equivalence.
\end{definition}

\begin{proposition}
The commutative square
\begin{equation}\label{s2-pushout} \tag{S}
\xymatrix{
B\Omega S^2 \ar[r] \ar[d] & BOrp(3) \ar[d] \\
BSO(2) \ar[r] & BSO(3)
}
\end{equation}
is a $4$-pushout.
\end{proposition}
\begin{proof}
We need to show that the map $P := BSO(2) \cup_{S^2} BOrp(3) \ra BSO(3)$ is a 4-equivalence.  By the exact sequence of the pair $(P, BOrp(3))$, the homology of $P$ in degrees 1 through 5 is $0$, $\ZZ/2$, $0$, $\ZZ$, $\ZZ/4$.  Now consider the map $P \ra BSO(3)$ in homology.  The maps $H_2(BOrp(3)) \ra H_2(P)$ and $H_5(BOrp(3)) \ra H_5(P)$ are isomorphisms, and $H_2(BOrp(3)) \ra H_2(BSO(3))$ is an isomorphism and $H_5(BOrp(3)) \ra H_5(BSO(3))$ is surjective by the Serre spectral sequence of $K(\ZZ,3) \ra BOrp(3) \ra BSO(3)$.  The map $H_4(BSO(2)) \ra H_4(P)$ is an isomorphism, and the map $H_4(BSO(2)) \ra H_4(BSO(3))$ is an isomorphism (by for instance noting that the mod 2 reduction of $p_1$ is $w_2^2$ and that considered with $\ZZ[\frac{1}{2}]$ coefficients, the pullback of $p_1$ to $BU(1)$ is $c_1^2$).  It follows that $H_{\leq 5}(BSO(3),P) = 0$, thus $\pi_{\leq 5}(BSO(3),P) = 0$ and so the map $P \ra BSO(3)$ is a 4-equivalence, as desired.
\end{proof}

%\CD{I suggest that in writing we try to be fairly careful to denote the difference between a space $X$ and its $k$-truncation or $k$-quotient, ie never leave the truncation/quotient implicit.}

\begin{corollary} \label{cor-bso3trunc}
The minimal 4-truncation of $BSO(3)$ is the 4-cell complex $(S^2 \cup_2 e_3) \cup_q e_4 \cup_{\frac{\eta q}{2}} e_5$, where $q: S^3 \ra S^2$ and $\frac{\eta q}{2}: S^4 \ra (S^2 \cup_2 e_3)$ are generators.
\end{corollary}

\begin{proof}
By the proposition, the map $P := BSO(2) \cup_{S^2} BOrp(3) \ra BSO(3)$ is a 4-equivalence.  Note that $_4 BSO(2) \cup_{S^2} {}_4 BOrp(3)$ is a 4-truncation of $P$ and therefore provides a 4-truncation of $BSO(3)$.  Using Proposition~\ref{prop-borptrunc} the cell structure of that space is as described.  

That this truncation is uniquely minimal is seen as follows.  The first four homology groups of $BSO(3)$ force any truncation to have at least one cell in each of the dimensions 2, 3, and 4.  If the truncation has only three cells, the attaching map of the 3-cell must be multiplication by 2, and the third homotopy group of $BSO(3)$ forces the attaching map of the 4-cell to be a generator; call the resulting complex $H := (S^2 \cup_2 e_3) \cup_q e_4$.  The complex $H$ cannot be a 4-truncation of $BSO(3)$ because $\pi_4(BSO(3)) = \ZZ$ whereas $\pi_4(H)$ contains a nontrivial torsion element, namely the image of $\frac{\eta q}{2}$ under the map $\pi_4(\Sigma \RP^2) \xra{\mathrm{inc}_*} \pi_4(H)$.  (This element is nontrivial because the composite $\ZZ/4 = \pi_4(\Sigma \RP^2) \xra{\mathrm{inc}_*} \pi_4(H) \ra \pi_4(S^3) = \ZZ/2$ is surjective, where the map $H \ra S^3$ kills the 2- and 4-cells.)

We therefore know that a minimal truncation must have exactly 4-cells.  If it has two 2-cells or two 3-cells, the homology in dimension 2 or 3 respectively will not match that of $BSO(3)$, preventing it from being a 4-truncation.  If the putative 4-cell truncation $T$ has two 4-cells, the homotopy exact sequence of the pair $(BSO(3), T)$ shows that $\pi_4(T) \ra \pi_4(BSO(3))$ cannot be an isomorphism.  A minimal truncation must therefore have a single cell in each of dimensions 2, 3, 4, and 5; the second and third homotopy groups of $BSO(3)$ force the 4-skeleton of this truncation to be $(S^2 \cup e_3) \cup_q e_4$ as before, and in order to reproduce $\pi_4(BSO(3))$, the 5-cell must be attached by plus or minus the image of the class $\frac{\eta q}{2}$ under the map $\pi_4(\Sigma \RP^2) \xra{\mathrm{inc}_*} \pi_4(H)$, as desired.  % Notice that the class \eta q / 2 cannot be divisible in  H because then it wouldn't map nontrivially onto \pi_4 S^3.  So it is nontrivial and you have to kill it, only way to do that is plus or minus the class itself.
\end{proof}

Because the fourth and fifth cohomology groups of $BQuad$ are trivial, the composite $BQuad \ra BSO(2) \ra BSO(3) \xra{p_1} K(\ZZ,4)$ is trivial and the map $BQuad \ra BSO(3)$ lifts uniquely to a map $BQuad \ra BOrp(3)$.

\begin{corollary}
The commutative square
\begin{equation} \label{quad-pushout} \tag{Q}
\xymatrix{
BQuad \ar[r] \ar[d] & BOrp(3) \ar[d] \\
BSO(2) \ar[r] & BSO(3)
}
\end{equation}
is also a $4$-pushout.
\end{corollary}
\begin{proof}
Though the map $B\Omega S^2 \ra BQuad$ is not a 4-equivalence, the difference of 4-types is a single additional 5-cell in $BQuad$, which results in an additional 6-cell of $BSO(2) \cup_{BQuad} BOrp(3)$ compared to $BSO(2) \cup_{B\Omega S^2} BOrp(3)$; that 6-cell does not affect the 4-type of the pushout.
\end{proof}

\nid Though the square~\eqref{s2-pushout} is more elementary, the square~\eqref{quad-pushout} encodes important conceptual information not visible in~\eqref{s2-pushout}.  Specifically, a cellular model of the pushout $BSO(2) \cup_{BQuad} BOrp(3)$ has, in addition to cells $e_2$, $e_3$, $e_4$, and $e_5$ described in Corollary~\ref{cor-bso3trunc}, a 6-cell $e_6$ attached as follows.  First note that the 5-cell of $BQuad$ maps by multiplication by $2$ to the 5-cell of $BOrp(3)$ and by the cone on $\eta$ to the 4-cell of $BSO(2)$.  % This is not at all clear, but is forced if you know 2 gamma = eta q.  Maybe rephrase in a less flippant way?
Now decompose the boundary $\partial e_6$ as the union $a_5 \cup (S^4 \times I) \cup b_5$, for 5-cells $a_5$ and $b_5$.  The cell $a_5$ wraps twice onto $e_5$, and the cell $b_5$ maps by the cone on $\eta: S^4 \ra S^3$ onto $e_4$, while the cylinder $S^4 \times I$ performs a homotopy between $2 \gamma : S^4 \ra \Sigma \RP^2$ and $\eta q: S^4 \ra S^2$ in $\Sigma \RP^2$.  In words, the 4-cell of $BSO(3)$ trivializes $q$ while the 5-cell trivializes $\gamma$, and the 6-cell witnesses the equivalence of $\eta$ times the trivialization of $q$ with $2$ times the trivialization of $\gamma$.

\begin{remark}
There is a commutative square (but not a 4-pushout),
\begin{equation}\label{quadmoore} \tag{M}
\xymatrix{
B\Omega S^2 \ar[r] \ar[d] & B\Omega \Sigma \RP^2 \ar[d] \\
BQuad \ar[r] & BOrp(3)
}
\end{equation}
% It isn't a 4-pushout because the pushout is Sigma RP2 with a 5-cell killing eta q, but you know eta q is twice the generator of pi_4 Sigma RP2, so there is pi_4 left in the pushout, which is no present in BOrp(3).
We saw that $B\Omega S^2$ and $BQuad$ were interchangable in the pushout square.  The space $BOrp(3)$ cannot be similarly replaced by $B\Omega \Sigma \RP^2$.  Nevertheless it will sometimes be useful to consider building an $Orp(3)$ action by first building an $\Omega \Sigma \RP^2$ action.
\end{remark}

\begin{remark}
We could have more briefly described a 4-truncation of $BSO(3)$ without reference to $BOrp(3)$, $BSO(2)$, $BQuad$, and whatnot.  But it will be important for our applications to topological field theories associated to fusion categories that we understand the 4-truncations of those structure groups as well, and the exact relationship of those truncations to that of $BSO(3)$.
\end{remark}

\CDcomm{Later on, we'll describe the meaning of a Quad action, etc, and then refer back to see that S leads to one description, while Q leads to a slightly lengthier description that ties together the trivialization of q and the condition on eta q / 2.}

\section{Homotopy actions}

\CDcomm{This section describes the meaning of a homotopy action for each group.  This is where the explicit translation to computable, i.e. categorically readable, expressions occurs.  Also, for each group, the map $G \ra Aut(X)$ is described. [Well, if you only describe that map for $SO(3)$, say so.]}

\begin{definition} \label{def-hoaction}
Given a space $B$ and a fibration $F \ra E \ra B$ with $E$ contractible, a homotopy action of $F$ on a space $X$ is a map $\alpha: B \ra B\Aut(X)$, where $\Aut(X)$ is the monoid of self-homotopy equivalences of $X$.
\end{definition}
\nid We will often refer to a homotopy action of $F$ simply as an ``action of $F$" or an ``$F$-action", as we will never be concerned with literal actions of groups.  We may also refer to this as an $F'$-action, when $F'$ is equipped with some implicit homotopy equivalence to $F$, or indeed when the whole fibration $F \ra E \ra B$ is itself implicit.  Moreover, we will use $G$ and $BG$ as generic symbols for the fibre $F$ and base $B$ of the fibration $F \ra E \ra B$ of definition~\ref{def-hoaction}, even though $G$ may not be a group and $BG$ may not be the classifying space of a group.

In case $G$ is an honest group with an honest action on $X$ given by $h: G \ra \Aut(X)$ (which is therefore forced to land in the subgroup of $\Aut(X)$ of homeomorphisms), then the corresponding homotopy action is the map $Bh: BG \ra B\Aut(X)$.


\subsection{By the groups $\Omega S^2$ and $\Quad$} \label{sec-quadaction}

\begin{proposition}
The space of $\Omega S^2$ actions on a 3-type $X$ is homotopy equivalent to $\Omega \Aut(X)$; in particular, the set of homotopy classes of actions is $\pi_1(\Aut(X))$.
\end{proposition}

\begin{proof}
By definition a homotopy action of $\Omega S^2$ on a 3-type $X$ is a map $B \Omega S^2 \ra B\Aut(X)$.  That map is a point of $\Omega^2 B\Aut(X) \simeq \Omega \Aut(X)$.
\end{proof}

\nid We typically use $S$ to denote a based map $S^1 \ra \Aut(X)$ determining the $\Omega S^2$ action on $X$, and we use $[S] \in \pi_1(\Aut(X))$ to denote the corresponding homotopy class.

\CDcomm{Um, what is a cellular 3-truncation of $\Omega \Sigma S^1$? ... $\Omega S^2 \ra \Aut(X)$. Omit??}

As mentioned, the map $S : S^1 \ra \Aut(X)$ can be viewed as a point $S \in \Omega \Aut(X)$.  The space $\Omega \Aut(X)$ is an $E_2$-space.  Letting $\ast$ denote automorphism composition and $\cdot$ denote loop composition, there is, for any two points $a, b \in \Omega \Aut(X)$ a well-defined-up-to-homotopy path $\beta_{a,b}$ from $a \cdot b$ to $b \cdot a$ implementing a half (righthanded) braiding.  Using this braiding, we may form the following loop in $\Omega \Aut(X)$:
\begin{equation} \label{eq-q}
q(S) := (I \ra S \cdot S^{-1} \xra{\beta_{S,S^{-1}}} S^{-1} \cdot S \ra I) \in \Omega (\Omega \Aut(X))
\end{equation}
Here $I$ denotes the trivial loop at the identity automorphism.  The first map of that composite is the standard loop-retraction null-homotopy of the composite of the loop $S \in \Omega \Aut(X)$ with its reverse loop $S^{-1}$, and the last map is similar for the composite of $S^{-1}$ with $S$.  Altogether, this construction provides a map $\pi_1(\Aut(X)) \xra{q} \pi_2(\Aut(X))$, and in fact a map $\Omega \Aut(X) \xra{q} \Omega^2 \Aut(X)$.  We will refer to this as the figure-8 construction, as it imitates the following normally framed embedding of $S^1$ in $\RR^3$:
\begin{equation} \nn
\begin{tikzpicture}
\draw[linestyle,looseness=1.25]
(-.75,0) to [out=-90,in=-120] (0,0)
	to [out=60,in=90] (.75,0);
\draw[linestylegray,coverline,looseness=1.25]
(.75,0) to [out=-90,in=-60] (0,0)
	to [out=120,in=90] (-.75,0);
\end{tikzpicture}
\end{equation} 
%\begin{tikzpicture}
%\draw[linestyle,fuzzleft,looseness=1.25]
%(-.75,0) to [out=-90,in=-120] (0,0)
%	to [out=60,in=90] (.75,0);
%\draw[linestyle,coverlineleft,fuzzleft,looseness=1.25]
%(.75,0) to [out=-90,in=-60] (0,0)
%	to [out=120,in=90] (-.75,0);
%\end{tikzpicture}
%\end{equation} 
%\CD{I do not know how to fix the tikz issue here --- the pre function and the raise function appear to be incompatible.}
% Note: the vertical direction in this picture is the loop direction, and the automorphism direction is into the page, and left right is the second loop direction.
%The gray fuzz indicates the first normal frame vector, and the second normal frame vector points uniformly out of the page.
The normal framing is implicitly determined as follows: the first normal vector is approximately in the plane of the page, pointing inward on the left side of the figure and outward on the right side, and the second normal vector points uniformly out of the page.  (Here the black line segment indicates the portion that can be interpreted as $S \in \Omega \Aut(X)$ and the gray line segment indicates the portion that can be interpreted as $S^{-1}$.)


\begin{lemma} \label{lemma-q}
The figure-8 constuction~\eqref{eq-q} represents precomposition with a generating element $q: S^3 \ra S^2$ of $\pi_3(S^2)$.  Specifically, the following square commutes up to homotopy, where ``$q$" refers to the above construction, and ``$q^\ast$" refers to pullback along the element $q$:
\begin{equation} \nn
\xymatrix{
\Omega^2 B\Aut(X) \ar[r]^{q^\ast} \ar@{<->}[d]_{\simeq} & \Omega^3 B\Aut(X)  \ar@{<->}[d]^{\simeq} \\
\Omega \Aut(X) \ar[r]_{q} & \Omega^2 \Aut(X)
}
\end{equation}
\end{lemma}

\begin{proof}
In fact, the square commutes with any space $Y$ in place of $B\Aut(X)$ and correspondingly $\Omega Y$ in place of $\Aut(X)$.  Because both the constructions $q$ and $q^\ast$ are natural, to prove the general case, it suffices to check in the universal case of $Y = S^2$ on the universal point $(S^2 \xra{\id} S^2) \in \Omega^2 S^2$.  In that case, construction~\eqref{eq-q} is the Pontryagin-Thom image of the normally framed embedding of $S^1$ into $\RR^3$ depicted above; in words, take the normally framed figure-8 immersion of $S^1$ into $\RR^2$, compose with the inclusion $\RR^2 \ra \RR^3$ equipped with the trivial normal framing, and deform the composite to an embedding (having a righthanded braid).  That normally framed embedding does indeed represent a generator of $\pi_3(S^2)$ as required.
%NB Though this might look like it only checks on $\pi_0$, this does provide a proof of the commutativity of the square.  Specifically, you choose a single universal homotopy in the universal case, and then the image of that homotopy under a map \Omega^2 S^2 \ra \Omega^2 Y is a continuously varying homotopy showing the diagram commutes.
\end{proof}

We may play the same game again.  We have $q(S) \in \Omega^2 \Aut(X)$.  Though it does not matter particularly, for definiteness let $\cdot$ as before denote the first loop composition, and $\circ$ now denote the second loop composite.  The space $\Omega^2 \Aut(X)$ is an $E_3$-space and therefore has a path $\tau_{a,b}$ from $a \circ b$ to $b \circ a$ implementing the symmetric switch.  For any point $A \in \Omega^2 \Aut(X)$ we may form the following composite
\begin{equation} \label{eq-eta}
\eta(A) := (I \ra A \circ A^{-1} \xra{\tau_{A,A^{-1}}} A^{-1} \circ A \ra I) \in \Omega(\Omega^2 \Aut(X))
\end{equation}
Here $I$ now denotes the trivial double loop at the identity automorphism.  This construction provides a map $\pi_2(\Aut(X)) \xra{\eta} \pi_3(\Aut(X))$, in fact a map $\Omega^2 \Aut(X) \xra{\eta} \Omega^3 \Aut(X)$, which may also be called the figure-8 construction.

\begin{lemma} \label{lemma-eta}
The figure-8 construction~\eqref{eq-eta} represents precomposition with a nontrivial element $\eta: S^4 \ra S^3$ of $\pi_4(S^3)$.  More precisely, the following square commutes up to homotopy, where ``$\eta$" refers to the above construction, and ``$\eta^\ast$" refers to pullback along the element $\eta$:
\begin{equation} \nn
\xymatrix{
\Omega^3 B\Aut(X) \ar[r]^{\eta^\ast} \ar@{<->}[d]_{\simeq} & \Omega^4 B\Aut(X)  \ar@{<->}[d]^{\simeq} \\
\Omega^2 \Aut(X) \ar[r]_{\eta} & \Omega^3 \Aut(X)
}
\end{equation}
\end{lemma}

\begin{proof}
As for the previous lemma, it suffices to check for the universal space, here $S^3$ in place of $B\Aut(X)$, and on the universal point, here $(S^3 \xra{\id} S^3) \in \Omega^3 S^3$.  In that case, construction~\eqref{eq-eta} is the Pontryagin-Thom image of the following normally framed embedding of $S^1$ into $\RR^4$: take the normally framed embedding of $S^1$ into $\RR^3$ from the proof of Lemma~\ref{lemma-q} and include it into $\RR^4$, adding a trivial normal framing direction.  This normally framed manifold represents a generator of $\pi_4(S^3)$ as required.
\end{proof}

We can now describe homotopy $Quad$ actions.

\begin{proposition} \label{prop-quadaction}
The space of $Quad$ actions on a 3-type $X$ is homotopy equivalent to the union of the components of $\Omega \Aut(X)$ corresponding to elements $[S] \in \pi_1(\Aut(X))$ such that $\eta(q([S])) = 0 \in \pi_3(\Aut(X))$, that is such that the double figure-8 construction on $[S]$ vanishes.
\end{proposition}
\begin{proof}
By Proposition~\ref{prop-quadtrunc}, the attaching map of the 5-cell of $BQuad$ is $\eta q: S^4 \ra S^2$, where $\eta : S^4 \ra S^3$ is nontrivial and $q: S^3 \ra S^2$ is a generator.  It suffices therefore to know that the double figure-8 construction $\pi_1(\Aut(X)) \xra{q \eta} \pi_3(\Aut(X))$ is equal to the pullback $\pi_2(B\Aut(X)) \xra{(\eta q)^\ast} \pi_4(B\Aut(X))$ once $\pi_{k}(\Aut(X))$ and $\pi_{k+1}(B\Aut(X))$ are identified.  This follows from Lemmas~\ref{lemma-q} and~\ref{lemma-eta}.
\end{proof}


\CDcomm{What is a cellular 3-truncation of $Quad$? ... $Quad \ra \Aut(X)$.  Maybe omit.}

\subsection{By the group $SO(2)$}

\begin{prop} \label{prop-so2action}
The homotopy classes of $SO(2)$ actions on a 3-type $X$ are the homotopy classes of pairs $(S,W)$, where $S: S^1 \ra \Aut(X)$ is a map and $W$ is a null homotopy of $q(S): S^2 \ra \Aut(X)$, the figure-8 construction applied to $S$.  More precisely, the space of $SO(2)$ actions on $X$ is homotopy equivalent to the homotopy pullback $\Omega \Aut(X) \times_{\Omega^2 \Aut(X)} \ast$, where the map is $\Omega \Aut(X) \xra{q} \Omega^2 \Aut(X)$.
\end{prop}
% The construction of q really does provide a map from \Omega \Aut(X) to \Omega^2 \Aut(X) so the statement of the proposition makes sense.
\begin{proof}
As mentioned, the 4-truncation of $BSO(2)$ is $S^2 \cup_q e_4$.  An $SO(2)$ action is therefore a map $S^2 \cup_q e_4 \ra B\Aut(X)$.  On the 2-cell, this map is provided by $S \in \Omega \Aut(X) \simeq \Omega^2 B\Aut(X)$.  An extension of the map $S^2 \xra{S} B\Aut(X)$ over the 4-cell is a null homotopy of the composite $S^3 \xra{q} S^2 \xra{S} B\Aut(X)$, that is a path in $\Omega^3 B\Aut(X)$ from $q^\ast S$ to the identity.  By Lemma~\ref{lemma-q}, providing such a null homotopy is equivalent to providing a null homotopy of $q(S) \in \Omega^2\Aut(X) \simeq \Omega^3 B\Aut(X)$.
\end{proof}

We can restrict attention to 3-types that happen to be 2-types.
\begin{corollary}
The homotopy classes of $SO(2)$ actions on a 2-type $X$ are the homotopy classes of map $S: S^1 \ra \Aut(X)$ such that the figure-8 construction applied to $S$ is null.
\end{corollary}

\subsection{By the groups $\Omega \Sigma \RP^2$ and $Orp(3)$}

Given a loop $S: S^1 \ra \Aut(X)$ of automorphisms of $X$, we can form $S \ast S : S^1 \ra \Aut(X)$ by sending each point $p \in S^1$ to the square $S(p) \ast S(p)$ of the automorphism $S(p) \in \Aut(X)$, or we can form $S \cdot S : S^1 \ra \Aut(X)$ by taking the loop composition of $S$ with itself.  Of course, $S \ast S$ and $S \cdot S$ are homotopic maps $S^1 \ra \Aut(X)$, but there is no canonical homotopy between them so it will occasionally be important to distinguish them.  We will use the expression ``$2S$" as shorthand for $S \ast S$.
\CDcomm{What is the group of self-homotopy equivalences of $\Sigma \RP^2$?  If it has more than just reflections, then there is a subtlety in the identification of $\Omega \Sigma \RP^2$ actions that we should address/comment on.}

\begin{prop} \label{prop-sigmarp2action}
The homotopy classes of $\Omega \Sigma \RP^2$ actions on a 3-type $X$ are the homotopy classes of pairs $(S,R)$, where $S: S^1 \ra \Aut(X)$ is a map and $R$ is a null homotopy of $2S: S^1 \ra \Aut(X)$.  More precisely, the space of actions is homotopy equivalent to the homotopy pullback $\Omega \Aut(X) \,_2\!\times_{\Omega \Aut(X)} \ast$, where the subscript $2$ indicates that map is $\Omega \Aut(X) \xra{2} \Omega \Aut(X)$.
\end{prop}
% For this to not involve a choice, you need the attaching map of \Sigma \RP^2 to have been doubled in the automorphism direction, that is in the "extra" loop direction.  If it had been doubled in one of the other directions, then there is a left or right choice in the identification of the proposition.  Weird. (??)  Actually, if there aren't any interesting self-equivalences of $\Sigma \RP^2$ then there isn't actually an issue.
\begin{proof}
This is immediate, as multiplication by $2$ commutes through the identification $\Omega \Aut(X) \simeq \Omega^2 B\Aut(X)$.
\end{proof}

Recall from Proposition~\ref{prop-borptrunc} that the 4-truncation of $BOrp(3)$ is $(S^2 \cup_2 e_3) \cup_{\frac{\eta q}{2}} e_5$.  From Proposition~\ref{prop-sigmarp2action}, we know that a $\Sigma \RP^2 = S^2 \cup_2 e_3$ action is a point of $\Map(\Sigma \RP^2,B\Aut(X)) \simeq \Omega\Aut(X) \,_2\! \times_{\Omega \Aut(X)} \ast$.  To precisely describe a full $Orp(3)$ action, we need to provide a geometric construction of a map $\Omega\Aut(X) \,_2\! \times_{\Omega \Aut(X)} \ast \ra \Omega^3 \Aut(X)$ such that the composite $\Map(\Sigma \RP^2,B\Aut(X)) \simeq \Omega\Aut(X) \,_2\! \times_{\Omega \Aut(X)} \ast \ra \Omega^3 \Aut(X) \simeq \Omega^4 B\Aut(X)$ is pullback along $\frac{\eta q}{2} \in \pi_4(\Sigma\RP^2)$.  

We first revisit the corresponding construction for $\eta q \in \pi_4(S^2)$.  Given an element $S \in \Omega \Aut(X)$, recall from Section~\ref{sec-quadaction} that we can read the standard normally framed figure 8 embedding of $S^1$ in $\RR^3$ by performing the element $S$ in a transverse 2-plane slice along the curve, to obtain construction~\eqref{eq-q}, a map $\Omega \Aut(X) \xra{q} \Omega^2 \Aut(X)$.  We can similarly read the following normally framed embedding of $S^1 \times S^1$ into $\RR^4$ as representing a map $\Omega \Aut(X) \xra{\eta q} \Omega^2 \Aut(X)$:
\begin{equation} \label{eq-etaq}
\emptyset 
\leadsto
\cb{\begin{tikzpicture}
\draw[linestyle,looseness=1.25]
(0,0) to [out=-90,in=-90] (.5,0);
\draw[linestylegray,looseness=1.25]
(.5,0) to [out=90,in=90] (0,0);
\end{tikzpicture}}
\leadsto
\cb{\begin{tikzpicture}
\draw[linestyle,looseness=1.25]
(0,0) to [out=-90,in=-120] (.5,0)
	to [out=60,in=120] (1,0)
	to [out=-60,in=-90] (1.5,0);
\draw[linestylegray,coverline,looseness=1.25]
(1.5,0) to [out=90,in=60] (1,0)
	to [out=-120,in=-60] (.5,0)
	to [out=120,in=90] (0,0);
\end{tikzpicture}}
\leadsto
\cb{\begin{tikzpicture}
\draw[linestyle,looseness=1.25]
(0,0) to [out=-90,in=-120] (.5,0)
	to [out=60,in=90] (1,0);
\draw[linestylegray,coverline,looseness=1.25]
(1,0) to [out=-90,in=-60] (.5,0)
	to [out=120,in=90] (0,0);
\draw[linestyle,looseness=1.25]
(2.25,0) to [out=-90,in=-60] (1.75,0)
	to [out=120,in=90] (1.25,0);
\draw[linestylegray,coverline,looseness=1.25]
(1.25,0) to [out=-90,in=-120] (1.75,0)
	to [out=60,in=90] (2.25,0);
\end{tikzpicture}}
\stackrel{\lcurvearrowse}{\leadsto}
\cb{\begin{tikzpicture}
\draw[linestyle,looseness=1.25]
(0,0) to [out=-90,in=-120] (.5,0)
	to [out=60,in=90] (1,0);
\draw[linestylegray,coverline,looseness=1.25]
(1,0) to [out=-90,in=-60] (.5,0)
	to [out=120,in=90] (0,0);
\draw[linestyle,looseness=1.25]
(1,-.5) to [out=-90,in=-60] (.5,-.5)
	to [out=120,in=90] (0,-.5);
\draw[linestylegray,coverline,looseness=1.25]
(0,-.5) to [out=-90,in=-120] (.5,-.5)
	to [out=60,in=90] (1,-.5);
\end{tikzpicture}}
\stackrel{\lcurvearrowsw}{\leadsto}
\cb{\begin{tikzpicture}
\draw[linestyle,looseness=1.25]
(1,0) to [out=-90,in=-60] (.5,0)
	to [out=120,in=90] (0,0);
\draw[linestylegray,coverline,looseness=1.25]
(0,0) to [out=-90,in=-120] (.5,0)
	to [out=60,in=90] (1,0);
\draw[linestyle,looseness=1.25]
(1.25,0) to [out=-90,in=-120] (1.75,0)
	to [out=60,in=90] (2.25,0);
\draw[linestylegray,coverline,looseness=1.25]
(2.25,0) to [out=-90,in=-60] (1.75,0)
	to [out=120,in=90] (1.25,0);
\end{tikzpicture}}
\leadsto
\cb{\begin{tikzpicture}
\draw[linestyle,looseness=1.25]
(1.5,0) to [out=90,in=60] (1,0)
	to [out=-120,in=-60] (.5,0)
	to [out=120,in=90] (0,0);
\draw[linestylegray,coverline,looseness=1.25]
(0,0) to [out=-90,in=-120] (.5,0)
	to [out=60,in=120] (1,0)
	to [out=-60,in=-90] (1.5,0);
\end{tikzpicture}}
\leadsto
\cb{\begin{tikzpicture}
\draw[linestylegray,looseness=1.25]
(0,0) to [out=-90,in=-90] (.5,0);
\draw[linestyle,looseness=1.25]
(.5,0) to [out=90,in=90] (0,0);
\end{tikzpicture}}
\leadsto
\emptyset
\end{equation} 
Here the normal framing is implicit and as before is determined as follows: the first nontrivial picture on the left has first normal vector pointing toward the center of the circle and second normal vector pointing out of the page; for the rest of the pictures, the first normal is determined by planar deformation from the first picture, and the second normal continues to point out of the page.  Also as before, the black line segments indicate the element $S \in \Omega \Aut(X)$, and the gray line segments indicate the element $S^{-1}$, where the inverse is taken in the loop direction.  That the map $\Omega \Aut(X) \xra{\eta q} \Omega^3 \Aut(X)$ produced by the above picture is given by pullback along $\eta q \in \pi_4(S^2)$ is the content of Lemmas~\ref{lemma-q} and ~\ref{lemma-eta}.

Given a point of $\Omega\Aut(X) \,_2\! \times_{\Omega \Aut(X)} \ast$, we have both a point $S \in \Omega\Aut(X)$ and a null homotopy $R$ of $2S$.  We represent this null homotopy and its inverse graphically by 
$\cb{
\begin{tikzpicture}
\draw[linestyle,looseness=1.75] (0,.15) to [out=0,in=0] (0,-.15);
\end{tikzpicture}
}$
and
$\cb{
\begin{tikzpicture}
\draw[linestyle,looseness=1.75] (0,.15) to [out=180,in=180] (0,-.15);
\end{tikzpicture}
}$
respectively.  Combining this notation with the previous notation of a normally framed surface, the following picture encodes a map $\Omega\Aut(X) \,_2\! \times_{\Omega \Aut(X)} \ast \xra{\frac{\eta q}{2}} \Omega^3 \Aut(X)$:
\begin{equation} \label{eq-etaq2}
\emptyset 
\leadsto
\cb{\begin{tikzpicture}
\draw[linestyle,looseness=1.25]
(0,0) to [out=-90,in=-90] (.5,0);
\draw[linestylegray,looseness=1.25]
(.5,0) to [out=90,in=90] (0,0);
\end{tikzpicture}}
\leadsto
\cb{\begin{tikzpicture}
\draw[linestyle,looseness=1.25]
(0,0) to [out=-90,in=-120] (.5,0)
	to [out=60,in=120] (1,0)
	to [out=-60,in=-90] (1.5,0);
\draw[linestylegray,coverline,looseness=1.25]
(1.5,0) to [out=90,in=60] (1,0)
	to [out=-120,in=-60] (.5,0)
	to [out=120,in=90] (0,0);
\end{tikzpicture}}
\leadsto
\cb{\begin{tikzpicture}
\draw[linestyle,looseness=1.25]
(0,0) to [out=-90,in=-120] (.5,0)
	to [out=60,in=90] (1,0);
\draw[linestylegray,coverline,looseness=1.25]
(1,0) to [out=-90,in=-60] (.5,0)
	to [out=120,in=90] (0,0);
\draw[linestyle,looseness=1.25]
(2.25,0) to [out=-90,in=-60] (1.75,0)
	to [out=120,in=90] (1.25,0);
\draw[linestylegray,coverline,looseness=1.25]
(1.25,0) to [out=-90,in=-120] (1.75,0)
	to [out=60,in=90] (2.25,0);
\end{tikzpicture}}
\stackrel{\lcurvearrowse}{\leadsto}
\cb{\begin{tikzpicture}
\draw[linestyle,looseness=1.25]
(0,0) to [out=-90,in=-120] (.5,0)
	to [out=60,in=90] (1,0);
\draw[linestylegray,coverline,looseness=1.25]
(1,0) to [out=-90,in=-60] (.5,0)
	to [out=120,in=90] (0,0);
\draw[linestyle,looseness=1.25]
(1,-.5) to [out=-90,in=-60] (.5,-.5)
	to [out=120,in=90] (0,-.5);
\draw[linestylegray,coverline,looseness=1.25]
(0,-.5) to [out=-90,in=-120] (.5,-.5)
	to [out=60,in=90] (1,-.5);
\end{tikzpicture}}
\leadsto
\cb{\begin{tikzpicture}
\draw[linestyle,looseness=1.25]
(0,0) to [out=-90,in=90] (.15,-.25)
	to [out=-90,in=90] (0,-.5);
\draw[linestyle,looseness=1.25]
(1,0) to [out=90,in=60] (.5,0)
	to [out=-120,in=90] (.3,-.25)
	to [out=-90,in=120] (.5,-.5)
	to [out=-60,in=-90] (1,-.5);
\draw[linestylegray,coverline,looseness=1.25]
(0,0) to [out=90,in=120] (.5,0)
	to [out=-60,in=-90] (1,0);
\draw[linestylegray,coverline,looseness=1.25]
(0,-.5) to [out=-90,in=-120] (.5,-.5)
	to [out=60,in=90] (1,-.5);
\end{tikzpicture}}
\leadsto
\cb{\begin{tikzpicture}
\draw[linestyle,looseness=1.25]
(.1,0) to [out=90,in=90] (.75,.25);
\draw[linestyle,looseness=1.25]
(.1,0) to [out=-90,in=-90] (.75,-.25);
\draw[linestylegray,looseness=1.75]
(.75,.25) to [out=-90,in=90] (.4,.15);
\draw[linestylegray,looseness=1.75]
(.75,-.25) to [out=90,in=-90] (.4,-.15);
\draw[linestyle,looseness=1.25]
(.4,.15) to [out=-90,in=90] (1,0)
	to [out=-90,in=90] (.4,-.15);
\end{tikzpicture}}
\leadsto
\cb{\begin{tikzpicture}
\draw[linestyle,looseness=1.25]
(0,0) to [out=-90,in=-90] (.5,0);
\draw[linestyle,looseness=1.25]
(.5,0) to [out=90,in=90] (0,0);
\end{tikzpicture}}
\leadsto
\emptyset
\end{equation}
\nid We refer to this as the figure-32 construction.

\begin{lemma} \label{lemma-etaq2}
The figure-32 construction~\eqref{eq-etaq2} represents precomposition with a generator $\frac{\eta q}{2}: S^4 \ra \Sigma \RP^2$ of $\pi_4(\Sigma \RP^2)$.  More precisely, the following square commutes up to homotopy, where $\frac{\eta q}{2}$ refers to the above construction, and $(\frac{\eta q}{2})^\ast$ refers to pullback along the element $\frac{\eta q}{2}$:
\begin{equation} \nn
\xymatrix{
\Map(\Sigma \RP^2,B\Aut(X)) \ar[r]^-{(\frac{\eta q}{2})^\ast} \ar@{<->}[d]_{\simeq} & \Omega^4 B\Aut(X)  \ar@{<->}[d]^{\simeq} \\
\Omega\Aut(X) \,_2\! \times_{\Omega \Aut(X)} \ast \ar[r]_-{\frac{\eta q}{2}} & \Omega^3 \Aut(X)
}
\end{equation}
\end{lemma}
\begin{proof}
By Lemmas~\ref{lemma-q} and~\ref{lemma-eta}, we know that construction~\eqref{eq-etaq} represents precomposition with $\eta q : S^4 \ra S^2 \xra{\text{inc}} \Sigma\RP^2$.  Observe that construction~\eqref{eq-etaq} is homotopic to the sum of construction~\eqref{eq-etaq2} with the following construction:
\begin{equation} \label{eq-etaq2end}
\emptyset 
\leadsto
\cb{\begin{tikzpicture}
\draw[linestyle,looseness=1.25]
(0,0) to [out=-90,in=-90] (.5,0);
\draw[linestyle,looseness=1.25]
(.5,0) to [out=90,in=90] (0,0);
\end{tikzpicture}}
\leadsto
\cb{\begin{tikzpicture}
\draw[linestyle,looseness=1.25]
(.1,0) to [out=90,in=90] (.75,.25);
\draw[linestyle,looseness=1.25]
(.1,0) to [out=-90,in=-90] (.75,-.25);
\draw[linestylegray,looseness=1.75]
(.75,.25) to [out=-90,in=90] (.4,.15);
\draw[linestylegray,looseness=1.75]
(.75,-.25) to [out=90,in=-90] (.4,-.15);
\draw[linestyle,looseness=1.25]
(.4,.15) to [out=-90,in=90] (1,0)
	to [out=-90,in=90] (.4,-.15);
\end{tikzpicture}}
\leadsto
\cb{\begin{tikzpicture}
\draw[linestyle,looseness=1.25]
(0,0) to [out=-90,in=90] (.15,-.25)
	to [out=-90,in=90] (0,-.5);
\draw[linestyle,looseness=1.25]
(1,0) to [out=90,in=60] (.5,0)
	to [out=-120,in=90] (.3,-.25)
	to [out=-90,in=120] (.5,-.5)
	to [out=-60,in=-90] (1,-.5);
\draw[linestylegray,coverline,looseness=1.25]
(0,0) to [out=90,in=120] (.5,0)
	to [out=-60,in=-90] (1,0);
\draw[linestylegray,coverline,looseness=1.25]
(0,-.5) to [out=-90,in=-120] (.5,-.5)
	to [out=60,in=90] (1,-.5);
\end{tikzpicture}}
\leadsto
\cb{\begin{tikzpicture}
\draw[linestyle,looseness=1.25]
(0,0) to [out=-90,in=-120] (.5,0)
	to [out=60,in=90] (1,0);
\draw[linestylegray,coverline,looseness=1.25]
(1,0) to [out=-90,in=-60] (.5,0)
	to [out=120,in=90] (0,0);
\draw[linestyle,looseness=1.25]
(1,-.5) to [out=-90,in=-60] (.5,-.5)
	to [out=120,in=90] (0,-.5);
\draw[linestylegray,coverline,looseness=1.25]
(0,-.5) to [out=-90,in=-120] (.5,-.5)
	to [out=60,in=90] (1,-.5);
\end{tikzpicture}}
\stackrel{\lcurvearrowsw}{\leadsto}
\cb{\begin{tikzpicture}
\draw[linestyle,looseness=1.25]
(1,0) to [out=-90,in=-60] (.5,0)
	to [out=120,in=90] (0,0);
\draw[linestylegray,coverline,looseness=1.25]
(0,0) to [out=-90,in=-120] (.5,0)
	to [out=60,in=90] (1,0);
\draw[linestyle,looseness=1.25]
(1.25,0) to [out=-90,in=-120] (1.75,0)
	to [out=60,in=90] (2.25,0);
\draw[linestylegray,coverline,looseness=1.25]
(2.25,0) to [out=-90,in=-60] (1.75,0)
	to [out=120,in=90] (1.25,0);
\end{tikzpicture}}
\leadsto
\cb{\begin{tikzpicture}
\draw[linestyle,looseness=1.25]
(1.5,0) to [out=90,in=60] (1,0)
	to [out=-120,in=-60] (.5,0)
	to [out=120,in=90] (0,0);
\draw[linestylegray,coverline,looseness=1.25]
(0,0) to [out=-90,in=-120] (.5,0)
	to [out=60,in=120] (1,0)
	to [out=-60,in=-90] (1.5,0);
\end{tikzpicture}}
\leadsto
\cb{\begin{tikzpicture}
\draw[linestylegray,looseness=1.25]
(0,0) to [out=-90,in=-90] (.5,0);
\draw[linestyle,looseness=1.25]
(.5,0) to [out=90,in=90] (0,0);
\end{tikzpicture}}
\leadsto
\emptyset
\end{equation} 
Next note that construction~\eqref{eq-etaq2} and the time reversal of construction~\eqref{eq-etaq2end} are vertical reflections of one another; it follows that the sum of \eqref{eq-etaq2} and $-\eqref{eq-etaq2end}$ is null, or said differently, that \eqref{eq-etaq2} and \eqref{eq-etaq2end} are homotopic.  Thus construction~\eqref{eq-etaq2} is a 2-divisor of construction~\eqref{eq-etaq}, and, because $\pi_4(\Sigma \RP^2) = \ZZ/4$, therefore represents $(\frac{\eta q}{2})^\ast$, up to sign.  As we will only be concerned with the vanishing of construction~\eqref{eq-etaq2}, the sign will not matter for us---we leave its determination to the interested reader.
\end{proof}
\CDcomm{[This last sentence about sign doesn't make sense unless we have specified a chosen element $\frac{\eta q}{2}$ of the homotopy group.  Really, Proposition 2.5 should specify the generator we use, not just up to sign, then at least the rest is precise.]  [If we don't have a good way to specify, we can take the figure 32 construction to specify the sign, but then the proposition statements need adjusting.]}

\begin{proposition} \label{prop-orpaction}
The homotopy classes of $Orp(3)$ actions on a 3-type $X$ are the homotopy classes of pairs $(S,R)$, where $S: S^1 \ra \Aut(X)$ is a map and $R$ is a null homotopy of $2S$, such that the figure-32 construction on $(S,R)$ is zero.  More precisely, the space of such actions is homotopy equivalent to the union of components of $\Omega \Aut(X) \,_2\! \times_{\Omega \Aut(X)} \ast$ on which the figure-32 construction vanishes.
\end{proposition}
\begin{proof}
This follows by combining Proposition~\ref{prop-borptrunc} and Lemma~\ref{lemma-etaq2}.
\end{proof}

\CDcomm{A cellular 3-truncation of $Orp(3)$ looks a little messy (cohom is $\ZZ, 0, \ZZ/2, \ZZ/2, \ZZ/2, \ZZ/4$ I believe.  Maybe omit.}

\subsection{By the group $SO(3)$}

\begin{theorem} \label{thm-so3action}
The homotopy classes of $SO(3)$ actions on a 3-type $X$ are the homotopy classes of triples $(S,R,W)$, where 
\begin{itemize}
\item $S: S^1 \ra \Aut(X)$ is a based loop of automorphisms of $X$,
\item $R: D^2 \ra \Aut(X)$ is a null homotopy of $S \ast S: S^1 \ra \Aut(X)$, the pointwise (that is, automorphism) square of $S$,
\item $W: D^3 \ra \Aut(X)$ is a null homotopy of $q(S): S^2 \ra \Aut(X)$, the figure-8 construction~\eqref{eq-q} applied to $S$,
\end{itemize}
such that
\begin{itemize}
\item $\frac{\eta q}{2} (S,R) : S^3 \ra \Aut(X)$, that is the figure-32 construction~\eqref{eq-etaq2} applied to $(S,R)$, is null homotopic.
\end{itemize}
\end{theorem}
\nid By this point needless to say, the space of $SO(3)$ is homotopy equivalent to the space
\begin{equation} \nn
\ast \,_{\Omega^2 \Aut(X)}\! \times_q \left(\Omega \Aut(X) \,_2\! \times_{\Omega \Aut(X)} \ast\right)_{[\frac{\eta q}{2}]=0},
\end{equation}
where the maps from $\Omega \Aut(X)$ are $q$ and $2$ respectively, and the subscript $[\frac{\eta q}{2}]=0$ indicates that the only components where the figure-32 construction vanishes are retained.
\begin{proof}
This follows by combining Corollary~\ref{cor-bso3trunc}, Lemma~\ref{lemma-q}, and Lemma~\ref{lemma-etaq2}.
\end{proof}

\subsection{The map $SO(3) \ra \Aut(X)$}

Given an $SO(3)$ action on $X$, that is a map $\alpha: BSO(3) \ra B\Aut(X)$, we have an associated map of spaces $\Omega \alpha: SO(3) \ra \Aut(X)$.  Roughly speaking, the map $\Omega \alpha$ knows everything about the action, except that it is in fact an action, i.e. a map of (homotopical) groups.  Given a point $p \in X$, we have a map $\Omega \alpha (p) : SO(3) \ra X$; it is sometimes informative to compute this map explicitly, and to that end convenient to know the map $\Omega \alpha$.  

A cellular 3-truncation of $SO(3)$ is, of course, $(S^1 \cup_2 e_2) \cup_p e_3$, where $p: S^2 \ra \RP^2$ is the antipodal quotient.  Given a map $\RP^2 \ra \Aut(X)$, we need to explicitly construct the pullback $S^2 \xra{p} \RP^2 \ra \Aut(X)$.  The map from $\RP^2$ provides a point $S \in \Omega \Aut(X)$ and a null homotopy $R$ of $2S$.  Using the previously described graphical notation for $S$ and $R$, the following picture defines a map $\Map(\RP^2,\Aut(X)) \xra{p} \Omega^2 \Aut(X)$:
\begin{equation} \label{eq-p}
\cb{\begin{tikzpicture}
\draw[linestyle,looseness=1.25]
(0,0) to [out=-90,in=-120] (.75,0)
	to [out=60,in=90] (1.5,0);
\draw[linestyle,coverline,looseness=1.25]
(1.5,0) to [out=-90,in=-60] (.75,0)
	to [out=120,in=90] (0,0);
\end{tikzpicture}}
\end{equation}
This map $p$ is homotopic to the map given by pullback along the antipodal projection $S^2 \ra \RP^2$, as desired.\CD{This isn't just obvious, I suppose.}  Theorem~\ref{thm-so3action} provides a null homotopy of construction~\eqref{eq-q} applied to $S$, whereas we seem to need a null homotopy of construction~\eqref{eq-p} applied to $S$ and $R$.  Though construction~\eqref{eq-p} is a priori rather different from the map $\Omega\Aut(X) \xra{q} \Omega^2 \Aut(X)$ defined by construction~\eqref{eq-q}, we will see that they are, in fact, closely related.

Note that the following square commutes up to homotopy, where ``$p$" refers to the above construction~\eqref{eq-p} and ``$(\Sigma p)^\ast$" refers to pullback along the suspension of the antipodal projection $p: S^2 \ra \RP^2$:
\begin{equation} \nn
\xymatrix{
\Map(\Sigma\RP^2,B\Aut(X)) \ar[r]^-{(\Sigma p)^\ast} \ar@{<->}[d]_{\simeq} & \Omega^3 B\Aut(X)  \ar@{<->}[d]^{\simeq} \\
\Map(\RP^2,\Aut(X)) \ar[r]_-{p} & \Omega^2 \Aut(X)
}
\end{equation}
That is, construction~\eqref{eq-p} represents pullback along $\Sigma p: S^3 \ra \Sigma \RP^2$ in the same sense that construction~\eqref{eq-q} represents pullback along $q: S^3 \ra S^2$.  Comparing the homotopy classes $\Sigma p \in \pi_3(\Sigma \RP^2)$ and $\inc_* q \in \pi_3(\Sigma \RP^2)$, for $\inc: S^2 \ra \Sigma \RP^2$, is therefore equivalent to comparing construction~\eqref{eq-p} and construction~\eqref{eq-q}.  

We will need the following lemma:
\begin{lemma}
The map $S^3 \ra \Sigma \RP^2$ defined by the normally framed manifold
\begin{equation} \label{eq-bean}
\cb{\begin{tikzpicture}
\draw[linestyle,out looseness=.75,in looseness=1]
	(-.5,-.25) to [out=-90,in=-90] (.5,0)
	(-.5,.25) to [out=90,in=90] (.5,0);
\draw[linestyle,looseness=.75]
	(-.5,-.25) to [out=90,in=-90] (0,0)
	(-.5,.25) to [out=-90,in=90] (0,0);
\end{tikzpicture}}
\end{equation}
represents the unique nontrivial 2-torsion class in $\pi_3(\Sigma \RP^2) = \ZZ/4$.
\end{lemma}
\nid We refer to the construction defined in this lemma as ``the Radford bean".  Note well that in this construction the left and right facing cups are applications of the nullhomotopy $R$ of $2S$, not applications of a cancellation between $S$ and $S^{-1}$.
\begin{proof}
Consider the following local deformation:
\begin{equation} \nn
\cb{\begin{tikzpicture}
\draw[linestyle,looseness=.75]
(0,0) to [out=0,in=-90] (1,.25)
	to [out=90,in=-90] (.5,.75)
	to [out=90,in=180] (1.5,1);
\end{tikzpicture}}
\leadsto
\cb{\begin{tikzpicture}
\draw[linestyle,looseness=.75]
(0,.125) to [out=0,in=-90] (1.25,.25)
	to [out=90,in=-90] (.6,.5);
\draw[linestylegraylight,looseness=.75]
(.6,.5) to [out=90,in=-90] (.9,.75);
\draw[linestyle,looseness=.75]
(.9,.75) to [out=90,in=-90] (.25,1)
	to [out=90,in=180] (1.5,1.125);
\end{tikzpicture}}
=
\cb{\begin{tikzpicture}
\draw[linestyle,looseness=.75]
(-.25,-.125) to [out=0,in=-90] (.5,.125)
	to [out=90,in=-90] (.15,.375);
\draw[linestylegraylight,out looseness=.75,in looseness=1.5]
(.15,.375) to [out=90,in=110] (.75,.125);
\draw[linestylegraylight,out looseness=1.5,in looseness=.75]
(.75,.125) to [out=-70,in=-90] (1.35,-.125);
\draw[linestyle,looseness=.75]
(1.35,-.125) to [out=90,in=-90] (1,.125)
	to [out=90,in=180] (1.75,.375);
\end{tikzpicture}}
\leadsto
\cb{\begin{tikzpicture}
\draw[linestyle,looseness=.75]
(-.75,-.25) to [out=0,in=90] (.5,-.375);
\draw[linestyle,looseness=.75]
(-.5,.375) to [out=-90,in=180] (.75,.25);
\draw[linestylegraylight,coverline,out looseness=.9,in looseness=1.5]
(.5,-.375) to [out=-90,in=-70] (0,0);
\draw[linestylegraylight,coverline,out looseness=1.5,in looseness=.9]
(0,0) to [out=110,in=90] (-.5,.375);
\end{tikzpicture}}
\end{equation}
Applying this deformation on the top part of the bean, we obtain the following homotopy:
\begin{equation} \nn
\cb{\begin{tikzpicture}
\draw[linestyle,out looseness=.75,in looseness=1]
	(-.5,-.25) to [out=-90,in=-90] (.5,0)
	(-.5,.25) to [out=90,in=90] (.5,0);
\draw[linestyle,looseness=.75]
	(-.5,-.25) to [out=90,in=-90] (0,0)
	(-.5,.25) to [out=-90,in=90] (0,0);
\end{tikzpicture}}
\leadsto
\cb{\begin{tikzpicture}
\draw[linestyle,looseness=.75]
(-.75,-.375) to [out=90,in=90] (.5,-.375);
\draw[linestyle,looseness=.75]
(-.5,.375) to [out=-90,in=90] (.85,-.125);
\draw[linestylegraylight,coverline,out looseness=.9,in looseness=1.5]
(.5,-.375) to [out=-90,in=-70] (0,0);
\draw[linestylegraylight,coverline,out looseness=1.5,in looseness=.9]
(0,0) to [out=110,in=90] (-.5,.375);
\draw[linestyle,out looseness=.75,in looseness=1.5]
(-.75,-.375) to [out=-90,in=-90] (.85,-.125);
\end{tikzpicture}}
\leadsto
\cb{\begin{tikzpicture}
\draw[linestyle,looseness=.75]
(-.75,-.375) to [out=90,in=90] (.5,-.375);
\draw[linestyle,looseness=.75]
(-.5,.375) to [out=-90,in=90] (.85,-.125);
\draw[linestylegraylight,coverline,out looseness=.9,in looseness=1.5]
(.5,-.375) to [out=-90,in=-70] (0,0);
\draw[linestylegraylight,coverline,out looseness=1.5,in looseness=.9]
(0,0) to [out=110,in=90] (-.5,.375);
\draw[linestyle,looseness=.75]
	(-.75,-.375) to [out=-90,in=90] (0,-.5);
\draw[linestylegraylight,looseness=.75]
	(0,-.5) to [out=-90,in=90] (-.5,-.75);
\draw[linestyle,out looseness=.75,in looseness=1.5]
	(-.5,-.75) to [out=-90,in=-90] (.85,-.125);
\end{tikzpicture}}
=
\cb{\begin{tikzpicture}
\draw[linestyle,out looseness=1,in looseness=.75]
	(-.75,-.5) to [out=-90,in=-90] (-.15,0);
\draw[linestyle,looseness=.75]
	(-.15,0) to [out=90,in=-90] (-.75,.5);
\draw[linestyle,out looseness=1,in looseness=.75]
	(.75,.5) to [out=90,in=90] (.15,0);
\draw[linestyle,looseness=.75]
	(.15,0) to [out=-90,in=90] (.75,-.5);
\draw[linestylegraylight,coverline,looseness=.75]
	(-.75,-.5) to [out=90,in=-90] (.75,-.5)
	(.75,.5) to [out=-90,in=90] (-.75,.5);
\end{tikzpicture}}
\leadsto
\cb{\begin{tikzpicture}
\draw[linestyle,looseness=1.25]
	(-.5,-.35) to [out=-90,in=90] (.5,-.35)
	(-.5,.35) to [out=-90,in=90] (.5,.35);
\draw[linestylegraylight,coverline,looseness=1.25]
	(-.5,-.35) to [out=90,in=-90] (.5,-.35)
	(-.5,.35) to [out=90,in=-90] (.5,.35);
\end{tikzpicture}}
\end{equation}
That homotopy shows that the bean is homotopic to the composite $S^3 \xra{q \cdot 2} S^2 \xra{\inc} \Sigma \RP^2$.  (Here the expression $q \cdot 2$ refers to the composite $S^3 \xra{2} S^3 \xra{q} S^2$; this would more typically be written and thought of as ``$2q$", but we want to be sure to distinguish it from the composite $S^3 \xra{q} S^2 \xra{2} S^2$, which is $q \cdot 4$.) The lemma follows, because $\pi_3(S^2) \xra{\inc_*} \pi_3(\Sigma \RP^2)$ is surjective.
\end{proof}
\begin{remark}
This lemma is of more significance that it might appear.  It means that the null homotopy $R$ of $2S$ does not, \emph{indeed cannot be chosen to}, satisfy the standard snake relation.
\end{remark}

\begin{lemma}
The suspension $\Sigma p: S^3 \ra \Sigma \RP^2$ of the antipodal projection $p: S^2 \ra \RP^2$ is homotopic to the composite of the \emph{negative} Hopf map $-q: S^3 \ra S^2$ with the inclusion $S^2 \ra \Sigma \RP^2$.
\end{lemma}
\begin{proof}
The desired equation $\Sigma p = -q$ follows if $\Sigma p + q = 0$, which follows if $\Sigma p - q = -2q = 2q$.  It therefore suffices to show that the union of construction~\eqref{eq-p} and the negative of construction~\eqref{eq-q} is homotopic to construction~\eqref{eq-bean}.  This is indeed the case:
\begin{equation} \nn
\cb{\begin{tikzpicture}
\draw[linestyle,looseness=1.25]
(0,0) to [out=-90,in=-120] (.5,0)
	to [out=60,in=90] (1,0);
\draw[linestyle,coverline,looseness=1.25]
(1,0) to [out=-90,in=-60] (.5,0)
	to [out=120,in=90] (0,0);
\draw[linestylegraylight,looseness=1.25]
(1,-.5) to [out=-90,in=-60] (.5,-.5)
	to [out=120,in=90] (0,-.5);
\draw[linestyle,coverline,looseness=1.25]
(0,-.5) to [out=-90,in=-120] (.5,-.5)
	to [out=60,in=90] (1,-.5);
\end{tikzpicture}}
\leadsto
\cb{\begin{tikzpicture}
\draw[linestyle,looseness=1.25]
(0,0) to [out=90,in=60] (-.5,0)
	to [out=-120,in=-90] (-1,0);
\draw[linestylegraylight,looseness=1.25]
(0,-.5) to [out=-90,in=-60] (-.5,-.5)
	to [out=120,in=90] (-1,-.5);
\draw[linestyle,coverline,looseness=1.25]
(0,0) to [out=-90,in=90] (-.15,-.25)
	to [out=-90,in=90] (0,-.5);
\draw[linestyle,coverline,looseness=1.25]
(-1,0) to [out=90,in=120] (-.5,0)
	to [out=-60,in=90] (-.3,-.25)
	to [out=-90,in=60] (-.5,-.5)
	to [out=-120,in=-90] (-1,-.5);
\end{tikzpicture}}
=
\cb{\begin{tikzpicture}
\draw[linestyle,looseness=.75]
	(.5,-.25) to [out=90,in=-90] (0,0)
	to [out=90,in=-90] (.5,.25)
	to [out=90,in=-90] (0,.5);
\draw[linestyle,looseness=1]
	(0,.5) to [out=90,in=90] (1,0)
	(0,-.5) to [out=-90,in=-90] (1,0);
\draw[linestylegraylight,looseness=.75]
	(.5,-.25) to [out=-90,in=90] (0,-.5);
\end{tikzpicture}}
=
\cb{\begin{tikzpicture}
\draw[linestyle,out looseness=.75,in looseness=1]
	(-.5,-.25) to [out=-90,in=-90] (.5,0)
	(-.5,.25) to [out=90,in=90] (.5,0);
\draw[linestyle,looseness=.75]
	(-.5,-.25) to [out=90,in=-90] (0,0)
	(-.5,.25) to [out=-90,in=90] (0,0);
\end{tikzpicture}}
\end{equation}
\end{proof}

\nid Thanks to this lemma we can \emph{pick} a homotopy, call it $H$, between $\Sigma p: S^3 \ra \Sigma \RP^2$ and $\inc_*(-q): S^3 \ra \Sigma \RP^2$.  Note well that this homotopy is not well defined---there is a $\pi_4(\Sigma\RP^2) = \ZZ/4$ ambiguity in the choice.  The homotopy $H$ also provides a distinguished homotopy between construction~\eqref{eq-p} and the negative of construction~\eqref{eq-q}, and therefore, for any pair $(S, R) \in \left(\Omega\Aut(X) \,_2\! \times_{\Omega\Aut(X)} \ast\right)$, between the map $p(S,R): S^2 \ra \Aut(X)$ and the map $-q(S): S^2 \ra \Aut(X)$.

Equipped with this homotopy, we can complete the description of the map from $SO(3)$ to the automorphisms of $X$.  Recall from Theorem~\ref{thm-so3action} that a homotopy $SO(3)$ action on a 3-type $X$ is defined by a triple $(S,R,W)$, where $S: S^1 \ra \Aut(X)$, $R: D^2 \ra \Aut(X)$ is a null homotopy of the square of $S$, and $W: D^3 \ra \Aut(X)$ is a null homotpoy of $q(S)$.
\begin{proposition} \label{prop-so3map}
Given a homotopy $SO(3)$ action on a 3-type $X$, defined by the triple $(S,R,W)$, the associated map of spaces $SO(3) \ra \Aut(X)$ is given on the 1-, 2-, and 3-cells of $SO(3)$ by, respectively, the map $S: S^1 \ra \Aut(X)$, the map $R: D^2 \ra \Aut(X)$, and the map $M: D^3 \ra \Aut(X)$, a null homotopy of $p(S,R): S^2 \ra \Aut(X)$ defined by $M:=H \cdot (-W)$.
\end{proposition}
\nid Here $-W: D^3 \ra \Aut(X)$ refers to the null homotopy of $-q(S)$ obtained by reflecting the null homotopy $W: D^3 \ra \Aut(X)$ across an axis, and $H$ is the homotopy defined above between $p(S,R)$ and $-q(S)$.  We leave the proof to the reader, with the note that the ambiguity in the choice of the homotopy $H$ does not affect the homotopy class of the resulting map $SO(3) \ra \Aut(X)$ precisely because of the fact that for a homotopy $SO(3)$ action, the map $\frac{\eta q}{2}: S^3 \ra \Aut(X)$ is null.

\begin{remark}
Though there is no a priori reason for this to be the case, it turns out that the map of spaces $SO(3) \ra \Aut(X)$ already contains all the information of a homotopy $SO(3)$ action on $X$.  More specifically, a map $SO(3) \ra \Aut(X)$ is given by a triple $(S, R, M)$, where $S$ is a map $S^1 \ra \Aut(X)$, $R$ is a null homotopy of $2S$, and $M$ is a null homotopy of $p(S,R)$; this triple deloops (and if it does deloops uniquely) to a homotopy $SO(3)$ action if and only if $\frac{\eta q}{2}(S,R) = 0$.
\end{remark}

\begin{remark}
Along the way, we have determined an alternative 4-truncation of $BSO(3)$, namely $(S^2 \cup_2 e_3) \cup_{\Sigma p} e_4 \cup_{\frac{\eta q}{2}} e_5$---this 4-truncation is, of course, canonically homotopy equivalent to the earlier 4-truncation.  % Use the homotopy H and a reflection to construct the homotopy equivalence to the earlier 4-truncation.  Note the 5-cell ensures that the choice of H didn't affect the resulting homotopy equivalence
\end{remark}

\begin{remark}
Though we have already repeatedly used the fact that $\Sigma p$, the suspension of the antipodal projection $p: S^2 \ra \RP^2$, is 4-torsion in $\pi_3(\Sigma \RP^2) = \ZZ/4$, we cannot resist including a proof:
\begin{equation} \nn
...
\end{equation}
\end{remark}


\section{Homotopy fixed points}

\CDcomm{This section describes the meaning of a homotopy fixed point explicitly for each group.}

The classifying space $B\Aut(X)$ carries a universal fibration with fiber $X$, namely $U_X := E\Aut(X) \times_{\Aut(X)} X \ra B\Aut(X)$.  \CD{Should we say something about when that really is a classifying fibration?  Can the universal fibration really be written as that bundle? Ie is $E\Aut(X)$ kosher?  If not just cite the existence of such a fibration?}  Given a homotopy action of $G$ on a space $X$, that is a map $\alpha: BG \ra B\Aut(X)$, there is an associated fibration $EG(X):=\alpha^\ast(U_X) \ra BG$, again with fiber $X$.

\begin{definition}
A homotopy fixed point of a homotopy $G$-action on $X$ is a section of the fibration $X \ra EG(X) \ra BG$.
\end{definition}

\begin{remark}
In case $G$ is an honest group with an honest action on $X$, the fibration $EG(X) \ra BG$ can be taken to be the bundle projection $EG \times_G X \ra BG$.  The data of a section of that bundle is precisely the same as the data of a $G$-equivariant map $EG \ra X$, that is of a homotopy fixed point in the ordinary sense.
\end{remark}

\begin{proposition} \label{prop-hofptrunc}
Given a homotopy action $BG \ra B\Aut(X)$ on a 3-type $X$, the natural map from the space of sections of $X \ra EG(X) \ra BG$ to the space of sections of that fibration over the 4-truncation ${}_4 BG \ra BG$, is a homotopy equivalence.
\end{proposition}
\begin{proof}
Given a section over ${}_4 BG$, extensions of the section to $BG$ are controlled by the obstruction groups $H^k(BG;\pi_{k-1} X)$ for $k \geq 5$, and choices of extensions are controlled by the groups $H^k(BG;\pi_k X)$ for $k \geq 5$.  All of those groups vanish.
\end{proof}

In this section, for brevity we sometimes use ``is" to mean ``is homotopy equivalent to".

\subsection{For $\Omega S^2$ and $\Quad$ actions}

Recall that an $\Omega S^2$ action on a 3-type $X$, which a priori is a based map $\alpha: S^2 \ra B\Aut(X)$ may instead be viewed as a based map $S: S^1 \ra \Aut(X)$.

\begin{proposition}
The space of homotopy fixed points of the $\Omega S^2$ action on a 3-type $X$, given by $S: S^1 \ra \Aut(X)$, is the space of pairs $(x,P)$ where $x \in X$ is a point and $P: D^2 \ra X$ is a null homotopy of the loop $S(x): S^1 \ra X$.
\end{proposition}
\begin{proof}
By definition a homotopy fixed point of the action is a section of the fibration $X \ra E\Omega S^2(X) \ra S^2$.  We can describe this fibration more explicitly as $X \cup_{X \times S^1} X \times D^2 \ra \ast \cup_{S^1} D^2 = S^2$, where the map $X \times S^1 \ra X \times D^2$ is the inclusion and the map $X \times S^1 \ra X$ is the adjoint of $\tilde{S}: S^1 \ra \Aut(X)$; here $\tilde{S}$ is the pointwise inverse of $S$.  %To be really pedantic, the clutching function S: S^1 \ra \Aut(X) constructs the bundle classified by \Sigma S^1 \ra \Sigma\Aut(X) \ra B\Aut(X) --- that map _is_ the image under the equivalence \Omega \Aut(X) to \Omega^2 B\Aut(X).

The point $x$ provides the section on the basepoint $\ast$ of $\ast \cup_{S^1} D^2$, which provides the section $S(x): S^1 \ra X$ on $\partial D^2$; the null homotopy $P$ extends the section over the 2-cell of $S^2$.
\end{proof}

Given a $Quad$ action $\alpha: BQuad \ra B\Aut(X)$ on a 3-type $X$, there is of course an induced $\Omega S^2$ action.

\begin{proposition}
The homotopy fixed points of a $Quad$ action on a 3-type $X$ are the same as the homotopy fixed points of the underlying $\Omega S^2$ action on $X$.
\end{proposition}
\begin{proof}
By Proposition~\ref{prop-hofptrunc}, to understand the homotopy fixed points of $Quad$ it suffices to consider sections of the bundle $X \ra EQuad(X) \ra BQuad$ over the 4-truncation of $BQuad$, which by Proposition~\ref{prop-quadtrunc} is $S^2 \cup_{\eta q} e_5$.  An $\Omega S^2$ homotopy fixed point is a section of the bundle over the 2-skeleton of ${}_4 BQuad$.  The obstruction to extending this section over the 5-cell of ${}_4 BQuad$ lives in $H^5({}_4 BQuad;\pi_4(X)) = 0$ and the homotopy classes of choices of lift is a torsor for $H^5({}_4 BQuad;\pi_5(X)) = 0$.
\end{proof}
\CD{That proof could probably be phrased better.}


\subsection{For $SO(2)$ actions}


\subsection{For $\Omega \Sigma \RP^2$ and $Orp(3)$ actions}


\subsection{For $SO(3)$ actions}

\subsection{For $Spin(3)$ actions} %and other groups?
.






\end{document}











