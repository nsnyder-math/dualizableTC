%%% Authors: Christopher Douglas, Christopher Schommer-Pries, and Noah Snyder

\documentclass{amsart}


%%%%%%% Standard Packages
\usepackage{amsmath}       % I think this gives me some symbols
\usepackage{amsthm}        % Does theorem stuff
\usepackage{amssymb}       % more symbols and fonts
\usepackage{amsfonts}
\usepackage[all]{xy}
\usepackage{xspace}
\usepackage{calc}



\setlength{\topskip}{0pt}
\setlength{\footskip}{30pt}
\headheight=0pt
\topmargin=0pt
\headsep=18pt
\textheight=603pt %% 792pt to page, 648 is 9in
\textwidth=420pt  %% 612pt to page, 468pt is 6.5in
\oddsidemargin=25pt
\evensidemargin=25pt

\pagestyle{plain}


%%%%%% Adds hyperlinks
\usepackage[colorlinks, linkcolor=black, citecolor=blue,
	% pagebackref,
 	%bookmarksnumbered=true
	]{hyperref}
	
	
	
%%%%%% Tikz !!! Commands and Macros %%%%%%%%%%%%%
\usepackage{tikz}
\usetikzlibrary{matrix}


%%%% These draw triple or quadruple set of arrows of length 0.5 cm
\DeclareMathOperator{\righttriplearrows} {{\; \tikz{ \foreach \y in {0, 0.1, 0.2} { \draw [-stealth] (0, \y) -- +(0.5, 0);}} \; }}
\DeclareMathOperator{\lefttriplearrows} {{\; \tikz{ \foreach \y in {0, 0.1, 0.2} { \draw [stealth-] (0, \y) -- +(0.5, 0);}} \; }}
\DeclareMathOperator{\rightquadarrows} {{\; \tikz{ \foreach \y in {0, 0.1, 0.2, 0.3} { \draw [-stealth] (0, \y) -- +(0.5, 0);}} \; }}
\DeclareMathOperator{\leftquadarrows} {{\; \tikz{ \foreach \y in {0, 0.1, 0.2, 0.3} { \draw [stealth-] (0, \y) -- +(0.5, 0);}} \; }}

%%%%%%% End TikZ Commands and Macros %%%%%%%%%%%%%



%%%%%%%%%%%%%%%%%%%%%% Theorem Styles and Counters %%%%%%%%%%%%%%%%%%%%%%%%%%
% These all use the same "theorem" counter. 
\theoremstyle{plain} %%% Plain Theorem Styles.
\newtheorem{theorem}{Theorem}[section]
\newtheorem{lemma}[theorem]{Lemma}
\newtheorem{corollary}[theorem]{Corollary}          
\newtheorem{proposition}[theorem]{Proposition}              

\theoremstyle{definition} %%%% Definition-like Commands  
\newtheorem{definition}[theorem]{Definition}

\theoremstyle{remark}  %%%% Remark-like Commands
\newtheorem{remark}[theorem]{Remark}
\newtheorem{example}[theorem]{Example}
%%%%%%%%%%%%%%%%%%%%%% End Theorem Styles and Counters %%%%%%%%%%%%%%%%%%%%%%%%%%

%%%% Misc symbols %%%%%

\newcommand{\nn}{\nonumber}
\newcommand{\nid}{\noindent}
\newcommand{\ra}{\rightarrow}
\newcommand{\la}{\leftarrow}
\newcommand{\xra}{\xrightarrow}
\newcommand{\xla}{\xleftarrow}

\newcommand{\Bord}{\mathrm{Bord}}
\newcommand{\Vect}{\mathrm{Vect}}
\newcommand{\TC}{\mathrm{TC}}

\def\cA{\mathcal A}\def\cB{\mathcal B}\def\cC{\mathcal C}\def\cD{\mathcal D}
\def\cE{\mathcal E}\def\cF{\mathcal F}\def\cG{\mathcal G}\def\cH{\mathcal H}
\def\cI{\mathcal I}\def\cJ{\mathcal J}\def\cK{\mathcal K}\def\cL{\mathcal L}
\def\cM{\mathcal M}\def\cN{\mathcal N}\def\cO{\mathcal O}\def\cP{\mathcal P}
\def\cQ{\mathcal Q}\def\cR{\mathcal R}\def\cS{\ess}\def\cT{\mathcal T}
\def\cU{\mathcal U}\def\cV{\mathcal V}\def\cW{\mathcal W}\def\cX{\mathcal X}
\def\cY{\mathcal Y}\def\cZ{\mathcal Z}

\def\AA{\mathbb A}\def\BB{\mathbb B}\def\CC{\mathbb C}\def\DD{\mathbb D}
\def\EE{\mathbb E}\def\FF{\mathbb F}\def\GG{\mathbb G}\def\HH{\mathbb H}
\def\II{\mathbb I}\def\JJ{\mathbb J}\def\KK{\mathbb K}\def\LL{\mathbb L}
\def\MM{\mathbb M}\def\NN{\mathbb N}\def\OO{\mathbb O}\def\PP{\mathbb P}
\def\QQ{\mathbb Q}\def\RR{\mathbb R}\def\SS{\mathbb S}\def\TT{\mathbb T}
\def\UU{\mathbb U}\def\VV{\mathbb V}\def\WW{\mathbb W}\def\XX{\mathbb X}
\def\YY{\mathbb Y}\def\ZZ{\mathbb Z}

%%%%%%%%%















\begin{document}

\title{``Dualizable Tensor Categories II: homotopy $SO(3)$-actions"}

\author{Christopher L. Douglas}
\address{Mathematical Institute\\ University of Oxford\\ Oxford OX1 3LB\\ United Kingdom}
\email{cdouglas@maths.ox.ac.uk}
\urladdr{http://people.maths.ox.ac.uk/cdouglas}
      	
\author{Christopher Schommer-Pries}
\address{Department of Mathematics\\ Massachusetts Institute of Technology\\ Cambridge, MA 02139\\ USA}
\email{schommerpries.chris.math@gmail.com}
\urladdr{http://sites.google.com/site/chrisschommerpriesmath}

\author{Noah Snyder}
\address{Department of Mathematics\\ Columbia University\\ New York, NY 10027\\ USA}
\email{nsnyder@math.columbia.edu}
\urladdr{http://www.math.columbia.edu/\!\raisebox{-1mm}{~}nsnyder/}


\maketitle

\tableofcontents

\section{Sketch of paper contents}

This purpose of this paper is to explain what it means, cellularly, to give a homotopy $SO(3)$-action on a homotopy 3-type.  That is, you say, $BSO(3)$ has cells $e_2, e_3, e_4, e_5$, where $e_3$ is attached to $e_2$ by $2$, and $e_4$ is attached to $e_2$ by $\eta$ and $e_5$ is attached to $e_3$ by yadda.  And to give a map $BSO(3) \rightarrow BAut(X)$ is to give an element $S \in \pi_1(Aut(X))$ and a trivialization $R$ of $2S$ such that ... and to give ... %the element $\eta(q(S))$ of $\pi_3(Aut(X))$ is trivial, and to give a trivialization $a$ of $q(S) \in \pi_2(Aut(X))$.  <--- ???

Along the way, we also discuss what it means to have actions by $Orp(3)$, $SO(2)$, $\Omega S^2$, $\Omega \Sigma \RR P^2$.

\CDcomm{We should also precisely describe the data required to give a G-homotopy fixed point, for each of the discussed groups.}

\subsection{General remarks}

Why? The cobordism hypothesis tells us that $Hom^{\otimes}(GBord_n,C) = ((dC)^{\sim})^{hG}$, where $dC$ is the maximal symmetric monoidal sub-n-category of C with duals, and the $\sim$ denotes the maximal sub-n-groupoid.  Note that $(dC)^\sim$ is a homotopy $n$-type.  Thus we want to describe explicitly the structure of $G$ actions on $n$-types.  (In fact, the description we give will apply to actions on discrete $n$-categories that are not necessarily $n$-groupoids \CDcomm{(right?)}, but such a case is not relevant for cobordism hypothesis applications and we will not mention it explicitly.)

If $C$ is a discrete $n$-category (eg an $n$-type), let $Aut(C)$ denote the $n$-groupoid of automorphisms of $C$ --- the objects are the automorphisms of $C$ in the discrete $(n+1)$-category of $n$-categories, the morphisms are the automorphisms of automorphisms, etc.  Thus $Aut(C)$ is an $n$-type and $BAut(C)$ is an $(n+1)$-type.  By definition, an action of $G$ on $C$ is a map $BG \ra BAut(C)$.

\subsection{Warmup: $SO(2)$ actions}

An $SO(2)$ action on a 3-type $X$ is a map $\alpha: BSO(2) \ra BAut(X)$.  The 5-skeleton of $BSO(2)$ has cells $e_2$ and $e_4$.  The data of the map $\alpha$ on $e_2$ is an element of $\pi_2(BAut(X)) = \pi_1(Aut(X))$.  We denote this element $S$ and call it the Serre automorphism.  The 4-cell is attached to the 2-cell by $\eta : S^3 \ra S^2$.  The data of an extension of $S$ to the 4-cell and therefore to all of $BSO(2)$ is given by a trivialization of the composite $\eta S : S^3 \ra BAut(X)$.

\begin{proposition}
The composite $\eta S : S^3 \ra BAut(X)$ can be computed as follows: draw a figure 8, read the cap and cup as the equivalences $1 \cong S \cdot S^{-1}$ and $S^{-1} \cdot S \cong 1$ and the cross as the braiding of elements in $2Aut(X)$, and evaluate the picture in $\Omega^2(Aut(X)) \simeq \Omega^3 BAut(X)$.
\end{proposition}

\begin{proof}
\CDcomm{This would be clear stably, but requires explanation unstably.}
\end{proof}

To summarize:

\begin{proposition}
The set of homotopy classes of $SO(2)$ actions on a 3-type $X$ is the set of homotopy classes of pairs $(S,E)$, where $S: S^1 \ra Aut(X)$ is a map and $E : D^3 \ra Aut(X)$ is a null homotopy of the figure 8 composite on $S$.
\end{proposition}

\CDcomm{More explanation is really necessary to even make the statement clear.  Eg, make clear in what sense is the figure 8 composite an actual map to be trivialized rather than merely a homotopy class of maps.  (If it were just a homotopy class, it would look like E is an element of $\pi_3$ of $Aut(X)$, which is wrong, since it lives in a torsor.)  For instance, do this by picking equivalences $\Omega^2(Aut(X)) \simeq \Omega^3 BAut(X)$ etc?}

We can restrict attention to 3-types that happen to be 2-types.

\begin{corollary}
The set of homotopy classes of $SO(2)$ actions on a 2-type $X$ is the set of homotopy classes of maps $S: S^1 \ra Aut(X)$ such that the figure 8 composite on $S$ is null.
\end{corollary}






\end{document}











