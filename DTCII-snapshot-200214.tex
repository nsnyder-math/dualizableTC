%%% Authors: Christopher Douglas, Christopher Schommer-Pries, and Noah Snyder

\documentclass{amsart}

\include{DTCII-Preamble}
\usepackage{MnSymbol} % Needed this for a couple symbols, but it breaks \dtimes, so is not included in the general preamble.


%\tikzexternaldisable


\begin{document}

\title{``Dualizable Tensor Categories II: homotopy $SO(3)$-actions"}

\author{Christopher L. Douglas}
\address{Mathematical Institute\\ University of Oxford\\ Oxford OX1 3LB\\ United Kingdom}
\email{cdouglas@maths.ox.ac.uk}
\urladdr{http://people.maths.ox.ac.uk/cdouglas}
      	
\author{Christopher Schommer-Pries}
\address{Department of Mathematics\\ University of Notre Dame \\ Notre Dame, IN 46556\\ USA}
\email{cschomme@nd.edu}
\urladdr{http://sites.nd.edu/chris-schommer-pries/}

\author{Noah Snyder}
\address{Department of Mathematics\\ Indiana University\\ Bloomington, IN 47405\\ USA}
\email{nsnyder@math.indiana.edu}


\begin{abstract}
We give a finite explicit descriptions of what it means to give a homotopy $SO(3)$-action on a homotopy $3$-type and what it means to give a homotopy fixed point for such an action.  To answer these questions we give a cellular description of the $4$-type of $BSO(3)$ and use Pontryagin--Thom to understand the attaching maps explicitly.  We also consider actions on $3$-types and their homotopy fixed points for a number of groups closely related to $SO(3)$.  These questions are motivated by questions in topological quantum field theory and the structure of tensor categories, and applications in those directions will be given in later papers.
\end{abstract}

\maketitle

\tikzexternaldisable

\tableofcontents

\section{Introduction}



\section{Cell structures}



\section{Bordism models}




\section{Actions and fixed points}



\section{Decompressed actions and fixed points}












\CDcomm{OLD MATERIAL BELOW FOR REPURPOSING}

%%%%%

\CDcomm{To do items.
\begin{itemize}
\item Mention DTCI plus this implies we have constructed an $\Omega S^2$ action and an $\Omega \Sigma \RP^2$ action on $TC$.  DTCIII will go further.
\item Similarly, in ho fixed point section, discuss that pivotality is $\Omega S^2$ fixed point.  For sphericality, either do it ($\Omega \Sigma \RP^2$ fixed point) or forward ref to DTCIV, considering it may require translating the usual notion of sphericality to our notion.  (Same fixed points for $\Omega S^2$ and $Quad$ and for $\Omega \Sigma \RP^2$ and $\Orp(3)$.
\item Add category theory perspective/interpretation, building on Joyal/Street view of figure 8.
\item (Research to do: what is W?)
\end{itemize}
}

\section*{Sketch scratch}

This purpose of this paper is to explain what it means, cellularly, to give a homotopy $SO(3)$-action on a homotopy 3-type.  That is, you say, $BSO(3)$ has cells $e_2, e_3, e_4, e_5$, where $e_3$ is attached to $e_2$ by $2$, and $e_4$ is attached to $e_2$ by a generator $q: S^3 \ra S^2$ and $e_5$ is attached to $e_3$ by yadda.  And to give a map $BSO(3) \rightarrow BAut(X)$ is to give an element $S \in Map(S^2,BAut(X))$ and a trivialization $A$ of $q S \in Map(S^3,BAut(X))$ and a trivialization $R$ of $2S$ such that a canonical element called $\frac{\eta q S}{2}$ of $\pi_4(BAut(X))$ vanishes.  (Note that the choice of 2-divisor of $\eta q S$ depends on $R$.  Here $\eta: S^4 \ra S^3$ is th 	e generator.)

Along the way, we also discuss what it means to have actions by $Orp(3)$, $SO(2)$, $\Omega S^2$, $\Omega \Sigma \RP^2$.

\CDcomm{We should also precisely describe the data required to give a G-homotopy fixed point, for each of the discussed groups.  And for each group we should also explicitly describe the map $G \ra Aut(X)$ in terms of the given data defining $BG \ra BAut(X)$.}


\begin{tikzpicture}
\node (start) at (0,0) {$\mathrm{Id}_x$};
\node (first top) at (4,6) {$\cS_x \cS_x^{-1}$};
\node (second top) at (12,6) {$\cS_x^{-1} \cS_x$};
\node (end) at (16,0) {$\mathrm{Id}_x$};

\node (first middle) at (4,0) {$\cS_x \cS_x$};
\node (second middle) at (12,0) {$\cS_x \cS_x$};

\node (first bottom) at (4,-4) {$\cS_x^{-1} \cS_x$};
\node (second bottom) at (12,-4) {$\cS_x \cS_x^{-1} $};

\node (mate top left) at (6,4) {$\cS_x \cS_x \cS_x^{-1} \cS_x^{-1}$};
\node (mate bottom left) at (6,2) {$\cS_x \cS_x \cS_x \cS_x^{-1}$};
\node (mate top right) at (10,4) {$\cS_x \cS_x^{-1} \cS_x^{-1} \cS_x $};
\node (mate bottom right) at (10,2) {$\cS_x \cS_x \cS_x^{-1} \cS_x$};


\draw [->] (start) .. controls (3,14) and (13,14) .. node[above] {$\mathrm{Id}_{\mathrm{Id}_x}$} (end);

\node [shape=ellipse, draw] at (8,8.5) {$W_x$};

\draw [->]	(start) -- node[above, left] {$\mathrm{coev}$} (first top);
\draw [->]	(first top) -- node[above] {$\cS_{\cS_x^{-1}}$} (second top);
\draw [->]	(second top) -- node[above, right] {$\mathrm{ev}$} (end);

\draw[->] (first middle) -- node[above] {$\cS_{\cS_x}$} (second middle);

\draw[->] (first middle) -- node[left] {$\mathrm{coev}_{34}$} (mate bottom left);
\draw[->] (second middle) edge [bend left =30]  node[left]{$\mathrm{coev}_{23}$} (mate bottom right);
\draw[->] (mate bottom right) edge [bend left =30]  node[right]{$\mathrm{ev}_{34}$} (second middle);
\draw[->] (mate bottom left) -- node[left]{$(\cR_x^{-1})_{3}$} (mate top left);
\draw[->] (mate bottom right) -- node[right]{$(\cR_x^{-1})_{2}$} (mate top right);
\draw[->] (mate bottom left) -- node[above]{$\cS_{\cS_x}$} (mate bottom right);
\draw[->] (mate top left) -- node[above]{$\cS_{\cS_x \cS_x \cS_x^{-1}}$} (mate top right);
\draw[->] (mate top left) -- node[right]{$\mathrm{ev}_{23}$} (first top);
\draw[->] (mate top right) -- node[left]{$\mathrm{ev}_{12}$} (second top);


\draw [->]	(start) -- node[above, left] {$\mathrm{coev}$} (first bottom);
\draw [->]	(first bottom) -- node[above] {$\cS^{-1}_{\cS_x}$} (second bottom);
\draw [->]	(second bottom) -- node[above, right] {$\mathrm{ev}$} (end);


\node [shape=ellipse, draw]  at (11,1) {ZZ};
\node [shape=ellipse, draw] at (8,1) {$\cS_{\mathrm{coev}_{23}}$};
\node [shape=ellipse, draw] at (8,3) {$\cS_{(\cR_x^{-1})_2}$};
\node [shape=ellipse, draw] at (8,5) {$\cS_{\mathrm{ev}_{12}}$};
\end{tikzpicture}






\subsection{General remarks}

Why? The cobordism hypothesis tells us that $Hom^{\otimes}(GBord_n,C) = ((dC)^{\sim})^{hG}$, where $dC$ is the maximal symmetric monoidal sub-n-category of C with duals, and the $\sim$ denotes the maximal sub-n-groupoid.  Note that $(dC)^\sim$ is a homotopy $n$-type.  Thus we want to describe explicitly the structure of $G$ actions on $n$-types.  (In fact, the description we give will apply to actions on discrete $n$-categories that are not necessarily $n$-groupoids \CDcomm{(right?)}, but such a case is not relevant for cobordism hypothesis applications and we will not mention it explicitly.)

If $C$ is a discrete $n$-category (eg an $n$-type), let $Aut(C)$ denote the $n$-groupoid of automorphisms of $C$ --- the objects are the automorphisms of $C$ in the discrete $(n+1)$-category of $n$-categories, the morphisms are the automorphisms of automorphisms, etc.  Thus $Aut(C)$ is an $n$-type and $BAut(C)$ is an $(n+1)$-type.  By definition, an action of $G$ on $C$ is a map $BG \ra BAut(C)$.  Observe that if a map $A \ra B$ is a $k$-equivalence (i.e. an isomorphism on $\pi_i$ for $i \leq k$) and $X$ is a $k$-type (i.e. has $\pi_i = 0$ for $i > k$), then the induced map $[B,X] \ra [A,X]$ is an isomorphism.  As a result, we will only be concerned with the $(n+1)$-type of $BG$ when we are considering actions on $n$-categories.

... the techniques are for the most part elementary, but we believe the results are of fundamental importance for the study of 3-dimensional quantum topology.


\section{Introduction}

In prior work the authors investigated a 3-category of tensor categories, bimodule categories, functors, and natural transformations. We identified the multifusion categories as the so-called fully-dualizable objects. The data witnessing this full-dualizability gave rise to new structures which we claimed were part of an $SO(3)$-action on the 3-groupoid of multifusion categories, and we used these structures to give a new proof of a celebrated theorem of Etingof--Nikshych--Ostrik. This arose out of the authors' desire to understand and use the cobordism hypothesis to study the relationship between tensor categories and 3-dimensional topological quantum field theories. 

The cobordism hypothesis is a classification result for extended topological quantum field theories, arising in two parts. Extended topological quantum field theories, as the reader shall recall, are functors from an $n$-category of cobordisms to a target $n$-category. They come in different flavors depending on what sort of manifolds are used (i.e. what tangential structures they are given). For example one can use unoriented manifolds, oriented manifolds, manifolds with spin structure, tangentially framed manifolds, etc. The first part of the cobordism hypothesis states that the $n$-category of extended topological quantum field theories for the tangentially framed cobordism $n$-category with values in a given $n$-category $\cC$ is equivalent to the $n$-groupoid of fully-dualizable objects in $\cC$. The second part states that the  extended topological quantum field theories for other tangential structures are obtained from this $n$-groupoid of fully-dualizable objects by taking homotopy fixed points under an induced action. 

So for example, in the case of tensor categories, the cobordism hypothesis first predicts that the 3-groupoid of multifusion categories should admit an $SO(3)$-action. Then furthermore the oriented extended topological quantum field theories correspond to the homotopy fixed points for this action. This then prompts one to ask the natural question: just what \emph{is} an $SO(3)$-action on a 3-groupoid? Afterall one is a Lie group and the other a 3-category. At first blush they seem to live in such disparate realms of mathematics that even the phrase confuses the mind.
That, however, is the purpose of the current paper. We will give an explicit description of precisely what it means to give an $SO(3)$-action on a 3-groupoid, as well as an explicit description of the $SO(3)$ homotopy fixed points for such an action.




 


\section{Introduction}

This purpose of this paper is to give an explicit description of what it means to give a homotopy $SO(3)$-action on a homotopy 3-type and to give an explicit description of the $SO(3)$ homotopy fixed points for such an action.  These descriptions will be based on the cellular structure of the $4$-type of $BSO(3)$ with each cell giving a piece of data.  To describe this structure in sufficient detail we will need to give explicit descriptions of the attaching maps using Pontryagin--Thom.  We also explain what it means to give an action and what the homotopy fixed points are for a number of closely related groups like $SO(2)$, $\Omega S^2$, $Spin(3)$, and $Orpo(3)$.  These groups are either subgroups of $SO(3)$, groups given by killing certain characteristic classes in $SO(3)$, or loop groups of truncations of a cellular description of $BSO(3)$.

All of the results in this paper are purely homotopy theoretic and for the most part elementary, but the motivation for them comes from topological field theory.  Namely, the cobordism hypothesis says that the space of fully local framed $3$-dimensional TFTs with values in a symmetric monoidal $3$-category $\cC$ is equivalent to the space $X$ of fully dualizable objects in $\cC$, that $O(n)$ has a homotopy coherent action on $X$ by changing framing, and that the $SO(n)$ homotopy fixed points classify oriented TFTs.  If $\cC$ is a discrete $3$-category (as opposed to an $(\infty,3)$-category), then $X$ is a homotopy $3$-type.  Thus it is natural to ask exactly how much information is contained in the statement that $SO(3)$ acts on $X$, and exactly how much information you need to give an $SO(3)$ homotopy fixed point.  In later papers we will give applications of the purely homotopy theoretic results in this paper to the example of finite tensor categories.  Most significantly, we will translate each part of the data of giving an action of $SO(3)$ into a theorem about finite tensor categories (one of which is Radford's theorem), and we will see that a spherical structure on a fusion category gives an $SO(3)$ homotopy fixed point and thus a fully local TFT which extends the Turaev-Viro theories down to points, but we will also see that there are other $SO(3)$ homotopy fixed points.

An action of $SO(3)$ on $X$ is the same thing as map of homotopy coherent groups $SO(3) \rightarrow \mathrm{Aut}(X)$ which is turn the same thing as a map of spaces $BSO(3) \rightarrow B\mathrm{Aut}(X)$.  To give an explicit collection of data describing such a map we need a CW description of $BSO(3)$.  Since $X$ is a $3$-type, $B\mathrm{Aut}(X)$ is a $4$-type, so it is enough to understand $BSO(3)$ as a $4$-type.  Recall that in order to get the right $\pi_4$ we will need to include $5$-cells in our description of $BSO(3)$.

\begin{theorem}
$BSO(3)$ has the same $4$-type as the CW complex with one $2$-cell, one $3$-cell attached to $S^2$ via the degree $2$ map (yielding $\Sigma \mathbb{R}P^2$), one $4$-cell attached to $S^2$ via the Hopf map, and one $5$-cell attached to $\Sigma \mathbb{R}P^2$ via the generator of $\pi_4(\Sigma \mathbb{R}P^2) = \mathbb{Z}/4\mathbb{Z}$.
\end{theorem}

Using Pontryagin--Thom each of these attaching maps have explicit descriptions.  This yield the following explicit description of a homotopy coherent $SO(3)$ action on a $3$-type.

\begin{itemize}
\item An assignment to every point $x$ a $2$-loop $S_x: \mathrm{Id}_x \rightarrow \mathrm{Id}_x$.
\item An assignment to every path $p:x \rightarrow y$ a $3$-path $S_p: S_x \circ p \rightarrow p \circ S_y$.
\item For every $2$-path $\alpha: p \rightarrow q$ the condition $S_\alpha$ that $(\mathrm{Id} \ast \alpha^{-1}) \circ S_p$ is homotopic to $(\alpha \ast \mathrm{Id}) \circ S_p$.
\end{itemize}

\begin{itemize}
\item An assignment to every point $x$ a $3$-path $R_x: (S_x)^{-1}  \rightarrow S_x$.
\item For every path $p: x \rightarrow y$ the condition $R_p$ that the two $3$-paths ??? $S_x \circ S_x \circ p \rightarrow q$ are homotopic.
\end{itemize}



\CDcomm{I think much of what is currently in section I of DTCIV, introducing and discussing the 4-types of BQuad and BOrp(3), will need to be moved here for those spaces to seem less crazy.  How much?}

\CDcomm{Annoyingly, we probably also need some of the facts in the Appendix to DTCIV.  Nevertheless, I think that appendix should stay there, and we will just state the occasional fact we need here independently.}


\subsection*{Acknowledgments}
CSP would like to thank Peter Teichner for several fruitful conversations about the difference between $\Omega \Sigma \RR \PP^2$-actions and $SO(3)\langle p_1 \rangle$-actions.  NS would like to thank Prasit Bhattacharya for many long conversations in the tea room without which I would not have been able to understand much of this paper.  GRANT STUFF



\section{Cell structures}

\CDcomm{This section is ``pure homotopy theory", ie focuses on determining the cell structures only.  All issues about interpreting the attaching maps and computing them explicitly are left to later sections.}

As we only care about the 4-types of our spaces, for each space we will describe a cell complex that captures the 4-type in question.  More specifically, for concreteness we will construct 4-truncations of our spaces, in the following sense.

\begin{definition}
By a \emph{$k$-truncation} of a space $X$, we will mean a $(k+1)$-dimensional cell complex, denoted $_k X$, together with a $k$-equivalence $_{k} X \ra X$.
\end{definition}
\nid Note that the homotopy type of a $k$-truncation is not uniquely determined, but the $k$-type is.  Moreover the homotopy class of the $k$-equivalence $_k X \ra X$ is not, in general, uniquely determined.  Though it is by no means a necessity, it is sometimes reassuring to know that a given truncation is as small as possible: we call a $k$-truncation \emph{minimal} if it has the minimum number of cells among all $k$-truncations.  Note that the cell structure of a minimal $k$-truncation is not in general uniquely determined, though it happens that it is in all the examples we consider.

%\CD{For better or worse the notation here is meant to suggest that $_4 X$ is a subobject, as opposed to the usual ``4-quotient" $X_4$ obtained by adding cells to $X$.  This notation is a bit questionable.  Note that $_4 X$ is not a 4-type.}

\subsection{Of the spaces $B\Omega S^2$ and $B\Quad$}.

Needless to say, the 4-truncation of $B\Omega S^2 \simeq S^2$ is simply $S^2$, and this is obviously minimal.

\begin{definition}
The space $B\Quad$ is the homotopy fiber of the map $c_1^2: BSO(2) \ra K(\ZZ,4)$ classifying the square of the first chern class.
\end{definition}
%\CD{Maybe Quad will already have been introduced earlier.}


\begin{proposition} \label{prop-quadtrunc}
The minimal 4-trunctation of $BQuad$ is the 2-cell complex $S^2 \cup_{\eta q} e_5$.  Here the attaching map is the composite of the generator $\eta: S^4 \ra S^3$ and the generator $q: S^3 \ra S^2$.
\end{proposition}

\begin{proof}
From the long exact sequence of the fibration $K(\ZZ,3) \ra B\Quad \ra BSO(2)$, we see that the first four homotopy groups ($\pi_1$--$\pi_4$) of $B\Quad$ are $0$, $\ZZ$, $\ZZ$, $0$, and from the corresponding Serre spectral sequence, that the first six cohomology groups ($H^1$--$H^6$) are $0$, $\ZZ$, $0$, $0$, $0$, $\ZZ/2$.  From the cohomology, we see there is a 4-truncation with a single 2-cell and a single 5-cell, and the homotopy groups force the attaching map to be $\eta q$.  This truncation is clearly minimal.
\end{proof}

\subsection{Of the space $BSO(2)$}.

The 4-truncation of $BSO(2)$ is $S^2 \cup_q e_4$, and this is certainly minimal.

\subsection{Of the spaces $B \Omega \Sigma \RP^2$ and $B Orp(3)$}.

The space $B \Omega \Sigma \RP^2 \simeq \Sigma \RP^2 \simeq S^2 \cup_2 e_3$ is its own minimal 4-truncation.

\begin{definition}
The space $BOrp(3)$ is the homotopy fiber of the composite $BSO(3) \ra BSO \xra{p_1} K(\ZZ,4)$, where $p_1$ is the first Pontryagin class.
\end{definition}

We will need the following homotopy-theoretic fact.
\begin{proposition}
The fourth homotopy group of the suspension of the real projective plane is $\ZZ/4$, generated by a class called $\frac{\eta q}{2}$ in deference to the fact that $2 \cdot \frac{\eta q}{2}$ is equal to the homotopy class of the composite $S^4 \xra{\eta} S^3 \xra{q} S^2 \xra{\text{inc}} \Sigma \RP^2$.
\end{proposition}
\CD{Do we have a homotopy theoretic description of the generator?} %Later we will certainly give the geometric description.
\CDcomm{Need to add a reference or proof here.}

\begin{proposition} \label{prop-borptrunc}
The minimal 4-truncation of $BOrp(3)$ is the 3-cell complex $(S^2 \cup_2 e_3) \cup_{\frac{\eta q}{2}} e_5$.
\end{proposition}
\begin{proof}
The homotopy groups of $S^3$ determine the homotopy groups of $SO(3)$ which determine the homotopy groups of $BSO(3)$ as $0$, $\ZZ/2$, $0$, $\ZZ$, $\ZZ/2$ for $\pi_1$ through $\pi_5$.  Using the slightly surprising fact that the composite $BSO(3) \ra BSO \xra{p_1} K(\ZZ,4)$ induces multiplication by 4 on $\pi_4$, this in turn determines the homotopy groups of $BOrp(3)$ as $0$, $\ZZ/2$, $\ZZ/4$, $0$, $\ZZ/2$.  

The Serre spectral sequence for $SO(3) \ra * \ra BSO(3)$ shows the cohomology of $BSO(3)$ is $0$, $0$, $\ZZ/2$, $\ZZ$, $0$, $\ZZ/2$ for $H^1$ through $H^6$, and the Serre spectral sequence for $K(\ZZ,3) \ra BOrp(3) \ra BSO(3)$ then gives the cohomology of $BOrp(3)$ as $0$, $0$, $\ZZ/2$, $0$, $0$, $\ZZ/4$.  (The extension in degree 6 is resolved by the corresponding mod 2 spectral sequence.)

From the cohomology of $BOrp(3)$ we see that there is a map $\Sigma \RP^2 \ra BOrp(3)$ which is an isomorphism on $H_{\leq 4}$.  By Hurewicz it follows that this map is an isomorphism on $\pi_{\leq 3}$.  To correct the $\pi_4$ discrepancy, we need to attach a 5-cell to kill $\pi_4(\Sigma \RP^2)$.  Because $\pi_4(BOrp(3))$ is zero, we may pick an extension of the map $\Sigma \RP^2 \ra BOrp(3)$ to a map $(S^2 \cup_2 e_3) \cup_{\frac{\eta q}{2}} e_5 \ra BOrp(3)$.  This latter map is now a 4-equivalence, as desired.  This truncation is clearly minimal.  Note that there were two possible choices of the extension, but both yield 4-equivalences and we will have no need to differentiate them.
\end{proof}

\subsection{Of the space $BSO(3)$}.


\begin{definition}
A commutative square is a pushout of $n$-types, also called an $n$-pushout, if the canonical map from the pushout to the target object is an $n$-equivalence.
\end{definition}

\begin{proposition}
The commutative square
\begin{equation}\label{s2-pushout} \tag{S}
\xymatrix{
B\Omega S^2 \ar[r] \ar[d] & BOrp(3) \ar[d] \\
BSO(2) \ar[r] & BSO(3)
}
\end{equation}
is a $4$-pushout.
\end{proposition}
\begin{proof}
We need to show that the map $P := BSO(2) \cup_{S^2} BOrp(3) \ra BSO(3)$ is a 4-equivalence.  By the exact sequence of the pair $(P, BOrp(3))$, the homology of $P$ in degrees 1 through 5 is $0$, $\ZZ/2$, $0$, $\ZZ$, $\ZZ/4$.  Now consider the map $P \ra BSO(3)$ in homology.  The maps $H_2(BOrp(3)) \ra H_2(P)$ and $H_5(BOrp(3)) \ra H_5(P)$ are isomorphisms, and $H_2(BOrp(3)) \ra H_2(BSO(3))$ is an isomorphism and $H_5(BOrp(3)) \ra H_5(BSO(3))$ is surjective by the Serre spectral sequence of $K(\ZZ,3) \ra BOrp(3) \ra BSO(3)$.  The map $H_4(BSO(2)) \ra H_4(P)$ is an isomorphism, and the map $H_4(BSO(2)) \ra H_4(BSO(3))$ is an isomorphism (by for instance noting that the mod 2 reduction of $p_1$ is $w_2^2$ and that considered with $\ZZ[\frac{1}{2}]$ coefficients, the pullback of $p_1$ to $BU(1)$ is $c_1^2$).  It follows that $H_{\leq 5}(BSO(3),P) = 0$, thus $\pi_{\leq 5}(BSO(3),P) = 0$ and so the map $P \ra BSO(3)$ is a 4-equivalence, as desired.
\end{proof}

%\CD{I suggest that in writing we try to be fairly careful to denote the difference between a space $X$ and its $k$-truncation or $k$-quotient, ie never leave the truncation/quotient implicit.}

\begin{corollary} \label{cor-bso3trunc}
The minimal 4-truncation of $BSO(3)$ is the 4-cell complex $(S^2 \cup_2 e_3) \cup_q e_4 \cup_{\frac{\eta q}{2}} e_5$, where $q: S^3 \ra S^2$ and $\frac{\eta q}{2}: S^4 \ra (S^2 \cup_2 e_3)$ are generators.
\end{corollary}

\begin{proof}
By the proposition, the map $P := BSO(2) \cup_{S^2} BOrp(3) \ra BSO(3)$ is a 4-equivalence.  Note that $_4 BSO(2) \cup_{S^2} {}_4 BOrp(3)$ is a 4-truncation of $P$ and therefore provides a 4-truncation of $BSO(3)$.  Using Proposition~\ref{prop-borptrunc} the cell structure of that space is as described.  

That this truncation is uniquely minimal is seen as follows.  The first four homology groups of $BSO(3)$ force any truncation to have at least one cell in each of the dimensions 2, 3, and 4.  If the truncation has only three cells, the attaching map of the 3-cell must be multiplication by 2, and the third homotopy group of $BSO(3)$ forces the attaching map of the 4-cell to be a generator; call the resulting complex $H := (S^2 \cup_2 e_3) \cup_q e_4$.  The complex $H$ cannot be a 4-truncation of $BSO(3)$ because $\pi_4(BSO(3)) = \ZZ$ whereas $\pi_4(H)$ contains a nontrivial torsion element, namely the image of $\frac{\eta q}{2}$ under the map $\pi_4(\Sigma \RP^2) \xra{\mathrm{inc}_*} \pi_4(H)$.  (This element is nontrivial because the composite $\ZZ/4 = \pi_4(\Sigma \RP^2) \xra{\mathrm{inc}_*} \pi_4(H) \ra \pi_4(S^3) = \ZZ/2$ is surjective, where the map $H \ra S^3$ kills the 2- and 4-cells.)

We therefore know that a minimal truncation must have exactly 4-cells.  If it has two 2-cells or two 3-cells, the homology in dimension 2 or 3 respectively will not match that of $BSO(3)$, preventing it from being a 4-truncation.  If the putative 4-cell truncation $T$ has two 4-cells, the homotopy exact sequence of the pair $(BSO(3), T)$ shows that $\pi_4(T) \ra \pi_4(BSO(3))$ cannot be an isomorphism.  A minimal truncation must therefore have a single cell in each of dimensions 2, 3, 4, and 5; the second and third homotopy groups of $BSO(3)$ force the 4-skeleton of this truncation to be $(S^2 \cup e_3) \cup_q e_4$ as before, and in order to reproduce $\pi_4(BSO(3))$, the 5-cell must be attached by plus or minus the image of the class $\frac{\eta q}{2}$ under the map $\pi_4(\Sigma \RP^2) \xra{\mathrm{inc}_*} \pi_4(H)$, as desired.  % Notice that the class \eta q / 2 cannot be divisible in  H because then it wouldn't map nontrivially onto \pi_4 S^3.  So it is nontrivial and you have to kill it, only way to do that is plus or minus the class itself.
\end{proof}

Because the fourth and fifth cohomology groups of $BQuad$ are trivial, the composite $BQuad \ra BSO(2) \ra BSO(3) \xra{p_1} K(\ZZ,4)$ is trivial and the map $BQuad \ra BSO(3)$ lifts uniquely to a map $BQuad \ra BOrp(3)$.

\begin{corollary}
The commutative square
\begin{equation} \label{quad-pushout} \tag{Q}
\xymatrix{
BQuad \ar[r] \ar[d] & BOrp(3) \ar[d] \\
BSO(2) \ar[r] & BSO(3)
}
\end{equation}
is also a $4$-pushout.
\end{corollary}
\begin{proof}
Though the map $B\Omega S^2 \ra BQuad$ is not a 4-equivalence, the difference of 4-types is a single additional 5-cell in $BQuad$, which results in an additional 6-cell of $BSO(2) \cup_{BQuad} BOrp(3)$ compared to $BSO(2) \cup_{B\Omega S^2} BOrp(3)$; that 6-cell does not affect the 4-type of the pushout.
\end{proof}

\nid Though the square~\eqref{s2-pushout} is more elementary, the square~\eqref{quad-pushout} encodes important conceptual information not visible in~\eqref{s2-pushout}.  Specifically, a cellular model of the pushout $BSO(2) \cup_{BQuad} BOrp(3)$ has, in addition to cells $e_2$, $e_3$, $e_4$, and $e_5$ described in Corollary~\ref{cor-bso3trunc}, a 6-cell $e_6$ attached as follows.  First note that the 5-cell of $BQuad$ maps by multiplication by $2$ to the 5-cell of $BOrp(3)$ and by the cone on $\eta$ to the 4-cell of $BSO(2)$.  % This is not at all clear, but is forced if you know 2 gamma = eta q.  Maybe rephrase in a less flippant way?
Now decompose the boundary $\partial e_6$ as the union $a_5 \cup (S^4 \times I) \cup b_5$, for 5-cells $a_5$ and $b_5$.  The cell $a_5$ wraps twice onto $e_5$, and the cell $b_5$ maps by the cone on $\eta: S^4 \ra S^3$ onto $e_4$, while the cylinder $S^4 \times I$ performs a homotopy between $2 \gamma : S^4 \ra \Sigma \RP^2$ and $\eta q: S^4 \ra S^2$ in $\Sigma \RP^2$.  In words, the 4-cell of $BSO(3)$ trivializes $q$ while the 5-cell trivializes $\gamma$, and the 6-cell witnesses the equivalence of $\eta$ times the trivialization of $q$ with $2$ times the trivialization of $\gamma$.

\begin{remark}
There is a commutative square (but not a 4-pushout),
\begin{equation}\label{quadmoore} \tag{M}
\xymatrix{
B\Omega S^2 \ar[r] \ar[d] & B\Omega \Sigma \RP^2 \ar[d] \\
BQuad \ar[r] & BOrp(3)
}
\end{equation}
% It isn't a 4-pushout because the pushout is Sigma RP2 with a 5-cell killing eta q, but you know eta q is twice the generator of pi_4 Sigma RP2, so there is pi_4 left in the pushout, which is no present in BOrp(3).
We saw that $B\Omega S^2$ and $BQuad$ were interchangable in the pushout square.  The space $BOrp(3)$ cannot be similarly replaced by $B\Omega \Sigma \RP^2$.  Nevertheless it will sometimes be useful to consider building an $Orp(3)$ action by first building an $\Omega \Sigma \RP^2$ action.
\end{remark}

\begin{remark}
We could have more briefly described a 4-truncation of $BSO(3)$ without reference to $BOrp(3)$, $BSO(2)$, $BQuad$, and whatnot.  But it will be important for our applications to topological field theories associated to fusion categories that we understand the 4-truncations of those structure groups as well, and the exact relationship of those truncations to that of $BSO(3)$.
\end{remark}

\CDcomm{Later on, we'll describe the meaning of a Quad action, etc, and then refer back to see that S leads to one description, while Q leads to a slightly lengthier description that ties together the trivialization of q and the condition on eta q / 2.}

\section{Homotopy actions}

\CDcomm{This section describes the meaning of a homotopy action for each group.  This is where the explicit translation to computable, i.e. categorically readable, expressions occurs.  Also, for each group, the map $G \ra Aut(X)$ is described. [Well, if you only describe that map for $SO(3)$, say so.]}

\begin{definition} \label{def-hoaction}
Given a space $B$ and a fibration $F \ra E \ra B$ with $E$ contractible, a homotopy action of $F$ on a space $X$ is a map $\alpha: B \ra B\Aut(X)$, where $\Aut(X)$ is the monoid of self-homotopy equivalences of $X$.
\end{definition}
\nid We will often refer to a homotopy action of $F$ simply as an ``action of $F$" or an ``$F$-action", as we will never be concerned with literal actions of groups.  We may also refer to this as an $F'$-action, when $F'$ is equipped with some implicit homotopy equivalence to $F$, or indeed when the whole fibration $F \ra E \ra B$ is itself implicit.  Moreover, we will use $G$ and $BG$ as generic symbols for the fibre $F$ and base $B$ of the fibration $F \ra E \ra B$ of definition~\ref{def-hoaction}, even though $G$ may not be a group and $BG$ may not be the classifying space of a group.

In case $G$ is an honest group with an honest action on $X$ given by $h: G \ra \Aut(X)$ (which is therefore forced to land in the subgroup of $\Aut(X)$ of homeomorphisms), then the corresponding homotopy action is the map $Bh: BG \ra B\Aut(X)$.


\subsection{By the groups $\Omega S^2$ and $\Quad$} \label{sec-quadaction}

\begin{proposition}
The space of $\Omega S^2$ actions on a 3-type $X$ is homotopy equivalent to $\Omega \Aut(X)$; in particular, the set of homotopy classes of actions is $\pi_1(\Aut(X))$.
\end{proposition}

\begin{proof}
By definition a homotopy action of $\Omega S^2$ on a 3-type $X$ is a map $B \Omega S^2 \ra B\Aut(X)$.  That map is a point of $\Omega^2 B\Aut(X) \simeq \Omega \Aut(X)$.
\end{proof}

\nid We typically use $S$ to denote a based map $S^1 \ra \Aut(X)$ determining the $\Omega S^2$ action on $X$, and we use $[S] \in \pi_1(\Aut(X))$ to denote the corresponding homotopy class.

\CDcomm{Um, what is a cellular 3-truncation of $\Omega \Sigma S^1$? ... $\Omega S^2 \ra \Aut(X)$. Omit??}

As mentioned, the map $S : S^1 \ra \Aut(X)$ can be viewed as a point $S \in \Omega \Aut(X)$.  The space $\Omega \Aut(X)$ is an $E_2$-space.  Letting $\ast$ denote automorphism composition and $\cdot$ denote loop composition, there is, for any two points $a, b \in \Omega \Aut(X)$ a well-defined-up-to-homotopy path $\beta_{a,b}$ from $a \cdot b$ to $b \cdot a$ implementing a half (righthanded) braiding.  Using this braiding, we may form the following loop in $\Omega \Aut(X)$:
\begin{equation} \label{eq-q}
q(S) := (I \ra S \cdot S^{-1} \xra{\beta_{S,S^{-1}}} S^{-1} \cdot S \ra I) \in \Omega (\Omega \Aut(X))
\end{equation}
Here $I$ denotes the trivial loop at the identity automorphism.  The first map of that composite is the standard loop-retraction null-homotopy of the composite of the loop $S \in \Omega \Aut(X)$ with its reverse loop $S^{-1}$, and the last map is similar for the composite of $S^{-1}$ with $S$.  Altogether, this construction provides a map $\pi_1(\Aut(X)) \xra{q} \pi_2(\Aut(X))$, and in fact a map $\Omega \Aut(X) \xra{q} \Omega^2 \Aut(X)$.  We will refer to this as the figure-8 construction, as it imitates the following normally framed embedding of $S^1$ in $\RR^3$:
\begin{equation} \nn
\begin{tikzpicture}
\draw[linestyle,looseness=1.25]
(-.75,0) to [out=-90,in=-120] (0,0)
	to [out=60,in=90] (.75,0);
\draw[linestylegray,coverline,looseness=1.25]
(.75,0) to [out=-90,in=-60] (0,0)
	to [out=120,in=90] (-.75,0);
\end{tikzpicture}
\end{equation} 
%\begin{tikzpicture}
%\draw[linestyle,fuzzleft,looseness=1.25]
%(-.75,0) to [out=-90,in=-120] (0,0)
%	to [out=60,in=90] (.75,0);
%\draw[linestyle,coverlineleft,fuzzleft,looseness=1.25]
%(.75,0) to [out=-90,in=-60] (0,0)
%	to [out=120,in=90] (-.75,0);
%\end{tikzpicture}
%\end{equation} 
%\CD{I do not know how to fix the tikz issue here --- the pre function and the raise function appear to be incompatible.}
% Note: the vertical direction in this picture is the loop direction, and the automorphism direction is into the page, and left right is the second loop direction.
%The gray fuzz indicates the first normal frame vector, and the second normal frame vector points uniformly out of the page.
The normal framing is implicitly determined as follows: the first normal vector is approximately in the plane of the page, pointing inward on the left side of the figure and outward on the right side, and the second normal vector points uniformly out of the page.  (Here the black line segment indicates the portion that can be interpreted as $S \in \Omega \Aut(X)$ and the gray line segment indicates the portion that can be interpreted as $S^{-1}$.)


\begin{lemma} \label{lemma-q}
The figure-8 constuction~\eqref{eq-q} represents precomposition with a generating element $q: S^3 \ra S^2$ of $\pi_3(S^2)$.  Specifically, the following square commutes up to homotopy, where ``$q$" refers to the above construction, and ``$q^\ast$" refers to pullback along the element $q$:
\begin{equation} \nn
\xymatrix{
\Omega^2 B\Aut(X) \ar[r]^{q^\ast} \ar@{<->}[d]_{\simeq} & \Omega^3 B\Aut(X)  \ar@{<->}[d]^{\simeq} \\
\Omega \Aut(X) \ar[r]_{q} & \Omega^2 \Aut(X)
}
\end{equation}
\end{lemma}

\begin{proof}
In fact, the square commutes with any space $Y$ in place of $B\Aut(X)$ and correspondingly $\Omega Y$ in place of $\Aut(X)$.  Because both the constructions $q$ and $q^\ast$ are natural, to prove the general case, it suffices to check in the universal case of $Y = S^2$ on the universal point $(S^2 \xra{\id} S^2) \in \Omega^2 S^2$.  In that case, construction~\eqref{eq-q} is the Pontryagin-Thom image of the normally framed embedding of $S^1$ into $\RR^3$ depicted above; in words, take the normally framed figure-8 immersion of $S^1$ into $\RR^2$, compose with the inclusion $\RR^2 \ra \RR^3$ equipped with the trivial normal framing, and deform the composite to an embedding (having a righthanded braid).  That normally framed embedding does indeed represent a generator of $\pi_3(S^2)$ as required.
%NB Though this might look like it only checks on $\pi_0$, this does provide a proof of the commutativity of the square.  Specifically, you choose a single universal homotopy in the universal case, and then the image of that homotopy under a map \Omega^2 S^2 \ra \Omega^2 Y is a continuously varying homotopy showing the diagram commutes.
\end{proof}

We may play the same game again.  We have $q(S) \in \Omega^2 \Aut(X)$.  Though it does not matter particularly, for definiteness let $\cdot$ as before denote the first loop composition, and $\circ$ now denote the second loop composite.  The space $\Omega^2 \Aut(X)$ is an $E_3$-space and therefore has a path $\tau_{a,b}$ from $a \circ b$ to $b \circ a$ implementing the symmetric switch.  For any point $A \in \Omega^2 \Aut(X)$ we may form the following composite
\begin{equation} \label{eq-eta}
\eta(A) := (I \ra A \circ A^{-1} \xra{\tau_{A,A^{-1}}} A^{-1} \circ A \ra I) \in \Omega(\Omega^2 \Aut(X))
\end{equation}
Here $I$ now denotes the trivial double loop at the identity automorphism.  This construction provides a map $\pi_2(\Aut(X)) \xra{\eta} \pi_3(\Aut(X))$, in fact a map $\Omega^2 \Aut(X) \xra{\eta} \Omega^3 \Aut(X)$, which may also be called the figure-8 construction.

\begin{lemma} \label{lemma-eta}
The figure-8 construction~\eqref{eq-eta} represents precomposition with a nontrivial element $\eta: S^4 \ra S^3$ of $\pi_4(S^3)$.  More precisely, the following square commutes up to homotopy, where ``$\eta$" refers to the above construction, and ``$\eta^\ast$" refers to pullback along the element $\eta$:
\begin{equation} \nn
\xymatrix{
\Omega^3 B\Aut(X) \ar[r]^{\eta^\ast} \ar@{<->}[d]_{\simeq} & \Omega^4 B\Aut(X)  \ar@{<->}[d]^{\simeq} \\
\Omega^2 \Aut(X) \ar[r]_{\eta} & \Omega^3 \Aut(X)
}
\end{equation}
\end{lemma}

\begin{proof}
As for the previous lemma, it suffices to check for the universal space, here $S^3$ in place of $B\Aut(X)$, and on the universal point, here $(S^3 \xra{\id} S^3) \in \Omega^3 S^3$.  In that case, construction~\eqref{eq-eta} is the Pontryagin-Thom image of the following normally framed embedding of $S^1$ into $\RR^4$: take the normally framed embedding of $S^1$ into $\RR^3$ from the proof of Lemma~\ref{lemma-q} and include it into $\RR^4$, adding a trivial normal framing direction.  This normally framed manifold represents a generator of $\pi_4(S^3)$ as required.
\end{proof}

We can now describe homotopy $Quad$ actions.

\begin{proposition} \label{prop-quadaction}
The space of $Quad$ actions on a 3-type $X$ is homotopy equivalent to the union of the components of $\Omega \Aut(X)$ corresponding to elements $[S] \in \pi_1(\Aut(X))$ such that $\eta(q([S])) = 0 \in \pi_3(\Aut(X))$, that is such that the double figure-8 construction on $[S]$ vanishes.
\end{proposition}
\begin{proof}
By Proposition~\ref{prop-quadtrunc}, the attaching map of the 5-cell of $BQuad$ is $\eta q: S^4 \ra S^2$, where $\eta : S^4 \ra S^3$ is nontrivial and $q: S^3 \ra S^2$ is a generator.  It suffices therefore to know that the double figure-8 construction $\pi_1(\Aut(X)) \xra{q \eta} \pi_3(\Aut(X))$ is equal to the pullback $\pi_2(B\Aut(X)) \xra{(\eta q)^\ast} \pi_4(B\Aut(X))$ once $\pi_{k}(\Aut(X))$ and $\pi_{k+1}(B\Aut(X))$ are identified.  This follows from Lemmas~\ref{lemma-q} and~\ref{lemma-eta}.
\end{proof}


\CDcomm{What is a cellular 3-truncation of $Quad$? ... $Quad \ra \Aut(X)$.  Maybe omit.}

\subsection{By the group $SO(2)$}

\begin{prop} \label{prop-so2action}
The homotopy classes of $SO(2)$ actions on a 3-type $X$ are the homotopy classes of pairs $(S,W)$, where $S: S^1 \ra \Aut(X)$ is a map and $W$ is a null homotopy of $q(S): S^2 \ra \Aut(X)$, the figure-8 construction applied to $S$.  More precisely, the space of $SO(2)$ actions on $X$ is homotopy equivalent to the homotopy pullback $\Omega \Aut(X) \times_{\Omega^2 \Aut(X)} \ast$, where the map is $\Omega \Aut(X) \xra{q} \Omega^2 \Aut(X)$.
\end{prop}
% The construction of q really does provide a map from \Omega \Aut(X) to \Omega^2 \Aut(X) so the statement of the proposition makes sense.
\begin{proof}
As mentioned, the 4-truncation of $BSO(2)$ is $S^2 \cup_q e_4$.  An $SO(2)$ action is therefore a map $S^2 \cup_q e_4 \ra B\Aut(X)$.  On the 2-cell, this map is provided by $S \in \Omega \Aut(X) \simeq \Omega^2 B\Aut(X)$.  An extension of the map $S^2 \xra{S} B\Aut(X)$ over the 4-cell is a null homotopy of the composite $S^3 \xra{q} S^2 \xra{S} B\Aut(X)$, that is a path in $\Omega^3 B\Aut(X)$ from $q^\ast S$ to the identity.  By Lemma~\ref{lemma-q}, providing such a null homotopy is equivalent to providing a null homotopy of $q(S) \in \Omega^2\Aut(X) \simeq \Omega^3 B\Aut(X)$.
\end{proof}

We can restrict attention to 3-types that happen to be 2-types.
\begin{corollary}
The homotopy classes of $SO(2)$ actions on a 2-type $X$ are the homotopy classes of map $S: S^1 \ra \Aut(X)$ such that the figure-8 construction applied to $S$ is null.
\end{corollary}

\subsection{By the groups $\Omega \Sigma \RP^2$ and $Orp(3)$}

Given a loop $S: S^1 \ra \Aut(X)$ of automorphisms of $X$, we can form $S \ast S : S^1 \ra \Aut(X)$ by sending each point $p \in S^1$ to the square $S(p) \ast S(p)$ of the automorphism $S(p) \in \Aut(X)$, or we can form $S \cdot S : S^1 \ra \Aut(X)$ by taking the loop composition of $S$ with itself.  Of course, $S \ast S$ and $S \cdot S$ are homotopic maps $S^1 \ra \Aut(X)$, but there is no canonical homotopy between them so it will occasionally be important to distinguish them.  We will use the expression ``$2S$" as shorthand for $S \ast S$.
\CDcomm{What is the group of self-homotopy equivalences of $\Sigma \RP^2$?  If it has more than just reflections, then there is a subtlety in the identification of $\Omega \Sigma \RP^2$ actions that we should address/comment on.}

\begin{prop} \label{prop-sigmarp2action}
The homotopy classes of $\Omega \Sigma \RP^2$ actions on a 3-type $X$ are the homotopy classes of pairs $(S,R)$, where $S: S^1 \ra \Aut(X)$ is a map and $R$ is a null homotopy of $2S: S^1 \ra \Aut(X)$.  More precisely, the space of actions is homotopy equivalent to the homotopy pullback $\Omega \Aut(X) \,_2\!\times_{\Omega \Aut(X)} \ast$, where the subscript $2$ indicates that map is $\Omega \Aut(X) \xra{2} \Omega \Aut(X)$.
\end{prop}
% For this to not involve a choice, you need the attaching map of \Sigma \RP^2 to have been doubled in the automorphism direction, that is in the "extra" loop direction.  If it had been doubled in one of the other directions, then there is a left or right choice in the identification of the proposition.  Weird. (??)  Actually, if there aren't any interesting self-equivalences of $\Sigma \RP^2$ then there isn't actually an issue.
\begin{proof}
This is immediate, as multiplication by $2$ commutes through the identification $\Omega \Aut(X) \simeq \Omega^2 B\Aut(X)$.
\end{proof}

Recall from Proposition~\ref{prop-borptrunc} that the 4-truncation of $BOrp(3)$ is $(S^2 \cup_2 e_3) \cup_{\frac{\eta q}{2}} e_5$.  From Proposition~\ref{prop-sigmarp2action}, we know that a $\Sigma \RP^2 = S^2 \cup_2 e_3$ action is a point of $\Map(\Sigma \RP^2,B\Aut(X)) \simeq \Omega\Aut(X) \,_2\! \times_{\Omega \Aut(X)} \ast$.  To precisely describe a full $Orp(3)$ action, we need to provide a geometric construction of a map $\Omega\Aut(X) \,_2\! \times_{\Omega \Aut(X)} \ast \ra \Omega^3 \Aut(X)$ such that the composite $\Map(\Sigma \RP^2,B\Aut(X)) \simeq \Omega\Aut(X) \,_2\! \times_{\Omega \Aut(X)} \ast \ra \Omega^3 \Aut(X) \simeq \Omega^4 B\Aut(X)$ is pullback along $\frac{\eta q}{2} \in \pi_4(\Sigma\RP^2)$.  

We first revisit the corresponding construction for $\eta q \in \pi_4(S^2)$.  Given an element $S \in \Omega \Aut(X)$, recall from Section~\ref{sec-quadaction} that we can read the standard normally framed figure 8 embedding of $S^1$ in $\RR^3$ by performing the element $S$ in a transverse 2-plane slice along the curve, to obtain construction~\eqref{eq-q}, a map $\Omega \Aut(X) \xra{q} \Omega^2 \Aut(X)$.  We can similarly read the following normally framed embedding of $S^1 \times S^1$ into $\RR^4$ as representing a map $\Omega \Aut(X) \xra{\eta q} \Omega^2 \Aut(X)$:
\begin{equation} \label{eq-etaq}
\emptyset 
\leadsto
\cb{\begin{tikzpicture}
\draw[linestyle,looseness=1.25]
(0,0) to [out=-90,in=-90] (.5,0);
\draw[linestylegray,looseness=1.25]
(.5,0) to [out=90,in=90] (0,0);
\end{tikzpicture}}
\leadsto
\cb{\begin{tikzpicture}
\draw[linestyle,looseness=1.25]
(0,0) to [out=-90,in=-120] (.5,0)
	to [out=60,in=120] (1,0)
	to [out=-60,in=-90] (1.5,0);
\draw[linestylegray,coverline,looseness=1.25]
(1.5,0) to [out=90,in=60] (1,0)
	to [out=-120,in=-60] (.5,0)
	to [out=120,in=90] (0,0);
\end{tikzpicture}}
\leadsto
\cb{\begin{tikzpicture}
\draw[linestyle,looseness=1.25]
(0,0) to [out=-90,in=-120] (.5,0)
	to [out=60,in=90] (1,0);
\draw[linestylegray,coverline,looseness=1.25]
(1,0) to [out=-90,in=-60] (.5,0)
	to [out=120,in=90] (0,0);
\draw[linestyle,looseness=1.25]
(2.25,0) to [out=-90,in=-60] (1.75,0)
	to [out=120,in=90] (1.25,0);
\draw[linestylegray,coverline,looseness=1.25]
(1.25,0) to [out=-90,in=-120] (1.75,0)
	to [out=60,in=90] (2.25,0);
\end{tikzpicture}}
\stackrel{\lcurvearrowse}{\leadsto}
\cb{\begin{tikzpicture}
\draw[linestyle,looseness=1.25]
(0,0) to [out=-90,in=-120] (.5,0)
	to [out=60,in=90] (1,0);
\draw[linestylegray,coverline,looseness=1.25]
(1,0) to [out=-90,in=-60] (.5,0)
	to [out=120,in=90] (0,0);
\draw[linestyle,looseness=1.25]
(1,-.5) to [out=-90,in=-60] (.5,-.5)
	to [out=120,in=90] (0,-.5);
\draw[linestylegray,coverline,looseness=1.25]
(0,-.5) to [out=-90,in=-120] (.5,-.5)
	to [out=60,in=90] (1,-.5);
\end{tikzpicture}}
\stackrel{\lcurvearrowsw}{\leadsto}
\cb{\begin{tikzpicture}
\draw[linestyle,looseness=1.25]
(1,0) to [out=-90,in=-60] (.5,0)
	to [out=120,in=90] (0,0);
\draw[linestylegray,coverline,looseness=1.25]
(0,0) to [out=-90,in=-120] (.5,0)
	to [out=60,in=90] (1,0);
\draw[linestyle,looseness=1.25]
(1.25,0) to [out=-90,in=-120] (1.75,0)
	to [out=60,in=90] (2.25,0);
\draw[linestylegray,coverline,looseness=1.25]
(2.25,0) to [out=-90,in=-60] (1.75,0)
	to [out=120,in=90] (1.25,0);
\end{tikzpicture}}
\leadsto
\cb{\begin{tikzpicture}
\draw[linestyle,looseness=1.25]
(1.5,0) to [out=90,in=60] (1,0)
	to [out=-120,in=-60] (.5,0)
	to [out=120,in=90] (0,0);
\draw[linestylegray,coverline,looseness=1.25]
(0,0) to [out=-90,in=-120] (.5,0)
	to [out=60,in=120] (1,0)
	to [out=-60,in=-90] (1.5,0);
\end{tikzpicture}}
\leadsto
\cb{\begin{tikzpicture}
\draw[linestylegray,looseness=1.25]
(0,0) to [out=-90,in=-90] (.5,0);
\draw[linestyle,looseness=1.25]
(.5,0) to [out=90,in=90] (0,0);
\end{tikzpicture}}
\leadsto
\emptyset
\end{equation} 
Here the normal framing is implicit and as before is determined as follows: the first nontrivial picture on the left has first normal vector pointing toward the center of the circle and second normal vector pointing out of the page; for the rest of the pictures, the first normal is determined by planar deformation from the first picture, and the second normal continues to point out of the page.  Also as before, the black line segments indicate the element $S \in \Omega \Aut(X)$, and the gray line segments indicate the element $S^{-1}$, where the inverse is taken in the loop direction.  That the map $\Omega \Aut(X) \xra{\eta q} \Omega^3 \Aut(X)$ produced by the above picture is given by pullback along $\eta q \in \pi_4(S^2)$ is the content of Lemmas~\ref{lemma-q} and ~\ref{lemma-eta}.

Given a point of $\Omega\Aut(X) \,_2\! \times_{\Omega \Aut(X)} \ast$, we have both a point $S \in \Omega\Aut(X)$ and a null homotopy $R$ of $2S$.  We represent this null homotopy and its inverse graphically by 
$\cb{
\begin{tikzpicture}
\draw[linestyle,looseness=1.75] (0,.15) to [out=0,in=0] (0,-.15);
\end{tikzpicture}
}$
and
$\cb{
\begin{tikzpicture}
\draw[linestyle,looseness=1.75] (0,.15) to [out=180,in=180] (0,-.15);
\end{tikzpicture}
}$
respectively.  Combining this notation with the previous notation of a normally framed surface, the following picture encodes a map $\Omega\Aut(X) \,_2\! \times_{\Omega \Aut(X)} \ast \xra{\frac{\eta q}{2}} \Omega^3 \Aut(X)$:
\begin{equation} \label{eq-etaq2}
\emptyset 
\leadsto
\cb{\begin{tikzpicture}
\draw[linestyle,looseness=1.25]
(0,0) to [out=-90,in=-90] (.5,0);
\draw[linestylegray,looseness=1.25]
(.5,0) to [out=90,in=90] (0,0);
\end{tikzpicture}}
\leadsto
\cb{\begin{tikzpicture}
\draw[linestyle,looseness=1.25]
(0,0) to [out=-90,in=-120] (.5,0)
	to [out=60,in=120] (1,0)
	to [out=-60,in=-90] (1.5,0);
\draw[linestylegray,coverline,looseness=1.25]
(1.5,0) to [out=90,in=60] (1,0)
	to [out=-120,in=-60] (.5,0)
	to [out=120,in=90] (0,0);
\end{tikzpicture}}
\leadsto
\cb{\begin{tikzpicture}
\draw[linestyle,looseness=1.25]
(0,0) to [out=-90,in=-120] (.5,0)
	to [out=60,in=90] (1,0);
\draw[linestylegray,coverline,looseness=1.25]
(1,0) to [out=-90,in=-60] (.5,0)
	to [out=120,in=90] (0,0);
\draw[linestyle,looseness=1.25]
(2.25,0) to [out=-90,in=-60] (1.75,0)
	to [out=120,in=90] (1.25,0);
\draw[linestylegray,coverline,looseness=1.25]
(1.25,0) to [out=-90,in=-120] (1.75,0)
	to [out=60,in=90] (2.25,0);
\end{tikzpicture}}
\stackrel{\lcurvearrowse}{\leadsto}
\cb{\begin{tikzpicture}
\draw[linestyle,looseness=1.25]
(0,0) to [out=-90,in=-120] (.5,0)
	to [out=60,in=90] (1,0);
\draw[linestylegray,coverline,looseness=1.25]
(1,0) to [out=-90,in=-60] (.5,0)
	to [out=120,in=90] (0,0);
\draw[linestyle,looseness=1.25]
(1,-.5) to [out=-90,in=-60] (.5,-.5)
	to [out=120,in=90] (0,-.5);
\draw[linestylegray,coverline,looseness=1.25]
(0,-.5) to [out=-90,in=-120] (.5,-.5)
	to [out=60,in=90] (1,-.5);
\end{tikzpicture}}
\leadsto
\cb{\begin{tikzpicture}
\draw[linestyle,looseness=1.25]
(0,0) to [out=-90,in=90] (.15,-.25)
	to [out=-90,in=90] (0,-.5);
\draw[linestyle,looseness=1.25]
(1,0) to [out=90,in=60] (.5,0)
	to [out=-120,in=90] (.3,-.25)
	to [out=-90,in=120] (.5,-.5)
	to [out=-60,in=-90] (1,-.5);
\draw[linestylegray,coverline,looseness=1.25]
(0,0) to [out=90,in=120] (.5,0)
	to [out=-60,in=-90] (1,0);
\draw[linestylegray,coverline,looseness=1.25]
(0,-.5) to [out=-90,in=-120] (.5,-.5)
	to [out=60,in=90] (1,-.5);
\end{tikzpicture}}
\leadsto
\cb{\begin{tikzpicture}
\draw[linestyle,looseness=1.25]
(.1,0) to [out=90,in=90] (.75,.25);
\draw[linestyle,looseness=1.25]
(.1,0) to [out=-90,in=-90] (.75,-.25);
\draw[linestylegray,looseness=1.75]
(.75,.25) to [out=-90,in=90] (.4,.15);
\draw[linestylegray,looseness=1.75]
(.75,-.25) to [out=90,in=-90] (.4,-.15);
\draw[linestyle,looseness=1.25]
(.4,.15) to [out=-90,in=90] (1,0)
	to [out=-90,in=90] (.4,-.15);
\end{tikzpicture}}
\leadsto
\cb{\begin{tikzpicture}
\draw[linestyle,looseness=1.25]
(0,0) to [out=-90,in=-90] (.5,0);
\draw[linestyle,looseness=1.25]
(.5,0) to [out=90,in=90] (0,0);
\end{tikzpicture}}
\leadsto
\emptyset
\end{equation}
\nid We refer to this as the figure-32 construction.

\begin{lemma} \label{lemma-etaq2}
The figure-32 construction~\eqref{eq-etaq2} represents precomposition with a generator $\frac{\eta q}{2}: S^4 \ra \Sigma \RP^2$ of $\pi_4(\Sigma \RP^2)$.  More precisely, the following square commutes up to homotopy, where $\frac{\eta q}{2}$ refers to the above construction, and $(\frac{\eta q}{2})^\ast$ refers to pullback along the element $\frac{\eta q}{2}$:
\begin{equation} \nn
\xymatrix{
\Map(\Sigma \RP^2,B\Aut(X)) \ar[r]^-{(\frac{\eta q}{2})^\ast} \ar@{<->}[d]_{\simeq} & \Omega^4 B\Aut(X)  \ar@{<->}[d]^{\simeq} \\
\Omega\Aut(X) \,_2\! \times_{\Omega \Aut(X)} \ast \ar[r]_-{\frac{\eta q}{2}} & \Omega^3 \Aut(X)
}
\end{equation}
\end{lemma}
\begin{proof}
By Lemmas~\ref{lemma-q} and~\ref{lemma-eta}, we know that construction~\eqref{eq-etaq} represents precomposition with $\eta q : S^4 \ra S^2 \xra{\text{inc}} \Sigma\RP^2$.  Observe that construction~\eqref{eq-etaq} is homotopic to the sum of construction~\eqref{eq-etaq2} with the following construction:
\begin{equation} \label{eq-etaq2end}
\emptyset 
\leadsto
\cb{\begin{tikzpicture}
\draw[linestyle,looseness=1.25]
(0,0) to [out=-90,in=-90] (.5,0);
\draw[linestyle,looseness=1.25]
(.5,0) to [out=90,in=90] (0,0);
\end{tikzpicture}}
\leadsto
\cb{\begin{tikzpicture}
\draw[linestyle,looseness=1.25]
(.1,0) to [out=90,in=90] (.75,.25);
\draw[linestyle,looseness=1.25]
(.1,0) to [out=-90,in=-90] (.75,-.25);
\draw[linestylegray,looseness=1.75]
(.75,.25) to [out=-90,in=90] (.4,.15);
\draw[linestylegray,looseness=1.75]
(.75,-.25) to [out=90,in=-90] (.4,-.15);
\draw[linestyle,looseness=1.25]
(.4,.15) to [out=-90,in=90] (1,0)
	to [out=-90,in=90] (.4,-.15);
\end{tikzpicture}}
\leadsto
\cb{\begin{tikzpicture}
\draw[linestyle,looseness=1.25]
(0,0) to [out=-90,in=90] (.15,-.25)
	to [out=-90,in=90] (0,-.5);
\draw[linestyle,looseness=1.25]
(1,0) to [out=90,in=60] (.5,0)
	to [out=-120,in=90] (.3,-.25)
	to [out=-90,in=120] (.5,-.5)
	to [out=-60,in=-90] (1,-.5);
\draw[linestylegray,coverline,looseness=1.25]
(0,0) to [out=90,in=120] (.5,0)
	to [out=-60,in=-90] (1,0);
\draw[linestylegray,coverline,looseness=1.25]
(0,-.5) to [out=-90,in=-120] (.5,-.5)
	to [out=60,in=90] (1,-.5);
\end{tikzpicture}}
\leadsto
\cb{\begin{tikzpicture}
\draw[linestyle,looseness=1.25]
(0,0) to [out=-90,in=-120] (.5,0)
	to [out=60,in=90] (1,0);
\draw[linestylegray,coverline,looseness=1.25]
(1,0) to [out=-90,in=-60] (.5,0)
	to [out=120,in=90] (0,0);
\draw[linestyle,looseness=1.25]
(1,-.5) to [out=-90,in=-60] (.5,-.5)
	to [out=120,in=90] (0,-.5);
\draw[linestylegray,coverline,looseness=1.25]
(0,-.5) to [out=-90,in=-120] (.5,-.5)
	to [out=60,in=90] (1,-.5);
\end{tikzpicture}}
\stackrel{\lcurvearrowsw}{\leadsto}
\cb{\begin{tikzpicture}
\draw[linestyle,looseness=1.25]
(1,0) to [out=-90,in=-60] (.5,0)
	to [out=120,in=90] (0,0);
\draw[linestylegray,coverline,looseness=1.25]
(0,0) to [out=-90,in=-120] (.5,0)
	to [out=60,in=90] (1,0);
\draw[linestyle,looseness=1.25]
(1.25,0) to [out=-90,in=-120] (1.75,0)
	to [out=60,in=90] (2.25,0);
\draw[linestylegray,coverline,looseness=1.25]
(2.25,0) to [out=-90,in=-60] (1.75,0)
	to [out=120,in=90] (1.25,0);
\end{tikzpicture}}
\leadsto
\cb{\begin{tikzpicture}
\draw[linestyle,looseness=1.25]
(1.5,0) to [out=90,in=60] (1,0)
	to [out=-120,in=-60] (.5,0)
	to [out=120,in=90] (0,0);
\draw[linestylegray,coverline,looseness=1.25]
(0,0) to [out=-90,in=-120] (.5,0)
	to [out=60,in=120] (1,0)
	to [out=-60,in=-90] (1.5,0);
\end{tikzpicture}}
\leadsto
\cb{\begin{tikzpicture}
\draw[linestylegray,looseness=1.25]
(0,0) to [out=-90,in=-90] (.5,0);
\draw[linestyle,looseness=1.25]
(.5,0) to [out=90,in=90] (0,0);
\end{tikzpicture}}
\leadsto
\emptyset
\end{equation} 
Next note that construction~\eqref{eq-etaq2} and the time reversal of construction~\eqref{eq-etaq2end} are vertical reflections of one another; it follows that the sum of \eqref{eq-etaq2} and $-\eqref{eq-etaq2end}$ is null, or said differently, that \eqref{eq-etaq2} and \eqref{eq-etaq2end} are homotopic.  Thus construction~\eqref{eq-etaq2} is a 2-divisor of construction~\eqref{eq-etaq}, and, because $\pi_4(\Sigma \RP^2) = \ZZ/4$, therefore represents $(\frac{\eta q}{2})^\ast$, up to sign.  As we will only be concerned with the vanishing of construction~\eqref{eq-etaq2}, the sign will not matter for us---we leave its determination to the interested reader.
\end{proof}
\CDcomm{[This last sentence about sign doesn't make sense unless we have specified a chosen element $\frac{\eta q}{2}$ of the homotopy group.  Really, Proposition 2.5 should specify the generator we use, not just up to sign, then at least the rest is precise.]  [If we don't have a good way to specify, we can take the figure 32 construction to specify the sign, but then the proposition statements need adjusting.]}

\begin{proposition} \label{prop-orpaction}
The homotopy classes of $Orp(3)$ actions on a 3-type $X$ are the homotopy classes of pairs $(S,R)$, where $S: S^1 \ra \Aut(X)$ is a map and $R$ is a null homotopy of $2S$, such that the figure-32 construction on $(S,R)$ is zero.  More precisely, the space of such actions is homotopy equivalent to the union of components of $\Omega \Aut(X) \,_2\! \times_{\Omega \Aut(X)} \ast$ on which the figure-32 construction vanishes.
\end{proposition}
\begin{proof}
This follows by combining Proposition~\ref{prop-borptrunc} and Lemma~\ref{lemma-etaq2}.
\end{proof}

\CDcomm{A cellular 3-truncation of $Orp(3)$ looks a little messy (cohom is $\ZZ, 0, \ZZ/2, \ZZ/2, \ZZ/2, \ZZ/4$ I believe.  Maybe omit.}

\subsection{By the group $SO(3)$}

\begin{theorem} \label{thm-so3action}
The homotopy classes of $SO(3)$ actions on a 3-type $X$ are the homotopy classes of triples $(S,R,W)$, where 
\begin{itemize}
\item $S: S^1 \ra \Aut(X)$ is a based loop of automorphisms of $X$,
\item $R: D^2 \ra \Aut(X)$ is a null homotopy of $S \ast S: S^1 \ra \Aut(X)$, the pointwise (that is, automorphism) square of $S$,
\item $W: D^3 \ra \Aut(X)$ is a null homotopy of $q(S): S^2 \ra \Aut(X)$, the figure-8 construction~\eqref{eq-q} applied to $S$,
\end{itemize}
such that
\begin{itemize}
\item $\frac{\eta q}{2} (S,R) : S^3 \ra \Aut(X)$, that is the figure-32 construction~\eqref{eq-etaq2} applied to $(S,R)$, is null homotopic.
\end{itemize}
\end{theorem}
\nid By this point needless to say, the space of $SO(3)$ is homotopy equivalent to the space
\begin{equation} \nn
\ast \,_{\Omega^2 \Aut(X)}\! \times_q \left(\Omega \Aut(X) \,_2\! \times_{\Omega \Aut(X)} \ast\right)_{[\frac{\eta q}{2}]=0},
\end{equation}
where the maps from $\Omega \Aut(X)$ are $q$ and $2$ respectively, and the subscript $[\frac{\eta q}{2}]=0$ indicates that the only components where the figure-32 construction vanishes are retained.
\begin{proof}
This follows by combining Corollary~\ref{cor-bso3trunc}, Lemma~\ref{lemma-q}, and Lemma~\ref{lemma-etaq2}.
\end{proof}

\subsection{The map $SO(3) \ra \Aut(X)$}

Given an $SO(3)$ action on $X$, that is a map $\alpha: BSO(3) \ra B\Aut(X)$, we have an associated map of spaces $\Omega \alpha: SO(3) \ra \Aut(X)$.  Roughly speaking, the map $\Omega \alpha$ knows everything about the action, except that it is in fact an action, i.e. a map of (homotopical) groups.  Given a point $p \in X$, we have a map $\Omega \alpha (p) : SO(3) \ra X$; it is sometimes informative to compute this map explicitly, and to that end convenient to know the map $\Omega \alpha$.  

A cellular 3-truncation of $SO(3)$ is, of course, $(S^1 \cup_2 e_2) \cup_p e_3$, where $p: S^2 \ra \RP^2$ is the antipodal quotient.  Given a map $\RP^2 \ra \Aut(X)$, we need to explicitly construct the pullback $S^2 \xra{p} \RP^2 \ra \Aut(X)$.  The map from $\RP^2$ provides a point $S \in \Omega \Aut(X)$ and a null homotopy $R$ of $2S$.  Using the previously described graphical notation for $S$ and $R$, the following picture defines a map $\Map(\RP^2,\Aut(X)) \xra{p} \Omega^2 \Aut(X)$:
\begin{equation} \label{eq-bean}
\cb{\begin{tikzpicture}
\draw[linestyle,out looseness=.75,in looseness=1]
	(-.5,-.25) to [out=-90,in=-90] (.5,0)
	(-.5,.25) to [out=90,in=90] (.5,0);
\draw[linestyle,looseness=.75]
	(-.5,-.25) to [out=90,in=-90] (0,0)
	(-.5,.25) to [out=-90,in=90] (0,0);
\end{tikzpicture}}
\end{equation}
We refer to the construction defined in this diagram~\ref{eq-bean} as ``the Radford bean".  Note well that in this construction the left and right facing cups are applications of the nullhomotopy $R$ of $2S$, not applications of a cancellation between $S$ and $S^{-1}$.  This map $p$ is homotopic to the map given by pullback along the antipodal projection $S^2 \ra \RP^2$, as desired.  (This can be verified by direct geometric argument.)  Theorem~\ref{thm-so3action} provides (in $W$) a null homotopy of construction~\eqref{eq-q} applied to $S$, whereas we seem to need a null homotopy of construction~\eqref{eq-bean} applied to $S$ and $R$.  Though construction~\eqref{eq-bean} is a priori rather different from the map $\Omega\Aut(X) \xra{q} \Omega^2 \Aut(X)$ defined by construction~\eqref{eq-q}, we will see that they are, in fact, closely related.

Note that the following square commutes up to homotopy, where ``$p$" refers to the above construction~\eqref{eq-bean} and ``$(\Sigma p)^\ast$" refers to pullback along the suspension of the antipodal projection $p: S^2 \ra \RP^2$:
\begin{equation} \nn
\xymatrix{
\Map(\Sigma\RP^2,B\Aut(X)) \ar[r]^-{(\Sigma p)^\ast} \ar@{<->}[d]_{\simeq} & \Omega^3 B\Aut(X)  \ar@{<->}[d]^{\simeq} \\
\Map(\RP^2,\Aut(X)) \ar[r]_-{p} & \Omega^2 \Aut(X)
}
\end{equation}
That is, construction~\eqref{eq-bean} represents pullback along $\Sigma p: S^3 \ra \Sigma \RP^2$ in the same sense that construction~\eqref{eq-q} represents pullback along $q: S^3 \ra S^2$.  Comparing the homotopy classes $\Sigma p \in \pi_3(\Sigma \RP^2)$ and $\inc_* q \in \pi_3(\Sigma \RP^2)$, for $\inc: S^2 \ra \Sigma \RP^2$, is therefore equivalent to comparing construction~\eqref{eq-bean} and construction~\eqref{eq-q}.  \CDcomm{(In what follows, I've written as though the Radford bean construction and the suspension of the antipodal projects are \emph{equal}; really, we should in the above paragraph or two \emph{choose} a homotopy between them, and then we can implicitly identify them via that homotopy, and everything that follows is fine.)}

We will need the following lemma:
\begin{lemma}
The map $S^3 \ra \Sigma \RP^2$ defined by the normally framed manifold
\begin{equation} \nn
\cb{\begin{tikzpicture}
\draw[linestyle,out looseness=.75,in looseness=1]
	(-.5,-.25) to [out=-90,in=-90] (.5,0)
	(-.5,.25) to [out=90,in=90] (.5,0);
\draw[linestyle,looseness=.75]
	(-.5,-.25) to [out=90,in=-90] (0,0)
	(-.5,.25) to [out=-90,in=90] (0,0);
\end{tikzpicture}}
\end{equation}
represents the unique nontrivial 2-torsion class in $\pi_3(\Sigma \RP^2) = \ZZ/4$.
\end{lemma}
\begin{proof}
Consider the following local deformation:
\begin{equation} \nn
\cb{\begin{tikzpicture}
\draw[linestyle,looseness=.75]
(0,0) to [out=0,in=-90] (1,.25)
	to [out=90,in=-90] (.5,.75)
	to [out=90,in=180] (1.5,1);
\end{tikzpicture}}
\leadsto
\cb{\begin{tikzpicture}
\draw[linestyle,looseness=.75]
(0,.125) to [out=0,in=-90] (1.25,.25)
	to [out=90,in=-90] (.6,.5);
\draw[linestylegraylight,looseness=.75]
(.6,.5) to [out=90,in=-90] (.9,.75);
\draw[linestyle,looseness=.75]
(.9,.75) to [out=90,in=-90] (.25,1)
	to [out=90,in=180] (1.5,1.125);
\end{tikzpicture}}
=
\cb{\begin{tikzpicture}
\draw[linestyle,looseness=.75]
(-.25,-.125) to [out=0,in=-90] (.5,.125)
	to [out=90,in=-90] (.15,.375);
\draw[linestylegraylight,out looseness=.75,in looseness=1.5]
(.15,.375) to [out=90,in=110] (.75,.125);
\draw[linestylegraylight,out looseness=1.5,in looseness=.75]
(.75,.125) to [out=-70,in=-90] (1.35,-.125);
\draw[linestyle,looseness=.75]
(1.35,-.125) to [out=90,in=-90] (1,.125)
	to [out=90,in=180] (1.75,.375);
\end{tikzpicture}}
\leadsto
\cb{\begin{tikzpicture}
\draw[linestyle,looseness=.75]
(-.75,-.25) to [out=0,in=90] (.5,-.375);
\draw[linestyle,looseness=.75]
(-.5,.375) to [out=-90,in=180] (.75,.25);
\draw[linestylegraylight,coverline,out looseness=.9,in looseness=1.5]
(.5,-.375) to [out=-90,in=-70] (0,0);
\draw[linestylegraylight,coverline,out looseness=1.5,in looseness=.9]
(0,0) to [out=110,in=90] (-.5,.375);
\end{tikzpicture}}
\end{equation}
Applying this deformation on the top part of the bean, we obtain the following homotopy:
\begin{equation} \label{htpybean2e}
\cb{\begin{tikzpicture}
\draw[linestyle,out looseness=.75,in looseness=1]
	(-.5,-.25) to [out=-90,in=-90] (.5,0)
	(-.5,.25) to [out=90,in=90] (.5,0);
\draw[linestyle,looseness=.75]
	(-.5,-.25) to [out=90,in=-90] (0,0)
	(-.5,.25) to [out=-90,in=90] (0,0);
\end{tikzpicture}}
\leadsto
\cb{\begin{tikzpicture}
\draw[linestyle,looseness=.75]
(-.75,-.375) to [out=90,in=90] (.5,-.375);
\draw[linestyle,looseness=.75]
(-.5,.375) to [out=-90,in=90] (.85,-.125);
\draw[linestylegraylight,coverline,out looseness=.9,in looseness=1.5]
(.5,-.375) to [out=-90,in=-70] (0,0);
\draw[linestylegraylight,coverline,out looseness=1.5,in looseness=.9]
(0,0) to [out=110,in=90] (-.5,.375);
\draw[linestyle,out looseness=.75,in looseness=1.5]
(-.75,-.375) to [out=-90,in=-90] (.85,-.125);
\end{tikzpicture}}
\leadsto
\cb{\begin{tikzpicture}
\draw[linestyle,looseness=.75]
(-.75,-.375) to [out=90,in=90] (.5,-.375);
\draw[linestyle,looseness=.75]
(-.5,.375) to [out=-90,in=90] (.85,-.125);
\draw[linestylegraylight,coverline,out looseness=.9,in looseness=1.5]
(.5,-.375) to [out=-90,in=-70] (0,0);
\draw[linestylegraylight,coverline,out looseness=1.5,in looseness=.9]
(0,0) to [out=110,in=90] (-.5,.375);
\draw[linestyle,looseness=.75]
	(-.75,-.375) to [out=-90,in=90] (0,-.5);
\draw[linestylegraylight,looseness=.75]
	(0,-.5) to [out=-90,in=90] (-.5,-.75);
\draw[linestyle,out looseness=.75,in looseness=1.5]
	(-.5,-.75) to [out=-90,in=-90] (.85,-.125);
\end{tikzpicture}}
=
\cb{\begin{tikzpicture}
\draw[linestyle,out looseness=1,in looseness=.75]
	(-.75,-.5) to [out=-90,in=-90] (-.15,0);
\draw[linestyle,looseness=.75]
	(-.15,0) to [out=90,in=-90] (-.75,.5);
\draw[linestyle,out looseness=1,in looseness=.75]
	(.75,.5) to [out=90,in=90] (.15,0);
\draw[linestyle,looseness=.75]
	(.15,0) to [out=-90,in=90] (.75,-.5);
\draw[linestylegraylight,coverline,looseness=.75]
	(-.75,-.5) to [out=90,in=-90] (.75,-.5)
	(.75,.5) to [out=-90,in=90] (-.75,.5);
\end{tikzpicture}}
\leadsto
\cb{\begin{tikzpicture}
\draw[linestyle,looseness=1.25]
	(-.5,-.35) to [out=-90,in=90] (.5,-.35)
	(-.5,.35) to [out=-90,in=90] (.5,.35);
\draw[linestylegraylight,coverline,looseness=1.25]
	(-.5,-.35) to [out=90,in=-90] (.5,-.35)
	(-.5,.35) to [out=90,in=-90] (.5,.35);
\end{tikzpicture}}
\end{equation}
That homotopy shows that the bean is homotopic to the composite $S^3 \xra{q \cdot 2} S^2 \xra{\inc} \Sigma \RP^2$.  (Here the expression $q \cdot 2$ refers to the composite $S^3 \xra{2} S^3 \xra{q} S^2$; this would more typically be written and thought of as ``$2q$", but we want to be sure to distinguish it from the composite $S^3 \xra{q} S^2 \xra{2} S^2$, which is $q \cdot 4$.) The lemma follows, because $\pi_3(S^2) \xra{\inc_*} \pi_3(\Sigma \RP^2)$ is surjective.
\end{proof}
\begin{remark}
This lemma is of more significance that it might appear.  It means that the null homotopy $R$ of $2S$ does not, \emph{indeed cannot be chosen to}, satisfy the standard snake relation.
\end{remark}
The sequence of deformations in equation~\ref{htpybean2e} defines a homotopy, call it $H$, between the Radford bean map, which we have previously identified with the suspension of the antipodal projection $\Sigma p : S^3 \ra \Sigma \RP^2$, and twice the figure-8 map $\inc_*(q \cdot 2) : S^3 \ra \Sigma \RP^2$.  Note well that this homotopy could have been chosen differently---there is a $\pi_4(\Sigma\RP^2) = \ZZ/4$ ambiguity in the choice.  The homotopy $H$ also provides a distinguished homotopy between the corresponding constructions, namely the Radford bean construction and twice the figure-8 construction, that is for any pair $(S, R) \in \left(\Omega\Aut(X) \,_2\! \times_{\Omega\Aut(X)} \ast\right)$, it provides a homotopy between the map $p(S,R): S^2 \ra \Aut(X)$ and the map $2 q(S): S^2 \ra \Aut(X)$.


Equipped with the homotopy $H$ between $p(S,R)$ and $2 q(S)$, we can complete the description of the map from $SO(3)$ to the automorphisms of $X$.  Recall from Theorem~\ref{thm-so3action} that a homotopy $SO(3)$ action on a 3-type $X$ is defined by a triple $(S,R,W)$, where $S: S^1 \ra \Aut(X)$, $R: D^2 \ra \Aut(X)$ is a null homotopy of the square of $S$, and $W: D^3 \ra \Aut(X)$ is a null homotopy of $q(S)$.
\begin{proposition} \label{prop-so3map}
Given a homotopy $SO(3)$ action on a 3-type $X$, defined by the triple $(S,R,W)$, the associated map of spaces $SO(3) \ra \Aut(X)$ is given on the 1-, 2-, and 3-cells of $SO(3)$ by, respectively, the map $S: S^1 \ra \Aut(X)$, the map $R: D^2 \ra \Aut(X)$, and the map $M: D^3 \ra \Aut(X)$, a null homotopy of $p(S,R): S^2 \ra \Aut(X)$ defined by $M:=H \cdot (2 W)$.
\end{proposition}
\nid Here $2 W: D^3 \ra \Aut(X)$ refers to the null homotopy of $2 q(S)$ obtained by concatenating the null homotopy $W: D^3 \ra \Aut(X)$ with itself, and $H$ is the homotopy defined above between $p(S,R)$ and $2 q(S)$.  We leave the proof to the reader, with the note that the ambiguity in the choice of the homotopy $H$ does not affect the homotopy class of the resulting map $SO(3) \ra \Aut(X)$ precisely because of the fact that for a homotopy $SO(3)$ action, the map $\frac{\eta q}{2}: S^3 \ra \Aut(X)$ is null. \CD{I didn't really check this Proposition carefully, but I can't see any other way it could be.}

\CDcomm{The following discussion of the Radford-8 actually doesn't appear immediately needed for anything, but I bet it will come in handy at some point so we should include it somewhere.}
Now consider the map $e: S^3 \ra \Sigma\RP^2$ defined by the (Pontryagin-Thom) construction on the following picture:
\begin{equation} \label{eq-rad8}
\cb{\begin{tikzpicture}
\draw[linestyle,looseness=1.25]
(0,0) to [out=-90,in=-120] (.75,0)
	to [out=60,in=90] (1.5,0);
\draw[linestyle,coverline,looseness=1.25]
(1.5,0) to [out=-90,in=-60] (.75,0)
	to [out=120,in=90] (0,0);
\end{tikzpicture}}
\end{equation}
Equivalently, this picture defines a map $Map(\RP^2,\Aut(X)) \xra{e} \Omega^2 \Aut(X)$.  We refer to this construction~\ref{eq-rad8} as the ``Radford-8".
\begin{lemma}
The map $e: S^3 \ra \Sigma \RP^2$ is homotopic to the composite of the \emph{negative} Hopf map $-q: S^3 \ra S^2$ with the inclusion $S^2 \ra \Sigma \RP^2$.
\end{lemma}
\begin{proof}
The desired equation $e = -q$ follows if $e + q = 0$, which follows if $e - q = -2q = 2q$.  It therefore suffices to show that the union of construction~\eqref{eq-rad8} and the negative of construction~\eqref{eq-q} is homotopic to construction~\eqref{eq-bean}.  This is indeed the case:
\begin{equation} \nn
\cb{\begin{tikzpicture}
\draw[linestyle,looseness=1.25]
(0,0) to [out=-90,in=-120] (.5,0)
	to [out=60,in=90] (1,0);
\draw[linestyle,coverline,looseness=1.25]
(1,0) to [out=-90,in=-60] (.5,0)
	to [out=120,in=90] (0,0);
\draw[linestylegraylight,looseness=1.25]
(1,-.5) to [out=-90,in=-60] (.5,-.5)
	to [out=120,in=90] (0,-.5);
\draw[linestyle,coverline,looseness=1.25]
(0,-.5) to [out=-90,in=-120] (.5,-.5)
	to [out=60,in=90] (1,-.5);
\end{tikzpicture}}
\leadsto
\cb{\begin{tikzpicture}
\draw[linestyle,looseness=1.25]
(0,0) to [out=90,in=60] (-.5,0)
	to [out=-120,in=-90] (-1,0);
\draw[linestylegraylight,looseness=1.25]
(0,-.5) to [out=-90,in=-60] (-.5,-.5)
	to [out=120,in=90] (-1,-.5);
\draw[linestyle,coverline,looseness=1.25]
(0,0) to [out=-90,in=90] (-.15,-.25)
	to [out=-90,in=90] (0,-.5);
\draw[linestyle,coverline,looseness=1.25]
(-1,0) to [out=90,in=120] (-.5,0)
	to [out=-60,in=90] (-.3,-.25)
	to [out=-90,in=60] (-.5,-.5)
	to [out=-120,in=-90] (-1,-.5);
\end{tikzpicture}}
=
\cb{\begin{tikzpicture}
\draw[linestyle,looseness=.75]
	(.5,-.25) to [out=90,in=-90] (0,0)
	to [out=90,in=-90] (.5,.25)
	to [out=90,in=-90] (0,.5);
\draw[linestyle,looseness=1]
	(0,.5) to [out=90,in=90] (1,0)
	(0,-.5) to [out=-90,in=-90] (1,0);
\draw[linestylegraylight,looseness=.75]
	(.5,-.25) to [out=-90,in=90] (0,-.5);
\end{tikzpicture}}
=
\cb{\begin{tikzpicture}
\draw[linestyle,out looseness=.75,in looseness=1]
	(-.5,-.25) to [out=-90,in=-90] (.5,0)
	(-.5,.25) to [out=90,in=90] (.5,0);
\draw[linestyle,looseness=.75]
	(-.5,-.25) to [out=90,in=-90] (0,0)
	(-.5,.25) to [out=-90,in=90] (0,0);
\end{tikzpicture}}
\end{equation}
\end{proof}

%\CDcomm{This next paragraph is correct, but I think no longer needed for anything. Thus we used the notation H for something else in the text} \nid Thanks to this lemma we can \emph{pick} a homotopy, call it $H$, between $e: S^3 \ra \Sigma \RP^2$ and $\inc_*(-q): S^3 \ra \Sigma \RP^2$.  Note well that this homotopy is not well defined---there is a $\pi_4(\Sigma\RP^2) = \ZZ/4$ ambiguity in the choice.  The homotopy $H$ also provides a distinguished homotopy between construction~\eqref{eq-rad8} and the negative of construction~\eqref{eq-q}, and therefore, for any pair $(S, R) \in \left(\Omega\Aut(X) \,_2\! \times_{\Omega\Aut(X)} \ast\right)$, between the map $e(S,R): S^2 \ra \Aut(X)$ and the map $-q(S): S^2 \ra \Aut(X)$.


\begin{remark}
The above considerations determined an alternative 4-truncation of $BSO(3)$, namely $(S^2 \cup_2 e_3) \cup_{e} e_4 \cup_{\frac{\eta q}{2}} e_5$---this 4-truncation is, of course, canonically homotopy equivalent to the earlier 4-truncation.  % Use the homotopy between e and -q and a reflection to construct the homotopy equivalence to the earlier 4-truncation.  Note the 5-cell ensures that the choice of homotopy didn't affect the resulting homotopy equivalence.
\end{remark}

\begin{remark}
Though we have already seen the fact that the Radford-8 $e: S^3 \ra \Sigma\RP^2$ is 4-torsion in $\pi_3(\Sigma \RP^2) = \ZZ/4$, we cannot resist including a proof: \CDcomm{insert proof here?}
\begin{equation} \nn
...
\end{equation}
\end{remark}


\section{Homotopy fixed points}

\CDcomm{This section describes the meaning of a homotopy fixed point explicitly for each group.}

The classifying space $B\Aut(X)$ carries a universal fibration with fiber $X$, namely $U_X := E\Aut(X) \times_{\Aut(X)} X \ra B\Aut(X)$.  \CD{Should we say something about when that really is a classifying fibration?  Can the universal fibration really be written as that bundle? Ie is $E\Aut(X)$ kosher?  If not just cite the existence of such a fibration?}  Given a homotopy action of $G$ on a space $X$, that is a map $\alpha: BG \ra B\Aut(X)$, there is an associated fibration $EG(X):=\alpha^\ast(U_X) \ra BG$, again with fiber $X$.

\begin{definition}
A homotopy fixed point of a homotopy $G$-action on $X$ is a section of the fibration $X \ra EG(X) \ra BG$.
\end{definition}

\begin{remark}
In case $G$ is an honest group with an honest action on $X$, the fibration $EG(X) \ra BG$ can be taken to be the bundle projection $EG \times_G X \ra BG$.  The data of a section of that bundle is precisely the same as the data of a $G$-equivariant map $EG \ra X$, that is of a homotopy fixed point in the ordinary sense.
\end{remark}

\begin{proposition} \label{prop-hofptrunc}
Given a homotopy action $BG \ra B\Aut(X)$ on a 3-type $X$, the natural map from the space of sections of $X \ra EG(X) \ra BG$ to the space of sections of that fibration over the 4-truncation ${}_4 BG \ra BG$, is a homotopy equivalence.
\end{proposition}
\begin{proof}
Given a section over ${}_4 BG$, extensions of the section to $BG$ are controlled by the obstruction groups $H^k(BG;\pi_{k-1} X)$ for $k \geq 5$, and choices of extensions are controlled by the groups $H^k(BG;\pi_k X)$ for $k \geq 5$.  All of those groups vanish.
\end{proof}

In this section, for brevity we sometimes use ``is" to mean ``is homotopy equivalent to".

\subsection{For $\Omega S^2$ and $\Quad$ actions}

Recall that an $\Omega S^2$ action on a 3-type $X$, which a priori is a based map $\alpha: S^2 \ra B\Aut(X)$ may instead be viewed as a based map $S: S^1 \ra \Aut(X)$.

\begin{proposition}
The space of homotopy fixed points of the $\Omega S^2$ action on a 3-type $X$, given by $S: S^1 \ra \Aut(X)$, is the space of pairs $(x,P)$ where $x \in X$ is a point and $P: D^2 \ra X$ is a null homotopy of the loop $S(x): S^1 \ra X$.
\end{proposition}
\begin{proof}
By definition a homotopy fixed point of the action is a section of the fibration $X \ra E\Omega S^2(X) \ra S^2$.  We can describe this fibration more explicitly as $X \cup_{X \times S^1} X \times D^2 \ra \ast \cup_{S^1} D^2 = S^2$, where the map $X \times S^1 \ra X \times D^2$ is the inclusion and the map $X \times S^1 \ra X$ is the adjoint of $\tilde{S}: S^1 \ra \Aut(X)$; here $\tilde{S}$ is the pointwise inverse of $S$.  %To be really pedantic, the clutching function S: S^1 \ra \Aut(X) constructs the bundle classified by \Sigma S^1 \ra \Sigma\Aut(X) \ra B\Aut(X) --- that map _is_ the image under the equivalence \Omega \Aut(X) to \Omega^2 B\Aut(X).

The point $x$ provides the section on the basepoint $\ast$ of $\ast \cup_{S^1} D^2$, which provides the section $S(x): S^1 \ra X$ on $\partial D^2$; the null homotopy $P$ extends the section over the 2-cell of $S^2$.
\end{proof}

Given a $Quad$ action $\alpha: BQuad \ra B\Aut(X)$ on a 3-type $X$, there is of course an induced $\Omega S^2$ action.

\begin{proposition} \label{prop-quadvs2fp}
The homotopy fixed points of a $Quad$ action on a 3-type $X$ are the same as the homotopy fixed points of the underlying $\Omega S^2$ action on $X$.
\end{proposition}
\begin{proof}
By Proposition~\ref{prop-hofptrunc}, to understand the homotopy fixed points of $Quad$ it suffices to consider sections of the bundle $X \ra EQuad(X) \ra BQuad$ over the 4-truncation of $BQuad$, which by Proposition~\ref{prop-quadtrunc} is $S^2 \cup_{\eta q} e_5$.  An $\Omega S^2$ homotopy fixed point is a section of the bundle over the 2-skeleton of ${}_4 BQuad$.  The obstruction to extending this section over the 5-cell of ${}_4 BQuad$ lives in $H^5({}_4 BQuad;\pi_4(X)) = 0$ and the homotopy classes of choices of lift is a torsor for $H^5({}_4 BQuad;\pi_5(X)) = 0$.
\end{proof}
\CD{That proof could probably be phrased better.}


\subsection{For $SO(2)$ actions}


\subsection{For $\Omega \Sigma \RP^2$ and $Orp(3)$ actions}


\subsection{For $SO(3)$ actions}

\subsection{For $Spin(3)$ actions} %and other groups?
.
\appendix
\renewcommand{\thetheorem}{A.\arabic{theorem}}
\setcounter{theorem}{0}


\section{Structure groups of 3-manifolds} \label{sec-lft-struc}

\CDcomm{The point of this appendix is to explain why we only care about n-types (see the Proposition), and to motivate from the perspective of structured field theory why we'd look at spaces over $BSO(3)$, and to introduce the most important of these spaces ($BOrpo(3)$, $BQuad$, $BSO(2)$, $S^2$, $Spin(3)$, etc) and their relationships.}

\subsection{Structure groups for discrete field theories}

The second half of the cobordism hypothesis says that TFTs with other topological structures correspond to homotopy fixed points of certain group actions on the space of fully dualizable objects.  Thus different topological structures correspond to different kinds of structures on dualizable tensor categories.

\begin{definition}
An \emph{structure} for $n$-dimensional vector bundles is a space $T$ together with a fibration $f: T \ra BO(n)$.  A $T$-structure on an n-dimensional vector bundle $V$ on $X$ is a lift of the classifying map $X \xra{V} BO(n)$ along $f$ to $T$.
\end{definition}

Because we are concentrating on structures on manifolds of low dimension, the full homotopy type of the structure spaces is not relevant---it suffices to keep track only of the low-dimensional homotopical information of $BO(3)$ and the other structure spaces.  Recall that the $k$-homotopy type (or simply the ``$k$-type") of a space $X$ refers to the homotopical information contained in the $k$-Postnikov section $P_k X$ of the space; the $k$-Postnikov section $P_k X$ is the initial space in the category of spaces $Y$ equipped with a map $X \ra Y$ inducing an isomorphism on $\pi_i$ for $i \leq k$ and such that $\pi_i(Y) = 0$ for $i > k$. Said directly, the $k$-Postnikov section is the space you get by killing all the homotopy groups above dimension $k$. %!%\CD{Prefer to define $P_k X$ otherwise?}

The following proposition explains how much of the homotopy type of the structure space you need to keep track of in order to recover the homotopy type of the space of structured field theories:
\begin{proposition} \label{prop-fixedpointntypes}
Suppose $S \ra BO(n)$ and $T \ra BO(n)$ are $n$-dimensional structures, and $p: S \ra T$ is a map of structures, that is a map over $BO(n)$.  Let $\cC$ be a symmetric monoidal $n$-category.  The map of structures induces a map of bordism categories $\phi: \mathrm{SBord}_0^n \ra \mathrm{TBord}_0^n$, which in turn induces a map $\phi^*$ between the spaces of $\cC$-valued field theories.  If the structure map $p: S \ra T$ is an isomorphism on $\pi_i$ for $i \leq n$ and is surjective on $\pi_{n+1}$, then the induced map $\phi^*: \Hom(\mathrm{TBord}_0^n,\cC) \ra \Hom(\mathrm{SBord}_0^n,\cC)$ is a homotopy equivalence of spaces.
\end{proposition}

\nid We leave the proof to the reader.  Note that this proposition certainly fails to be true if the discrete symmetric monoidal $n$-category $\cC$ is replaced by a symmetric monoidal $(\infty,n)$-category with nontrivial homotopy in its spaces of $n$-morphisms.  As we are concentrating on 3-dimensional bordism categories and using the discrete 3-category $\TC$ as our target, it suffices to understand the 4-types of our various structure spaces.

\CD{This proposition is phrased in terms of the field theories, but maybe should be also explicitly translated into the fixed point view.  Unclear if / where we will state the cobordism hypothesis.}

\subsection{The `sub'-structures}


We will be primarily interested in five structures for 3-dimensional vector bundles, namely \emph{orientation}, \emph{combing}, \emph{oriented $p_1$ structure}, \emph{quadratic structure}, and \emph{framing}.  The structure spaces for an orientation, combing, and framing are respectively $BSO(3)$, $BSO(2)$, and $BSO(1) = *$, all with the obvious maps to $BO(3)$.  The structure spaces for oriented $p_1$ structure and quadratic structure are as follows:
\begin{definition}
The following two spaces, together with the obvious maps to $BO(3)$, define the aforementioned bundle structures:
\begin{itemize}
\item[Orpo:] $BOrpo(3) := \hofib(BSO(3) \ra BSO \xra{p_1} K(\ZZ,4))$
\item[Quad:] $BQuad := \hofib(BSO(2) = BU(1) \xra{c_1^2} K(\ZZ,4))$
\end{itemize}
\end{definition}

\begin{remark}
``Orpo structure" is short for ``oriented $p_1$ structure", and sometimes goes under the name ``$p_1$ structure" in the literature.  We reserve the name ``$p_1$ structure" for the structure group where only $p_1$ has been killed, rather than $w_1$ and $p_1$.  The terminology ``quadratic" refers to the fact that the single k-invariant of the structure space $BQuad$ is the fundamental quadratic map.
\end{remark}


We now precisely identify the 4-homotopy type of the four nontrivial structure spaces under consideration.  The framing structure space is of course contractible.  The homotopy type of the combing structure space is simply $K(\ZZ,2)$, so in particular that is also its 4-type.

\begin{proposition}
The following fibrations constitute the Postnikov towers of the 4-homotopy types of the structure spaces $BSO(3)$, $BOrpo(3)$, and $BQuad$.
\begin{align}
K(\ZZ,4) \ra & \; BSO(3)_4 \ra K(\ZZ/2,2) \overset{\gamma}{\dashrightarrow} K(\ZZ,5) \nn \\
K(\ZZ/4,3) \ra & \; BOrpo(3)_4 \ra K(\ZZ/2,2) \overset{\delta}{\dashrightarrow} K(\ZZ/4,4) \nn \\
K(\ZZ,3) \ra & \; BQuad_4 \ra K(\ZZ,2) \overset{c_1^2}{\dashrightarrow} K(\ZZ,4) \nn
\end{align}
\nid Here the dashed arrow indicates the k-invariant of the preceding fibration.  The class $\delta \in H^4(K(\ZZ/2,2);\ZZ/4)$ is a generator, and the class $\gamma \in H^5(K(\ZZ/2,2);\ZZ) = \ZZ/4$ is, up to sign, the integral Bockstein of $\delta$. % The sign is not well defined, because you can always modify \gamma by an automorphism of K(Z,5); ie this is the best you can do.
\end{proposition}

\begin{proof}
The quadratic structure group is defined as the homotopy fiber of the square of the first chern class, so in that case there is nothing to prove.

The first four homotopy groups of $BSO(3)$ are $\pi_1 = 0$, $\pi_2 = \ZZ/2$, $\pi_3 = 0$, $\pi_4 = \ZZ$.  The Postnikov tower of the 4-type of $BSO(3)$ is therefore as indicated for some k-invariant in $H^5(K(\ZZ/2,2);\ZZ) = \ZZ/4$.  This k-invariant isn't trivial: the cohomology group $H^5(K(\ZZ/2,2) \times K(\ZZ,4);\ZZ)$ has 4-torsion elements, while the group $H^5(BSO(3);\ZZ)$ does not.  Similarly, the k-invariant isn't twice a generator of $H^5(K(\ZZ/2,2);\ZZ)$: the group $H^4(K(\ZZ/2,2) \ltimes_{2\gamma} K(\ZZ,4);\ZZ/2)$ is rank two, while $H^4(BSO(3);\ZZ/2)$ is rank one.  The k-invariant for the 4-type of $BSO(3)$ is therefore a generator, as claimed.

The structure space $BOrpo(3)$ is defined as the homotopy fiber of the composite $BSO(3) \ra BSO \xra{p_1} K(\ZZ,4)$; as before we refer to this composite simply as $p_1$.  In the appendix, it is shown that this map $p_1: BSO(3) \ra K(\ZZ,4)$ is a generator of $H^4(BSO(3);\ZZ)$, and, somewhat non-obviously, that it induces multiplication by \emph{four} in homotopy.  The first four homotopy groups of $BOrpo(3)$ are therefore $\pi_1 = 0$, $\pi_2 = \ZZ/2$, $\pi_3 = \ZZ/4$, $\pi_4 = 0$.  The Postnikov tower of the 4-type of $BOrpo(3)$ is therefore as indicated for some k-invariant in $H^4(K(\ZZ/2,2);\ZZ/4) = \ZZ/4$.  The Postnikov tower of a space is functorial, so the existence of a map $BOrpo(3) \ra BSO(3)$ inducing an isomorphism on $\pi_2$ forces the k-invariant of $BOrpo(3)$ to be a generator $\delta \in H^4(K(\ZZ/2,2);\ZZ/4)$, and in fact shows that $\pm \beta_{\ZZ} \delta = \gamma$. \CD{Probably that proof could be improved, it's just the first thing that came to mind.}
\end{proof}

We have the following diagram of structure spaces:
\begin{equation} \nn
\xymatrix{
BQuad_{K(\ZZ,2)}^{K(\ZZ,3)} \ar[r] \ar[d] & BSO(2)_{K(\ZZ,2)} \ar[d] \\
BOrpo(3)_{K(\ZZ/2,2)}^{K(\ZZ/4,3)} \ar[r] & BSO(3)_{K(\ZZ/2,2)}^{K(\ZZ,4)}
}
\end{equation}
Here we have indicated in the subscripts and superscripts the stages of the respective Postnikov towers of the homotopy 4-types.  The two horizontal maps exist by definition, and the righthand vertical map is the standard inclusion.  The lefthand vertical map may be constructed as follows.  At the first layer of the Postnikov tower, the map $\pi: K(\ZZ,2) \ra K(\ZZ/2,2)$ is induced by the projection $\ZZ \ra \ZZ/2$.  To produce a map $BQuad \ra BOrpo(3)$ it suffices to factor the composite $K(\ZZ,2) \xra{\pi} K(\ZZ/2,2) \xra{\delta} K(\ZZ/4,4)$ through the map $K(\ZZ,2) \xra{c_1^2} K(\ZZ,4)$; this is accomplished by (plus or minus) the projection $K(\ZZ,4) \ra K(\ZZ/4,4)$.  %%% {This is because the path loop construction is functorial so any map K(\ZZ,4) \ra K(\ZZ/4,4) induces a map of fibrations, which pulls back across c_1^2, resp \delta, to give a map of fibrations whose total spaces are BQuad and BOrpo(3) respectively.  To check the factorization, just look at the induced maps in cohomology.

By construction, the four maps of structure spaces above induce the `obvious' maps on homotopy, namely $\pi_2(BQuad \ra BSO(2))$ is an isomorphism, $\pi_2(BQuad \ra BOrpo(3))$ and $\pi_3(BQuad \ra BOrpo(3))$ are both surjective, $\pi_2(BSO(2) \ra BSO(3))$ is surjective, and $\pi_2(BOrpo(3) \ra BSO(3))$ is an isomorphism.  

\begin{remark}
There is a map of structure spaces $S^2 \ra BQuad$ which is an isomorphism on $\pi_i$ for $i \leq 3$ and a surjection on $\pi_4$.  The homotopy fixed points of a $Quad$ action and the corresponding $\Omega S^2$ action are therefore the same, by Proposition~\ref{prop-fixedpointsntypes} or as explicitly recorded in Proposition~\ref{prop-quadvs2fp}.
\end{remark}

\subsection{The `super'-structures}.

\CDcomm{This subsection discusses those structure groups that factor through $BSpin(3)$. }

In this section we describe a few auxiliary structure groups, including the spin and string groups.

\begin{definition}
The following spaces, together with the obvious maps to $BO(3)$, define 3-dimensional bundle structures:
\begin{itemize}
\item[Spin:] $BSpin(3) := \hofib(BSO(3) \xra{w_2} K(\ZZ/2,2))$
\item[Spinpo:] $BSpinpo(3) := \hofib(BSpin(3) \xra{p_1} K(\ZZ,4)$
\item[String:] $BString(3) := \hofib(BSpin(3) \xra{(p_1)/2} K(\ZZ,4))$
\item[hString:] $BhString(3) := \hofib(BSpin(3) \xra{(p_1)/4} K(\ZZ,4))$
\end{itemize}
\end{definition}

\nid In order for the definitions of $BString(3)$ and $BhString(3)$ to be sensible, the composite $BSpin(3) \ra BSO(3) \ra BSO \xra{p_1} K(\ZZ,4)$, which we have referred to simply as $p_1$, must be uniquely 4-divisible; we will see presently that this is indeed the case.  Note that for our purposes it suffices to define the first Pontryagin class $p_1 \in H^4(BSO;\ZZ)$ as a generator.

\begin{proposition} \label{prop-H4}
The diagram of structure spaces
\begin{equation} \nn
\xymatrix@R=5pt{
& BSpin \ar[dr] & & \\
BSpin(3) \ar[ur] \ar[dr] && BSO \ar[r]^{p_1} & K(\ZZ,4) \\
& BSO(3) \ar[ur] & &
}
\end{equation}
induces on fourth homotopy and fourth homology respectively the maps
\begin{equation} \nn
\xymatrix@R=5pt{
\save[]+<-1cm,0cm>*{\pi_4:}\restore & \ZZ \ar[dr]^{1} & & \\
\ZZ \ar[ur]^{2} \ar[dr]_{1} && \ZZ \ar[r]^{2} & \ZZ \\
& \ZZ \ar[ur]_{2} & & \\
&&& \\
\save[]+<-1cm,0cm>*{H^4:}\restore & \ZZ \ar[dl]_{2} & & \\
\ZZ && \ZZ \ar[ul]_{2} \ar[dl]^{1} & \ZZ \ar[l]_{1} \\
& \ZZ \ar[ul]^{4} & &
}
\end{equation}
\end{proposition}

We emphasize from this proposition two facts that were directly needed for the structure group computations in Section~\ref{sec-lft-struc}: the composite $BSO(3) \ra BSO \xra{p_1} K(\ZZ,4)$ is a generator of $H^4(BSO(3);\ZZ)$, which we refer to also as $p_1$, and the map $\pi_4(BSO(3)) \xra{\pi_4(p_1)} \pi_4(K(\ZZ,4))$ is multiplication by $4$.

\begin{proof}
The defining long exact sequences show that $\pi_4(BSpin(3) \ra BSO(3))$ and $\pi_4(BSpin \ra BSO)$ are isomorphisms.  The map $\pi_4(BSpin(3) \ra BSpin)$ is the same as $\pi_3(Spin(3) \ra Spin(5))$; the map $\pi_3(Spin(3) \ra Spin(4))$ is diagonal, and $\pi_3(Spin(4) \ra Spin(5))$ is addition. %!% Is there a nice clean way to see those two facts?
The map $\pi_4(BSO(3) \ra BSO)$ follows by commutativity, and the map $H^4(BSpin(3) \ra BSpin)$ follows by Hurewicz.

That the map $H^4(BSpin \ra BSO)$ is multiplication by 2 can be seen in the integral cohomology Serre spectral sequence for the fibration $BSpin \ra BSO \ra K(\ZZ/2,2)$, given knowledge of the cohomology of $BSO$ and $BSpin$.  Similarly that $H^4(BSpin(3) \ra BSO(3))$ is multiplication by 4 follows from the fibration $BSpin(3) \ra BSO(3) \ra K(\ZZ/2,2)$.  That $H^4(BSO(3) \ra BSO)$ is an isomorphism follows by commutativity.

The map $H^4(BSO \xra{p_1} K(\ZZ,4))$ is an isomorphism by definition of $p_1$.  The map $H^4(BSpin \xra{p_1} K(\ZZ,4))$ is thus multiplication by 2.  By Hurewicz it follows that $\pi_4(BSpin \xra{p_1} K(\ZZ,4)$ is multiplication by 2, and therefore that $\pi_4(BSO \xra{p_1} K(\ZZ,4))$ is also multiplication by 2.
\end{proof}


Figure~\ref{fig-structuregroups} organizes some of the 3-dimensional structure groups we have considered.
\begin{figure}[!ht]
\begin{equation} \nn
\xymatrix@!R@!C@R=10pt@C=5pt{
&& \ast \ar[dl] \ar[ddrr] && \\
& BString(3) \ar[dl] &&& \\
BSpinpo(3) \ar[ddrr] \ar[dddd] &&&& BQuad \ar[ddll] \ar[dddd] \\
&&&& \\
&& BOrpo(3) \ar[dddd] && \\
&&&& \\
BSpin(3) \ar[ddrr] &&&& BSO(2) \ar[ddll] \\
&&&& \\
&& BSO(3) &&
}
\end{equation}
\caption{Three-dimensional structure groups} \label{fig-structuregroups}
\end{figure}

\subsection{Miscellaneous structures}

\begin{definition}
The following spaces, together with the obvious maps to $BO(3)$, define 3-dimensional bundle structures:
\begin{itemize}
\item[2-Frame:] $BTwFrame(3) := \hofib(BSO(3) \ra BSpin(6)) = B\Omega(Spin(6)/SO(3))$
\item[StFrame:] $BStFrame(3) := \hofib(BO(3) \ra BO(\infty)) = B\Omega(O(\infty)/O(3))$
\end{itemize}
\end{definition}

\nid In the definition of 2-Frame, the inclusion from $SO(3)$ to $Spin(6)$ is the canonical lift of the map $SO(3) \ra SO(6)$ sending $\phi$ to $\phi \oplus \phi$; hence the idea of framing twice the vector bundle, hence the name.  


\section{Homotopy types of 3-dimensional structure groups}

\CDcomm{The point of this appendix is to say a bunch about the cohomology and homotopy groups of all the structure spaces encountered.}  

In this appendix we record some computations regarding the homotopy and homology of the various structure spaces.
%!% Need to credit Andre with helping understand all this.

In Table~\ref{table-structurecalc}, we record the low-dimensional homology and homotopy of the structure spaces we have considered.  All these structure spaces are simply connected.  The remaining homology and homotopy groups follow from the above proposition, along with a number of classical computations, using a flurry of interconnected spectral sequences.  

\begin{table}[!ht]
\begin{tabular}{|r||c|c|c|c||c|c|c|c|}
\hline
& $H^2$ & $H^3$ & $H^4$ & $H^5$ & $\pi_2$ & $\pi_3$ & $\pi_4$ & $\pi_5$ \\
\hline
$BSO(2)$ % is K(Z,2)
& $\ZZ$ & $0$ & $\ZZ$ & $0$ 
& $\ZZ$ & $0$ & $0$ & $0$ \\
$BQuad$ % K(Z,3) --> BQuad --> BSO(2)
& $\ZZ$ & $0$ & $0$ & $0$
& $\ZZ$ & $\ZZ$ & $0$ & $0$ \\
$BSO(3)$ % SO(3) --> * --> BSO(3) for Z and Z/2
& $0$ & $\ZZ/2$ & $\ZZ$ & $0$
& $\ZZ/2$ & $0$ & $\ZZ$ & $\ZZ/2$ \\ % Mimura
$BSO$
& $0$ & $\ZZ/2$ & $\ZZ$ & $\ZZ/2$ % cell structure from mod 2 cohom
& $\ZZ/2$ & $0$ & $\ZZ$ & $0$ \\ % Bott
$BOrpo(3)$ % K(Z,3) --> BOrpo(3) --> BSO(3)
& $0$ & $\ZZ/2$ & $0$ & $0$ 
& $\ZZ/2$ & $\ZZ/4$ & $0$ & $\ZZ/2$  \\
$Spin(6)/SO(3)$ % SO(3) --> Spin(6) --> Spin(6)/SO(3) in htpy
& $0$ & $\ZZ/2$ & $0$ & $\ZZ$ % Spin(6) is torsion free (Mimura), from this can use SO(3) --> Spin(6) --> Spin(6)/SO(3) integrally and mod 2 to calculation the cohom.
& $\ZZ/2$ & $\ZZ/4$ & $0$ & \CDcomm{$??$} \\
$BSpin(3)$ % S^3 --> * --> BSpin(3)
& $0$ & $0$ & $\ZZ$ & $0$
& $0$ & $0$ & $\ZZ$ & $\ZZ/2$ \\
$BSpin$
& $0$ & $0$ & $\ZZ$ & $0$ % BZ/2 --> BSpin --> BSO painful
& $0$ & $0$ & $\ZZ$ & $0$ \\ % Z/2 --> Spin --> SO
$BSpinpo(3)$
& $0$ & $0$ & $\ZZ/4$ & $0$
& $0$ & $\ZZ/4$ & $0$ & $\ZZ/2$ \\
$BString(3)$ % K(Z,3) --> BString(3) --> BSpin(3)
& $0$ & $0$ & $\ZZ/2$ & $0$
& $0$ & $\ZZ/2$ & $0$ & $\ZZ/2$ \\
$O(\infty)/O(3)$ % O/O(3) --> BSO(3) --> BSO, htpy first shows quotient is 2-connected and that pi_3 is torsion, implies cohom starts in degree 4; knowing that the gen of H^4 of BSO pulls back to a gen of BSO(3) forces H^4 of the quotient to be Z/2, killing the H^5 of BSO.  Knowing also H^6 of BSO pulls back isomorphically to BSO(3) forces the rest of the cohomology.
& $0$ & $0$ & $\ZZ/2$ & $0$
& $0$ & $\ZZ/2$ & $0$ & $\ZZ/2$ \\
$BhString(3)$ % K(Z,3) --> BhString(3) --> BSpin(3)
& $0$ & $0$ & $0$ & $0$
& $0$ & $0$ & $0$ & $\ZZ/2$ \\
\hline 
\end{tabular} \vspace*{8pt}
\caption{Low-dimensional cohomology and homotopy of structure spaces.} \label{table-structurecalc}
\end{table} \CD{Someone should recheck the cohomology of $Spin(6)/SO(3)$.}
% H^5 and pi_5 are necessary for computing H^4 and pi_5 of some of the other groups, so you can't just delete those whole columns.

The maps on $H^{\leq 4}$ and $\pi_{\leq 4}$, induced by maps between the structure spaces in the table, are as expected, aside from those already recorded in Proposition~\ref{prop-H4}.  Specifically, in addition to the homotopy of the maps between $BQuad$, $BSO(2)$, $BOrpo(3)$, and $BSO(3)$ recorded in Section~\ref{sec-lft-struc}, we have the following: $H^2(BQuad \ra BSO(2))$ is iso; $H^4(BSO(2) \ra BSO(3))$ is iso; $H^3(BSO(3) \ra BSO)$ is iso; $H^3(BOrpo(3) \ra BSO(3))$ is iso; $H^4(BString(3) \ra BSpinpo(3))$ and $H^4(BSpinpo(3) \ra BSpin(3))$ are surjective; $\pi_3(BString(3) \ra BSpinpo(3))$ is injective; and $\pi_3(BSpinpo(3) \ra BOrpo(3))$ is iso. %%% The map BSO(2) --> BSO(3) is from knowing that the mod 2 reduction of p_1 is w_2^2.

Note that in this table $BSO$ and $BSpin$ are not, per se, 3-dimensional structure spaces, as they do not map to $BO(3)$, and we mention them only as useful intermediary spaces.  One could define a \emph{stable} $BSO$-structure or $BSpin$-structure on a 3-dimensional vector bundle $M \ra BO(3)$ as a lift to $BSO$, respectively $BSpin$, of the composite $M \ra BO(3) \ra BO$; these might be called a stable orientation and stable spin structure respectively.  Note, though, that the pullback $\holim(BO(3) \ra BO \la BSO)$ is $BSO(3)$ and the pullback $\holim(BO(3) \ra BO \la BSpin)$ is $BSpin(3)$; it follows that a stable orientation is the same data as an orientation, and a stable spin structure is the same data as a spin structure.  (Though we will not need this fact, the same is true for orpo structures and stable orpo structures.)  This is certainly not the relation between stable framing, which is an $O(\infty)/O(3)$-structure, and framing, which is a $*$-structure.

The above table suggests three homotopical coincidences.  The most apparent is that $BhString(3)$ is 4-connected.  The natural forgetful map from the set of homotopy classes of framings of a 3-manifold $M$ to the set of homotopy classes of $hString(3)$-structures on $M$ is therefore an isomorphism.  We also see a coincidence between the low-dimensional invariants of $BOrpo(3)$ and $Spin(6)/SO(3)$ and between those of $BString(3)$ and $O(\infty)/O(3)$.

Orpo structures and 2-framings are different; for instance, the set of homotopy classes of parametrized families of the two structures on a vector bundle can be genuinely distinct.  However, as far individual tangential structures on a single 3-manifold are concerned, they are homotopically indistinguishable---this convergence no doubt has contributed to some of the confusion in the literature.
\begin{proposition}
There is a canonical 4-connected map $Spin(6)/SO(3) \ra BOrpo(3)$ lifting the natural map $Spin(6)/SO(3) \ra BSO(3)$.  In particular, for a closed 3-manifold $M$, there is a canonical bijection between the orpo structures and the 2-framings on $M$. 
\end{proposition}
\begin{proof}
From Proposition~\ref{prop-H4}, we know that $p_1$ is the generator of $H^4(BSO(3);\ZZ)$, and that $(p_1)/2$ is the generator of $H^4(BSpin(6);\ZZ)$---the fourth cohomology is of $BSpin(6)$ is stable.  Observe that the map $BSO(3) \ra BSpin(6)$ induces an isomorphism on $H^4$.  The first Pontryagin class $BSO(3) \xra{p_1} K(\ZZ,4)$ therefore factors as $BSO(3) \ra BSpin(6) \xra{(p_1)/2} K(\ZZ,4)$.  The composite $Spin(6)/SO(3) \ra BSO(3) \xra{p_1} K(\ZZ,4)$ then factors as $Spin(6)/SO(3) \ra BSO(3) \ra BSpin(6) \xra{(p_1)/2} K(\ZZ,4)$.  This last composite has a canonical null-homotopy, because the first two maps form a homotopy fibration.  There is therefore a canonical lift of $Spin(6)/SO(3) \ra BSO(3)$ to a map $Spin(6)/SO(3) \ra BOrpo(3)$.

As both $Spin(6)/SO(3) \ra BSO(3)$ and $BOrpo(3) \ra BSO(3)$ are isomorphisms on $\pi_2$, the lift is certainly an isomorphism on $\pi_2$.  The map $Spin(6)/SO(3) \ra BOrpo(3)$ induces a map from $Spin(6) = \hofib(Spin(6)/SO(3) \ra BSO(3))$ to $K(\ZZ,3) = \hofib(BOrpo(3) \ra BSO(3))$ that is an isomorphisms on $\pi_3$.  Since both $Spin(6) \ra Spin(6)/SO(3)$ and $K(\ZZ,3) \ra BOrpo(3)$ are surjective on $\pi_3$, it follows that $Spin(6)/SO(3) \ra BOrpo(3)$ is an isomorphism on $\pi_3$, as needed.
\end{proof} % "Observe" means "do some unpleasant spectral sequence computations"
% It is implicitly used---really used---in this proof that the third homotopy groups are the same, and that depends on $BSO(3) \xra{\widetilde{\phi \oplus \phi}} BSpin(6)$ being multiplication by $4$ on $\pi_4$ --- of course this is recorded in the table.

Similarly, as far as individual tangential structures on 3-manifolds are concerned, there is a homotopical coincidence between stably framed structures and string structures.
\begin{proposition}
There is a canonical 4-connected map $O(\infty)/O(3) \ra BString(3)$ lifting the natural map $O(\infty)/O(3) \ra BO(3)$.  In particular, for a closed 3-manifold $M$, there is a canonical bijection between the stable framings and the string structures on $M$.
\end{proposition} 
\begin{proof}
As $O(\infty)/O(3)$ is the same as $SO(\infty)/SO(3)$, and the map $O(\infty)/O(3) \ra BO(3)$ is the composite $SO(\infty)/SO(3) \ra BSO(3) \ra BO(3)$, it suffices to work with the special orthogonal quotient throughout.

Let $P$ denote the homotopy fiber of the generating map $K(\ZZ/2,2) \ra K(\ZZ,5)$.  By construction we have a homotopy fibration $BString(3) \ra BSO(3) \ra P$.  Moreover, appealing in part to the homotopy calculation in Proposition~\ref{prop-H4}, we see that the map $BSO(3) \ra P$ factors as $BSO(3) \ra BSO(\infty) \ra P$.  Note that the map $BSO(\infty) \ra P$ is an isomorphism on $\pi_4$.  The map $SO(\infty)/SO(3) \ra BSO(3) \ra P$ can now be factored as $SO(\infty)/SO(3) \ra BSO(3) \ra BSO(\infty) \ra P$.  The first two maps in this composite form a homotopy fibration and therefore admit a canonical null-homotopy.  There is therefore a canonical lift of $SO(\infty)/SO(3) \ra BSO(3)$ to a map $SO(\infty)/SO(3) \ra BString(3)$.

The map $SO(\infty)/SO(3) \ra BString(3)$ induces a map from $SO(\infty) = \hofib(SO(\infty)/SO(3) \ra BSO(3))$ to $\Omega P = \hofib(BString(3) \ra BSO(3))$.  This map $SO(\infty) \ra \Omega P$ is an isomorphism on $\pi_3$, because the map $BSO(\infty) \ra P$ was an isomorphism on $\pi_3$.  As both the maps $SO(\infty) \ra SO(\infty)/SO(3)$ and $\Omega P \ra BString(3)$ are surjective on $\pi_3$, it follows that $SO(\infty)/SO(3) \ra BString(3)$ is an isomorphism on $\pi_3$, as needed.
\end{proof}

This proposition associates to each stable framing of a 3-manifold $M$ a string structure.  This association may be thought of as follows: because the fibration $BString(3) \ra BO(3)$ is the pullback of the fibration $BString \ra BO$ along the inclusion $BO(3) \ra BO$, the data of a lift of $M \ra BO(3)$ to $BString(3)$ is the same as the data of a lift of $M \ra BO(3) \ra BO$ to $BString$.  That is, string structures and stable string structures are identical notions.  Now note that there is a natural forgetful map from stable framings to stable string structures, since a trivialization of the composition $M \ra BO(3) \ra BO$ certainly provides a lift of that composite to, in particular, $BString$.

% \CDcomm{The following proposition feels completely irrelevant now.}
%In the main text, we have concentrated on the structure groups $BQuad$, $BSO(2)$, $BOrpo(3)$, and $BSO(3)$.  As we have seen, $hString(3)$-structures are essentially just framings, but $String(3)$-structures appear to be rather different.  It turns out, though, that provided we focus on fusion categories over the complex numbers, all $\TC$-valued field theories are in fact string:
%\begin{proposition} \label{prop-string}
%Every framed local field theory $\cF : {\FrBord}_0^3 \ra \TC$ with target tensor categories over $\CC$ descends to a string local field theory $\overline{\cF} : \StrBord_0^3 \ra \TC$.
%\end{proposition}
%\nid This result is a direct analog of the corresponding result for conformal-net-valued local field theories, given in~\cite{0912.5307}, and the proof is the same.






\end{document}











